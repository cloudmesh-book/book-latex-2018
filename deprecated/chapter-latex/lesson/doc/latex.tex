\begin{fileremark}\url{https://github.com/cloudmesh/classes/blob/master/docs/source/lesson/doc/latex.rst}\end{fileremark}
\section{LaTeX}\label{latex}


\url{https://wch.github.io/latexsheet/latexsheet.pdf}

\url{http://tug.ctan.org/info/lshort/english/lshort.pdf}


\subsection{Introduction}\label{introduction}

Mastering a text processing system is an essential part of a
researcher's life. Not knowing how to use a text processing system can
slow down the productivity of research drastically.

The information provided here is not intended to replace one of the many
text books available about LaTeX. For the beginning, you might be just
fine with the documentation provided here. For serious users, I
recommend purchasing a book. Examples for books include

\begin{itemize}
\tightlist
\item
  LaTeX Users and Reference Guide, by Leslie Lamport
\item
  LaTeX an Introduction, by Helmut Kopka
\item
  The LaTeX Companion, by Frank Mittelbach
\end{itemize}

If you do not want to buy a book you can find a lot of useful
information in the LaTeX reference manual.

\subsection{LaTeX vs. X}\label{latex-vs.-x}

We will refrain from providing a detailed analysis on why we use LaTeX
in many cases versus other technologies. In general, we find that LaTeX:

\begin{itemize}
\tightlist
\item
  is incredibly stable
\item
  produces high-quality output
\item
  is platform independent
\item
  has lots of templates
\item
  has been around for many years so it works well
\item
  removes you from the pain of figure placements
\item
  focusses you on content rather tan the appearance of the paper
\item
  integrates well with code repositories such as git to write
  collaborative papers.
\item
  has superior bibliography integration
\item
  has a rich set of tools that make using LaTeX easier
\item
  authors do not play with layouts much so papers in a format are
  uniform
\end{itemize}

In case you need a graphical view to edit LaTeX or LateX exportable
files you also find AucTeX and Lyx.

\subsubsection{Word}\label{word}

Word is arguably available to many, but if you work on Linux you may be
out of luck. Also Word often focusses not on structure of the text but
on its apperance. Many students abuse Word and the documents in Word
become a pain to edit with multiple users. Recently Microsoft has
offered online services to collaborate on writing documents in groups
which work well. Integration with bibliography managers such as endnote
or Mendeley is possible.

However, we ran into issues whenever we use word:

\begin{itemize}
\tightlist
\item
  Word tends sometimes to crash for unknown reasons and we lost a lot of
  work
\item
  Word has some issues with the bibliography managers and tends to crash
  sometimes for unknown reasons.
\item
  Word is slow with integration to large bibliographies.
\item
  Figure placement in Word in some formats is a disaster and you will
  spend many hours to correct things just to find out that if you make
  small changes you have to spend additional many hours to get used to
  the new placement. We have not yet experienced a word version where we
  have not lost images. Maybe that has changed, so let us know
\end{itemize}

However, we highly recommend the collaborative editing features of Word
that work on a paragraph and not letter level. Thus saving is essential
so you do not block other people from editing the paragraph.

\subsubsection{Google Docs}\label{google-docs}

Unfortunately, many useful features got lost in the new google docs.
However, it is great to collaborate quickly online, share thoughts and
even write your latex documents together if you like (just copy your
work in a file offline and use latex to compile it ;-) )

The biggest issue we have with Google Docs is that it does not allow the
support of 2 column formats, that the bibliography integration is
non-existent and that paste and copy from web pages and images
encourages unintended plagiarism when collecting information without
annotations (LaTeX and Word are prone to this too, but we found from
experience that it tends to happen more with Google docs users.

\subsubsection{A Place for Each}\label{a-place-for-each}

When looking at the tools we find a place for each:

\begin{description}
\item[Google docs:]
Short meeting notes, small documents, quick online collaborations to
develop documents collaboratively at the same time.
\item[Word:]
Available to many, supports 2 column format, supports paragraph based
collaborative editing, Integrates with bibliography managers.
\item[LaTeX:]
Reduces failures, great offline editing, superior bibliography
management, superior image placement, runs everywhere. Great
collaborative editing with sharelatex, allows easy generation of
proceedings written by hundreds of people with shared index.
\item[The best choice for your class:]
LaTeX
\end{description}

\subsection{Editing}\label{editing}

\subsubsection{Emacs}\label{emacs}

The text editor emacs provides a great basis for editing TeX and LaTeX
documents. Both modes are supported. In addition there exists a color
highlight module enabling the color display of LaTeX and TeX commands.
On OSX aquaemacs and carbon emacs have build in support for LaTeX. Spell
checking is done with flyspell in emacs.

\subsubsection{Vi/Vim}\label{vivim}

Another popular editor is vi or vim. It is less feature rich but many
programmers ar using it. As it can edit ASCII text you can edit LaTeX.
With the LaTeX add-ons to vim, vim becomes similar powerful while
offering help and syntax highlighting for LaTeX as emacs does. (The
authors still prefer emacs)

\subsubsection{TeXshop}\label{texshop}

Other editors such as TeXshop are available which provide a more
integrated experience. However, we find them at times to stringent and
prefer editors such as emacs/

\subsection{LyX}\label{lyx}

We have made very good experiences with Lyx. You must assure that the
team you work with uses it consistently and that you all use the same
version.

Using the ACM templates is documented here:

\begin{itemize}
\tightlist
\item
  \url{https://wiki.lyx.org/Examples/AcmSiggraph}
\end{itemize}

On OSX it is important that you have a new version of LaTeX and Lyx
installed. As it takes up quite some space, you ma want to delete older
versions. The new version of LyX comes with the acmsigplan template
included. However on OSX and other platforms the .cls file is not
included by default. However the above link clearly documents how to fix
this.

\subsection{WYSIWYG locally}\label{wysiwyg-locally}

We have found that editors such as Lyx and Auctex provide very good
WYSIWYG alike features. However, we found an even easier way while using
skim, a pdf previewer, in conjunction with emacs and latexmk. This can
be achieved while using the following command assuming your latex file
is called `report.tex`:

\begin{verbatim}
latexmk -pvc -view=pdf report
\end{verbatim}

This command will update your pdf previewer (make sure to use skim)
whenever you edit the file report.tex and save it. It will maintain via
skim the current position, thus you have a real great way of editing in
one window, while seeing the results in the other.

Skim can be found at: \url{http://skim-app.sourceforge.net/}

\subsection{Installation}\label{installation}

\subsubsection{Local Install}\label{local-install}

Installing LaTeX is trivial, and is documented on the internet very
well. However, it requires sufficient space and time as it is a large
environment. A system such as TeX Live takes in full install about 5.5
GB. In addition to LaTeX we recommend that you install jabref and use it
for bibliography management.

Thus you will have the most of them on your system.

\begin{itemize}
\tightlist
\item
  pdflatex: the latex program producing pdf
\item
  bibtex: to create bibliographies
\item
  jabref: GUI application to bibtex files (\url{http://www.jabref.org/})
\end{itemize}

Make sure you check that these programs are there, for example with the
Linux commands:

\begin{verbatim}
which pdflatex
which bibtex
which jabref (on OSX you may have an icon for it)
\end{verbatim}

If these commands are missing, please install them. For the newest
documentation on instalation of LaTeX we recommend you look up the
instalation for your specific OS.

\subsubsection{Install on Ubuntu 16.04}\label{install-on-ubuntu-16.04}

The easiest way to install it on ubuntu is to use the terminal and type
in (make sure you have enough space):

\begin{verbatim}
sudo apt-get install texlive-full
\end{verbatim}

One of the best editors for LaTeX is emacs as you can also do
bibliography management with it and not just LaTeX. However, other
editors are avaialable including:

\begin{itemize}
\tightlist
\item
  Kile, TeXworks, JLatexEditor, Gedit LaTeX Plugin, TeXMaker
\end{itemize}

Please look up how to install them if you like to use them. TeXMaker is
popular, However I find the combination of emacs and latexmk superior.
TeXmaker is installed with:

\begin{verbatim}
sudo apt-get install texmaker
\end{verbatim}

Other instalations:

\begin{itemize}
\tightlist
\item
  kile is installed by default
\item
  \url{https://www.tug.org/texworks/} (Works on ubuntu, Windows, OSX)
\end{itemize}

\subsubsection{LaTeX for OSX}\label{latex-for-osx}

\begin{itemize}
\tightlist
\item
  \url{https://www.latex-project.org/get/}
\end{itemize}

\subsubsection{LaTeX for Windows}\label{latex-for-windows}

\begin{itemize}
\tightlist
\item
  \url{https://www.latex-project.org/get/}
\end{itemize}

\subsubsection{Online Services}\label{online-services}

\paragraph{Sharelatex}\label{sharelatex}

Those that like to use latex, but do not have it installed on their
computers may want to look at the following video:

Video: \url{https://youtu.be/PfhSOjuQk8Y}

Video with cc: \url{https://www.youtube.com/watch?v=8IDCGTFXoBs}

ShareLaTeX not only allows you to edit online, but allows you to share
your documents in a group of up to three. Licenses are available if you
need more than three people in a team.

\paragraph{Overleaf}\label{overleaf}

Overleaf.com is a collaborative latex editor. In its free version it has
a very limited disk space. However it comes with a Rich text mode that
allows you to edit the document in a preview mode. The free templates
provided do not include ACM template, put you are allowed to use the OSA
template.

Features of overleaf are documented at:
\url{https://www.overleaf.com/benefits}

\subsection{Paperia}\label{paperia}

We do not know where this service is located. However it offers similar
services as Sharelatex and Overleaf.

\begin{itemize}
\tightlist
\item
  \url{https://papeeria.com/}
\end{itemize}

\subsection{The LaTeX Cycle}\label{the-latex-cycle}

To create a PDF file from latex yo need to generate it following a
simple development and improvement cycle.

First, Create/edit ASCII source file with \texttt{file.tex} file:

\begin{verbatim}
emacs file.tex
\end{verbatim}

Create/edit bibliography file:

\begin{verbatim}
jabref refs.bib
\end{verbatim}

Create the PDF:

\begin{verbatim}
pdflatex file
bibtex file
pdflatex file
pdflatex file
\end{verbatim}

View the PDF:

\begin{verbatim}
open file
\end{verbatim}

It not only showcases you an example file in ACM 2 column format, but
also integrates with a bibliography. Furthermore, it provides a sample
Makefile that you can use to generate view and recompile, or even
autogenerate. A compilation would look like:

\begin{verbatim}
make
make view
\end{verbatim}

If however you want to do things on change in the tex file you can do
this automatically simply with:

\begin{verbatim}
make watch
\end{verbatim}

for make watch its best to use skim as pdf previewer

\subsection{Generating Images}\label{generating-images}

To produce high quality images the programs PowerPoint and omnigraffle
on OSX are recommended. When using powerpoint please keep the image
ratio to 4x3 as they produce nice size graphics which you also can use
in your presentations. When using other rations they may not fit in
presentations and thus you may increase unnecessarily your work. We do
not recommend vizio as it is not universally available and produces
images that in case you have to present them in a slide presentation
does not easily reformat if you do not use 4x3 aspect ratio.

Naturally, graphics should be provided in SVG or PDF format so they can
scale well when we look at the final PDF. Including PNG, gif, or jpeg
files often do not result in the necessary resolution or the files
become real big. For this reason we for example can also not recommend
tools such as tablaeu as they do not provide proper exports to high
quality publication formats. For interactive display such tool may be
good, but for publications it produces inferior formatted images.

\subsection{Bibliographies}\label{bibliographies}

LaTeX integrates very well with bibtex. There are several preformatted
styles available. It includes also styles for ACM and IEEE
bibliographies. For the ACM style we recommend that you replace
abbrv.bst with abbrvurl.bst, add hyperref to your usepackages so you can
also display URLs in your citations:

\begin{verbatim}
\bibliographystyle{IEEEtran}
\bibliography{references.bib}
\end{verbatim}

Then you have to run latex and bibtex in the following order:

\begin{verbatim}
latex  file
bibtex file
latex  file
latex  file
\end{verbatim}

or simply call make from our makefile.

The reason for the multiple execution of the latex program is to update
all cross-references correctly. In case you are not interested in
updating the library every time in the writing progress just postpone it
till the end. Missing citations are viewed as {[}?{]}.

Two programs stand out when managing bibliographies: emacs and jabref:

\begin{itemize}
\tightlist
\item
  \url{http://www.jabref.org/}
\end{itemize}

Other programs such as Mendeley, Zotero, and even endnote integrate with
bibtex. However their support is limited, so we recommend that you just
use jabref. Furthermore its free and runs on all platforms.

\subsubsection{jabref}\label{jabref}

Jabref is a very simple to use bibliography manager for LaTeX and other
systems. It can create a multitude of bibliography file formats and
allows upload in other online bibliography managers.

\begin{itemize}
\tightlist
\item
  Installation: Go to \url{http://www.jabref.org/} and click download
\item
  Video: \url{https://youtu.be/cMtYOHCHZ3k}
\item
  Video with cc: \url{https://www.youtube.com/watch?v=QVbifcLgMic}
\end{itemize}

\subsubsection{jabref and MSWord}\label{jabref-and-msword}

According to others it is possible to integrate jabref references
directly into MSWord. This has been conducted so far however only on a
Windows computer.

We have not tried this ourselves, but give it as a potential option.

Here are the steps the need to be done:

\begin{enumerate}
\def\labelenumi{\arabic{enumi}.}
\tightlist
\item
  Create the Jabref bibliography just like in presented in the Jabref
  video
\item
  After finishing adding your sources in Jabref, click
  File -\textgreater{} export
\item
  Name your bibliography and choose MS Office 2007(*.xml) as the file
  format. Remember the location of where you saved your file.
\item
  Open up your word document. If you are using the ACM template, go
  ahead and remove the template references listed under
  Section 7. References
\item
  In the MS Word ribbon choose `References'
\item
  Choose `Manage Sources'
\item
  Click `Browse' and locate/select your Jabref xml file
\item
  You should now see your references appear in the left side window.
  Select the references you want to add to your document and click the
  `copy' button to move them from the left side window to the right
  window.
\item
  Click the `Close' button
\item
  In the MS Word Ribbon, select `Bibliography' under the References tab
\item
  Click `Insert Bibliography' and your references should appear in the
  document
\item
  Ensure references are of Style: IEEE. Styles are located in the
  References tab under `Manage Sources'
\end{enumerate}

As you can see there is significant effort involve, so we do recommend
you use LaTeX as you can focus there on content rather than dealing with
complex layout decisions. This is especially true, if your papers have
figures or tables, or you need to add references.

\subsubsection{Other Reference Managers}\label{other-reference-managers}

Please note that you should first decide which reference manager you
like to use. In case you for example install zotero and mendeley, that
may not work with word or other programs.

\paragraph{Endnote}\label{endnote}

Endnote os a reference manager that works with Windows. Many people use
Endnote. However, in the past, Endnote has caused complications when
dealing with collaborative management of references. Its price is
considerable. We have lost many hours of work because of unstability of
Endnote in some cases. As a student, you may be able to use Endnote for
free at Indiana University.

\begin{itemize}
\tightlist
\item
  \url{http://endnote.com/}
\end{itemize}

\paragraph{Mendeley}\label{mendeley}

Mendeley is a free reference manager compatible with Windows Word 2013,
Mac Word 2011, LibreOffice, BibTeX. Videos on how to use it are
available at:

\begin{itemize}
\tightlist
\item
  \url{https://community.mendeley.com/guides/videos}
\end{itemize}

Installation instructions are available at

\begin{itemize}
\tightlist
\item
  \url{https://www.mendeley.com/features/reference-manager/}
\end{itemize}

When dealing with large databases, we found the integration of Mendeley
into word slow.

\paragraph{Zotero}\label{zotero}

Zotero is a free tool to help you collect, organize, cite, and share
your research sources. Documentation is available at

\begin{itemize}
\tightlist
\item
  \url{https://www.zotero.org/support/}
\end{itemize}

The download link is available from

\begin{itemize}
\tightlist
\item
  \url{https://www.zotero.org/}
\end{itemize}

We have limited experience with Zotero

\subsection{Slides}\label{slides}

Slides are best produced with the seminar package:

\begin{verbatim}
\documentclass{seminar}

\begin{slide}

    Hello World on slide 1

\end{slide}

The text between slides is ignored

\begin{slide}

    Hello World on slide 2

\end{slide}
\end{verbatim}

However, in case you need to have a slide presentation we recommend you
use ppt. Just paste and copy content from your PDF or your LaTeX source
file into the ppt.

\subsection{Links}\label{links}

\begin{itemize}
\tightlist
\item
  The
  \href{http://texdoc.net/texmf-dist/doc/latex/latex2e-help-texinfo/latex2e.pdf}{LaTeX
  Reference Manual} provides a good introduction to Latex.
\end{itemize}

LaTeX is available on all modern computer systems. A very good
installation for OSX is available at:

\begin{itemize}
\tightlist
\item
  \url{https://tug.org/mactex/}
\end{itemize}

However, if you have older versions on your systems you may have to
first completely uninstall them.

\subsection{Tips}\label{tips}

Including figures over two columns:

\begin{itemize}
\item
  \url{http://tex.stackexchange.com/questions/30985/displaying-a-wide-figure-in-a-two-column-document}
\item
  positioning figures with textwidth and columnwidth
  \url{https://www.sharelatex.com/learn/Positioning_images_and_tables}
\item
  An organization as the author. Assume the author is National Institute
  of Health and want to have the author show up, please do:

\begin{verbatim}
key= {National Institute of Health},
author= {{National Institute of Health}},
\end{verbatim}

  Please note the \{\{ \}\}
\item
  words containing `fi' or `ffi' showing blank places like below after
  recompiling it: find as nd efficiency as e ciency

  You copied from word or PDF ff which is actually not an ff, but a
  condensed character, change it to ff and ffi, you may find other such
  examples such as any non ASCII character. A degree is for example
  another common issue in data science.
\item
  do not use \textbar{} \& and other latex characters in bibtex
  references, instead use , and the word and
\item
  If you need to use \_ it is \_ but if you use urls leave them as is
\item
  We do recommend that you use sharelatex and jabref for writing papers.
  This is the easiest solution and beats in many cases MSWord as you can
  focus on writing and not on formatting.
\end{itemize}
