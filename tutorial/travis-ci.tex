\MDNAME\
%%%%%%%%%%%%%%%%%%%%%%%%%%%%%%%%%%%%%%%%%%%%%%%%%%%%%%%%%%%%%%%%%%%%%%%%%%%%%%%
% DO NOT MODIFY THIS FILE
%%%%%%%%%%%%%%%%%%%%%%%%%%%%%%%%%%%%%%%%%%%%%%%%%%%%%%%%%%%%%%%%%%%%%%%%%%%%%%%

\subsection{Travis CI}

hid-sp18-503

status: 70

\subsubsection{Repository Build status}

\href{https://travis-ci.org/seashiva94/hid-sp18-503}{\includegraphics[width=0.8\columnwidth]{images/hid-sp18-503.png}}

Travis Continuous Integration (Travis CI) is a continuous integration
service that synchronizes with github. It allows developers to automate
the process of testing an deploying code. Travis CI allows users to test
code for various different environments and allows the use of various
languages, which can be easily specified by adding a travis.yml file in
the repository.

We demonstrate how to use it in Python. To use Travis CI on github you
need to have a github account and a a travis account. Please go to the
Web sites and creat your account.

\begin{itemize}
\item
  \textless{}travis.com\textgreater{}
\item
  \textless{}github.com\textgreater{}
\end{itemize}

To access it within github, you need to allow travis to synchronize the
repositories. To do so please go to the configuration .

Next select the repository that you want to use with Travis CI on the
travis website. Now go to github and locally clone the repository.

\subsubsection{The .travis.yml file}

In the github repository you need to place a \texttt{.travis.yml} file.
The \texttt{.travis.yml} file describes to travis, which language is
used, what environments to test the code on, and to install any
dependencies that may be required.

The first line of the \texttt{yml} file specifies the language used

\begin{lstlisting}
language: python
\end{lstlisting}

Next we need to specify the versions of python we need to test the code
for. Here we need to specify the versions in the same manner as we use
to install the language, as travis creates a virtual environment for
each of these versions.

\begin{lstlisting}
python:
  - 3.6.4
\end{lstlisting}

Dependencies can be specified in the \texttt{yml} file under using he
\texttt{install:} tag. From our Python section \ref{??} we mentioned
that the use of pip is recommended for python. Thus let us assume we
have the requirements in a requirements.txt file within the git
repository. Hence we can use it in the install tag as follows:

\begin{lstlisting}
install:
  - pip install -r requirements.txt
\end{lstlisting}

If tests should be performed for different versions of libraries, this
can also be specified. Let us give an example for different versions of
django. Here we define the versions in the \texttt{env:} tag. Hence the
the install is run multiple times but for each environment we set.

\begin{lstlisting}
env:
  - django_version=1.10
  - django_version=1.11
install:
  - pip install Django==$django_version
\end{lstlisting}

To specify what commands to execute to run the tests, we use the
\texttt{script:} tag. If a make file is used for this purpose, this can
be done a follows however travis allows the use of pytest for this
purpose as well.

\begin{lstlisting}
script: make test
\end{lstlisting}

Travis allows various services to be run as required at the time of
testing. This can be enabled using the \texttt{services} tag in the
\texttt{travis.yml} file. To start a mongodb service for testing your
code on travis add the following to the \texttt{travis.yml} file

\begin{lstlisting}
services:
  - mongodb
\end{lstlisting}

To show, and keep track of the build status of the repository, travis
allows users to add build status badges in the markdown file. Once a
test is complete, click on the badge on the travis page, and select
markdown. Then copy the markdown code and add it to the required
markdown file. On subsequent pushes to the repository, this status keeps
updating.

\subsubsection{An Example}

\TODO{Arnav: the example is not linked. So we need to make sure that
the example is also copied. I am not sure what the best way of doing
this is, maybe we need a new example directory under the user
cloudmesh such as cloudmesh/travis-example. In any case link to Repo
is missing}

To showcase the use of travis, the \texttt{.travis.yml} file in this
repository uses a make file (test\_build) to test the swagger code.
Swagger auto generates basic tests that check for correct response codes
as specified in the swagger specification yml file. The
\texttt{.travis.yml} file specifies that mongodb is needed for testing.
The script tag in \texttt{.travis.yml} file invokes the make command
which in turn performs the following steps.

\begin{itemize}
\item
  It dowmloads swagger cli
\item
  It uses the swagger/Makefile to generate the swagger code and install
  requirements
\item
  It install test requirements provided by swagger
\item
  It uses pytest to perform tests that were generated by swagger
\item
  Finally it cleans the repository.
\end{itemize}

\subsubsection{Travis and docker}

\TODO{Help needed: we need an extension to showcase how to use docker}

\subsubsection{Class examples}

A significant example is located at

\begin{itemize}
\item
  \url{https://github.com/cloudmesh/book/blob/master/.travis.yml}
\end{itemize}

Here we have published an image to docker hub and use it to verify if
this LaTeX is compiling whenever we check in a change to git. Due to the
big image, this check will take significant amount of time.

The Dockerfile to create the image is located at

\begin{itemize}
\item
  \url{https://github.com/cloudmesh/book/tree/master/examples/docker/tex}
\end{itemize}

