\MDNAME\
%%%%%%%%%%%%%%%%%%%%%%%%%%%%%%%%%%%%%%%%%%%%%%%%%%%%%%%%%%%%%%%%%%%%%%%%%%%%%%%
% DO NOT MODIFY THIS FILE
%%%%%%%%%%%%%%%%%%%%%%%%%%%%%%%%%%%%%%%%%%%%%%%%%%%%%%%%%%%%%%%%%%%%%%%%%%%%%%%

\section{Python Apache Avro}

Min Chen (hid-sp18-405)

Apache Avro is a data serialization system, which provides rich data
structures, remote procedure call (RPC), a container file to store
persistent data and simple integration with dynamic languages
\cite{hid-sp18-405-tutorial-avro-doc}. Avro depends on schemas, which
are defined with JSON. This facilitates implementation in other
languages that have the JSON libraries. The key advantages of Avro are
schema evolution - Avro will handle the missing/extra/modified fields,
dynamic typing - serialization and deserialization without code
generation, untagged data - data encoding and faster data processing by
allowing data to be written without overhead.

The following steps illustrate using Avro to serialize and deserialize
data with example modified from Apache Avro 1.8.2 Getting Started
(Python) \cite{hid-sp18-405-tutorial-avro-python}.

\subsection{Download, Unzip and Install}

The zipped installation file \emph{avro-1.8.2.tar.gz} could be
downloaded from
\href{http://mirrors.ocf.berkeley.edu/apache/avro/avro-1.8.2/py/}{here}

To unzip, using linux:

\begin{lstlisting}
tar xvf avro-1.8.2.tar.gz
\end{lstlisting}

using MacOS:

\begin{lstlisting}
gunzip -c avro-1.8.2.tar.gz | tar xopf -
\end{lstlisting}

cd into the directory and install using the following:

\begin{lstlisting}
cd avro-1.8.2
python setup.py install
\end{lstlisting}

To check successful installation, import avro in python without error
message:

\begin{lstlisting}
python
>>> import avro
\end{lstlisting}

This above instruction is for Python2. The Python3 counterpart,
\emph{avro-python3-1.8.2.tar.gz} could be downloaded from
\href{http://mirrors.sonic.net/apache/avro/avro-1.8.2/py3/}{here} and
the unzip and install procedure is the same.

\subsection{Defining a schema}

Use a simple schema for students contributed in cloudmesh as an example:
paste the following lines into an empty text file with the name it
\emph{student.avsc}

\begin{lstlisting}
{"namespace": "cloudmesh.avro",
 "type": "record",
 "name": "Student",
 "fields": [
    {"name": "name", "type": "string"},
    {"name": "hid",  "type": "string"},
    {"name": "age", "type": ["int", "null"]},
    {"name": "project_name", "type": ["string", "null"]}
    ]
}
\end{lstlisting}

This schema defines a record representing a hypothetical student, which
is defined to be a record with the name \emph{Student} and 4 fields,
namely name, hid, age and project name. The type of each of the field
needs to be provided. If any field is optional, one could use the list
including \emph{null} to define the type as shown in age and project
name in the example schema. Further, a namespace \emph{cloudmesh.avro}
is also defined, which together with the name attribute defines the full
name of the schema (cloudmesh.avro.Student in this case).

\subsection{Serializing}

The following piece of python code illustrates serialization of some
data

\begin{lstlisting}
import avro.schema
from avro.datafile import DataFileWriter
from avro.io import DatumWriter

schema = avro.schema.parse(open("student.avsc", "rb").read())

writer = DataFileWriter(open("students.avro", "wb"), DatumWriter(), schema)
writer.append({"name": "Min Chen", "hid": "hid-sp18-405", "age": 29,
       "project_name": "hadoop with docker"})
writer.append({"name": "Ben Smith", "hid": "hid-sp18-309",
       "project_name": "spark with docker"})
writer.append({"name": "Alice Johnson", "hid": "hid-sp18-208", "age": 27})
writer.close()
\end{lstlisting}

The code does the following:

\begin{itemize}
\item
  Imports required modules
\item
  Reads the schema \emph{student.avsc} (make sure that the schema file
  is placed in the same directory as the python code)
\item
  Create a \emph{DataFileWriter} called writer, for writing serialized
  items to a data file on disk
\item
  Use \emph{DataFileWriter.append()} to add data points to the data
  file. Avro records are represented as Python dicts.
\item
  The resulting data file saved on the disk is named
  \emph{students.avro}
\item
  This above instruction is for Python2. If one is using Python3, change

\begin{lstlisting}
schema = avro.schema.parse(open("student.avsc", "rb").read())
\end{lstlisting}

  to:

\begin{lstlisting}
schema = avro.schema.Parse(open("student.avsc", "rb").read())
\end{lstlisting}

  since the method name has a different case in Python3.
\end{itemize}

\subsection{Deserializing}

The following python code illustrates deserialization

\begin{lstlisting}
from avro.datafile import DataFileReader
from avro.io import DatumReader

reader = DataFileReader(open("students.avro", "rb"), DatumReader())
for student in reader:
print (student)
reader.close()
\end{lstlisting}

The code does the following:

\begin{itemize}
\item
  Imports required modules
\item
  Use \emph{DatafileReader} to read the serilaized data file
  \emph{students.avro}, it is an iterator
\item
  Returns the data in a python dict
\end{itemize}

The output should look like:

\begin{lstlisting}
{'name': 'Min Chen', 'hid': 'hid-sp18-405', 
'age': 29, 'project_name': 'hadoop with docker'}
{'name': 'Ben Smith', 'hid': 'hid-sp18-309',
 'age': None, 'project_name': 'spark with docker'}
{'name': 'Alice Johnson', 'hid': 'hid-sp18-208',
 'age': 27, 'project_name': None}
\end{lstlisting}

\subsection{Resources}

\begin{itemize}
\item
  The steps and instructions are modified from
  \href{http://avro.apache.org/docs/1.8.2/gettingstartedpython.html}{Apache
  Avro 1.8.2 Getting Started (Python)}
  \cite{hid-sp18-405-tutorial-avro-python}.
\item
  The Avro Python library does not support code generation, while Avro
  used with Java supports code generation, see
  \href{http://avro.apache.org/docs/1.8.2/gettingstartedjava.html}{Apache
  Avro 1.8.2 Getting Started (Java)}
  \cite{hid-sp18-405-tutorial-avro-java}.
\item
  Avro provides a convenient way to represent complex data structures
  within a Hadoop MapReduce job. Details about Avro are documented in
  \href{http://avro.apache.org/docs/1.8.2/mr.html}{Apache Avro 1.8.2
  Hadoop MapReduce guide} \cite{hid-sp18-405-tutorial-avro-mapreduce}.
\item
  For more information on schema files and how to specify name and type
  of a record can be found at
  \href{http://avro.apache.org/docs/1.8.2/spec.html\#schema_record}{record
  specification} \cite{hid-sp18-405-tutorial-avro-record}.
\end{itemize}

