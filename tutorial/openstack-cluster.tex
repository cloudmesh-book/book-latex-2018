\MDNAME\
%%%%%%%%%%%%%%%%%%%%%%%%%%%%%%%%%%%%%%%%%%%%%%%%%%%%%%%%%%%%%%%%%%%%%%%%%%%%%%%
% DO NOT MODIFY THIS FILE
%%%%%%%%%%%%%%%%%%%%%%%%%%%%%%%%%%%%%%%%%%%%%%%%%%%%%%%%%%%%%%%%%%%%%%%%%%%%%%%

\section{Tutorial: OpenStack Cluster Setup}

Authored by spathan1985 (hid-sp18-516) -
https://github.com/cloudmesh-community/hid-sp18-516/blob/master/tutorial/openstackcluster.md

\subsection{Overview}

OpenStack is an Infrastructure as a Service platform that allows to
create and manage virtual environments. It is a free and opensource
software for cloud computing, that controls large pools of compute,
storage and networking resources throughout a datacenter managing it
either via dashboard or through the OpenStack
API\textasciitilde{}\cite{hid-sp18-516-www-openstack}. The OpenStack
software can be easily accessed via the OpenStack Web Interface called
\texttt{Horizon} provided by Chameleon. Chameleon provides an
installation of OpenStack version 2015.1 (Kilo) using the KVM
virtualization
technology\textasciitilde{}\cite{hid-sp18-516-www-chameleon}.

In this tutorial, we are going to create some instances using Openstack
Command Line Interface (CLI) and create a cluster from those instances.
For the purpose of this tutorial, we will just create 3 instances to
show how a cluster can be set-up.

\subsection{Creating the RC File}

The first step is to create the OpenStack RC. This can be done either
via the editor or via GUI. We will see how to create via the
editor\textasciitilde{}\cite{hid-sp18-516-las17handbook}.

\begin{itemize}
\item
  Open a terminal on your VM and create a directory

  \texttt{mkdir\ -p\ \textasciitilde{}/.cloudmesh/chameleon}
\item
  Download the open RC template from the following link using wget:

  \texttt{wget\ \ https://raw.githubusercontent.com/cloudmesh/book/master/examples/chameleon/cc-openrc.sh\ \ -O\ \textasciitilde{}/.cloudmesh/chameleon/cc-openrc.sh}
\item
  Open the cc-openrc.sh file and edit the following items and save it:

  \texttt{export\ CC\_PREFIX="\textless{}yourusername-hidnumber\textgreater{}"}

  \texttt{export\ OS\_USERNAME="\textless{}put\ your\ chameleon\ cloud\ username\ here\textgreater{}"}
\item
  After saving the file, you need to \texttt{source} the file by
  running:

  \texttt{source\ \textasciitilde{}/.cloudmesh/chameleon/cc-openrc.sh}
\end{itemize}

\subsection{Installing the OpenStack CLI}

The next step is to install the OpenStack CLI and novaclient to create
your instances. * To install the OpenStack client, run the command:

\begin{lstlisting}
`pip install python-openstackclient`
\end{lstlisting}

\begin{itemize}
\item
  Since the openstack client, has limited functionalities, it is also
  better to install \texttt{nova} API. This can be done by running:

  \texttt{pip\ install\ python-novaclient}
\end{itemize}

Now you are ready to create your instances on OpenStack. This tutorial
is executed on the following versions:

\texttt{(ENV2)\ spathan@spathan-VirtualBox:\textasciitilde{}/.cloudmesh/chameleon\$\ openstack\ -\/-version\ \ \ openstack\ 3.14.0}

\texttt{(ENV2)\ spathan@spathan-VirtualBox:\textasciitilde{}/.cloudmesh/chameleon\$\ nova\ -\/-version\ \ \ 10.1.0}

\subsection{Creating the Instance}

In order to create the instances, you need to check the flavor list to
be used. To do that, run the following command:

\texttt{openstack\ flavor\ list}

Example output:

\begin{lstlisting}
(ENV2) spathan@spathan-VirtualBox:~/.cloudmesh/chameleon$ openstack flavor list
+----+----------------+-------+------+-----------+-------+-----------+
| ID | Name           |   RAM | Disk | Ephemeral | VCPUs | Is Public |
+----+----------------+-------+------+-----------+-------+-----------+
| 1  | m1.tiny        |   512 |    1 |         0 |     1 | True      |
| 2  | m1.small       |  2048 |   20 |         0 |     1 | True      |
| 3  | m1.medium      |  4096 |   40 |         0 |     2 | True      |
| 4  | m1.large       |  8192 |   80 |         0 |     4 | True      |
| 5  | m1.xlarge      | 16384 |  160 |         0 |     8 | True      |
| 6  | storage.medium |  4096 | 2048 |         0 |     1 | True      |
| 7  | m1.xxlarge     | 32768 |  160 |         0 |     8 | True      |
| 8  | m1.xxxlarge    | 32768 |  160 |         0 |    16 | True      |
+----+----------------+-------+------+-----------+-------+-----------+
\end{lstlisting}

To check the images available, run the following command:

\texttt{openstack\ image\ list\ \textbar{}\ grep\ \textquotesingle{}Ubuntu16.04\textquotesingle{}}

Example output:
\texttt{(ENV2)\ spathan@spathan-VirtualBox:\textasciitilde{}/.cloudmesh/chameleon\$\ openstack\ image\ list\ \textbar{}\ grep\ \textquotesingle{}Ubuntu16.04\textquotesingle{}\ \textbar{}\ 1895dc93-7cd8-4b46-a252-4eb55b80859f\ \textbar{}\ CC-Ubuntu16.04\ \ \ \ \ \ \ \ \ \ \ \ \ \ \ \ \ \ \ \ \ \ \ \ \ \ \ \ \ \ \ \ \ \ \ \textbar{}\ active\ \textbar{}\ \textbar{}\ 01a6d6d1-e692-4e86-b776-dbb387c64958\ \textbar{}\ CC-Ubuntu16.04-20160610\ \ \ \ \ \ \ \ \ \ \ \ \ \ \ \ \ \ \ \ \ \ \ \ \ \ \textbar{}\ active\ \textbar{}\ \textbar{}\ 58e3df11-a1ff-4b6b-bbdf-3835dd0b20a2\ \textbar{}\ CC-Ubuntu16.04-20161010\ \ \ \ \ \ \ \ \ \ \ \ \ \ \ \ \ \ \ \ \ \ \ \ \ \ \textbar{}\ active\ \textbar{}\ \textbar{}\ cce5fe00-ef3d-4d42-9ec9-39ab57b5488e\ \textbar{}\ CC-Ubuntu16.04-20161214\ \ \ \ \ \ \ \ \ \ \ \ \ \ \ \ \ \ \ \ \ \ \ \ \ \ \textbar{}\ active\ \textbar{}\ \textbar{}\ 2808416b-f1b6-4429-9a4f-a03a1b0da4c3\ \textbar{}\ CC-Ubuntu16.04-20170410\ \ \ \ \ \ \ \ \ \ \ \ \ \ \ \ \ \ \ \ \ \ \ \ \ \ \textbar{}\ active\ \textbar{}\ \textbar{}\ c80567e8-12f9-44fc-8335-f62831ee73a6\ \textbar{}\ CC-Ubuntu16.04-20171028\ \ \ \ \ \ \ \ \ \ \ \ \ \ \ \ \ \ \ \ \ \ \ \ \ \ \textbar{}\ active\ \textbar{}\ \textbar{}\ 9d5745c1-caee-4a42-ae4e-4c810589c749\ \textbar{}\ CC-Ubuntu16.04-20171110\ \ \ \ \ \ \ \ \ \ \ \ \ \ \ \ \ \ \ \ \ \ \ \ \ \ \textbar{}\ active\ \textbar{}\ \textbar{}\ 30adb881-f921-404d-a2de-cf530795b2aa\ \textbar{}\ CC-Ubuntu16.04-20171221\ \ \ \ \ \ \ \ \ \ \ \ \ \ \ \ \ \ \ \ \ \ \ \ \ \ \textbar{}\ active\ \textbar{}\ \textbar{}\ 1decb35b-dd9b-4d88-b9e2-62bdbe68124a\ \textbar{}\ das\_arun\_deeplearning\_CC-Ubuntu16.04\ \ \ \ \ \ \ \ \ \ \ \ \ \textbar{}\ active\ \textbar{}}

To check the security groups list, run the following command. We are
going to use the \texttt{default} security group that's already created
for us.

\begin{lstlisting}
(ENV2) spathan@spathan-VirtualBox:~/.cloudmesh/chameleon$ openstack security group list
+--------------------------------------+-----------------+--------------------------------+-----------+
| ID                                   | Name            | Description                    | Project   |
+--------------------------------------+-----------------+--------------------------------+-----------+
| 00555968-5dad-4544-88c9-91cff2a390e6 | default         | Default security group         | CH-819337 |
| c3b48218-80f5-4243-919b-581a3a7103e6 | rrufael-default | Security group rrufael-default | CH-819337 |
+--------------------------------------+-----------------+--------------------------------+-----------+
\end{lstlisting}

In order to access the instance from your own VM, you need to create a
keypair. If you have keypair already available on your VM and if you
would like to use it, go ahead. Else you can create it using
\texttt{ssh-keygen\ -t\ rsa} command. To add your public key on
OpenStack, run the following command:

\texttt{nova\ keypair-add\ -\/-pub-key\ \textasciitilde{}/.ssh/\textless{}yourpublickeyname\textgreater{}.pub\ \$CC\_PREFIX-key}

To list the keypair added, run the following command:

\texttt{nova\ keypair-list}

You should see your \$CC\_PREFIX-key added.

Now to create the instances, all 3 instances can be launched at a time
using the \texttt{nova\ boot} command:

\texttt{nova\ boot\ -\/-image\ CC-Ubuntu16.04\ -\/-min-count\ 1\ -\/-max-count\ 3\ -\/-key-name\ \$CC\_PREFIX-key\ -\/-flavor\ m1.small\ \$CC\_PREFIX}

To see the list of instances created, type the following command:

\texttt{nova\ list} or \texttt{openstack\ server\ list}

Example output:

\begin{lstlisting}
(ENV2) spathan@spathan-VirtualBox:~/.cloudmesh/chameleon$ nova list
+--------------------------------------+-------------------------------+---------+------------+-------------+----------------------------+
| ID                                   | Name                          | Status  | Task State | Power State | Networks                   |
+--------------------------------------+-------------------------------+---------+------------+-------------+----------------------------+
| 1420b190-5995-4cae-a4c5-985691a20a63 | rcarmick-hid404-Ubuntu16.04.3 | SHUTOFF | -          | Shutdown    | CH-819337-net=192.168.0.19 |
| 8fb6376e-48e0-44cc-aa36-c89ead7d5092 | rrufael-703-01                | SHUTOFF | -          | Shutdown    | CH-819337-net=192.168.0.11 |
| fb858c85-8702-4ae2-8e4f-33ce672c817b | spathan-516-1                 | ACTIVE  | -          | Running     | CH-819337-net=192.168.0.33 |
| 10f5f689-0975-4e8b-9e1e-0fdbe968b086 | spathan-516-2                 | ACTIVE  | -          | Running     | CH-819337-net=192.168.0.34 |
| 7e443fc1-f1ce-4821-8710-e461a1f4a27f | spathan-516-3                 | ACTIVE  | -          | Running     | CH-819337-net=192.168.0.35 |
+--------------------------------------+-------------------------------+---------+------------+-------------+----------------------------+
\end{lstlisting}

\subsection{Associating Floating IP}

In order to ssh to the instance from your own VM, you need to assign a
floating IP to the instance. For this we first need to generate floating
IPs.

To check if there are any floating IPs already available, run the
command:

\texttt{openstack\ floating\ ip\ list}

If there are floating IPs available, you can choose to associate them
with your instance. If the floating ip list does not return any output,
then you need to generate the floating IPs in order to associate them to
your instances.

To generate the floating IPs, run the following command:

\texttt{openstack\ floating\ ip\ create\ ext-net}

As we have 3 instances to associate with, we need to generate 3 floating
IPs. Hence run the above command 3 times.

Now when you run the \texttt{openstack\ floating\ ip\ list} command, you
should see 3 floating IPs created. Example output:

\begin{lstlisting}
(ENV2) spathan@spathan-VirtualBox:~/.cloudmesh/chameleon$ openstack floating ip list
+--------------------------------------+---------------------+------------------+------+--------------------------------------+-----------+
| ID                                   | Floating IP Address | Fixed IP Address | Port | Floating Network                     | Project   |
+--------------------------------------+---------------------+------------------+------+--------------------------------------+-----------+
| 09a3393e-6a7c-41b7-b2ad-cc00d518c8b0 | 129.114.111.86      | None             | None | dbf29083-ff90-406e-b69a-b1d93b8f0a2d | CH-819337 |
| e9d85da5-ede0-4460-bd1e-908901c1d9ce | 129.114.111.90      | None             | None | dbf29083-ff90-406e-b69a-b1d93b8f0a2d | CH-819337 |
| f7cf14ef-7601-4172-93c2-d259b10cd47e | 129.114.111.85      | None             | None | dbf29083-ff90-406e-b69a-b1d93b8f0a2d | CH-819337 |
+--------------------------------------+---------------------+------------------+------+--------------------------------------+-----------+
\end{lstlisting}

Before we associate the floating IP to the instance, we need to assign
port to our instance ID. To do this, run the following command:

\texttt{openstack\ port\ list\ -\/-device-id\ \textless{}InstanceID\textgreater{}}

Example output:

\begin{lstlisting}
(ENV2) spathan@spathan-VirtualBox:~/.cloudmesh/chameleon$ openstack server list
+--------------------------------------+-------------------------------+---------+----------------------------+----------------+-----------+
| ID                                   | Name                          | Status  | Networks                   | Image          | Flavor    |
+--------------------------------------+-------------------------------+---------+----------------------------+----------------+-----------+
| 7e443fc1-f1ce-4821-8710-e461a1f4a27f | spathan-516-3                 | ACTIVE  | CH-819337-net=192.168.0.35 | CC-Ubuntu16.04 | m1.small  |
| 10f5f689-0975-4e8b-9e1e-0fdbe968b086 | spathan-516-2                 | ACTIVE  | CH-819337-net=192.168.0.34 | CC-Ubuntu16.04 | m1.small  |
| fb858c85-8702-4ae2-8e4f-33ce672c817b | spathan-516-1                 | ACTIVE  | CH-819337-net=192.168.0.33 | CC-Ubuntu16.04 | m1.small  |
| 1420b190-5995-4cae-a4c5-985691a20a63 | rcarmick-hid404-Ubuntu16.04.3 | SHUTOFF | CH-819337-net=192.168.0.19 | CC-Ubuntu16.04 | m1.medium |
| 8fb6376e-48e0-44cc-aa36-c89ead7d5092 | rrufael-703-01                | SHUTOFF | CH-819337-net=192.168.0.11 | CC-Ubuntu16.04 | m1.medium |
+--------------------------------------+-------------------------------+---------+----------------------------+----------------+-----------+

(ENV2) spathan@spathan-VirtualBox:~/.cloudmesh/chameleon$ openstack port list --device-id fb858c85-8702-4ae2-8e4f-33ce672c817b
+--------------------------------------+------+-------------------+-----------------------------------------------------------------------------+--------+
| ID                                   | Name | MAC Address       | Fixed IP Addresses                                                          | Status |
+--------------------------------------+------+-------------------+-----------------------------------------------------------------------------+--------+
| 4a9de659-c83f-4818-ab37-a066f7529e3d |      | fa:16:3e:36:4b:e5 | ip_address='192.168.0.33', subnet_id='508eb1f9-1237-4c02-a3ff-313960136c8a' | ACTIVE |
+--------------------------------------+------+-------------------+-----------------------------------------------------------------------------+--------+
\end{lstlisting}

Now we can associate the floating IP to the instance by running the
following command:

\texttt{openstack\ floating\ ip\ set\ -\/-port\ \textless{}PortID\textgreater{}\ \textless{}floatingIP\textgreater{}}

Example Output:

\begin{lstlisting}
(ENV2) spathan@spathan-VirtualBox:~/.cloudmesh/chameleon$ openstack floating ip set --port 4a9de659-c83f-4818-ab37-a066f7529e3d 129.114.111.85

(ENV2) spathan@spathan-VirtualBox:~/.cloudmesh/chameleon$ openstack server list
+--------------------------------------+-------------------------------+---------+--------------------------------------------+----------------+-----------+
| ID                                   | Name                          | Status  | Networks                                   | Image          | Flavor    |
+--------------------------------------+-------------------------------+---------+--------------------------------------------+----------------+-----------+
| 7e443fc1-f1ce-4821-8710-e461a1f4a27f | spathan-516-3                 | ACTIVE  | CH-819337-net=192.168.0.35                 | CC-Ubuntu16.04 | m1.small  |
| 10f5f689-0975-4e8b-9e1e-0fdbe968b086 | spathan-516-2                 | ACTIVE  | CH-819337-net=192.168.0.34                 | CC-Ubuntu16.04 | m1.small  |
| fb858c85-8702-4ae2-8e4f-33ce672c817b | spathan-516-1                 | ACTIVE  | CH-819337-net=192.168.0.33, 129.114.111.85 | CC-Ubuntu16.04 | m1.small  |
| 1420b190-5995-4cae-a4c5-985691a20a63 | rcarmick-hid404-Ubuntu16.04.3 | SHUTOFF | CH-819337-net=192.168.0.19                 | CC-Ubuntu16.04 | m1.medium |
| 8fb6376e-48e0-44cc-aa36-c89ead7d5092 | rrufael-703-01                | SHUTOFF | CH-819337-net=192.168.0.11                 | CC-Ubuntu16.04 | m1.medium |
+--------------------------------------+-------------------------------+---------+--------------------------------------------+----------------+-----------+
\end{lstlisting}

Make sure you associate all your instances to the floating IP in the
same manner. As you can see from the \texttt{openstack\ server\ list}
command above, there are 2 addresses shown for the
\texttt{spathan-516-1} instance. One is the Private IP and the another
is the floating IP that you associated your instance with.

\subsection{Access the Instances from your own VM}

Once you have associated all instances with floating IPs and all your
instances are up and running. You should now be able to SSH to all your
instances using the floating IP from the terminal of your own VM. This
is because of the SSH-Keypair you had added in the earlier steps.

Run the following commands to SSH to the instances:

\texttt{ssh\ cc@\textless{}floatingIPofInstance1\textgreater{}}

\texttt{ssh\ cc@\textless{}floatingIPofInstance2\textgreater{}}

\texttt{ssh\ cc@\textless{}floatingIPofInstance3\textgreater{}}

\subsection{Generating Passwordless SSH Key-Pairs on Instances}

The next step is to copy the public keys on one instance into the
authorized\_keys file of the other instances. * For this, open 3
terminals. run the commands from your own VM:

\texttt{ssh\ cc@\textless{}floatingIPofInstance1\textgreater{}}\\
\texttt{ssh\ cc@\textless{}floatingIPofInstance2\textgreater{}}~\\
\texttt{ssh\ cc@\textless{}floatingIPofInstance3\textgreater{}}

\begin{itemize}
\item
  Open the public key file contents of instance1(id\_rsa\_01.pub) using
  cat command:
\end{itemize}

\texttt{cat\ \textasciitilde{}/.ssh/id\_rsa\_01.pub}

\begin{lstlisting}
Copy the contents and paste it into the ~/.ssh/authorized_keys file of
instance2 and instance3.
\end{lstlisting}

\begin{itemize}
\item
  Then, open the public key file contents of instance2(id\_rsa\_02.pub)
  using cat command:
\end{itemize}

\texttt{cat\ \textasciitilde{}/.ssh/id\_rsa\_02.pub}

\begin{lstlisting}
Copy the contents and paste it into the ~/.ssh/authorized_keys file of
instance1 and instance3.
\end{lstlisting}

\begin{itemize}
\item
  Next, open the public key file contents of instance3(id\_rsa\_02.pub)
  using cat command:
\end{itemize}

\texttt{cat\ \textasciitilde{}/.ssh/id\_rsa\_03.pub}

\begin{lstlisting}
Copy the contents and paste it into the ~/.ssh/authorized_keys file of
instance1 and instance2.
\end{lstlisting}

Once this is done, the authorized\_keys file on one instance should have
the public keys of all the other instances, including the public key of
your own VM (since you had added that earlier while creating the
instance).

\subsection{Accessing instances}

Now, you should be able to ssh between these instances using either
floatingIP or the PrivateIP. But be sure to provide the private key of
the instance that you are logging in from. Run the following commands to
verify: * First login to one of the instance e.g.~instance1 using

\texttt{ssh\ cc@\textless{}floatingIPofInstance1\textgreater{}} from
your own VM.

\begin{itemize}
\item
  Now ssh to another instance from instance1, either 2 or 3 using:
\end{itemize}

\texttt{ssh\ -i\ \textless{}privateKeyofInstance1:e.g.id\_rsa\_01\textgreater{}\ \ \ \ cc@\textless{}floatingIPofInstance2/PrivateIPofInstance2\textgreater{}}

\begin{itemize}
\item
  The prompt should now change to instance2.
\item
  Try to ssh from instance1 to instance3, from instance2 to
  instance1/instance3 or from instance3 to instance1/instance2.
\item
  It should login in successfully to these instances.
\item
  Instead of providing the private\_key, you can avoid that by starting
  ssh\_agent process and then adding the private key using ssh add. To
  do this, run the following commands on all instances with the correct
  private\_key name:
\end{itemize}

\texttt{eval\ "\$(ssh-agent\ -s)"}

\texttt{ssh-add\ \textasciitilde{}/.ssh/id\_rsa}

Now you should be able to login from one instance to another using ssh
and without providing the private key.

Our goal was to set up a cluster using the instances created on
Openstack. Now you are able to successfully login to each of these
instances and the login from one instance to another is also possible.

\subsection{References}

\url{https://docs.openstack.org/python-openstackclient/pike/}

\url{https://docs.openstack.org/mitaka/cli-reference/common/cli_install_openstack_command_line_clients.html\#install-the-openstack-client}

\url{https://docs.openstack.org/mitaka/cli-reference/common/cli_set_environment_variables_using_openstack_rc.html}

\url{https://docs.openstack.org/python-openstackclient/pike/cli/command-objects/floating-ip.html}

\url{https://docs.openstack.org/releasenotes/python-novaclient/queens.html}

\url{https://www.tecmint.com/create-deploy-and-launch-virtual-machines-in-openstack/}

\url{https://www.tecmint.com/ssh-passwordless-login-using-ssh-keygen-in-5-easy-steps/}

\subsection{Running it on Container}

git clone the tutorial directory on your machine

Make sure you have your \texttt{cc-openrc.sh} file in the same directory
as your Dockerfile

To build the image, simply run the following command from your terminal:

\texttt{sudo\ docker\ build\ -t\ openstack-tutorial\ .}

To login to the container in interactive mode, run the command:

\texttt{sudo\ docker\ run\ -p\ 8080:8080\ -\/-rm\ -it\ openstack-tutorial\ bash}

Once you login, you have to source your cc-openrc.sh file:

\texttt{source\ cc-openrc.sh}

Now you will be able to execute all the openstack commands on container
and create the cluster as shown above.

Example output:

\begin{lstlisting}
(ENV2) spathan@spathan-VirtualBox:~/github/cloudmesh-community/hid-sp18-516/tutorial$ sudo docker run -p 8080:8080 --rm -it openstack-tutorial bash
root@c41a045021ff:/app# ls
cc-openrc.sh
root@c41a045021ff:/app# source cc-openrc.sh 
Please enter your OpenStack Password: 
root@c41a045021ff:/app# nova list
+--------------------------------------+-----------------+---------+-------------+-------------+---------------------------------------------+
| ID                                   | Name            | Status  | Task State  | Power State | Networks                                    |
+--------------------------------------+-----------------+---------+-------------+-------------+---------------------------------------------+
| 1420b190-5995-4cae-a4c5-985691a20a63 | rcarmick-404-01 | SHUTOFF | -           | Shutdown    | CH-819337-net=192.168.0.19                  |
| e27a1cc6-a698-4257-8561-636ed7bb1d08 | rlambadi-514-01 | SHUTOFF | -           | Shutdown    | CH-819337-net=192.168.0.58, 129.114.111.59  |
| 70b6f5cb-686f-49f9-8ac0-caf3ad0907a5 | rlambadi-514-02 | SHUTOFF | -           | Shutdown    | CH-819337-net=192.168.0.54, 129.114.33.98   |
| d37292ad-7172-45f6-bef4-e3e795c98e05 | rrufael-703-01  | SHUTOFF | powering-on | Shutdown    | CH-819337-net=192.168.0.55, 129.114.110.186 |
| 78538d78-1bb0-434f-8605-413997cc55ef | rrufael-703-02  | SHUTOFF | -           | Shutdown    | CH-819337-net=192.168.0.59, 129.114.110.10  |
| 01621d47-ced4-4a04-9a63-f74a1ae0098e | rrufael-703-03  | SHUTOFF | -           | Shutdown    | CH-819337-net=192.168.0.38, 129.114.33.5    |
| ccca4aff-777d-4856-98b9-55d3a37df4d7 | skhande-511-01  | SHUTOFF | -           | Shutdown    | CH-819337-net=192.168.0.44, 129.114.33.75   |
| d61fdc1f-9622-4937-9d3a-8b52228e442b | skhande-511-02  | SHUTOFF | -           | Shutdown    | CH-819337-net=192.168.0.45                  |
| f92dc70d-6389-42dd-814e-fb674d6665e2 | skhande-511-03  | SHUTOFF | -           | Shutdown    | CH-819337-net=192.168.0.46                  |
| fb858c85-8702-4ae2-8e4f-33ce672c817b | spathan-516-1   | SHUTOFF | -           | Shutdown    | CH-819337-net=192.168.0.33, 129.114.111.85  |
| 10f5f689-0975-4e8b-9e1e-0fdbe968b086 | spathan-516-2   | SHUTOFF | -           | Shutdown    | CH-819337-net=192.168.0.34, 129.114.111.86  |
| 7e443fc1-f1ce-4821-8710-e461a1f4a27f | spathan-516-3   | SHUTOFF | -           | Shutdown    | CH-819337-net=192.168.0.35, 129.114.111.90  |
+--------------------------------------+-----------------+---------+-------------+-------------+---------------------------------------------+
root@c41a045021ff:/app# exit
\end{lstlisting}

