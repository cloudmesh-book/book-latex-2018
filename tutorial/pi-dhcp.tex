\MDNAME\
%%%%%%%%%%%%%%%%%%%%%%%%%%%%%%%%%%%%%%%%%%%%%%%%%%%%%%%%%%%%%%%%%%%%%%%%%%%%%%%
% DO NOT MODIFY THIS FILE
%%%%%%%%%%%%%%%%%%%%%%%%%%%%%%%%%%%%%%%%%%%%%%%%%%%%%%%%%%%%%%%%%%%%%%%%%%%%%%%

\section{PI DHCP Server Tutorial}

\TODO{Min, Bertholt: could som of the actions not be done when we burn
the SD Cards. E.g. I wonder if i plugin an SDcard can we not set up
all the file actions. Also even whne we are locally on the machine, as
this is the first install, could we not just copy or cat the
information into the right files. Maybe we can werie a cmd5 command
cms pi dhcp ... to make this easy?}

\subsection{Note: This tutorial aims at providing some introduction to DHCP and also how to set it up on Pi clusters by hand. The automation (i.e.~include the set-up directly in the image) will be provided by Bertolt in a separate tutorial.}

\begin{itemize}
\item
  Min Chen (hid-sp18-405)
\end{itemize}

We describe how to set up \textbf{D}ynamic \textbf{H}ost
\textbf{C}onfiguration \textbf{P}rotocol (DHCP) server on a Raspberry Pi
Cluster. The OS on these Pi's is RASPBIAN STRETCH WITH DESKTOP released
on 2017-11-29.

\subsection{Introduction}

``The Dynamic Host Configuration Protocol (DHCP) enables any of the
computers on the local area network (LAN) to be given a network
configuration automatically as soon as the boot process on the machine
gets underway'' \cite{hid-sp18-405-tutorial-pidhcp-pdf}. Instead of
using a router, we use one of the raspbery PI's to fulfill this
function. The other Pi's are configured in such a way that the serve as
clients and obtain the network address form our raspberry providing the
addresses. They are DHCP Clients, and to use DHCP they need to have
their networking setup properly configured.

Hence, we set up one of the five Pi's in the cluster (hostname: red00)
to be the DHCP server, and the rest four with Pi's with the names red01,
red02, red03, red04 will then be DHCP clients.

The information which is passed from DHCP Server to its clients
includes\cite{hid-sp18-405-tutorial-pidhcp-pdf}:

\begin{itemize}
\item
  a suitable IP Address;
\item
  the address of your router;
\item
  an address of one or more Domain Name Servers (DNS)
\end{itemize}

If you want to get an introduction about the logical process followed by
a DHCP service, please follow
\href{https://www.raspberrypi.org/learning/networking-lessons/lesson-3/plan/}{this
link} \cite{hid-sp18-405-tutorial-pidhcp-lesson3}

\subsection{Setting up the DHCP server}

Choose one of the Pi's as the DHCP server, using the Pi with hostname
\emph{red00} as an example here. Log into this Pi and open a terminal.
The following steps are all processed in the terminal of this chosen Pi.

\subsubsection{Software installation}

The first step is to install a package dhcpd, which is a popular DHCP
server for the Pi. In the terminal

\begin{lstlisting}
sudo apt-get update
sudo apt-get install isc-dhcp-server
    
\end{lstlisting}

At the end of the installation process, the DHCP server daemon will be
started and it will \textbf{fail}, because the configuration has not
been done. It will get fixed in later steps.

\subsubsection{Configure the DHCP server}

The configuration file for the DHCP server is at
\emph{/etc/dhcp/dhcpd.conf} Start the editing process with \emph{nano}
as follows:

\begin{lstlisting}
sudo nano /etc/dhcp/dhcpd.conf
    
\end{lstlisting}

Define subnet which will be the network that all the other Pi's will
connect to. Add the following lines to the file
\emph{/etc/dhcp/dhcpd.conf}:

\begin{lstlisting}
subnet 192.168.2.0 netmask 255.255.255.0 {
    range 192.168.2.100 192.168.2.200;
    option broadcast-address 192.168.2.255;
    option routers 192.168.2.1;
    max-lease-time 7200;
    option domain-name "red00";
    option domain-name-servers 8.8.8.8;
}
\end{lstlisting}

Note: The subnet and netmask are IP values required for assisting
communications across your LAN.

\begin{itemize}
\item
  \texttt{subnet}: ``you can obtain the IP Address of a computer on your
  LAN using the Linux ifconfig command: take the \emph{Inet Addr} value
  and replace its final octet with a zero to get your subnet''
  \cite{hid-sp18-405-tutorial-pidhcp-pdf};
\item
  \texttt{range}: this is the range of IP Addresses distributed by this
  DHCP Service. You may have two ranges such as

\begin{lstlisting}
range 192.168.2.100 192.168.2.120
range 192.168.2.150 192.168.2.200
\end{lstlisting}

  to refrain the DHCP server from handling out some of the addresses
  (from 121 to 149)
\item
  \texttt{domain-name-server}: If you have a DNS service for machines on
  your LAN, enter the server IP address or you can use public DNS
  Services such as Google's, which are at \texttt{8.8.8.8}. and
  \texttt{8.8.4.4}.
\end{itemize}

Finally, Save your changes to the file with \texttt{Ctrl-O} and exit
nano with \texttt{Ctrl-X}.

\subsubsection{Change the interface of DHCP service}

Now you need to tell the DHCP service the interface to hand out
addresses on. Edit the following file:

\begin{lstlisting}
sudo nano /etc/default/isc-dhcp-server
    
\end{lstlisting}

Find the following section:

\begin{lstlisting}
# On what interfaces should the DHCP server (dhcpd) serve DHCP requests?
#       Separate multiple interfaces with spaces, e.g. "eth0 eth1".
INTERFACES=""
    
\end{lstlisting}

And change the last line to:

\begin{lstlisting}
INTERFACES="eth0"
\end{lstlisting}

\subsubsection{Set static IP Address for the server}

The next step is to set a static IP address on the Raspberry pi as this
won't be able to start the DHCP service without it. We use \emph{nano}
to edit the file at \emph{/etc/network/interfaces}:

\begin{lstlisting}
sudo nano /etc/network/interfaces
    
\end{lstlisting}

Add the following lines:

\begin{lstlisting}
auto eth0
iface eth0 inet static
    address 192.168.2.1
    netmask 255.255.255.0
        
\end{lstlisting}

If you have a line like:

\begin{lstlisting}
iface eth0 inet dhcp
    
\end{lstlisting}

be sure to disable this line. It is used to set up the Pi as a DHCP
client, however, we want this Pi to be a DHCP server.

Now this Raspberry Pi will now always have the IP address
\texttt{192.168.2.1}. You can double-check this by entering the command
\texttt{ifconfig}; the IP address should be shown on the second line
just after \texttt{inet\ addr}.

\subsubsection{Restart the DCHP service}

Finally, to complete the set-up, restart the DHCP service by the
following command:

\begin{lstlisting}
sudo service isc-dhcp-server stop
sudo service isc-dhcp-server start
    
\end{lstlisting}

\subsubsection{Checking the currently leased addresses}

Run the following command to check the currently assigned addresses:

\begin{lstlisting}
cat /var/lib/dhcp/dhcpd.leases
    
\end{lstlisting}

You should expect something like the following:

\begin{lstlisting}
lease 192.168.2.102 {
  starts 0 2018/02/25 21:36:16;
  ends 0 2018/02/25 21:46:16;
  tstp 0 2018/02/25 21:46:16;
  cltt 0 2018/02/25 21:36:16;
  binding state active;
  next binding state free;
  rewind binding state free;
  hardware ethernet b8:27:eb:42:c9:e9;
  uid "\001\270'\353B\311\351";
  set vendor-class-identifier = "dhcpcd-6.11.5:Linux-4.9.59-v7+:armv7l:BCM2835";
  client-hostname "red01";
}
lease 192.168.2.100 {
  starts 0 2018/02/25 21:41:09;
  ends 0 2018/02/25 21:51:09;
  tstp 0 2018/02/25 21:51:09;
  cltt 0 2018/02/25 21:41:09;
  binding state active;
  next binding state free;
  rewind binding state free;
  hardware ethernet b8:27:eb:60:b8:8e;
  uid "\001\270'\353`\270\216";
  set vendor-class-identifier = "dhcpcd-6.11.5:Linux-4.9.59-v7+:armv7l:BCM2835";
  client-hostname "red03";
}
lease 192.168.2.101 {
  starts 0 2018/02/25 21:41:10;
  ends 0 2018/02/25 21:51:10;
  tstp 0 2018/02/25 21:51:10;
  cltt 0 2018/02/25 21:41:10;
  binding state active;
  next binding state free;
  rewind binding state free;
  hardware ethernet b8:27:eb:9a:55:13;
  uid "\001\270'\353\232U\023";
  set vendor-class-identifier = "dhcpcd-6.11.5:Linux-4.9.59-v7+:armv7l:BCM2835";
  client-hostname "red02";
}
lease 192.168.2.103 {
  starts 0 2018/02/25 21:41:11;
  ends 0 2018/02/25 21:51:11;
  tstp 0 2018/02/25 21:51:11;
  cltt 0 2018/02/25 21:41:11;
  binding state active;
  next binding state free;
  rewind binding state free;
  hardware ethernet b8:27:eb:3a:7c:7c;
  uid "\001\270'\353:||";
  set vendor-class-identifier = "dhcpcd-6.11.5:Linux-4.9.59-v7+:armv7l:BCM2835";
  client-hostname "red04";
}
\end{lstlisting}

Note that in this case, the 4 Pi's with hostnames: red01, red02, red03,
and red04 are assigned IP addresses \texttt{192.168.2.102},
\texttt{192.168.2.101}, \texttt{192.168.2.100} and
\texttt{192.168.2.103} respectively

An alternative way is to use the following command:

\begin{lstlisting}
    dhcp-lease-list --lease PATH_TO_LEASE_FILE
    
\end{lstlisting}

which would give a cleaner look such as:

\begin{lstlisting}
Reading leases from /var/lib/dhcp/dhcpd.leases
MAC                IP            hostname valid until manufacturer
===================================================================
b8:27:eb:3a:7c:7c  192.168.2.103 red04    2018-02-25 21:56:13 -NA-
b8:27:eb:42:c9:e9  192.168.2.102 red01    2018-02-25 21:50:30 -NA-
b8:27:eb:60:b8:8e  192.168.2.100 red03    2018-02-25 21:56:11 -NA-
b8:27:eb:9a:55:13  192.168.2.101 red02    2018-02-25 21:56:12 -NA-
    
\end{lstlisting}

\subsection{Configure fixed IP's for clients}

It is sometimes needed to have the dhcp server assign fixed addresses to
each node in the cluster so that it is easy to remember the node by IP
addresses. For instance next node in the cluster is red01 and it would
be helpful to have a fixed IP for example \texttt{192.168.2.50}.

To do this, we modify the \emph{/etc/dhcp/dhcpd.conf} by:

\begin{lstlisting}
sudo nano /etc/dhcp/dhcpd.conf
       
\end{lstlisting}

and add the following lines:

\begin{lstlisting}
 host red01 {
    hardware ethernet b8:27:eb:42:c9:e9;
    fixed-address 192.168.2.50;
}
\end{lstlisting}

Notice that \texttt{b8:27:eb:42:c9:e9} is the so-called MAC Address of
the Ethernet interface (network adapter) of the machine which you wish
to name red01. ``It provides a hardware reference on the client for the
server to use in network communications.You can find the MAC Address(es)
of the Ethernet interface(s) on any computer using the \emph{ifconfig}
command'' \cite{hid-sp18-405-tutorial-pidhcp-pdf}. You may also notice
that the previous two commands of checking the currently leased
addresses will also provide MAC Addresses.

\textbf{Warning}: ``if you wish to have your DHCP Server award a fixed
IP Address it should be one outside the DHCP normally assigned range of
IP Addresses'' \cite{hid-sp18-405-tutorial-pidhcp-pdf}.

Another thing to notice is that the previous two commands of checking
the currently leased addresses does not include the clients given a
fixed address. For example, if one runs the command after the fixed IP
config for red01:

\begin{lstlisting}
dhcp-lease-list --lease PATH_TO_LEASE_FILE
    
\end{lstlisting}

the result would be:

\begin{lstlisting}
Reading leases from /var/lib/dhcp/dhcpd.leases
MAC                IP            hostname valid until manufacturer
===================================================================
b8:27:eb:3a:7c:7c  192.168.2.103 red04    2018-02-25 21:56:13 -NA-
b8:27:eb:60:b8:8e  192.168.2.100 red03    2018-02-25 21:56:11 -NA-
b8:27:eb:9a:55:13  192.168.2.101 red02    2018-02-25 21:56:12 -NA-
    
\end{lstlisting}

red01 is no longer in the list because it is assigned a fixed-address
\texttt{192.168.2.50}

\subsubsection{Checking the currently leased addresses for fixed IP clients}

To check the currently leased addresses for fixed IP clients:

\begin{lstlisting}
    cat /var/lib/dhcp/dhclient.eth0.leases
\end{lstlisting}

\subsection{Resources}

The steps and instruction presented here are combined from several web
resources:

\begin{itemize}
\item
  \href{https://www.raspberrypi.org/learning/networking-lessons/lesson-3/plan/}{Lesson
  3 - Dynamic Host Configuration Protocol (DHCP)}
  \cite{hid-sp18-405-tutorial-pidhcp-lesson3}
\item
  \href{http://my-music.mine.nu/images/rpi_raspbianwheezy_dhcp_server.pdf}{Configuring
  the Raspberry Pi as a DHCP Server under Raspbian Wheezy}
  \cite{hid-sp18-405-tutorial-pidhcp-pdf}
\item
  \href{https://tekmarathon.com/2017/02/16/hadoop-and-spark-installation-on-raspberry-pi-3-cluster-part-3/}{Hadoop
  and Spark Installation on Raspberry Pi-3 Cluster}
  \cite{hid-sp18-405-tutorial-pidhcp-hadoopinstall}
\item
  \href{https://blog.monotok.org/setup-raspberry-pi-dhcp-server/}{Setup
  Raspberry pi as a dhcp server}
  \cite{hid-sp18-405-tutorial-pidhcp-setup}
\end{itemize}

