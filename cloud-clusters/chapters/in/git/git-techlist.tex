\section{How to use Git in this class}\label{how-to-use-git-in-this-class}

\FILENAME\

Let us use the first homework `TechList 1' as an example to illstrate
why and how we choose Git as a essential tool in this class.

In the `Technologies' section on the class website, we will
collaboratively edit the web page contents. Each student shall add their
paragraphs about certain tech topics. This leads to a question: multiple
authors are working on a same document, whose version should be used as
the final version? After all, there can only be one version published
online.

Git can largely (not completely) help use sort it out. And here is our
Git workflow:

\subsection{Step 1}\label{step-1}

In the class official repo, we start from an empty TechList file. Here
is the file we start from in our class repo:
\url{https://github.com/cloudmesh/classes/blob/master/docs/source/i524/technologies.rst}

\subsection{Step 2}\label{step-2}

Each student is going to create a repository exactly the same as the
official repo. This step is called to `fork' the repo. You do so by
click the `fork' button on the top right corner of the page. Once done,
you have your own copy of the class repo.

\subsection{Step 3}\label{step-3}

You create this repo in your Github account. This means you still store
everything online remotely.

\subsection{Step 4}\label{step-4}

To conveniently edit the files, you need a copy on your laptop. This is
to `clone' your personal repo to your local disk. Look at your Github
repo and find a green button saying `clone or download'. Click this
button and you will have a pulldown menu where you should copy the URL.
In your terminal/commandline, run `git clone \textless{}the URL you just
copied\textgreater{}'. This step copies the project to your own disk for
editing.

\subsection{Step 5}\label{step-5}

Your local copy is a `live' version, since you are going to keep editing
it. Put your contents in the `technologies.rst' file in the
corresponding sections.

\subsection{Step 6}\label{step-6}

At any point during your editing, if you want to secure the changes you
made to the file, you can always `commit'. Refer to the `git commit'
instruction. Obviously, when you finish all the typing, a `commit'
operation must be performed. Consider a `commit' operation a fancy
`save' operation you may always do during file editing. A major
advantage is, `commit' will keep a history of all the versions you
committed. All the committed texts will not be lost. This reduces a lot
of risks during long term, large scale development.

\subsection{Step 7}\label{step-7}

When you complete your work, it is the time to submit your homework back
to the graders. Do this in two steps. First, you have to `push' your
local disk copy to the repo you created under your acount in step 4).
Follow closely after this link
\url{https://github.com/cloudmesh/classes\#submitting-changes} to get
your local disk and your personal remote a.k.a. the `origin'
synchronized. Second step is done via your browser. At the very top of
your repo web page, switch to the `Pull requests' tab, and submit a pull
request to the class repo. This way, the instructor will get notified
and merge your repo into the class official repo. After this merge being
performed, all your contributions will be included into the public site.
