\part{Syllabus}

\chapter{Course Overview}


\section{Justification}

Intelligent Systems Engineering is a department and a new major at
Indiana University. This course is for students who are interested in
any component of the Masters or Ph.D. in Intelligent Systems
Engineering. It is an advanced elective class.


%%%%%%%%%%%%%%%%%%%%%%%%%%%%%%%%%%%%%%%%%%%%%%%%%%%%%%%%%%%%%%%%%%%%%%
\chapter{E516: Introduction to Cloud Computing}
%%%%%%%%%%%%%%%%%%%%%%%%%%%%%%%%%%%%%%%%%%%%%%%%%%%%%%%%%%%%%%%%%%%%%%

We present the syllabus for the introduction to Cloud Computing course
taught at Indiana University that uses in part the material presented
in this document. The information includes the section or chapter
number, the name of the chapter or section, and the timeframe when it
is recommended to work through the material and the page number where
to find the material in this document. Please note that we will update
the document throughout the semester thus, pagenumbers will change. A
specific date of publication when the last update occurred is

\TODAY

\section{Course Description}

This course describes the emerging cloud and big data technologies
within the world of distributed intelligent systems where each system
component has internal, and external sources of intelligence that are
subject to collective control. Examples are given from a wide variety
of applications. A project will allow students to dive into practical
issues after they have obtained a theoretical background.

\section{Course pre-requisites}

Python will be used as a Programming Languages. In some cases Java may
also be useful however it's use in class will be marginal. The class
will be using Linux commandline tools. Prior knowledge of Linux is of
advantage but not required.  Studnets are expected to have access to a
computer on which they can execute Linux easily. As the OS
requirements have recently increased we recommend a computer with 8GB
main memory and the ability to run virtualbox and/or containers. If it
turns out your machine is not capable enough we attempt to provide
access to IU linux machines.

\section{Teaching and learning methods}

\begin{itemize}
\item Lectures
\item Assignments including specific lab activities
\item Final project 
\end{itemize}

\section{Covered Topics}

The class will cover a number of topics as part of tracks that are
executed in parallel throughout the class. Although we assume that at
a graduate course level the communication track has already been
covered elsewhere, we made sure we also include it here in case you dod
not yet have such experiences.

\begin{description}

\item[Introduction] Overview and status of cloud computing
\item[Big Data Applications] Introduction to the Apache Big Data
  Satck. Selective presentation of the members of the Big Data set
\item[New Big Data Application Technologies] Students will
  explore the Apache Web page and report on them.
\item[Hadoop] Introduction to Hadoop
\item[Virtual Machines] Introduction to OpenStack
\item[Cloud 3.0] Microservices, Events, Functions  
\item[Project] Delivery of a reproducible substantial student project
 
\end{description}


\section{Student learning outcomes}

When students complete this course, they should be able to:

\begin{itemize}
\item Have an advanced understanding of issues involved in designing
  and Software Defined Systems using the latest network and cloud
  technologies.
\item Gain hands-on laboratory experience with several examples from
  academia and industry.
\item Understand the application of Apache Big Data Software Stack and
  DevOps to Software Defined systems.
\item Apply knowledge of mathematics, science, and engineering
\item	Understand research challenges and important issues with Software Defined Systems.
\item	Have advanced skills in teamwork with peers.
\end{itemize}

\section{Grading}

Grade Item	Percentage

\begin{tabular}{lr}
Assignments	  & 30\% \\
Final Project	& 60\% \\
Participation	& 10\% \\
\end{tabular}




\section{Representative bibliography}

\begin{itemize}
\item Cloud Computing for Science and Engineering By Ian Foster and
  Dennis
  B. Gannon\URL{https://mitpress.mit.edu/books/cloud-computing-science-and-engineering}
\item Ansible Tutorials
\item	\url{http://bigdataopensourceprojects.soic.indiana.edu/}
\item	The backdrop for course is the ~350 software subsystems illustrated at \url{http://hpc-abds.org/kaleidoscope/}
\item	\url{http://cloudmesh.github.io/introduction_to_cloud_computing/}
\item	There are a huge number of web resources
\end{itemize}

\section{Lectures and Lecture Material}

\subsection{Communication Track}

\WHERE{c:doc}{Month 1}
\WHERE{S:plagiarism}{Month 1}
\WHERE{S:writing}{Month 1}
\WHERE{C:latex}{Month 1}
\WHERE{C:bibtex}{Month 1}
\WHERE{C:emacs}{Optional, highly recommended}
\WHERE{S:markdown}{Month 1}
\WHERE{S:results}{Week 1}

\subsection{Theory Track}



\subsection{Programming Track}

\subsection{Cluster Track}

%%%%%%%%%%%%%%%%%%%%%%%%%%%%%%%%%%%%%%%%%%%%%%%%%%%%%%%%%%%%%%%%%%%%%%
\chapter{E616: Advanced Cloud Computing}
%%%%%%%%%%%%%%%%%%%%%%%%%%%%%%%%%%%%%%%%%%%%%%%%%%%%%%%%%%%%%%%%%%%%%%

We present the syllabus for the E616 course taught at Indiana
University that uses in part the material presented in this
document. The information includes the section or chapter number, the
name of the chapter or section, and the timeframe when it is
recommended to work through the material and the page number where to
find the material in this document. Please note that we will update
the document throughout the semester thus, page numbers will change. A
specific date of publication when the last update occurred is

\TODAY

\section{Course Description }

This course describes Cloud 3.0 in which DevOps, Microservices, and
Function as a Service is added to basic cloud computing. The
discussion is centered around the Apache Big Data Stack and a major
student project aimed at demonstrating integration of cloud
capabilities.

\section{Course pre-requisites}

Python will be used as a Programming Languages. It is expected that
you know a programming language. ENGR-E516 or an introduction to cloud
computing is recommended. Studnets are expected to have access to a
computer on which they can execute Linux easily. As the OS
requirements have recently increased we recommend a computer with 8GB
main memory and the ability to run virtualbox and/or containers. If it
turns out your machine is not capable enough we attempt to provide
access to IU linux machines.

\section{Teaching and learning methods}

\begin{itemize}
\item	Lectures
\item	Assignments including specific lab activities
\item	Final project
\item Class will use software mainly written in Python
  and Linux Shell.
\end{itemize}



\section{Covered Topics}

The class will cover a number of topics as part of tracks that are
executed in parallel throughout the class. Although we assume that at
a graduate course level the communication track has already been
covered elsewhere, we made sure we also include it here in case you dod
not yet have such experiences.

\begin{description}

\item[Introduction] Overview and status of cloud computing
\item[Cloud and Big Data Applications] Introduction to the Apache Big Data
  Satck. Selective presentation of the members of the Big Data set
\item[New Cloud Big Data Application Technologies] Students will
  explore the Apache Web page and report on them.
\item[DevOps] Introduction to DevOps with Dokerfile and ansible
\item[Virtual Machines] Introduction to OpenStack
\item[Containers] Introduction to Container technology
\item[Advanced Containers] Building clusters with containers.
\item[Cloud 3.0] Microservices, Events, Functions  
\item[Project] Delivery of a reproducible substantial student project 
 (you are allowed to substantially enhance a project that you started
  in E516)

\end{description}

\section{Student learning outcomes}

When students complete this course, they should be able to:

\begin{itemize}
\item	Have an advanced understanding of issues involved in designing and applying modern cloud technologies using the latest developments.
\item	Gain hands-on laboratory experience.
\item	Understand the Apache Big Data Software Stack.
\item	Apply knowledge of mathematics, science, and engineering.
\item	Understand research challenges and important application areas of clouds
\item	Have advanced skills in teamwork with peers.
\item Be able to use DevOps technologies.
\end{itemize}

\section{Grading}

Grade Item	Percentage
\begin{tabular}{lr}
Assignments	  & 30\% \\
Final Project	& 60\% \\
Participation	& 10\% \\
\end{tabular}



\section{Representative bibliography}

\begin{itemize}
\item	Machine to machine protocols \url{https://en.wikipedia.org/wiki/MQTT}
\item	Cloud software systems \url{http://hpc-abds.org/kaleidoscope/}
\item	GE Software defined machines \url{https://www.ge.com/digital/blog/software-defined-machines}
\item	Software Defined Networks \url{https://en.wikipedia.org/wiki/Software-defined_networking}
\item	There are a huge number of other web resources
\end{itemize}

\section{Lectures and Lecture Material}

\subsection{Communication Track}

\WHERE{c:doc}{Month 1}
\WHERE{S:plagiarism}{Week 1}
\WHERE{S:results}{Week 1}

\WHERE{S:markdown}{Week 1}
\WHERE{C:emacs}{Week 2}

\WHERE{S:writing}{Week 3}
\WHERE{C:latex}{Week 3}
\WHERE{C:bibtex}{Week 4}

\begin{description}
\item[Evaluation Paper1:] Create a paper about a cloud technology with our give
class template in the git repository. If a paper is plagiarised you
will receive an ``F'' and it is reported based on University
policies.
\end{description}


\subsection{Theory Track}

\WHERE{S:o-workflow}{Week 1}
\WHERE{S:o-application}{Week 2}
\WHERE{S:o-programming}{Week 3}
\WHERE{S:o-streams}{Week 4}
\WHERE{S:o-prg-model}{Week 5}
\WHERE{S:o-process-communication}{Week 5}
\WHERE{S:o-db-memory}{Week 6}
\WHERE{S:o-db-object}{Week 6}
\WHERE{S:o-Tools}{Week 7}
\WHERE{S:o-sql}{Week 6}
\WHERE{S:o-NoSQL}{Week 6}
\WHERE{S:o-file-management}{Week 8}
\WHERE{S:o-data-transport}{Week 8}
\WHERE{S:o-cluster}{Week 9}
\WHERE{S:o-file-systems}{Week 8}
\WHERE{S:o-interoperability}{Week 3}
\WHERE{S:o-DevOps}{Week 10}
\WHERE{S:o-hypervisors}{Week 11}
\WHERE{S:o-cross-cutting-functions}{Week 12}
\WHERE{S:o-protocols}{Week 13}
\WHERE{S:o-todo}{Week 4}


\begin{description}
\item[Evaluation Paper1:] Create a paper about a cloud technology with our give
class template in the git repository. If a paper is plagiarized you
will receive an ``F'' and it is reported based on University
policies. The paper is in a directory called paper1. All images are in
the directory paper1/images, the report is in report.tex, the content
is in content.tex. It follows the template we provided. Submission of
report.pdf is not allowed. We will create the report and check
completness that way. 
\end{description}

\subsection{Programming Track}

\subsubsection{Development Environment}

\WHERE{C:linux}{Week 2}
\WHERE{C:github}{Week 3}
\WHERE{S:virtual-box}{Week 3}

\begin{description}
\item[Evaluation Experiment 1:] Create a virtual machine and take a 
  photo with your laptop or computer and the virtual box running on
  the screen. Showcase the virtual box interface and in non full
  screen mode at the same time the operating system you run. We
  recommend yo use Ubuntu.
\end{description}

\subsubsection{Python}

\WHERE{C:python}{Week 3 - 4}
\WHERE{C:python-install}{Week ?}
\WHERE{C:python-language}{Week ?}
\WHERE{C:python-lib}{Week ?}
\WHERE{C:python-cmd5}{Week ?}


\WHERE{c:numpy}{}
\WHERE{c:scipy}{} 
\WHERE{c:opencv}{}
\WHERE{c:secchi-disk}{}

\begin{description}
\item[Evaluation Experiment 1:] Create a program in python that
  identifies a termination criteria for the Secchi disk
  problem. E.g. at what image can we no longer see the disk?
  Describe your solution in md and submit to the git repository in a
  directory called {\em secchi}. The program is called secchi.py, the
  description is in README.md. It uses cmd5 for creating a command
  shell that can load the data and analyze it. 
\end{description}

\subsection{Cloud}

\subsubsection{Chameleon Cloud and OpenStack}

\WHERE{C:chameleon}{Week 4}
\WHERE{C:cc-charge}{Week 4}
\WHERE{C:cc-hardware}{Week 4}
\WHERE{C:cc-start}{Week 4}
\WHERE{C:cc-horizon}{Week 5}
\WHERE{C:cc-guide}{Week 4}
\WHERE{C:cc-heat}{Optional}
\WHERE{C:cc-baremetal}{Optional}



\subsection{Cluster Track}

\WHERE{c:pi-cluster-form-factor}{Month 1}
