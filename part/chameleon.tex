\chapterimage{images/chameleon.jpeg} % Chapter heading image

\chapter{Chameleon Cloud}
\label{C:chameleon}
\index{Chameleon}
\FILENAME\

\section{Overview}

\FILENAME

This chapter is copied from the Chameleon Web page. However we have
included where appropriate some updates. 

Chameleon is an experimental testbed for Computer Science funded by the
NSF FutureCloud program. Chameleon is built over two sites, University
of Chicago and TACC, offering a total of over 550 nodes and 5 PB of
space in twelve
\href{https://www.chameleoncloud.org/about/hardware-description/}{Standard
Cloud Unit (SCU) racks}. To effectively support Computer Science
experiments Chameleon offers bare metal reconfigurability on most of the
hardware. To provide easy access to educational users, three SCUs at
TACC (a quarter of the testbed) are configured with OpenStack KVM. You
can read more about
Chameleon~\href{https://www.chameleoncloud.org/about/chameleon/}{here}.

Chameleon is broadly available to members of the US Computer Science
research community and its international collaborators working in the
open community on cloud research. The expectation is that any research
performed on Chameleon will result in publication in a broadly
available journal or conference.

In order to promote fairness to all users, we have~the following set of
Best Practices for using Chameleon~bare metal partitions:

\begin{description}
\item
  \item[Think Small for Development and class use:] If you are just developing or
  prototyping a system, and not yet running experiments at scale, use
  only as many nodes as you actually need (e.g., many projects can be
  developed and tested on 3-4 nodes), and try to take short
  reservations. Start with hours first and do not let your VMs
  unnecessarily run for long unused periods. 
\item
  \item[Automate deployments:] You can always snapshot your work/images
  between sessions using the
  \href{https://www.chameleoncloud.org/docs/user-guides/ironic/\#snapshotting_an_instance}{snapshotting
  instructions} to simplify the redeployment of your environment during
  the next work session. You can also use scripting and environment
  customization to make it easier to redeploy images. An additional
  benefit of automation is that it makes it easier for you to reproduce
  your work and eventually~share it~with colleagues within your lab and
  other collaborators.
\item [Think Big for Experimentation:] Once you are ready to experiment you
  will want to test your experimental setup on increasingly larger
  scales. This is possible by taking an advance reservation for many
  resources for a relatively short time. The more resources you need,
  the more likely it is that you will need to run experiments at a less
  attractive time (e.g., during the weekend) --- here's where automation
  will also help. In justified cases, we will support reserving even the
  whole bare metal testbed.
\end{description}



\FILENAME

\hypertarget{main-nav}{}
\begin{itemize}
\item
  \href{https://www.chameleoncloud.org/}{Home}
\item
  \href{openstack-kvm-user-guide.html\#}{Documentation }

  \begin{itemize}
    \item
    \href{https://www.chameleoncloud.org/docs/getting-started/}{Getting
    Started}
  \item
    \href{https://www.chameleoncloud.org/docs/bare-metal-user-guide/}{Bare
    Metal}
  \item
    \href{https://www.chameleoncloud.org/docs/complex-appliances/}{Complex
    Appliances}
  \item
    \href{https://www.chameleoncloud.org/docs/openstack-kvm-cloud/}{OpenStack
    KVM Cloud}
  \item
    \href{https://www.chameleoncloud.org/docs/user-faq/}{User FAQ}
  \item
    \href{https://www.chameleoncloud.org/docs/community/}{Community}
  \end{itemize}
\item
  \href{https://www.chameleoncloud.org/appliances/}{Appliances}
\item
  \href{https://www.chameleoncloud.org/hardware/}{Hardware}
\item
  \href{https://www.chameleoncloud.org/news/}{News}
\item
  \href{openstack-kvm-user-guide.html\#}{About }

  \begin{itemize}
    \item
    \href{https://www.chameleoncloud.org/about/chameleon/}{About
    Chameleon}
  \item
    \href{https://www.chameleoncloud.org/about/hardware-description/}{Hardware
    Description}
  \item
    \href{https://www.chameleoncloud.org/talks/}{Talks}
  \item
    \href{https://www.chameleoncloud.org/about/newsletter/}{Stay in
    Touch}
  \item
    \href{https://www.chameleoncloud.org/about/media-resources/}{Media
    Resources}
  \end{itemize}
\end{itemize}

\begin{itemize}
\item
  \href{https://www.chameleoncloud.org/login/}{Log in}
\item
  \href{openstack-kvm-user-guide.html\#}{Users}

  \begin{itemize}
    \item
    \href{https://www.chameleoncloud.org/user/register/}{Register}
  \item
    \href{https://www.chameleoncloud.org/docs/getting-started/experiment-quickstart/}{Experiment}
  \item
    \href{https://www.chameleoncloud.org/user/help/ticket/new/guest/}{Help}
  \end{itemize}
\end{itemize}



\input{section/cloud/chameleon/hardware}
\FILENAME

\chapter{Getting Started}\label{getting-started}

Here we describe how you can get access to chameleon clude under the
assumption that you are a student or a researcher that joins an
existing project on Chameleon cloud. You will need to follow the
following steps:

\section{Step 1: Create~a Chameleon~account}\label{accounts}

To get started using Chameleon you will need
to\textbf{\href{https://www.chameleoncloud.org/register}{~create a
user~account}.}

You will be asked to agree to the
\href{https://www.chameleoncloud.org/terms/view/chameleon-user-terms/}{Chameleon
terms and conditions} which, among others, ask you to acknowledge the
use of Chameleon in your publications.

As part of creating an account you may request PI status, which means
that you will be able to create and lead Chameleon projects (see Step 2
below). 

\section{Step 2: Create or join a project}\label{allocations}

To use Chameleon, you will need to be associated with a
\href{https://www.chameleoncloud.org/docs/user-faq/\#toc-how-do-i-apply-for-a-chameleon-project-}{project}
that is assigned an
\href{https://www.chameleoncloud.org/docs/user-faq/\#toc-what-are-the-project-allocation-sizes-and-limits-}{allocation}.
This means that you either need to
(1) \textbf{\href{https://www.chameleoncloud.org/user/projects/new/}{apply
for a new project}} or (2)
\textbf{\href{https://www.chameleoncloud.org/docs/user-faq/\#toc-my-pi-professor-colleague-already-has-a-chameleon-project-how-do-i-get-added-as-a-user-on-the-project-}{ask
the PI of an existing Chameleon project to add you}.}

A project is headed by a project PI, typically
\href{https://www.chameleoncloud.org/docs/user-faq/\#toc-who-is-eligible-to-be-chameleon-pi-and-how-do-i-make-sure-that-my-pi-status-is-reflected-in-my-profile-}{a
faculty member or researcher scientist at a scientific institution}. If
you are a student we recommend that you ask your professor to work with
you on creating a project. Please note taht you must not create a
project by yourself and that you indeed need to work with your
proferrsor. 

A project application typically consists of about one paragraph
description of the intended research and takes one buisness day to
process. 

Enrolling you into an existing research or class project depends on
the time availability of the project lead or professor of your
class. It is important that you communicate your chameleon cloud
account name to the project lead so they can easily add you. Make sure
you realy give them only your chameleon coount name and potentially
your organizational e-mail, Firstname, and Lastname so they can check
you are realy eligible to get access.

\section{Step 3: Start using Chameleon!}\label{using-chameleon}

Now that you have enrolled nad are added to the project by your
project lead you cans tart using chameleon cloud. However be minded
that you ought to shut down the resources/VMs whenever they are not in
use to avoid unnecessary charging. Remember the project has imited
time on chameleon and any unused time will be charged against the project.

Chameleon provides two types of resources with links to their respective
users guides below:

\textbf{\href{https://www.chameleoncloud.org/docs/bare-metal-user-guide-old/}{Bare
Metal User Guide}} will tell you how to use Chameleon bare metal
resources which provide strong isolation and allow you maximum control
(reboot to new operating system, reboot the kernel, etc.)

\textbf{\href{https://www.chameleoncloud.org/docs/user-guides/openstack-kvm-user-guide/}{OpenStack
KVM User Guide}} will tell you how to get started with Chamemeleon's
OpenStack KVM cloud which is a multi-tenant environment providing weak
performance isolation. 

If you have any questions or encounter any problems, you can check out
our \href{https://www.chameleoncloud.org/docs/user-faq/}{User FAQ},
or \href{https://www.chameleoncloud.org/user/help/}{submit a ticket}.


\input{section/cloud/chameleon/charge}
\input{section/cloud/chameleon/user-guide}
\FILENAME\

\section{Horizon Graphical User Interface}
\label{C:cc-horizon}

\subsection{Configure resources}

Once your lease is started, you are almost ready to start an instance.
But first, you need to make sure that you will be able to connect to
it by setting up a key pair. This only has to be done once per user
per project.

Go to Project \textgreater{} Compute \textgreater{} Access \& Security,
then select the Key Pairs tab.

\includegraphics[width=0.8\columnwidth]{images/chameleon/Screen-Shot-2016-10-26-at-14-37-00.png}

Here you can either ask OpenStack to create an SSH key pair for you (via
the ``Create Key'' Pair~button), or, if you already have an SSH key pair
on your machine and are happy to use it, click on ``Import Key Pair''.

If you chose to import a key pair, you will be asked to enter a name for
the key pair, for example laptop. In the ``Public Key'' box, copy the
content of your SSH public key. Typically it will be at
\textasciitilde{}/.ssh/id\_rsa.pub. On Mac OS X, you can run in a
terminal:
~\texttt{cat\ \textasciitilde{}/.ssh/id\_rsa.pub\ \textbar{}\ pbcopy}\\
It copies the content of the public key to your copy/paste buffer. Then
you can simply paste in the ``Public Key'' box.

\includegraphics[width=0.8\columnwidth]{images/chameleon/Screen-Shot-2016-10-26-at-14-37-18.png}

Then, click on the blue ``Import Key~Pair'' button. This should show you
the list of key pairs, with the one you just added.

\includegraphics[width=0.8\columnwidth]{images/chameleon/Screen-Shot-2016-10-26-at-14-37-52.png}

For those already familiar with OpenStack, note that Security Groups are
not functional on bare-metal. All instances ports are open to the
Internet and any security group rule you add will not be~respected.

Now, go to the ``Instances'' panel.

\includegraphics[width=0.8\columnwidth]{images/chameleon/Screen-Shot-2016-10-26-at-14-39-56.png}

Click on the ``Launch Instance'' button in the top right corner. Select
a reservation in the Reservation box, pick an instance name (in this
example my-first-instance) and in~the Image Name list select our default
environment named CC-CentOS7. If you have multiple key pairs registered,
you need to select one in the ``Access \& Security''~tab. Finally, click
on the blue ``Launch'' button.

\includegraphics[width=0.8\columnwidth]{images/chameleon/Screen-Shot-2016-10-26-at-14-41-08.png}

The instance will show up in the instance list, at first in Build
status. It takes a few minutes to deploy the instance on bare-metal
hardware and reboot the machine.

\includegraphics[width=0.8\columnwidth]{images/chameleon/Screen-Shot-2016-10-26-at-15-53-31.png}

After a few minutes the instance should become in Active status and the
Power State should be Running.

\includegraphics[width=0.8\columnwidth]{images/chameleon/Screen-Shot-2016-10-26-at-16-22-38.png}

At this point the instance might still be booting: it might take a
minute or two to actually be accessible on the network and accept SSH
connections.~In the meantime, you can attach a floating IP to the
instance. Click on the~``Associate Floating IP'' button.~You should get
a screen like the one below:

\includegraphics[width=0.8\columnwidth]{images/chameleon/Screen-Shot-2016-10-26-at-16-25-04.png}

If there are no unused floating IP already allocated to your project,
click on the + button. In the window that opens, select the ext-net pool
if not already selected by default and click on the blue Allocate IP
button.

\includegraphics[width=0.1\columnwidth]{images/chameleon/Screen-Shot-2016-10-26-at-16-33-45-W05kOLQ.png}

You will be returned to the previous window. The correct value for
``Port to be associated'' should already be selected, so you only have
to click on ``Associate''.

\includegraphics[width=0.8\columnwidth]{images/chameleon/Screen-Shot-2016-10-26-at-16-25-10.png}

This should send you back to the instance list, where you can see the
floating IP attached to the instance (you may need to refresh your
browser to see the floating IP).

\includegraphics[width=0.8\columnwidth]{images/chameleon/Screen-Shot-2016-10-26-at-16-26-54.png}

\subsection{Interact with resources}

Now you should be able to connect to the instance via SSH using the cc
account. In a terminal, type ssh
cc@\textless{}floating\_ip\textgreater{}, in our example this would
be~\texttt{ssh\ cc@130.202.88.241}

SSH will probably tell you:

\begin{verbatim}
The authenticity of host \textquotesingle{}130.202.88.241
(130.202.88.241) can't be established. RSA key fingerprint 
is 5b:ca:f0:63:6f:22:c6:96:9f:c0:4a:d8:5e:dd:fd:eb. 
Are you sure you want to continue connecting (yes/no)?

\end{verbatim}

Type yes and press Enter. You should arrive to a prompt like this one:

\texttt{{[}cc@my-first-instance\ \textasciitilde{}{]}\$}

If you notice SSH errors such as connection refused, password requests,
or failures to accept your key, it is likely that the physical node is
still going through the boot process. In that case, please wait before
retrying. Also make sure that you use the~\textbf{cc}~account. If after
10 minutes you still cannot connect to the machine,
please~\href{https://www.chameleoncloud.org/user/help/}{open a ticket
with our help desk}.

You can now check whether the resource matches its known description in
the resource registry. For this, simply
run:~\texttt{sudo\ cc-checks\ -v}

{\centering \includegraphics[width=0.5\columnwidth]{images/chameleon/cc-checks.png}}

The cc-checks program prints the result of each check in green if it is
successful and~red if it failed.

You can now run your experiment directly on the machine via SSH. You can
run commands with root privileges by prefixing them with~\texttt{sudo}.
To completely~switch~user and become root, use
the~\texttt{sudo\ su\ -\ root}~command.

\subsubsection{Snapshot an instance}

All instances in Chameleon, whether KVM or bare-metal, are running off
disk images. The content of these disk images can be snapshotted at any
point in time, which allows you to save your work and launch new
instances from updated images later.

While OpenStack KVM has built-in support for snapshotting in the Horizon
web interface and via the command line, bare-metal instances require a
more complex process. To make this process easier,{ we developed the
\href{https://github.com/ChameleonCloud/ChameleonSnapshotting}{cc-snapshot}
tool, which implements snapshotting a bare-metal instance from command
line and uploads it to Glance, so that it can be immediately used to
boot a new bare-metal instance. The snapshot images created with this
tool are whole disk images.}

{For ease of use, \emph{cc-snapshot} has been installed in all the
appliances supported by the Chameleon project. If you would like to use
it in a different setting, it can be downloaded and installed from the
\href{https://github.com/ChameleonCloud/ChameleonSnapshotting}{github
repository}.}

{Once cc-snapshot is installed, to make a snapshot of a bare-metal
instance, run the following command from inside the instance:}

{\texttt{sudo\ cc-snapshot\ \textless{}snapshot\_name\textgreater{}}}

{You can verify that it has been uploaded to Glance by running the
following command:}

{\texttt{glance\ image-list}}

{If you prefer to use a series of standard Unix commands, or are
generally interested in more detail about image management, please refer
to our
\href{https://www.chameleoncloud.org/docs/user-guides/ironic/\#snapshotting_an_instance}{image
management guide}.}

\subsection{Use FPGAs}

Consult the
\href{https://www.chameleoncloud.org/docs/bare-metal-user-guide/fpga/}{dedicated
page}~if you would like to use the FPGAs available on Chameleon.

\subsection{Next Step}

Now that you have created some resources, it is time to interact with
them! You will find instructions to the next step by visiting the
following link:

\begin{itemize}

\item
  \href{https://www.chameleoncloud.org/monitor-and-collect/}{Monitor
  resources and collect results}
\end{itemize}


\input{section/cloud/chameleon/heat}
\FILENAME

\section{Bare Metal}\label{C:cc-baremetal}
\index{Chameleon!Bare Metal}

In this page you will find documentation guiding you through the
bare-metal deployment features available in~Chameleon. Chameleon gives
users administrative access to bare-metal compute resources to
run~{cloud computing~}experiments with a high degree of customization
and repeatability. Typically, an experiment will go through several
phases, as illustrated in the figure below:


\includegraphics[width=\columnwidth]{images/chameleon/baremetal.png}


The bare-metal user guide comes in two editions. The first is how to use
Chameleon resources~via the web interface, the recommended choice for
new users to quickly learn how to use our testbed:

\textbf{\href{https://www.chameleoncloud.org/discover-resources}{Get
started with Chamelon using the web interface}}

\begin{enumerate}
\item
  \href{https://www.chameleoncloud.org/discover-resources/}{Discover
  Resources}
\item
  \href{https://www.chameleoncloud.org/provision-resources/}{Provision
  Resources}~
\item
  \href{https://www.chameleoncloud.org/configure-and-interact/}{Configure
  and Interact}
\item
  \href{https://www.chameleoncloud.org/monitor-and-collect/}{Monitor and
  Collect Results}
\end{enumerate}

The second targets~advanced users who are already familiar with
Chameleon~and would like to learn how to use Chameleon from the~command
line or with scripts.

\href{https://www.chameleoncloud.org/discover-resources-command-lines}{Get
started with Chameleon using the command line~(advanced)}

\begin{enumerate}
\item
  \href{https://www.chameleoncloud.org/discover-resources-command-lines/}{Discover
  Resources}
\item
  \href{https://www.chameleoncloud.org/advanced-provision-resources/}{Provision
  Resources}
\item
  \href{https://www.chameleoncloud.org/advanced-configure-and-interact/}{Configure
  and Interact}
\item
  \href{https://www.chameleoncloud.org/monitor-and-collect/}{Monitor and
  Collect Results}
\end{enumerate}

You do not need to strictly follow the documentation~sequentially.
However, note that some steps assume that~previous ones have been
successfully performed.

You can also consult~documentation~describing how to use advanced
features of Chameleon not covered by the guides above:

\begin{itemize}
\item
  the
  \href{https://www.chameleoncloud.org/docs/bare-metal-user-guide/chameleon-object-store/}{Chameleon
  Object Store},
\item
  \href{https://www.chameleoncloud.org/docs/bare-metal-user-guide/network-isolation-bare-metal/}{network
  isolation for bare metal}.
\end{itemize}





\chapter{Frequently Asked Questions}\label{frequently-asked-questions}

\FILENAME

\section{Appliances}\label{appliances}

\subsection{What is an appliance?}\label{what-is-an-appliance}

An appliance is an application packaged together with the environment
that this application requires. For example, an appliance can consists
of the operating system, libraries and tools used by the application,
configuration features such as environment variable settings, and the
installation of the application itself. Examples of appliances might
include a KVM virtual machine image, a Docker image, or a bare metal
image. Chameleon appliance refers to bare metal images that can be
deployed on the Chameleon testbed. Since an appliance captures the
experimental environment exactly, it is a key element of
reproducibility; publishing an appliance used to obtain experimental
results will go a long way to allowing others to reproduce and build on
your research easily.

To deploy distributed applications on several~Chameleon instances,
complex appliances combine an~image and a~template describing how the
cluster should be configured and contextualized. You can read more about
them in the
\href{https://www.chameleoncloud.org/docs/complex-appliances/}{Complex
Appliances documentation}.

\subsection{What is the Appliance Catalog?}\label{what-is-the-chameleon-appliance-catalog}

\href{https://www.chameleoncloud.org/appliances/}{The Chameleon
Appliance Catalog}~is a repository that allows users to discover,
publish, and share appliances. The appliance catalog contains useful
images of both bare metal and virtual machine appliances supported by
the Chameleon team as well appliances contributed by users.

\subsection{How do I publish an appliance in the Appliance Catalog?}\label{how-do-i-publish-an-appliance-in-the-chameleon-appliance-catalog}

The new Appliance Catalog allows you to easily publish and share your
own appliances so that others can discover them and use them either to
reproduce the research of others or as a basis for their own research.
~Before creating your own appliance it is advisable to review other
appliances on
the~\href{https://www.chameleoncloud.org/appliances/}{Chameleon
Appliance Catalog}~in order to get an idea of the categories you will
want to contribute and what others have done.~

Once you are ready to proceed, an appliance can be contributed to
Chameleon in the following steps:

\begin{enumerate}
\tightlist
\item
  Create the appliance itself. You may want to test it as well as give
  some thought to what support you are willing to provide for the
  appliance (e.g., if your group developed and supports a software
  package, the appliance may be just a new way of packaging the software
  and making it available, in which case your standard support channels
  may be appropriate for the appliance as well).
\item
  Upload the appliance to the Chameleon Image Repository (Glance) and
  make the image public. In order to enter the appliance into the
  Catalog you will be asked to provide the Glance ID for the image.
  These IDs are per-cloud, so that there are three options right now:
  bare metal/CHI at University of Chicago, bare metal/CHI at TACC, and
  OpenStack/KVM at TACC. You will need to provide at least one
  appliance, but may want to provide all three.
\item
  Go to
  the~\href{https://www.chameleoncloud.org/appliances/create/}{Appliance
  Catalog Create Appliance web form}, fill out, and submit the form. Be
  prepared to provide the following information: a descriptive name
  (this sometimes requires some thought!), author and support contact,
  version, and an informative description. The description is a very
  important part of the appliance record; others will use it to evaluate
  if the appliance contains tools they need for their research so it
  makes sense to prepare it carefully. To make your description
  effective you may want to think of the following questions: what does
  the appliance contain? what are the specific packages and their
  versions? what is it useful for? where can it be deployed and/or what
  restrictions/limitations does it have? how should users connect to it
  / what accounts are enabled?
\end{enumerate}

If you are adding a complex appliance, skip the image ID fields and
enter your template instead in the dedicated text box.

As always, if you encounter any problems or want to share with us
additional improvements we should do to the process, please don't
hesitate to~\href{https://www.chameleoncloud.org/help/}{submit a
ticket}.~

\subsection{How can I manage an appliance on Appliance
Catalog?}\label{how-can-i-manage-an-appliance-on-chameleon-appliance-catalog}

If you are the owner of the appliance,~you can edit~the appliance
data,~such as the description or the support information. Browse to the
appliance that you want to edit and view its Details page. At the top
right of the page is an Edit button. You will be presented with the same
web form as when creating the appliance, pre-filled with the appliances
current information. Make changes as necessary and click Save at the
bottom of the page.

And finally, you can delete appliances you had made available.~{Browse
to the appliance that you want to delete~and click Edit on the
Appliance~Details page. At the bottom of the page is a Delete button.
You will be asked to confirm once more that you do want to delete this
appliance}. After confirming, the appliance will be removed and no
longer listed on the Appliance Catalog.

\subsection{Why are there different image IDs  for the same
appliance?}\label{why-are-there-different-image-ids-for-kvmtacc-chitacc-and-chiuc-for-the-same-appliance}

The three clouds forming the Chameleon testbed are fully separated, each
having its own Glance image repository. The same appliance
image~uploaded to the three clouds will produce three different image
IDs.

In addition, it is sometimes needed to customize an appliance image for
each site, resulting in slightly different image files.

\subsection{Can I use another operating system on bare-metal?}\label{can-i-useubuntudebian-oranother-operating-system-rather-than-centos-on-bare-metal}

The recommended appliance for Chameleon is CentOS 7~(supported by
Chameleon staff), or appliances built on top of it.\\
These appliances provide~Chameleon-specific customizations, such as
login using the~cc account, the~cc-checks utility to verify hardware
against our resource registry, gathering of metrics, etc.

Since 2016, we also provide and~support Ubuntu 14.04 and
16.04~appliances with the same functionality.

\section{Bare Metal Troubleshooting}\label{bare-metal-troubleshooting}

\subsection{Why are my Bare Metal instances failing to
launch?}\label{why-are-my-bare-metal-instances-failing-to-launch}

The Chameleon Bare Metal clouds require users to reserve resources
before allowing them to launch instances. Please follow the
\href{https://www.chameleoncloud.org/docs/bare-metal/}{documentation}
and make sure that:

\begin{enumerate}
\item
  You have created a lease and it has started (the associated
  reservation is shown as \textbf{Active})
\item
  You have selected your reservation in the \textbf{Launch Instance}
  panel
\end{enumerate}

If you still cannot start instances, please
\href{https://www.chameleoncloud.org/user/help/}{open a ticket with our
help desk}.

\section{OpenStack KVM Troubleshooting}\label{openstack-kvm-troubleshooting}

\subsection{Why are my OpenStack KVM instances failing to
launch?}\label{why-are-my-openstack-kvm-instances-failing-to-launch}

If you get an error stating that~\textbf{No valid host was found}, it
might be caused by a lack of resources in the cloud. The Chameleon staff
continuously monitors the utilization of the testbed, but there might be
times when no more resources are available. If the error persists,
please~\href{https://www.chameleoncloud.org/user/help/}{open a ticket
with our help desk}.

\subsection{Why can't I ping or SSH to my
instance?}\label{why-cant-i-ping-or-ssh-to-my-instance}

While the possibility that the system is being taking over by nanites
should not be discounted too easily, it is always prudent to first
check~for the~following issues:

\begin{itemize}
\item
  Do you have a floating IP associated with your instance?~By default,
  instances do not have publicly-accessible IP addresses assigned. See
  the \textbf{Managing Virtual Machine Instances} section in the
  \href{https://www.chameleoncloud.org/docs/user-guides/openstack-kvm-user-guide/}{User
  Guide}.
\item
  Does your security group allow incoming ICMP (e.g. ping) traffic?~By
  default, firewall rules do not allow ping to your instances. If you
  wish to enable it, see the \textbf{Firewall (Access Security)} section
  in the
  \href{https://www.chameleoncloud.org/docs/user-guides/openstack-kvm-user-guide/}{User
  Guide}.
\item
  Does your security group allow incoming SSH (TCP port 22) traffic?~By
  default, firewall rules do not allow SSH to your instances. If you
  wish to enable it, see the \textbf{Firewall (Access Security)} section
  in the
  \href{https://www.chameleoncloud.org/docs/user-guides/openstack-kvm-user-guide/}{User
  Guide}.
\end{itemize}

~If none of these~solve~your problem,
please~\href{https://www.chameleoncloud.org/user/help/}{open a ticket
with our help desk}, and send~us the results of the above (and any
evidence of nanites you find as well).

\section{Create your own SSH key pairs}\label{create-your-own-ssh-key-pairs}

The following document describes the procedure on how you can create an
SSH key pair on your Unix, Linux or Windows operating system.

\subsection{For Linux / Mac OS X}\label{for-linux-mac-os-x}

Open a terminal window:

\begin{itemize}
\tightlist
\item
  In a Mac OS X system, click on your launchpad and search for terminal
\item
  In an Ubuntu system you can use the keys Ctrl+Alt+T (for desktop
  version)
\end{itemize}

Access the SSH key pairs directory; {i}n your terminal type the command:

\begin{verbatim}
$ cd ~/.ssh
\end{verbatim}

Create your ssh key pair (public and private keys);~~in~the
\texttt{.ssh} directory, type the command:

\begin{verbatim}
$ ssh-keygen
\end{verbatim}

Press the enter key, then~enter a name for your key.

After completing the previous step, a message stating ``Enter file in
which to save the Key'' will be displayed. Enter the name of your
preference. I will use in this example the name ``sample-key''.{~}Then
press the enter key.

{Then,~}you will be requested to enter a passphrase for your
key.~Entering a passphrase is not necessary, so you can proceed to leave
it blank and press enter.{~}You will receive a message ``Enter same
passphrase again:'' so just leave it blank and press enter.

Since we are still in the \texttt{.ssh} directory, now you can see your
newly created key by typing:

\begin{verbatim}
$ ls
\end{verbatim}

You will see two files:

\begin{itemize}
\tightlist
\item
  sample-key (containing the private key)
\item
  sample-key.pub (containing the public key)
\end{itemize}

Then, provide the public key to your cloud system or individual
instance. To add a key pair in Chameleon,~access one of the resource
dashboards and go the following tabs:

~ ~ Compute \textgreater{} Access and Security \textgreater{} Key Pairs
\textgreater{} Import Key Pair

In this window, you only need to provide a name for your key pair and
paste your public key pair in the ``Public Key'' window. To obtain the
contents of your public key, access your local \texttt{.ssh} directory
through your terminal and use the command:

\begin{verbatim}
$ cat sample-key.pub
\end{verbatim}

Select and copy the contents displayed starting ssh-rsa all the way to
the end. Paste these contents into the ``Public Key'' window~mentioned
earlier.

Whenever you are creating an instance in Chameleon, you will have an
option to select the Public Key you just imported. Once selected, this
public key will be inserted into the instance's
\textasciitilde{}/.ssh/known\_hosts file. When a user attempts to
connect to the instance, the private key provided by the user will be
validated against this public key in the known\_hosts file.

Connect to an instance from your terminal

After you have created a key pair and imported it in
Chameleon, you can connect to~any instance configured with this key
pair. To do so you can use the
command:

\begin{verbatim}
$ ssh -i ~/.ssh/sample-key cc@<instance ip address>
\end{verbatim}

The full process can be viewed in the figure below:

{\includegraphics[width=\columnwidth]{images/chameleon/ssh1.png}}

\subsection{PuTTY}

Follow the instructions from Section \ref{S:putty}. 

Before closing this window, select the entire public key and copy it
with ``Control-C''. Please note that everything should be copied,
including ``ssh-rsa''. This will be used when importing the key pair to
Openstack.

At this time, the public key has been created and copied. Now you can
now follow the steps described above (starting with the line ``Provide
the public key to your cloud system or individual instance'') to import
the generated key pair for use with Chameleon!


