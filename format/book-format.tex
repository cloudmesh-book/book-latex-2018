\newcommand{\TODAY}{\today~\currenttime}

%----------------------------------------------------------------------------------------
% remove pandoc commands
%----------------------------------------------------------------------------------------
\newcommand{\tightlist}{}
\newenvironment{Shaded}{}{}
\newenvironment{Highlighting}{}{}
\newcommand{\ConstantTok}[1]{\textcolor[rgb]{0.53,0.00,0.00}{{#1}}}
\newcommand{\SpecialCharTok}[1]{\textcolor[rgb]{0.25,0.44,0.63}{{#1}}}
\newcommand{\VerbatimStringTok}[1]{\textcolor[rgb]{0.25,0.44,0.63}{{#1}}}
\newcommand{\SpecialStringTok}[1]{\textcolor[rgb]{0.73,0.40,0.53}{{#1}}}
\newcommand{\ImportTok}[1]{{#1}}
\newcommand{\DocumentationTok}[1]{\textcolor[rgb]{0.73,0.13,0.13}{\textit{{#1}}}}
\newcommand{\AnnotationTok}[1]{\textcolor[rgb]{0.38,0.63,0.69}{\textbf{\textit{{#1}}}}}
\newcommand{\CommentVarTok}[1]{\textcolor[rgb]{0.38,0.63,0.69}{\textbf{\textit{{#1}}}}}
\newcommand{\VariableTok}[1]{\textcolor[rgb]{0.10,0.09,0.49}{{#1}}}
\newcommand{\ControlFlowTok}[1]{\textcolor[rgb]{0.00,0.44,0.13}{\textbf{{#1}}}}
\newcommand{\OperatorTok}[1]{\textcolor[rgb]{0.40,0.40,0.40}{{#1}}}
\newcommand{\BuiltInTok}[1]{{#1}}
\newcommand{\ExtensionTok}[1]{{#1}}
\newcommand{\PreprocessorTok}[1]{\textcolor[rgb]{0.74,0.48,0.00}{{#1}}}
\newcommand{\AttributeTok}[1]{\textcolor[rgb]{0.49,0.56,0.16}{{#1}}}
\newcommand{\InformationTok}[1]{\textcolor[rgb]{0.38,0.63,0.69}{\textbf{\textit{{#1}}}}}
\newcommand{\WarningTok}[1]{\textcolor[rgb]{0.38,0.63,0.69}{\textbf{\textit{{#1}}}}}
\newcommand{\KeywordTok}[1]{\textcolor[rgb]{0.00,0.44,0.13}{\textbf{{#1}}}}
\newcommand{\DataTypeTok}[1]{\textcolor[rgb]{0.56,0.13,0.00}{{#1}}}
\newcommand{\DecValTok}[1]{\textcolor[rgb]{0.25,0.63,0.44}{{#1}}}
\newcommand{\BaseNTok}[1]{\textcolor[rgb]{0.25,0.63,0.44}{{#1}}}
\newcommand{\FloatTok}[1]{\textcolor[rgb]{0.25,0.63,0.44}{{#1}}}
\newcommand{\CharTok}[1]{\textcolor[rgb]{0.25,0.44,0.63}{{#1}}}
\newcommand{\StringTok}[1]{\textcolor[rgb]{0.25,0.44,0.63}{{#1}}}
\newcommand{\CommentTok}[1]{\textcolor[rgb]{0.38,0.63,0.69}{\textit{{#1}}}}
\newcommand{\OtherTok}[1]{\textcolor[rgb]{0.00,0.44,0.13}{{#1}}}
\newcommand{\AlertTok}[1]{\textcolor[rgb]{1.00,0.00,0.00}{\textbf{{#1}}}}
\newcommand{\FunctionTok}[1]{\textcolor[rgb]{0.02,0.16,0.49}{{#1}}}
\newcommand{\RegionMarkerTok}[1]{{#1}}
\newcommand{\ErrorTok}[1]{\textcolor[rgb]{1.00,0.00,0.00}{\textbf{{#1}}}}
\newcommand{\NormalTok}[1]{{#1}}

%----------------------------------------------------------------------------------------

\input{format/structure} % Insert the commands.tex file which contains the majority of the structure behind the template

\begin{document}

%----------------------------------------------------------------------------------------
%	TITLE PAGE
%----------------------------------------------------------------------------------------

\begingroup
\thispagestyle{empty}
\begin{tikzpicture}[remember picture,overlay]
\node[inner sep=0pt] (background) at (current page.center) {\includegraphics[width=\paperwidth]{background.png}};
\draw (current page.center) node [fill=blue!2!white,fill
opacity=0.9,text opacity=1,inner
sep=1cm]{\Huge\centering\bfseries\sffamily\parbox[c][][t]{\paperwidth}{\centering
    \TITLE \\[15pt] % Book title
    {\Large \SUBTITLE}\\[20pt] % Subtitle
    {\huge \AUTHOR \\ \Large \EMAIL} \\
    {\normalsize \TODAY} \\
    {\normalsize   \url{https://tinyurl.com/vonLaszewski-handbook} } \\
  }
}; % Author name
\end{tikzpicture}
\vfill
\endgroup

%----------------------------------------------------------------------------------------
%	COPYRIGHT PAGE
%----------------------------------------------------------------------------------------

\newpage
~\vfill
\thispagestyle{empty}

\noindent Copyright \copyright\ 2017\\
\AUTHOR \\

\noindent \EMAIL \\ % Copyright notice

% \noindent \textsc{Indiana University}\\ % Publisher

\noindent \url{https://github.com/cloudmesh/classes}\\

\noindent \url{https://tinyurl.com/vonLaszewski-handbook}\\


\begin{comment}
\noindent Licensed under the Creative Commons
Attribution-NonCommercial 3.0 Unported License (the ``License''). You
may not use this file except in compliance with the License. You may
obtain a copy of the License at
\url{http://creativecommons.org/licenses/by-nc/3.0}. Unless required
by applicable law or agreed to in writing, software distributed under
the License is distributed on an \textsc{``as is'' basis, without
  warranties or conditions of any kind}, either express or
implied. See the License for the specific language governing
permissions and limitations under the License.\\ % License information

\end{comment}

\noindent \textit{First printing by Gregor von Laszewski, October 2017} % Printing/edition date

%----------------------------------------------------------------------------------------
%	TABLE OF CONTENTS
%----------------------------------------------------------------------------------------

%\usechapterimagefalse % If you don't want to include a chapter image, use this to toggle images off - it can be enabled later with \usechapterimagetrue

\chapterimage{TOC.png} % Table of contents heading image

\pagestyle{empty} % No headers

\tableofcontents % Print the table of contents itself

\cleardoublepage % Forces the first chapter to start on an odd page so it's on the right

\pagestyle{fancy} % Print headers again



\newcommand{\FILENAME}{%
    %\begin{fileremark}%
    %  \currfiledir\currfilename%
    %\end{fileremark}%
    \rfoot{\footnotesize\em\currfiledir\currfilename}}
\newcommand{\CHANGE}{\begin{changeremark}THIS WILL CHANGE
    \currfilename\end{changeremark}}
\newcommand{\TODO}[1]{\todo[inline,bordercolor=yellow,color=yellow!5]{TODO: #1}}
\newcommand{\DONE}[1]{DONE: \todo[inline,color=green!5]{#1}}

\newcommand{\FIGURE}[5]{
\begin{figure}[#1] 
  \centering 
    \includegraphics[width=#2\columnwidth]{#3} 
  \caption{#4}\label{#5} 
\end{figure} 
}

\newcommand{\URL}[1]{ \begin{footnotesize} \begin{itemize} \item \url{#1} \end{itemize} \end{footnotesize}}
\newcommand{\HREF}[2]{ \begi<n{itemize} \item \href{#1}{#2} \end{itemize} }

\newcommand{\video}[4]{%
{\em \hfill \href{#4}{#3 (#2)~\includegraphics[width=\baselineskip]{images/video.png}}}    \index{Video!#1!#3 (#2)}
}

\newcommand{\slides}[4]{%
  {\em  \hfill \href{#4}{#3 (#2)~\includegraphics[width=\baselineskip]{images/slide.png}}} \index{Slides!#1!#3 (#2)}
}

\DefineVerbatimEnvironment{verbatim}{Verbatim}{fontsize=\footnotesize,xleftmargin=.5in}


\definecolor{codegreen}{rgb}{0,0.6,0}
\definecolor{codegray}{rgb}{0.5,0.5,0.5}
\definecolor{codepurple}{rgb}{0.58,0,0.82}
\definecolor{backcolour}{rgb}{0.97, 0.97, 0.97}
 
\lstdefinestyle{code}{
    backgroundcolor=\color{backcolour},   
    commentstyle=\color{codegreen},
    keywordstyle=\color{magenta},
    numberstyle=\tiny\color{codegray},
    stringstyle=\color{codepurple},
    basicstyle=\footnotesize,
    breakatwhitespace=false,         
    breaklines=true,                 
    captionpos=b,                    
    keepspaces=true,                 
    numbers=left,                    
    numbersep=5pt,                  
    showspaces=false,                
    showstringspaces=false,
    showtabs=false,                  
    tabsize=2,
    xleftmargin=.3in
}

\lstset{language=python,
        style=code,
        basicstyle=\small\ttfamily} 
%        basicstyle=\small\ttfamily} 

\newcommand{\WHERE}[3]{$\square$~ #1 \ref{#2}. \nameref{#2} \hfill ~ #3 ~ \pageref{#2}\\}

\newenvironment{NOTE2}[1]
    {\begin{tcolorbox}[colback=blue!5!white,colframe=blue!75!black,%
                       fonttitle=\bfseries,size=small,%
                       lefthand width=2cm,sidebyside,%
                       sidebyside gap=6mm,%
                       lower separated=false,title=#1]%
    {\begin{tikzpicture}[overlay]
\node[draw=ocre!60,line
width=1pt,circle,fill=ocre!25,font=\sffamily\bfseries,inner
sep=2pt,outer sep=0pt] at (25pt,0pt){\textcolor{ocre}{\LARGE Note}};\end{tikzpicture}}
    \tcblower}
    {\end{tcolorbox}}

\newenvironment{COLORNOTE}[2]
    {\begin{tcolorbox}[%
        coltitle=#2!70!black,
        colback=#2!3!white,colframe=#2!20!white,%
        fonttitle=\bfseries,size=small,%
        title=#1]}
    {\end{tcolorbox}}

\newenvironment{NOTE}
    {\begin{COLORNOTE}{Note}{blue}}
    {\end{COLORNOTE}}

\newenvironment{WARNING}
    {\begin{COLORNOTE}{Warning}{red}}
    {\end{COLORNOTE}}

\newenvironment{IU}
    {\begin{COLORNOTE}{Indiana University}{red}}
    {\end{COLORNOTE}}



\newcommand{\YES}{\ding{51}}%
\newcommand{\NO}{\ding{55}}%
\newcommand{\SQUARE}{$\square$~}