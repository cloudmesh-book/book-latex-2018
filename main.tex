%----------------------------------------------------------------------------------------
%	PACKAGES AND OTHER DOCUMENT CONFIGURATIONS
%----------------------------------------------------------------------------------------

\documentclass[11pt,fleqn]{book} % Default font size and left-justified equations

\usepackage{parskip}
\usepackage{comment}
\usepackage{longtable}
\parindent 0pt
\setlength{\parskip}{6pt}
\usepackage{graphicx}

\graphicspath{{./notebooks/facedetection_files/}{images/}}

%----------------------------------------------------------------------------------------
% remove pandoc commands
%----------------------------------------------------------------------------------------
\newcommand{\tightlist}{}
\newenvironment{Shaded}{}{}
\newenvironment{Highlighting}{}{}
\newcommand{\ConstantTok}[1]{\textcolor[rgb]{0.53,0.00,0.00}{{#1}}}
\newcommand{\SpecialCharTok}[1]{\textcolor[rgb]{0.25,0.44,0.63}{{#1}}}
\newcommand{\VerbatimStringTok}[1]{\textcolor[rgb]{0.25,0.44,0.63}{{#1}}}
\newcommand{\SpecialStringTok}[1]{\textcolor[rgb]{0.73,0.40,0.53}{{#1}}}
\newcommand{\ImportTok}[1]{{#1}}
\newcommand{\DocumentationTok}[1]{\textcolor[rgb]{0.73,0.13,0.13}{\textit{{#1}}}}
\newcommand{\AnnotationTok}[1]{\textcolor[rgb]{0.38,0.63,0.69}{\textbf{\textit{{#1}}}}}
\newcommand{\CommentVarTok}[1]{\textcolor[rgb]{0.38,0.63,0.69}{\textbf{\textit{{#1}}}}}
\newcommand{\VariableTok}[1]{\textcolor[rgb]{0.10,0.09,0.49}{{#1}}}
\newcommand{\ControlFlowTok}[1]{\textcolor[rgb]{0.00,0.44,0.13}{\textbf{{#1}}}}
\newcommand{\OperatorTok}[1]{\textcolor[rgb]{0.40,0.40,0.40}{{#1}}}
\newcommand{\BuiltInTok}[1]{{#1}}
\newcommand{\ExtensionTok}[1]{{#1}}
\newcommand{\PreprocessorTok}[1]{\textcolor[rgb]{0.74,0.48,0.00}{{#1}}}
\newcommand{\AttributeTok}[1]{\textcolor[rgb]{0.49,0.56,0.16}{{#1}}}
\newcommand{\InformationTok}[1]{\textcolor[rgb]{0.38,0.63,0.69}{\textbf{\textit{{#1}}}}}
\newcommand{\WarningTok}[1]{\textcolor[rgb]{0.38,0.63,0.69}{\textbf{\textit{{#1}}}}}
\newcommand{\KeywordTok}[1]{\textcolor[rgb]{0.00,0.44,0.13}{\textbf{{#1}}}}
\newcommand{\DataTypeTok}[1]{\textcolor[rgb]{0.56,0.13,0.00}{{#1}}}
\newcommand{\DecValTok}[1]{\textcolor[rgb]{0.25,0.63,0.44}{{#1}}}
\newcommand{\BaseNTok}[1]{\textcolor[rgb]{0.25,0.63,0.44}{{#1}}}
\newcommand{\FloatTok}[1]{\textcolor[rgb]{0.25,0.63,0.44}{{#1}}}
\newcommand{\CharTok}[1]{\textcolor[rgb]{0.25,0.44,0.63}{{#1}}}
\newcommand{\StringTok}[1]{\textcolor[rgb]{0.25,0.44,0.63}{{#1}}}
\newcommand{\CommentTok}[1]{\textcolor[rgb]{0.38,0.63,0.69}{\textit{{#1}}}}
\newcommand{\OtherTok}[1]{\textcolor[rgb]{0.00,0.44,0.13}{{#1}}}
\newcommand{\AlertTok}[1]{\textcolor[rgb]{1.00,0.00,0.00}{\textbf{{#1}}}}
\newcommand{\FunctionTok}[1]{\textcolor[rgb]{0.02,0.16,0.49}{{#1}}}
\newcommand{\RegionMarkerTok}[1]{{#1}}
\newcommand{\ErrorTok}[1]{\textcolor[rgb]{1.00,0.00,0.00}{\textbf{{#1}}}}
\newcommand{\NormalTok}[1]{{#1}}

%----------------------------------------------------------------------------------------

\input{structure} % Insert the commands.tex file which contains the majority of the structure behind the template

\begin{document}

%----------------------------------------------------------------------------------------
%	TITLE PAGE
%----------------------------------------------------------------------------------------

\begingroup
\thispagestyle{empty}
\begin{tikzpicture}[remember picture,overlay]
\node[inner sep=0pt] (background) at (current page.center) {\includegraphics[width=\paperwidth]{background}};
\draw (current page.center) node [fill=ocre!30!white,fill
opacity=0.6,text opacity=1,inner
sep=1cm]{\Huge\centering\bfseries\sffamily\parbox[c][][t]{\paperwidth}{\centering
    Big Data Applications and Analytics\\[15pt] % Book title
{\Large Theory and Practice}\\[20pt] % Subtitle
{\huge Gregor von Laszewski}}}; % Author name
\end{tikzpicture}
\vfill
\endgroup

%----------------------------------------------------------------------------------------
%	COPYRIGHT PAGE
%----------------------------------------------------------------------------------------

\newpage
~\vfill
\thispagestyle{empty}

\noindent Copyright \copyright\ 2017 Gregor von Laszewski\\ % Copyright notice

\noindent \textsc{Indiana University}\\ % Publisher

\noindent \textsc{https://github.com/cloudmesh/classes}\\ % URL

\noindent Licensed under the Creative Commons
Attribution-NonCommercial 3.0 Unported License (the ``License''). You
may not use this file except in compliance with the License. You may
obtain a copy of the License at
\url{http://creativecommons.org/licenses/by-nc/3.0}. Unless required
by applicable law or agreed to in writing, software distributed under
the License is distributed on an \textsc{``as is'' basis, without
  warranties or conditions of any kind}, either express or
implied. See the License for the specific language governing
permissions and limitations under the License.\\ % License information

\noindent \textit{First printing, March 2017} % Printing/edition date

%----------------------------------------------------------------------------------------
%	TABLE OF CONTENTS
%----------------------------------------------------------------------------------------

%\usechapterimagefalse % If you don't want to include a chapter image, use this to toggle images off - it can be enabled later with \usechapterimagetrue

\chapterimage{chapter_head_1.pdf} % Table of contents heading image

\pagestyle{empty} % No headers

\tableofcontents % Print the table of contents itself

\cleardoublepage % Forces the first chapter to start on an odd page so it's on the right

\pagestyle{fancy} % Print headers again


%----------------------------------------------------------------------------------------
%	PART
%----------------------------------------------------------------------------------------

\part{Notebooks}

\input{notebooks/facedetection.tex}
\input{notebooks/scikit-learn-k-means.tex}
\input{notebooks/fingerprint_matching.tex}


%----------------------------------------------------------------------------------------
%	PART
%----------------------------------------------------------------------------------------

\part{Preface}

%----------------------------------------------------------------------------------------
%	CHAPTER
%----------------------------------------------------------------------------------------

\chapterimage{chapter_head_2.pdf} % Chapter heading image

\chapter{Introduction}


\FILENAME

\section{About}\label{about}

The document is based on selected material published at the following
Web page

\begin{itemize}
\item
  \url{https://cloudmesh.github.io/classes/}
\end{itemize}

It is part of a class taught at Indiana University. The class
communication takes place at:

\begin{itemize}
\item
  \url{https://piazza.com/class/ix39m27czn5uw}
\end{itemize}

The PDF version will be made in future available at 

\begin{itemize}
\item
\url{https://github.com/laszewski/laszewski.github.io/raw/master/papers/vonLaszewski-bigdata.pdf}
\end{itemize}

This PDF document will be updated based on feedback from the students
and once we have now material available. For a more complete set of
information we recommend the students to visit the Web page.

\section{Citation}

The bibtex entry for this document is

\begin{verbatim}
@TechReport{las17handbook,
  author =       {Gregor von Laszewski},
  title =        {Handbook of Big Data Applications and Analytics},
  institution =  {Indiana University},
  year =         {2017},
  OPTtype =      {Draft},
  address =      {Smith Research Center, Bloomington, IN 47408},
  month =        dec,
  url={https://github.com/laszewski/laszewski.github.io/raw/master/papers/vonLaszewski-bigdata.pdf},
}
\end{verbatim}

\section{Contributors}

We like to acknowledge the following contributors that helped on this
document. Please notify us with your name and a brief commend on what
you contributed:

Descriptions provided in Section \ref{s:} wer contributed by the
following people that are either listed by full name or their
github.com id:

\begin{quotation}{\em
Abhijit Thakre, Abhishek Gupta, Abhishek Naik, Ajit Balaga, Anurag
Kumar Jain, Avadhoot Agasti, Badi' Abdul-Wahid, Cmbays, DIKSHA,
Dimitar Nikolov, Govind, Govind Mishra, Grace Li, Gregor von
Laszewski, Harshit Krishnakumar, Hyungro Lee, Jerome Mitchell, Jimmy
Ardiansyah, Jon, Jon Montgomery, Jordan Simmons, Juliette Zerick,
Karthik, Kumar Satyam, Mark McCombe, Matthew Lawson, Methkupalli
Vasanth, Miao Jiang, Miao Zhang, Milind Suryawanshi,
MilindSuryawanshi, Nandita Sathe, Naveen, Niteesh01, Piyush Rai,
Piyush Shinde, Prashanth, Pratik Jain, Rahul Raghatate, Rahul Singh,
Ribka Rufael, Ronak Parekh, Saber Sheybani, Sabyasachi Roy Choudhury,
Sagar Vora, Sahiti Korrapati, Scott McClary, Sean Shiverick,
SilviaKarim14, Sivaprasad Sushmita, Snehal Chemburkar, Sowmya Ravi,
Srikanth Ramanam, Sunanda Unni, SushmitaSivaprasad, Tony Liu, Vasanth
Methkupalli, Veera Marni, Vibhatha Abeykoon, Vibhatha Lakmal Abeykoon,
Vishwanath Kodre, William H Knapp III, acastrob, ak.15, alyez,
anveling, argetlam115, athakre, bhavesh37, cacoulte, cglmoocs,
elenadesigner, eunosm3, harkrish1, jemitchell, justbbusy, jzerick,
kartanba, karthick, karthick venkatesan, karthik-anba, kpvenkat,
ksrivatsav, lmundia, miaozhan, michaelsmith1983, mmccombe, nsathe,
piyurai, pratik11jain, ronak1182, sabyasachi087,
shah0112, sriramsitharaman, suunni, tifabi, tonythomascn, vasanth,
vibhatha, vkodre, vlabeyko, xl41, yatinsharma7
}\end{quotation}

\begin{description}
\item[John Doe] He contributed to none of teh sections as this is just
  an example.
\end{description}

\section{Conventions}

\subsection{Videos}

Videos to the class are refered to with embeded links into the PDF
document as follows: 

\video{About}{25:36}{Test Video}{https://www.youtube.com/watch?v=yC3PNkb_9mI}

An index will also be available in the index page
that lists on which page the video has been added.

\subsection{Slides}

Sides
\slides{About}{10}{Test slides}{PUT URL HERE}

\subsection{Images}

The video icon was copied from \url{http://www.freeiconspng.com/img/8039}.

\input{chapter/intro/disclaimer}
\begin{fileremark}\url{https://github.com/cloudmesh/classes/blob/master/docs/source/i523/2017/preface.rst}\end{fileremark}
\section{Preface}\label{preface}


%----------------------------------------------------------------------------------------
%	PART
%----------------------------------------------------------------------------------------

\part{I523}

%----------------------------------------------------------------------------------------
%	CHAPTER
%----------------------------------------------------------------------------------------

\input{chapter/intro/fall2017}


\chapter{Overview}

\section{Overview}\label{

\FILENAME

Using a paper as part of your project planing is an important learning
outcome. Instead of starting with a project we recommend taht you
start with a paper to direct your research.

This argument is made also by the following presentation.

\video{Writing}{57:39}{How to write a paper by Simon Peyton Jones}{https://www.youtube.com/watch?v=VK51E3gHENc}

We do recommend that you read the secitions in this part carefully as they will introduce you to important tools that make writing a paper relatively simple while allowing professional paper format and bibliography management tools.

To get a first impression we have also prepared a number of videos that may help you. However, note that the format for papers used in these videos is diffferent from the class and you must use the written documentation instead and use that format. Papaer not using our format will be returned without review. I suggest you start right from the beginning.

\begin{WARNING}
The videos that show you the ACM paper template that we do not use

\video{i524}{8:49}{ShareLaTeX }{https://youtu.be/PfhSOjuQk8Y}

\video{i524}{14:41}{jabref}{https://youtu.be/cMtYOHCHZ3k}

\end{WARNING}

\begin{exercise}\label{E:Documentation.1}
Watch the three lectures about How to write a paper, ShareLaTeX, and jabref.
\end{exercise}

\input{chapter/intro/organization}
\section{Git}\label{git}

This class uses open source technology and we like that you benefit from
material others in the class are developing or have developed. All
assignments are openly submitted to the class github for everyone to
see. As part of the goal of this class is to develop reusable
instructions, deployments, software, and examples. Such reuse is only
possible if the code is publicly available and others can benefit from
it. While using github.com we make sharing of information possible so
every one can benefit and achieve their best.

\subsection{Install}\label{install}

Information on how to install git can be found at

\URL{https://www.atlassian.com/git/tutorials/install-git}

\subsubsection{Config}\label{config}

Once you've got Git installed, several bits of configuration will
enhance your experience with the tool and better tune it to your
operating system. Let us tell you about settings for your username and
email address, line endings, and color, along with the settings'
associated configuration scopes.

\URL{https://www.youtube.com/watch?v=ZChtKFLiaNw}

It is important is that you always want to make sure that you want to
use the git config command to initialize git for the first time you use
it. This will ensure that you use on all resources the same Name and
e-mail so that git history and log will show consistently your checkins.
If you do not do this, your checkins in git do not show up in a
consistent fashion as a single user. This is done with the following
commands:

\begin{verbatim}
$ git config --global user.name "Albert Zweistein"
$ git config --global user.email albert.zweistein@gmail.com
\end{verbatim}

You can set also the editor with:

\begin{verbatim}
$ git config --global core.editor emacs
\end{verbatim}

You will also need to decide if you want to push branches individually
or all branches at the same time. It will be up to you to make what
whill work for you best. We found that the following seems to work best:

\begin{verbatim}
git config --global push.default matching
\end{verbatim}

More information about a first time setup is documented at:

\begin{verbatim}
* http://git-scm.com/book/en/Getting-Started-First-Time-Git-Setup
\end{verbatim}

To check your setup you can say:

\begin{verbatim}
$ git config --list
\end{verbatim}

In addition the tutorials from atlasian are a good source. However
remember that you may not use bitbucket as the repository, so ignore
those tutorials. We found the following useful

\begin{itemize}
\tightlist
\item
  What is git: \url{https://www.atlassian.com/git/tutorials/what-is-git}
\item
  Installing git:
  \url{https://www.atlassian.com/git/tutorials/install-git}
\item
  git config:
  \url{https://www.atlassian.com/git/tutorials/setting-up-a-repository\#git-config}
\item
  git clone:
  \url{https://www.atlassian.com/git/tutorials/setting-up-a-repository\#git-clone}
\item
  saving changes:
  \url{https://www.atlassian.com/git/tutorials/saving-changes}
\item
  collaborating with git:
  \url{https://www.atlassian.com/git/tutorials/syncing}
\end{itemize}

Please read the information on the screen when you set up

\subsection{Merge}\label{merge}

As we are allowing contribution by the community, they are best managed
through a merge with our upstream repository so you can update to the
newest status before you issue a pul request.

Make sure you have upstream repo defined:

\begin{verbatim}
$ git remote add upstream https://github.com/cloudmesh/classes
\end{verbatim}

Backup all your changed files - just in case you need them while merging
the changes back

Get latest from upstream:

\begin{verbatim}
$ git rebase upstream/master
\end{verbatim}

In this step, the conflicting file shows up (in my case it was
refs.bib):

\begin{verbatim}
$ git status
\end{verbatim}

should show the name of the conflicting file:

\begin{verbatim}
$ git diff <file name>
\end{verbatim}

should show the actual differences. May be in some cases, It is easy to
simply take latest version from upstream and reapply your changes.

So you can decide to checkout one version earlier of the specific file.
At this stage, the re-base should be complete. So, you need to commit
and push the changes to your fork:

\begin{verbatim}
$ git commit
$ git rebase origin/master
$ git push
\end{verbatim}

Then reapply your changes to refs.bib - simply use the backedup version
and use the editor to redo the changes.

At this stage, only refs.bib is changed:

\begin{verbatim}
$ git status
\end{verbatim}

should show the changes only in refs.bib. Commit this change using:

\begin{verbatim}
$ git commit -a -m "new:usr: <message>"
\end{verbatim}

And finally push the last commited change:

\begin{verbatim}
$ git push
\end{verbatim}

The changes in the file to resolve merge conflict automatically goes to
the original pull request and the pull request can be merged
automatically.

You still have to issue the pull request from the Github Web page so it
is registered with the upstream repository.

\input{chapter/intro/project}
\chapter{Assignments}\label{c:assignments}

\section{Assignments E222}
\label{s:e222-assignments}
\index{Assignments!E222}

All assignments are due Monday morning at 9:00 am est. No exceptions unless otherwise specified.  

\subsection{Bio Post}

\begin{exercise}\label{a:e222-bio-piazza}

{\bf Bio Post on Piazza.} Please post a formal bio to piazza

\end{exercise}
 
\begin{exercise} \label{a:e222-bio-googledocs}

{\bf Bio Post in Google doc.} Due Jan XX

After you have posted it to piazza copy your updated formal bios into the following document.  Make sure you use 3rd person and stay formal. This is a formal bio. Comment on the effectiveness of using the cloud service for this task. A the end of the document. This assignment does not replace the post of the bio to piazza, it is used to gather all bios in one document and to evaluate if google docs is a good tool for this kind of task. Remember we have lots of students and google is used often just with small groups.
 
 \smallskip

 {\hfill \href{https://docs.google.com/document/d/1ejzlKYqC3dLac8WXVpcPQsJh1j4BDqRxxgGg1cFQbeQ/edit?usp=sharing}{E222 Link to google doc $\mapsto$}}

\end{exercise}

<<<<<<< HEAD
\section{Assignments E516}\label{s:e516-assignments}
=======
\section{Assignments E516, I524, E616}
\label{s:e516/524/616-assignments}
>>>>>>> c665577c0fee6db15175c03403a5adaeef13b724
\index{Assignments!E516}

All assignments are due Monday morning at 9:00 am est. No exceptions unless otherwise specified.

\subsection{Bio Post}

\begin{exercise} \label{a:e516-bio-piazza}

{\bf Bio Post on Piazza.} Please post a formal bio to piazza

\end{exercise}

\begin{exercise} \label{a:e516-bio-googledocs}

 {\bf Bio Post in Google doc.} Due Jan XX
 
 After you have posted it to piazza copy your updated formal bios into the following document.  Make sure you use 3rd person and stay formal. This is a formal bio. Comment on the effectiveness of using the cloud service for this task. A the end of the document. This assignment does not replace the post of the bio to piazza, it is used to gather all bios in one document and to evaluate if google docs is a good tool for this kind of task. Remember we have lots of students and google is used often just with small groups.

 \smallskip

 {\hfill \href{https://docs.google.com/document/d/1ejzlKYqC3dLac8WXVpcPQsJh1j4BDqRxxgGg1cFQbeQ/edit?usp=sharing}{E516 Link to google doc $\mapsto$}}

 \end{exercise}

\subsection{Big Data Collaboration}

\begin{exercise} \label{a:e516-big-data-and-collaboration}

{\bf Big data and collaboration.} Due Jan 22

The purpose of this assignment is multifold; test the ability of Google docs to be used in collaborative fashion by more than a small group and report on the experience. Good Things and bad things, learn on how to use Google docs with headings and table of contents learn how to gather resources quickly with hyperlinks to web resources or articles and translate them into formal academic references. Most importantly convey some very important feature of big data.Contribute this into the handbook for everyone's benefit (done by TAs). Your task is to identify Big Data size related articles and Web resources and produce a historical development of the growth of this data

  {\hfill \href{https://docs.google.com/document/d/1ZHNdhX_Jx7uBQo0kthSYQ6TQR8_KNbgOwH2EuqBQcjY/edit?usp=sharing}{E516 Link to google doc $\mapsto$}}
 
\end{exercise}

\subsection{New Technology List}

\begin{exercise} \label{a:e516-new-tech-list}

{\bf New Technology List} Due: Jan 29

The handbook contains a large number of technologies to which an abstract is provided. Your task is to identify FIRST not to do an abstract but to collaboratively gather a LIST of new technologies that are important in Cloud and Big Data. We suggest doing this in a google docs document first. Write Lastname, Firstname, class id behind the technology so we know who contributed it. Indicate also if commercial, or open source, We are mostly interested in open source activities. Keep the list sorted by alphabet. Use a bullet so formatting is preserved

 
{\hfill \href{https://docs.google.com/document/d/1LeHGHTSBbaPXYVor0efhmi5W7JJjS7EQHABHqgRAPuU/edit?usp=sharing}{E516 Link to google doc $\mapsto$}}
 
Example: OpenWhisk,https://openwhisk.apache.org/, open source, Gregor von Laszewski, e516
 
\end{exercise}

\subsection{New Technology Abstract}

\begin{exercise} \label{a:e516-new-tech-abstracts}

<<<<<<< HEAD
{\bf New Technology Abstract}
Due date: Feb 5th

We have gathered the technology list \href{https://piazza.com/class/jbkvbp3ed3m2ez?cid=50}{tech list document} a number of technologies that are not yet covered in the handbook or need improvement in the handbook.\\

The TAs will be selecting about 5 technologies for each student. Each student will write high-quality non-plagiarized abstracts which bibtex references.\\
 
Details will be announced by the TAs\\
 
Learning outcomes:\\

\noindent Identify how to not plagiarize\\
Work in a large team (with coordination by TAs)\\
Use bibtex and jabref for reference management which you will be using for your final paper\\
Find new trends in big data and cloud computing\\

\end{exercise}

\section{Assignments I524} \label{s:i524-assignments}
\index{Assignments!I524}

All assignments are due Monday morning at 9:00 am est. No exceptions unless otherwise specified.

\subsection{Bio Post}

\begin{exercise} \label{a:i524-bio-piazza}

{\bf Bio Post on Piazza.} Please post a formal bio to piazza

\end{exercise}

\begin{exercise} \label{a:i524-bio-googledocs}

 {\bf Bio Post in Google doc.} Due Jan XX
 
 After you have posted it to piazza copy your updated formal bios into the following document.  Make sure you use 3rd person and stay formal. This is a formal bio. Comment on the effectiveness of using the cloud service for this task. A the end of the document. This assignment does not replace the post of the bio to piazza, it is used to gather all bios in one document and to evaluate if google docs is a good tool for this kind of task. Remember we have lots of students and google is used often just with small groups.

 \smallskip

{\hfill \href{https://docs.google.com/document/d/1ejzlKYqC3dLac8WXVpcPQsJh1j4BDqRxxgGg1cFQbeQ/edit?usp=sharing}{I524 Link to google doc $\mapsto$}}

 \end{exercise}

\subsection{Big Data Collaboration}

\begin{exercise} \label{a:i524-big-data-and-collaboration}

{\bf Big data and collaboration.} Due: Jan 22

The purpose of this assignment is multifold; test the ability of Google docs to be used in collaborative fashion by more than a small group and report on the experience. Good Things and bad things, learn on how to use Google docs with headings and table of contents learn how to gather resources quickly with hyperlinks to web resources or articles and translate them into formal academic references. Most importantly convey some very important feature of big data.Contribute this into the handbook for everyone's benefit (done by TAs). Your task is to identify Big Data size related articles and Web resources and produce a historical development of the growth of this data

\smallskip

{\hfill \href{https://docs.google.com/document/d/1ZHNdhX_Jx7uBQo0kthSYQ6TQR8_KNbgOwH2EuqBQcjY/edit?usp=sharing}{I524 Link to google doc $\mapsto$}}

\end{exercise}

\subsection{New Technology List}

\begin{exercise} \label{a:i524-new-tech-list}
{\bf New Technology List} Due: Jan 29

The handbook contains a large number of technologies to which an abstract is provided. Your task is to identify FIRST not to do an abstract but to collaboratively gather a LIST of new technologies that are important in Cloud and Big Data. We suggest doing this in a google docs document first. Write Lastname, Firstname, class id behind the technology so we know who contributed it. Indicate also if commercial, or open source, We are mostly interested in open source activities. Keep the list sorted by alphabet. Use a bullet so formatting is preserved

{\hfill \href{https://docs.google.com/document/d/1LeHGHTSBbaPXYVor0efhmi5W7JJjS7EQHABHqgRAPuU/edit?usp=sharing}{I524 Link to google doc $\mapsto$}}

\smallskip
 
Example: OpenWhisk,https://openwhisk.apache.org/, open source, Gregor von Laszewski, i524
 
\end{exercise}
=======
\subsection{New Technology List}

\begin{exercise} \label{E:new-tech}
Due: Jan 29

The handbook contains a large number of technologies to which an
abstract is provided.

Your task is to identify FIRST not to do an abstract but to
collaboratively gather a LIST of new technologies that are important
in Cloud and Big Data. We suggest doing this in a google docs document
first. Write Lastname, Firstname, class id behind the technology so we
know who contributed it. Indicate also if commercial, or open source,
We are mostly interested in open source activities. Keep the list
sorted by alphabet. Use a bullet so formatting is preserved
>>>>>>> c665577c0fee6db15175c03403a5adaeef13b724

\subsection{New Technology Abstract}

<<<<<<< HEAD
\begin{exercise} \label{a:i524-new-tech-abstracts}
 
{\bf New Technology Abstract} Due date: Feb 5th

We have gathered with the technology list (https://piazza.com/class/jbkvbp3ed3m2ez?cid=50) a number of technologies that are not yet covered in the handbook or need improvement in the handbook.\\

The TAs will be selecting about 5 technologies for each student. Each student will write high-quality non-plagiarized abstracts which bibtex references.\\
 
Details will be announced by the TAs\\
 
Learning outcomes:\\

\noindent Identify how to not plagiarize\\
Work in a large team (with coordination by TAs)\\
Use bibtex and jabref for reference management which you will be using for your final paper\\
Find new trends in big data and cloud computing\\

\end{exercise}


\section{Assignments E616} \label{s:e616-assignments}
\index{Assignments!E616}

All assignments are due Monday morning at 9:00 am est. No exceptions unless otherwise specified.

\subsection{Bio Post}

\begin{exercise}\label{a:e616-bio-piazza}
{\bf Bio Post on Piazza.} Please post a formal bio to piazza
\end{exercise}

\begin{exercise} \label{a:e616-bio-googledocs}

 {\bf Bio Post in Google doc.} Due Date: Jan XX
 
 After you have posted it to piazza copy your updated formal bios into the following document. Make sure you use 3rd person and stay formal. This is a formal bio. Comment on the effectiveness of using the cloud service for this task. A the end of the document. This assignment does not replace the post of the bio to piazza, it is used to gather all bios in one document and to evaluate if Google docs is a good tool for this kind of task. Remember we have lots of students and Google is used often just with small groups.
 
 \smallskip

 {\hfill \href{https://docs.google.com/document/d/1ejzlKYqC3dLac8WXVpcPQsJh1j4BDqRxxgGg1cFQbeQ/edit?usp=sharing} {E616 Link to google doc $\mapsto$}}

 \end{exercise}

\subsection{Big Data Collaboration}

\begin{exercise} \label{a:e616-big-data-and-collaboration}

{\bf Big data and collaboration.} Due Date: Jan XX

The purpose of this assignment is multifold; test the ability of Google docs to be used in collaborative fashion by more than a small group and report on the experience. Good Things and bad things, learn on how to use Google docs with headings and table of contents learn how to gather resources quickly with hyperlinks to web resources or articles and translate them into formal academic references. Most importantly convey some very important feature of big data.Contribute this into the handbook for everyone's benefit (done by TAs). Your task is to identify Big Data size related articles and Web resources and produce a historical development of the growth of this data

  {\hfill \href{https://docs.google.com/document/d/1ZHNdhX_Jx7uBQo0kthSYQ6TQR8_KNbgOwH2EuqBQcjY/edit?usp=sharing}{E516 Link to google doc $\mapsto$}}

\end{exercise}

\subsection{New Technology List}
=======
Example: 

OpenWhisk, \url{https://openwhisk.apache.org/}, open source, Gregor von Laszewski, e616

\end{exercise}

>>>>>>> c665577c0fee6db15175c03403a5adaeef13b724

\begin{exercise} \label{a:e6161-new-tech-list}

{\bf New Technology List} Due: Jan 29 

<<<<<<< HEAD
The handbook contains a large number of technologies to which an abstract is provided. Your task is to identify FIRST not to do an abstract but to collaboratively gather a LIST of new technologies that are important in Cloud and Big Data. We suggest doing this in a google docs document first. Write Lastname, Firstname, class id behind the technology so we know who contributed it. Indicate also if commercial, or open source, We are mostly interested in open source activities. Keep the list sorted by alphabet. Use a bullet so formatting is preserved
=======
\subsection{New Technology Abstract}
>>>>>>> c665577c0fee6db15175c03403a5adaeef13b724

\begin{exercise} \label{E:new-tech-abstract}
 
<<<<<<< HEAD
{\hfill \href{https://docs.google.com/document/d/1LeHGHTSBbaPXYVor0efhmi5W7JJjS7EQHABHqgRAPuU/edit?usp=sharing}{E616 Link to google doc $\mapsto$}}
 
Example: OpenWhisk,https://openwhisk.apache.org/, open source, Gregor von Laszewski, e616
 
\end{exercise}


\subsection{New Technology Abstract}

\begin{exercise}\label{a:e6161-new-tech-abstracts}
 
{\bf New Technology Abstract} Due date: Feb 5th

We have gathered with the technology list (https://piazza.com/class/jbkvbp3ed3m2ez?cid=50) a number of technologies that are not yet covered in the handbook or need improvement in the handbook.\\

The TAs will be selecting about 5 technologies for each student. Each student will write high-quality non-plagiarized abstracts which bibtex references.\\
 
Details will be announced by the TAs\\
 
Learning outcomes:

\noindent Identify how to not plagiarize\\
Work in a large team (with coordination by TAs)\\
Use bibtex and jabref for reference management which you will be using for your final paper\\
Find new trends in big data and cloud computing\\

\end{exercise}
=======
Due date: Feb 5th

We have gathered with the technology list 

\url{https://piazza.com/class/jbkvbp3ed3m2ez?cid=50}

a number of technologies that are not yet covered in the handbook or
need improvement in the handbook.

The TAs will be selecting about 5 technologies for each student. Each
student will write high-quality non-plagiarized abstracts which bibtex
references.
 
Learning outcomes:

\begin{itemize}

\item Identify how to not plagiarize
\item Work in a large team (with coordination by TAs)
\item Use bibtex and jabref for reference management which you will be using for your final paper
\item Find new trends in big data and cloud computing

\end{itemize}
>>>>>>> c665577c0fee6db15175c03403a5adaeef13b724

\end{exercise}




\chapter{Theory}

\FILENAME

\section{Introduction}\label{introduction}

\begin{description}
\item[You may find that some videos may have a different lesson,]
section or unit number. Please ignore this. In case the content does not
correspond to the title, please let us know.
\end{description}

This section has a technical overview of course followed by a broad
motivation for course hosted at www-cloudmesh-classes.

The course overview covers it's content and structure. It presents an
introduction to general field of Big Data and Analytics. We are
especially analysing the many different application areas in which Big
Data can be applied. As Big Datais typically not just used in isolation
but is part of a larger Informatics issue for a particular field we also
use the term X-Informatics, where X defines a usecase or area of
specialization in which Big Data is applied to. As such we organize the
class around the the \emph{Rallying Cry} of course: Use Clouds running
Data Analytics Collaboratively processing Big Data to solve problems in
X-Informatics.

The courses is set up as a number of lessons that are typically between
20 minutes to an hour. The lessons are either provided as written
documents or as video lectures. They are enhanced by an in person
meeting that takes place either in a lecture room for residential
students or as online meeting for online students.

The course covers a mix of applications (the X in X-Informatics) and
technologies needed to support the field electronically i.e. to process
the application data. The overview ends with a discussion of course
content at highest level. The course starts with a motivation
summarizing clouds and data science, then units describing applications
in areas such as Physics, e-Commerce, Web Search and Text mining,
Health, Sensors and Remote Sensing). These are interspersed with
discussions of infrastructure (clouds) and data analytics (algorithms
like clustering and collaborative filtering used in applications). The
course uses Python as primary programming language. We will be
introducing practical use of cloud resources so that you have the
oportunity to explore example analytics applications on smaller data
sets that you define.

The course motivation starts with striking examples of the data deluge
with examples from research, business and the consumer. The growing
number of jobs in data science is highlighted. He describes industry
trend in both clouds and big data. Then the cloud computing model
developed at amazing speed by industry is introduced. The 4 paradigms of
scientific research are described with growing importance of data
oriented version.He covers 3 major X-informatics areas: Physics,
e-Commerce and Web Search followed by a broad discussion of cloud
applications. Parallel computing in general and particular features of
MapReduce are described.

We discuss in this course include the following topics. We may change
the order of the topics to allow for maximal flexibility and parallel
learning experiences.

Writing Track:

\begin{itemize}
\item  Writing a short review article
\item  Writing a porject or term report
\end{itemize}

Theory Track:

\begin{itemize}
\item  Motivation: Big Data and the Cloud; Centerpieces of the Future Economy
\item  Introduction: What is Big Data, Data Analytics
\item  Use Cases: Big Data Use Cases Survey

  \begin{itemize}
  \item    Use Case, Physics Discovery of Higgs Particle
  \item    Use Case: e-Commerce and Lifestyle with recommender systems
  \item    Use Case: Web Search and Text Mining and their technologies
  \item    Use Case: Sports
  \item    Use Case: Health
  \item    Use Case: Sensors
  \item    Use Case: Radar for Remote Sensing.
  \end{itemize}

\item Parallel Computing Overview and familiar examples
\item Cloud Technology for Big Data Applications \& Analytics
\end{itemize}

Practice Track:

\begin{itemize}
\item
  Python for Big Data Applications and Analytics: NumPy, SciPy,
  MatPlotlib
\item
  Using FutureGrid for Big Data Applications and Analytics Course
\item
  Using Chameleon Cloud for Big Data Applications and Analytics Course
\item
  {[}optional{]} Using Plotviz Software for Displaying Point
  Distributions in 3D
\item
  Recommender Systems - K-Nearest Neighbors, Clustering and heuristic
  methods
\item
  PageRank
\item
  Kmeans
\item
  MapReduce
\item
  Kmeans and MapReduce Parallelism
\end{itemize}

\subsection{Course Motivation}\label{course-motivation}

We motivate the study of X-informatics by describing data science and
clouds. He starts with striking examples of the data deluge with
examples from research, business and the consumer. The growing number of
jobs in data science is highlighted. He describes industry trend in both
clouds and big data.

He introduces the cloud computing model developed at amazing speed by
industry. The 4 paradigms of scientific research are described with
growing importance of data oriented version. He covers 3 major
X-informatics areas: Physics, e-Commerce and Web Search followed by a
broad discussion of cloud applications. Parallel computing in general
and particular features of MapReduce are described. He comments on a
data science education and the benefits of using MOOC's.

\subsubsection{Emerging Technologies}\label{emerging-technologies}

This presents the overview of talk, some trends in computing and data
and jobs. Gartner's emerging technology hype cycle shows many areas of
Clouds and Big Data. We highlight 6 issues of importance: economic
imperative, computing model, research model, Opportunities in advancing
computing, Opportunities in X-Informatics, Data Science Education


\video{Introduction}{40:14}{Motivation}  {https://drive.google.com/file/d/0B1Of61fJF7WsV2RvMlFzSDNPZEU/view?usp=sharing}
  
\slides{Introduction}{30}  {Motivation}{https://drive.google.com/file/d/0B8936_ytjfjmOUZraHc4M1ptczA/view?usp=sharing}


\subsubsection{Data Deluge}\label{data-deluge}

We give some amazing statistics for total storage; uploaded video and
uploaded photos; the social media interactions every minute; aspects of
the business big data tidal wave; monitors of aircraft engines; the
science research data sizes from particle physics to astronomy and earth
science; genes sequenced; and finally the long tail of science. The next
slide emphasizes applications using algorithms on clouds. This leads to
the rallying cry ``Use Clouds running Data Analytics Collaboratively
processing Big Data to solve problems in X-Informatics educated in data
science'`with a catalog of the many values of X''Astronomy, Biology,
Biomedicine, Business, Chemistry, Climate, Crisis, Earth Science,
Energy, Environment, Finance, Health, Intelligence, Lifestyle,
Marketing, Medicine, Pathology, Policy, Radar, Security, Sensor, Social,
Sustainability, Wealth and Wellness''


\video{Introduction}{30:38}  {Data Deluge}{https://www.youtube.com/watch?v=7VHPXJv3DN4}


\slides{Introduction}{20}  {Data  Deluge}{https://drive.google.com/open?id=0B8936_ytjfjmUXY3anBaeU9lLVU}

\subsubsection{Jobs}\label{jobs}

Jobs abound in clouds and data science. There are documented shortages
in data science, computer science and the major tech companies advertise
for new talent.


\video{Introduction}{9:39}  {Jobs}{https://www.youtube.com/watch?v=KsjiQS8uXDA}


\slides{Introduction}{8}  {Jobs}{https://drive.google.com/open?id=0B8936_ytjfjmaG50YW9TeWdvUTg}


\subsubsection{Industrial Trends}\label{industrial-trends}

Trends include the growing importance of mobile devices and comparative
decrease in desktop access, the export of internet content, the change
in dominant client operating systems, use of social media, thriving
Chinese internet companies.


\video{Introduction}{19:25} 
  {Industrial Trends}{https://www.youtube.com/watch?v=32vD7uN7fqY}


\slides{Introduction}{16}
  {Industrial
  Trends}{https://drive.google.com/open?id=0B8936_ytjfjmWW1SdXgxWkRLYjg}



\video{Introduction}{16:54}   {Industrial Trends  II}{https://www.youtube.com/watch?v=O8fgXAQcnvw}

\slides{Introduction}{16}
  {Indusrial
  Trends II}{https://drive.google.com/open?id=0B8936_ytjfjmeEV2R19ORzhBQVE}



\video{Introduction}{30:13} 
  {Indusrial Trends
  III}{https://www.youtube.com/watch?v=kW38MG7ukzs}

\slides{Introduction}{21}
  {Industrial
  Trends III}{https://drive.google.com/open?id=0B8936_ytjfjmNDZKcE1MSU45ZG8}


\subsubsection{Digital Disruption of Old
Favorites}\label{digital-disruption-of-old-favorites}

Not everything goes up. The rise of the Internet has led to declines in
some traditional areas including Shopping malls and Postal Services.

\video{Introduction}{32:54} 
{Digital Distruption
and transformation}{https://www.youtube.com/watch?v=bw9yYXwe7Bs} 



\slides{Introduction}{28}
  {Digital
  Distruption and transformation}{https://drive.google.com/open?id=0B8936_ytjfjmdW5CYnBtME9FVTQ}


\subsubsection{Computing Model}\label{computing-model}

\emph{Industry adopted clouds which are attractive for data analytics}

Clouds and Big Data are transformational on a 2-5 year time scale.
Already Amazon AWS is a lucrative business with almost a \$4B revenue.
We describe the nature of cloud centers with economies of scale and
gives examples of importance of virtualization in server consolidation.
Then key characteristics of clouds are reviewed with expected high
growth in Infrastructure, Platform and Software as a Service.


\video{Introduction}{24:03} 
  {Computing Model I}{https://www.youtube.com/watch?v=oYKTCKFGTco}


\slides{Introduction}{14}
  {Computing
  Model I}{https://drive.google.com/open?id=0B8936_ytjfjmTU9nNml2bUlsUHM}



\video{Introduction}{28:18} 
  {Computing Model II}{https://www.youtube.com/watch?v=km_eXHq7m3o}


\slides{Introduction}{27}
  {Computing
  Model II}{https://drive.google.com/open?id=0B8936_ytjfjmNHhLYnI0X0YxdFE}

\subsubsection{Research Model}\label{research-model}

\emph{4th Paradigm; From Theory to Data driven science?}

We introduce the 4 paradigms of scientific research with the focus on
the new fourth data driven methodology.


\video{Introduction}{7:33}  {Research Model}{https://www.youtube.com/watch?v=xkeECe3mmjI}


\slides{Introduction}{4}  {Research  Model}{https//drive.google.com/open?id=0B8936_ytjfjma0pMbHJnek02dDA}


\subsubsection{Data Science Process}\label{data-science-process}

We introduce the DIKW data to information to knowledge to wisdom
paradigm. Data flows through cloud services transforming itself and
emerging as new information to input into other transformations.


\video{Introduction}{15:42} {Data Science Process}{https://www.youtube.com/watch?v=KstIH2aQ60Y}


\slides{Introduction}{10}
  {Data  Science Process}{https://drive.google.com/open?id=0B8936_ytjfjmVDVZa01keW0wQmc}


\subsubsection{Physics-Informatics}\label{physics-informatics}

\emph{Looking for Higgs Particle with Large Hadron Collider LHC}

We look at important particle physics example where the Large hadron
Collider has observed the Higgs Boson. He shows this discovery as a bump
in a histogram; something that so amazed him 50 years ago that he got a
PhD in this field. He left field partly due to the incredible size of
author lists on papers.


\video{Introduction}{13:27} 
  {Physics-informatics}{https://www.youtube.com/watch?v=2A7Z741FCHs}

\slides{Introduction}{6}
  {Physics-inforamtics}{https://drive.google.com/open?id=0B8936_ytjfjmc2J2TWgwWGRwaFk}


\subsubsection{Recommender Systems}\label{recommender-systems}

Many important applications involve matching users, web pages, jobs,
movies, books, events etc. These are all optimization problems with
recommender systems one important way of performing this optimization.
We go through the example of Netflix \textasciitilde{}\textasciitilde{}
everything is a recommendation and muses about the power of viewing all
sorts of things as items in a bag or more abstractly some space with
funny properties.


\video{Introduction}{12:21}
  {Recommender Systems  I}{https://www.youtube.com/watch?v=LXhng3fcG9o}



\slides{Introduction}{9}
  {Recommender  Systems I}{https://drive.google.com/open?id=0B8936_ytjfjmOXlVd2FsSUkwekk}



\video{Introduction}{9:44} 
  {Recommender Systems
  II}{https://www.youtube.com/watch?v=Y4S0jY0yfEE}

\slides{Introduction}{6}
  {Recommender
  Systems II}{https://drive.google.com/open?id=0B8936_ytjfjmMzM2M3RhMEJ4bjQ}


\subsubsection{Web Search and Information
Retrieval}\label{web-search-and-information-retrieval}

This course also looks at Web Search and here we give an overview of the
data analytics for web search, Pagerank as a method of ranking web pages
returned and uses material from Yahoo on the subtle algorithms for
dynamic personalized choice of material for web pages.


\video{Introduction}{12:05}   {Web Search and  Information Retrieval}{https://www.youtube.com/watch?v=p-0NtNTzoh8}


\slides{Introduction}{6}  {Web  Search and Information Retrieval}{https://drive.google.com/open?id=0B8936_ytjfjmSm8zNmZ5VFJxRms}


\subsubsection{Cloud Application in
Research}\label{cloud-application-in-research}

We describe scientific applications and how they map onto clouds,
supercomputers, grids and high throughput systems. He likes the cloud
use of the Internet of Things and gives examples.


\video{Introduction}{33:51}{Cloud Applications  in Research}{https://www.youtube.com/watch?v=U3ZG2qOFpxE}


\slides{Introduction}{20}  {Cloud  Applications in Research}{https://drive.google.com/open?id=0B8936_ytjfjma0RhdU0zdkxmczA}

\subsubsection{Parallel Computing and
MapReduce}\label{parallel-computing-and-mapreduce}

We define MapReduce and gives a homely example from fruit blending.


\video{Introduction}{14:02}  {Computing and  MapReduce}{https://www.youtube.com/watch?v=aQ8NMxe9IsU}


\slides{Introduction}{9}  {Computing  and MapReduce}{https://drive.google.com/open?id=0B8936_ytjfjmeTl4NWhHRjJMOGc}

\subsubsection{Data Science Education}\label{data-science-education}

We discuss one reason you are taking this course
\textasciitilde{}\textasciitilde{} Data Science as an educational
initiative and aspects of its Indiana University implementation. Then
general; features of online education are discussed with clear growth
spearheaded by MOOC's where we use this course and others as an example.
He stresses the choice between one class to 100,000 students or 2,000
classes to 50 students and an online library of MOOC lessons. In olden
days he suggested `'hermit's cage virtual university''
\textasciitilde{}\textasciitilde{} gurus in isolated caves putting
together exciting curricula outside the traditional university model.
Grading and mentoring models and important online tools are discussed.
Clouds have MOOC's describing them and MOOC's are stored in clouds; a
pleasing symmetry.


\video{Introduction}{28:08}   {Data Science  Education}{https://www.youtube.com/watch?v=bA_eNjJTmRQ}


\slides{Introduction}{19}  {Data  Science Education}{https://drive.google.com/open?id=0B8936_ytjfjmT0J1RjYwY1VwZ1k}


\subsubsection{Conclusions}\label{conclusions}

The conclusions highlight clouds, data-intensive methodology,
employment, data science, MOOC's and never forget the Big Data ecosystem
in one sentence ``Use Clouds running Data Analytics Collaboratively
processing Big Data to solve problems in X-Informatics educated in data
science''


\video{Introduction}{4:59}  {Conclusions}{https://www.youtube.com/watch?v=FmcR5mrhYvk}

\slides{Introduction}{4}  {Conclusions}{https://drive.google.com/open?id=0B8936_ytjfjmVjRNeG1pdUNnMlE}


\subsubsection{Resources}\label{resources}

\begin{itemize}
\item
  \url{http://www.gartner.com/technology/home.jsp} and many web links
\item
  Meeker/Wu May 29 2013 Internet Trends D11 Conference
  \url{http://www.slideshare.net/kleinerperkins/kpcb-internet-trends-2013}
\item
  \url{http://cs.metrostate.edu/~sbd/slides/Sun.pdf}
\item
  Taming The Big Data Tidal Wave: Finding Opportunities in Huge Data
  Streams with Advanced Analytics, Bill Franks Wiley ISBN:
  978-1-118-20878-6
\item
  Bill Ruh
  \url{http://fisheritcenter.haas.berkeley.edu/Big_Data/index.html}
\item
  \url{http://www.genome.gov/sequencingcosts/}
\item
  CSTI General Assembly 2012, Washington, D.C., USA Technical Activities
  Coordinating Committee (TACC) Meeting, Data Management, Cloud
  Computing and the Long Tail of Science October 2012 Dennis Gannon
\item
  \url{http://www.microsoft.com/en-us/news/features/2012/mar12/03-05CloudComputingJobs.aspx}
\item
  \url{http://www.mckinsey.com/mgi/publications/big_data/index.asp}
\item
  Tom Davenport
  \url{http://fisheritcenter.haas.berkeley.edu/Big_Data/index.html}
\item
  \url{http://research.microsoft.com/en-us/people/barga/sc09_cloudcomp_tutorial.pdf}
\item
  \url{http://research.microsoft.com/pubs/78813/AJ18_EN.pdf}
\item
  \url{http://www.google.com/green/pdfs/google-green-computing.pdf}
\item
  \url{http://www.wired.com/wired/issue/16-07}
\item
  \url{http://research.microsoft.com/en-us/collaboration/fourthparadigm/}
\item
  Jeff Hammerbacher
  \url{http://berkeleydatascience.files.wordpress.com/2012/01/20120117berkeley1.pdf}
\item
  \url{http://grids.ucs.indiana.edu/ptliupages/publications/Where\%20does\%20all\%20the\%20data\%20come\%20from\%20v7.pdf}
\item
  \url{http://www.interactions.org/cms/?pid=1032811}
\item
  \url{http://www.quantumdiaries.org/2012/09/07/why-particle-detectors-need-a-trigger/atlasmgg/}
\item
  \url{http://www.sciencedirect.com/science/article/pii/S037026931200857X}
\item
  \url{http://www.slideshare.net/xamat/building-largescale-realworld-recommender-systems-recsys2012-tutorial}
\item
  \url{http://www.ifi.uzh.ch/ce/teaching/spring2012/16-Recommender-Systems_Slides.pdf}
\item
  \url{http://en.wikipedia.org/wiki/PageRank}
\item
  \url{http://pages.cs.wisc.edu/~beechung/icml11-tutorial/}
\item
  \url{https://sites.google.com/site/opensourceiotcloud/}
\item
  \url{http://datascience101.wordpress.com/2013/04/13/new-york-times-data-science-articles/}
\item
  \url{http://blog.coursera.org/post/49750392396/on-the-topic-of-boredom}
\item
  \url{http://x-informatics.appspot.com/course}
\item
  \url{http://iucloudsummerschool.appspot.com/preview}
\item
  \url{https://www.youtube.com/watch?v=M3jcSCA9_hM}
\end{itemize}



\section{Overview}\label{

\FILENAME

Using a paper as part of your project planing is an important learning
outcome. Instead of starting with a project we recommend taht you
start with a paper to direct your research.

This argument is made also by the following presentation.

\video{Writing}{57:39}{How to write a paper by Simon Peyton Jones}{https://www.youtube.com/watch?v=VK51E3gHENc}

We do recommend that you read the secitions in this part carefully as they will introduce you to important tools that make writing a paper relatively simple while allowing professional paper format and bibliography management tools.

To get a first impression we have also prepared a number of videos that may help you. However, note that the format for papers used in these videos is diffferent from the class and you must use the written documentation instead and use that format. Papaer not using our format will be returned without review. I suggest you start right from the beginning.

\begin{WARNING}
The videos that show you the ACM paper template that we do not use

\video{i524}{8:49}{ShareLaTeX }{https://youtu.be/PfhSOjuQk8Y}

\video{i524}{14:41}{jabref}{https://youtu.be/cMtYOHCHZ3k}

\end{WARNING}

\begin{exercise}\label{E:Documentation.1}
Watch the three lectures about How to write a paper, ShareLaTeX, and jabref.
\end{exercise}

\FILENAME

\section{Health Informatics Case
Study}\label{health-informatics-case-study}

This section starts by discussing general aspects of Big Data and Health
including data sizes, different areas including genomics, EBI, radiology
and the Quantified Self movement. We review current state of health care
and trends associated with it including increased use of Telemedicine.
We summarize an industry survey by GE and Accenture and an impressive
exemplar Cloud-based medicine system from Potsdam. We give some details
of big data in medicine. Some remarks on Cloud computing and Health
focus on security and privacy issues.

We survey an April 2013 McKinsey report on the Big Data revolution in US
health care; a Microsoft report in this area and a European Union report
on how Big Data will allow patient centered care in the future. Examples
are given of the Internet of Things, which will have great impact on
health including wearables. A study looks at 4 scenarios for healthcare
in 2032. Two are positive, one middle of the road and one negative. The
final topic is Genomics, Proteomics and Information Visualization.

\subsection{X-Informatics Case Study: Health
Informatics}\label{x-informatics-case-study-health-informatics}

\subsubsection{Overview}\label{overview}

\slides{Health}{Health}{131}{https://drive.google.com/open?id=0B6wqDMIyK2P7UGRJNmlkYkNkQk0}

This section starts by discussing general aspects of Big Data and Health
including data sizes, different areas including genomics, EBI, radiology
and the Quantified Self movement. We review current state of health care
and trends associated with it including increased use of Telemedicine.
We summarize an industry survey by GE and Accenture and an impressive
exemplar Cloud-based medicine system from Potsdam. We give some details
of big data in medicine. Some remarks on Cloud computing and Health
focus on security and privacy issues.

We survey an April 2013 McKinsey report on the Big Data revolution in US
health care; a Microsoft report in this area and a European Union report
on how Big Data will allow patient centered care in the future. Examples
are given of the Internet of Things, which will have great impact on
health including wearables. A study looks at 4 scenarios for healthcare
in 2032. Two are positive, one middle of the road and one negative. The
final topic is Genomics, Proteomics and Information Visualization.

\subsubsection{Big Data and Health}\label{big-data-and-health}

This lesson starts with general aspects of Big Data and Health including
listing subareas where Big data important. Data sizes are given in
radiology, genomics, personalized medicine, and the Quantified Self
movement, with sizes and access to European Bioinformatics Institute.

\video{Health}{10:02}{Big Data and Health}{https://www.youtube.com/watch?v=ZkM-yZJQ1Cg} 

\subsubsection{Status of Healthcare
Today}\label{status-of-healthcare-today}

This covers trends of costs and type of healthcare with low cost genomes
and an aging population. Social media and government Brain initiative.

\video{Health}{16:09}{Status of Healthcare Today}{https://www.youtube.com/watch?v=x9TpdMBqYrk} 

\subsubsection{Telemedicine (Virtual
Health)}\label{telemedicine-virtual-health}

This describes increasing use of telemedicine and how we tried and
failed to do this in 1994.

\video{Health}{8:21}{Telemedicine}{https://www.youtube.com/watch?v=Pe4CVXQaL_U} 


\subsubsection{Big Data and Healthcare
Industry}\label{big-data-and-healthcare-industry}

Summary of an industry survey by GE and Accenture.

\video{Health}{10:02}{Big Data and Healthcare Indusry}{https://www.youtube.com/watch?v=64YOUnRJVZU}


\subsubsection{Medical Big Data in the
Clouds}\label{medical-big-data-in-the-clouds}

An impressive exemplar Cloud-based medicine system from Potsdam.

\video{Health}{15:02}{Medical Big Data in the Clouds}{https://www.youtube.com/watch?v=GldSVijkJcM} 


\subsubsection{Medical image Big Data}\label{medical-image-big-data}

\video{Health}{6:33}{Midical Image Big Data}{https://www.youtube.com/watch?v=GOcVtwx2R2k} 

\subsubsection{Clouds and Health}\label{clouds-and-health}

\video{Health}{4:35}{Clouds and Health}{http://youtu.be/9Whkl_UPS5g}


\subsubsection{McKinsey Report on the big-data revolution in US health
care}\label{mckinsey-report-on-the-big-data-revolution-in-us-health-care}

This lesson covers 9 aspects of the McKinsey report. These are the
convergence of multiple positive changes has created a tipping point for
innovation; Primary data pools are at the heart of the big data
revolution in healthcare; Big data is changing the paradigm: these are
the value pathways; Applying early successes at scale could reduce US
healthcare costs by \$300 billion to \$450 billion; Most new big-data
applications target consumers and providers across pathways; Innovations
are weighted towards influencing individual decision-making levers; Big
data innovations use a range of public, acquired, and proprietary data
types; Organizations implementing a big data transformation should
provide the leadership required for the associated cultural
transformation; Companies must develop a range of big data capabilities.

\video{Health}{14:53}{McKinsey Report}{https://www.youtube.com/watch?v=fu-TWnIk980} 

\subsubsection{Microsoft Report on Big Data in
Health}\label{microsoft-report-on-big-data-in-health}

This lesson identifies data sources as Clinical Data, Pharma \& Life
Science Data, Patient \& Consumer Data, Claims \& Cost Data and
Correlational Data. Three approaches are Live data feed, Advanced
analytics and Social analytics.

\video{Health}{2:26}{Microsoft Report on Big Data in Health}{http://youtu.be/PjffvVgj1PE}


\subsubsection{EU Report on Redesigning health in Europe for
2020}\label{eu-report-on-redesigning-health-in-europe-for-2020}

This lesson summarizes an EU Report on Redesigning health in Europe for
2020. The power of data is seen as a lever for change in My Data, My
decisions; Liberate the data; Connect up everything; Revolutionize
health; and Include Everyone removing the current correlation between
health and wealth.


\video{Health}{5:00}{EU Report on Redesigning health in Europe for 2020}{http://youtu.be/9mbt_ZSs0iw}


\subsubsection{Medicine and the Internet of
Things}\label{medicine-and-the-internet-of-things}

The Internet of Things will have great impact on health including
telemedicine and wearables. Examples are given.

\video{Health}{8:17}{Medicine and the Internet of Things}{https://www.youtube.com/watch?v=Jk3EeFzZnuU}


\subsubsection{Extrapolating to 2032}\label{extrapolating-to-2032}

A study looks at 4 scenarios for healthcare in 2032. Two are positive,
one middle of the road and one negative.

\video{Health}{15:13}{Extrapolating to 2032}{https://www.youtube.com/watch?v=a5G4HACeokg} 


\subsubsection{Genomics, Proteomics and Information
Visualization}\label{genomics-proteomics-and-information-visualization}

A study of an Azure application with an Excel frontend and a cloud BLAST
backend starts this lesson. This is followed by a big data analysis of
personal genomics and an analysis of a typical DNA sequencing analytics
pipeline. The Protein Sequence Universe is defined and used to motivate
Multi dimensional Scaling MDS. Sammon's method is defined and its use
illustrated by a metagenomics example. Subtleties in use of MDS include
a monotonic mapping of the dissimilarity function. The application to
the COG Proteomics dataset is discussed. We note that the MDS approach
is related to the well known chisq method and some aspects of nonlinear
minimization of chisq (Least Squares) are discussed.

\video{Health}{6:56}{Genomics, Proteomics and  Information Visualization}{https://www.youtube.com/watch?v=zGzBtxq1ZRE}

\video{Health}{6:56}{ CC) Genomics, Proteomics and  Information Visualization}{https://drive.google.com/file/d/0B5plU-u0wqMoVzduODM0Z2dFYWM/view?usp=sharing}


Next we continue the discussion of the COG Protein Universe introduced
in the last lesson. It is shown how Proteomics clusters are clearly seen
in the Universe browser. This motivates a side remark on different
clustering methods applied to metagenomics. Then we discuss the
Generative Topographic Map GTM method that can be used in dimension
reduction when original data is in a metric space and is in this case
faster than MDS as GTM computational complexity scales like N not N
squared as seen in MDS.

Examples are given of GTM including an application to topic models in
Information Retrieval. Indiana University has developed a deterministic
annealing improvement of GTM. 3 separate clusterings are projected for
visualization and show very different structure emphasizing the
importance of visualizing results of data analytics. The final slide
shows an application of MDS to generate and visualize phylogenetic
trees.

\video{Health}{10:33}{Genomics, Proteomics and Information Visualization I}{https://drive.google.com/file/d/0B5plU-u0wqMobXdEQWRHWl95UTA/view?usp=sharing}

\video{Health}{7:41}{Genomics, Proteomics and Information Visualization: II}{https://drive.google.com/file/d/0B5plU-u0wqModlhmdVUwdGlQNTA/view?usp=sharing}

\slides{Health}{Proteomics and Information Visualization}{131}{https://drive.google.com/open?id=0B8936_ytjfjmX0lEMWhMX2kwRHc}
  


\subsubsection{Resources}\label{resources}

\begin{itemize}
\tightlist
\item
  \url{https://wiki.nci.nih.gov/display/CIP/CIP+Survey+of+Biomedical+Imaging+Archives}
\item
  \url{http://grids.ucs.indiana.edu/ptliupages/publications/Where\%20does\%20all\%20the\%20data\%20come\%20from\%20v7.pdf}
\item
  \url{http://www.ieee-icsc.org/ICSC2010/Tony\%20Hey\%20-\%2020100923.pdf}
\item
  \url{http://quantifiedself.com/larry-smarr/}
\item
  \url{http://www.ebi.ac.uk/Information/Brochures/}
\item
  \url{http://www.kpcb.com/internet-trends}
\item
  \url{http://www.slideshare.net/drsteventucker/wearable-health-fitness-trackers-and-the-quantified-self}
\item
  \url{http://www.siam.org/meetings/sdm13/sun.pdf}
\item
  \url{http://en.wikipedia.org/wiki/Calico_\%28company\%29}
\item
  \url{http://www.slideshare.net/GSW_Worldwide/2015-health-trends}
\item
  \url{http://www.accenture.com/SiteCollectionDocuments/PDF/Accenture-Industrial-Internet-Changing-Competitive-Landscape-Industries.pdf}
\item
  \url{http://www.slideshare.net/schappy/how-realtime-analysis-turns-big-medical-data-into-precision-medicine}
\item
  \url{http://medcitynews.com/2013/03/the-body-in-bytes-medical-images-as-a-source-of-healthcare-big-data-infographic/}
\item
  \url{http://healthinformatics.wikispaces.com/file/view/cloud_computing.ppt}
\item
  \url{http://www.mckinsey.com/~/media/McKinsey/dotcom/Insights/Health\%20care/The\%20big-data\%20revolution\%20in\%20US\%20health\%20care/The\%20big-data\%20revolution\%20in\%20US\%20health\%20care\%20Accelerating\%20value\%20and\%20innovation.ashx}
\item
  \url{https://partner.microsoft.com/download/global/40193764}
\item
  \url{http://ec.europa.eu/information_society/activities/health/docs/policy/taskforce/redesigning_health-eu-for2020-ehtf-report2012.pdf}
\item
  \url{http://www.kpcb.com/internet-trends}
\item
  \url{http://www.liveathos.com/apparel/app}
\item
  \url{http://debategraph.org/Poster.aspx?aID=77}
\item
  \url{http://www.oerc.ox.ac.uk/downloads/presentations-from-events/microsoftworkshop/gannon}
\item
  \url{http://www.delsall.org}
\item
  \url{http://salsahpc.indiana.edu/millionseq/mina/16SrRNA_index.html}
\item
  \url{http://www.geatbx.com/docu/fcnindex-01.html}
\item
  \url{https://wiki.nci.nih.gov/display/CIP/CIP+Survey+of+Biomedical+Imaging+Archives}
\item
  \url{http://grids.ucs.indiana.edu/ptliupages/publications/Where\%20does\%20all\%20the\%20data\%20come\%20from\%20v7.pdf}
\item
  \url{http://www.ieee-icsc.org/ICSC2010/Tony\%20Hey\%20-\%2020100923.pdf}
\item
  \url{http://quantifiedself.com/larry-smarr/}
\item
  \url{http://www.ebi.ac.uk/Information/Brochures/}
\item
  \url{http://www.kpcb.com/internet-trends}
\item
  \url{http://www.slideshare.net/drsteventucker/wearable-health-fitness-trackers-and-the-quantified-self}
\item
  \url{http://www.siam.org/meetings/sdm13/sun.pdf}
\item
  \url{http://en.wikipedia.org/wiki/Calico_\%28company\%29}
\item
  \url{http://www.slideshare.net/GSW_Worldwide/2015-health-trends}
\item
  \url{http://www.accenture.com/SiteCollectionDocuments/PDF/Accenture-Industrial-Internet-Changing-Competitive-Landscape-Industries.pdf}
\item
  \url{http://www.slideshare.net/schappy/how-realtime-analysis-turns-big-medical-data-into-precision-medicine}
\item
  \url{http://medcitynews.com/2013/03/the-body-in-bytes-medical-images-as-a-source-of-healthcare-big-data-infographic/}
\item
  \url{http://healthinformatics.wikispaces.com/file/view/cloud_computing.ppt}
\item
  \url{http://www.mckinsey.com/~/media/McKinsey/dotcom/Insights/Health\%20care/The\%20big-data\%20revolution\%20in\%20US\%20health\%20care/The\%20big-data\%20revolution\%20in\%20US\%20health\%20care\%20Accelerating\%20value\%20and\%20innovation.ashx}
\item
  \url{https://partner.microsoft.com/download/global/40193764}
\item
  \url{http://ec.europa.eu/information_society/activities/health/docs/policy/taskforce/redesigning_health-eu-for2020-ehtf-report2012.pdf}
\item
  \url{http://www.kpcb.com/internet-trends}
\item
  \url{http://www.liveathos.com/apparel/app}
\item
  \url{http://debategraph.org/Poster.aspx?aID=77}
\item
  \url{http://www.oerc.ox.ac.uk/downloads/presentations-from-events/microsoftworkshop/gannon}
\item
  \url{http://www.delsall.org}
\item
  \url{http://salsahpc.indiana.edu/millionseq/mina/16SrRNA_index.html}
\item
  \url{http://www.geatbx.com/docu/fcnindex-01.html}
\end{itemize}



\section{e-Commerce and LifeStyle Case
Study}\label{e-commerce-and-lifestyle-case-study}

\FILENAME

Recommender systems operate under the hood of such widely recognized
sites as Amazon, eBay, Monster and Netflix where everything is a
recommendation. This involves a symbiotic relationship between vendor
and buyer whereby the buyer provides the vendor with information about
their preferences, while the vendor then offers recommendations tailored
to match their needs. Kaggle competitions h improve the success of the
Netflix and other recommender systems. Attention is paid to models that
are used to compare how changes to the systems affect their overall
performance. It is interesting that the humble ranking has become such a
dominant driver of the world's economy. More examples of recommender
systems are given from Google News, Retail stores and in depth Yahoo!
covering the multi-faceted criteria used in deciding recommendations on
web sites.

The formulation of recommendations in terms of points in a space or bag
is given where bags of item properties, user properties, rankings and
users are useful. Detail is given on basic principles behind recommender
systems: user-based collaborative filtering, which uses similarities in
user rankings to predict their interests, and the Pearson correlation,
used to statistically quantify correlations between users viewed as
points in a space of items. Items are viewed as points in a space of
users in item-based collaborative filtering. The Cosine Similarity is
introduced, the difference between implicit and explicit ratings and the
k Nearest Neighbors algorithm. General features like the curse of
dimensionality in high dimensions are discussed. A simple Python k
Nearest Neighbor code and its application to an artificial data set in 3
dimensions is given. Results are visualized in Matplotlib in 2D and with
Plotviz in 3D. The concept of a training and a testing set are
introduced with training set pre labeled. Recommender system are used to
discuss clustering with k-means based clustering methods used and their
results examined in Plotviz. The original labelling is compared to
clustering results and extension to 28 clusters given. General issues in
clustering are discussed including local optima, the use of annealing to
avoid this and value of heuristic algorithms.

\subsection{Recommender Systems:
Introduction}\label{recommender-systems-introduction}

We introduce Recommender systems as an optimization technology used in a
variety of applications and contexts online. They operate in the
background of such widely recognized sites as Amazon, eBay, Monster and
Netflix where everything is a recommendation. This involves a symbiotic
relationship between vendor and buyer whereby the buyer provides the
vendor with information about their preferences, while the vendor then
offers recommendations tailored to match their needs, to the benefit of
both.

There follows an exploration of the Kaggle competition site, other
recommender systems and Netflix, as well as competitions held to improve
the success of the Netflix recommender system. Finally attention is paid
to models that are used to compare how changes to the systems affect
their overall performance. It is interesting how the humble ranking has
become such a dominant driver of the world's economy.

\slides{Lifestyle}{Recommender}{45}{https://drive.google.com/open?id=0B6wqDMIyK2P7YkIwczVfQlJqVG8}{PDF}


\subsubsection{Recommender Systems as an Optimization
Problem}\label{recommender-systems-as-an-optimization-problem}

We define a set of general recommender systems as matching of items to
people or perhaps collections of items to collections of people where
items can be other people, products in a store, movies, jobs, events,
web pages etc. We present this as ``yet another optimization problem''.

\video{Lifestyle}{8:06}{Recommender Systems I}{https://www.youtube.com/watch?v=kO023BIW2dw} 


\subsubsection{Recommender Systems
Introduction}\label{recommender-systems-introduction-1}

We give a general discussion of recommender systems and point out that
they are particularly valuable in long tail of tems (to be recommended)
that aren't commonly known. We pose them as a rating system and relate
them to information retrieval rating systems. We can contrast
recommender systems based on user profile and context; the most familiar
collaborative filtering of others ranking; item properties; knowledge
and hybrid cases mixing some or all of these.

\video{Lifestyle}{12:56}{Recommender Systems Introduction}{https://youtu.be/KbjBKrzFYKg}


\subsubsection{Kaggle Competitions}\label{kaggle-competitions}

We look at Kaggle competitions with examples from web site. In
particular we discuss an Irvine class project involving ranking jokes.

\video{Lifestyle}{3:36}{Kaggle Competitions: }{https://youtu.be/DFH7GPrbsJA}


\subsubsection{Examples of Recommender
Systems}\label{examples-of-recommender-systems}

We go through a list of 9 recommender systems from the same Irvine
class.

\video{Lifestyle}{1:00}{Examples of Recommender Systems}{https://youtu.be/1Eh1epQj-EQ}


\subsubsection{Netflix on Recommender
Systems}\label{netflix-on-recommender-systems}

This is Part 1.

We summarize some interesting points from a tutorial from Netflix for
whom `'everything is a recommendation''. Rankings are given in multiple
categories and categories that reflect user interests are especially
important. Criteria used include explicit user preferences, implicit
based on ratings and hybrid methods as well as freshness and diversity.
Netflix tries to explain the rationale of its recommendations. We give
some data on Netflix operations and some methods used in its recommender
systems. We describe the famous Netflix Kaggle competition to improve
its rating system. The analogy to maximizing click through rate is given
and the objectives of optimization are given.

\video{Lifestyle}{14:20}{Netflix on Recommender Systems}{https://www.youtube.com/watch?v=ModhdIT9D24}


\subsubsection{Consumer Data Science}\label{consumer-data-science}

Here we go through Netflix's methodology in letting data speak for
itself in optimizing the recommender engine. An example iis given on
choosing self produced movies. A/B testing is discussed with examples
showing how testing does allow optimizing of sophisticated criteria.
This lesson is concluded by comments on Netflix technology and the full
spectrum of issues that are involved including user interface, data, AB
testing, systems and architectures. We comment on optimizing for a
household rather than optimizing for individuals in household.

\video{Lifestyle}{13:04}{Consumer Data Science}{https://youtu.be/B8cjaOQ57LI}


\subsubsection{Resources}\label{resources}

\begin{itemize}

\item
  \url{http://www.slideshare.net/xamat/building-largescale-realworld-recommender-systems-recsys2012-tutorial}
\item
  \url{http://www.ifi.uzh.ch/ce/teaching/spring2012/16-Recommender-Systems_Slides.pdf}
\item
  \url{https://www.kaggle.com/}
\item
  \url{http://www.ics.uci.edu/~welling/teaching/CS77Bwinter12/CS77B_w12.html}
\item
  Jeff Hammerbacher
  \url{https://berkeleydatascience.files.wordpress.com/2012/01/20120117berkeley1.pdf}
\item
  \url{http://www.techworld.com/news/apps/netflix-foretells-house-of-cards-success-with-cassandra-big-data-engine-3437514/}
\item
  \url{https://en.wikipedia.org/wiki/A/B_testing}
\item
  \url{http://www.infoq.com/presentations/Netflix-Architecture}
\end{itemize}

\subsection{Recommender Systems: Examples and
Algorithms}\label{recommender-systems-examples-and-algorithms}

We continue the discussion of recommender systems and their use in
e-commerce. More examples are given from Google News, Retail stores and
in depth Yahoo! covering the multi-faceted criteria used in deciding
recommendations on web sites. Then the formulation of recommendations in
terms of points in a space or bag is given.

Here bags of item properties, user properties, rankings and users are
useful. Then we go into detail on basic principles behind recommender
systems: user-based collaborative filtering, which uses similarities in
user rankings to predict their interests, and the Pearson correlation,
used to statistically quantify correlations between users viewed as
points in a space of items.

\slides{Lifestyle}{Recommender}{49}{https://drive.google.com/open?id=0B6wqDMIyK2P7UVloVElaZ2FXcTg}{PDF}


\subsubsection{Recap and Examples of Recommender
Systems}\label{recap-and-examples-of-recommender-systems}

We start with a quick recap of recommender systems from previous unit;
what they are with brief examples.

\video{Lifestyle}{5:48}{Recap and Examples of Recommender Systems}{https://www.youtube.com/watch?v=PwS8UE4TDS4}


\subsubsection{Examples of Recommender
Systems}\label{examples-of-recommender-systems-1}

We give 2 examples in more detail: namely Google News and Markdown in
Retail.

\video{Lifestyle}{8:34}{Examples of Recommender Systems}{https://youtu.be/og07mH9fU0M}


\subsubsection{Recommender Systems in Yahoo Use Case
Example}\label{recommender-systems-in-yahoo-use-case-example}

We describe in greatest detail the methods used to optimize Yahoo web
sites. There are two lessons discussing general approach and a third
lesson examines a particular personalized Yahoo page with its different
components. We point out the different criteria that must be blended in
making decisions; these criteria include analysis of what user does
after a particular page is clicked; is the user satisfied and cannot
that we quantified by purchase decisions etc. We need to choose
Articles, ads, modules, movies, users, updates, etc to optimize metrics
such as relevance score, CTR, revenue, engagement.These lesson stress
that if though we have big data, the recommender data is sparse. We
discuss the approach that involves both batch (offline) and on-line
(real time) components.

\video{Lifestyle}{8:46}{Recap of Recommender Systems II}{https://youtu.be/FBn7HpGFNvg}

\video{Lifestyle}{10:48}{Recap of Recommender Systems III}{https://youtu.be/VS2Y4lAiP5A}

\video{Lifestyle}{3:21}{Case Study of Recommender systems}{https://youtu.be/HrRJWEF8EfU}


\subsubsection{User-based nearest-neighbor collaborative
filtering}\label{user-based-nearest-neighbor-collaborative-filtering}

Collaborative filtering is a core approach to recommender systems. There
is user-based and item-based collaborative filtering and here we discuss
the user-based case. Here similarities in user rankings allow one to
predict their interests, and typically this quantified by the Pearson
correlation, used to statistically quantify correlations between users.

\video{Lifestyle}{7:20}{User-based nearest-neighbor collaborative filtering I}{https://youtu.be/lsf_AE-8dSk}

\video{Lifestyle}{7:29}{User-based nearest-neighbor collaborative filtering II}{https://youtu.be/U7-qeX2ItPk}


\subsubsection{Vector Space Formulation of Recommender
Systems}\label{vector-space-formulation-of-recommender-systems}

We go through recommender systems thinking of them as formulated in a
funny vector space. This suggests using clustering to make
recommendations.

\video{Lifestyle}{9:06}{Vector Space Formulation of Recommender Systems new}{https://youtu.be/IlQUZOXlaSU}


\subsubsection{Resources}\label{resources-1}

\begin{itemize}

\item
  \url{http://pages.cs.wisc.edu/~beechung/icml11-tutorial/}
\end{itemize}

\subsection{Item-based Collaborative Filtering and its
Technologies}\label{item-based-collaborative-filtering-and-its-technologies}

We move on to item-based collaborative filtering where items are viewed
as points in a space of users. The Cosine Similarity is introduced, the
difference between implicit and explicit ratings and the k Nearest
Neighbors algorithm. General features like the curse of dimensionality
in high dimensions are discussed.

\slides{Lifestyle}{Filtering}{18}{https://drive.google.com/open?id=0B6wqDMIyK2P7UExxVFc5YlpOZ28}{PDF}


\subsubsection{Item-based Collaborative
Filtering}\label{item-based-collaborative-filtering}

We covered user-based collaborative filtering in the previous unit. Here
we start by discussing memory-based real time and model based offline
(batch) approaches. Now we look at item-based collaborative filtering
where items are viewed in the space of users and the cosine measure is
used to quantify distances. WE discuss optimizations and how batch
processing can help. We discuss different Likert ranking scales and
issues with new items that do not have a significant number of rankings.

\video{Lifestyle}{11:18}{Item Based Filtering}{https://www.youtube.com/watch?v=HTdYGaOTlFI}

\video{Lifestyle}{7:16}{k Nearest Neighbors and High Dimensional Spaces}{https://youtu.be/SM8EJdAa4mw}


\subsubsection{k Nearest Neighbors and High Dimensional
Spaces}\label{k-nearest-neighbors-and-high-dimensional-spaces}

We define the k Nearest Neighbor algorithms and present the Python
software but do not use it. We give examples from Wikipedia and describe
performance issues. This algorithm illustrates the curse of
dimensionality. If items were a real vectors in a low dimension space,
there would be faster solution methods.

\video{Lifestyle}{10:03}{k Nearest Neighbors and High Dimensional Spaces}{https://youtu.be/2NqUsDGQDy8}




\chapter{Physics}
\label{c:physics}
\index{Physics}


\FILENAME

This section starts by describing the LHC accelerator at CERN and
evidence found by the experiments suggesting existence of a Higgs Boson.
The huge number of authors on a paper, remarks on histograms and Feynman
diagrams is followed by an accelerator picture gallery. The next unit is
devoted to Python experiments looking at histograms of Higgs Boson
production with various forms of shape of signal and various background
and with various event totals. Then random variables and some simple
principles of statistics are introduced with explanation as to why they
are relevant to Physics counting experiments. The unit introduces
Gaussian (normal) distributions and explains why they seen so often in
natural phenomena. Several Python illustrations are given. Random
Numbers with their Generators and Seeds lead to a discussion of Binomial
and Poisson Distribution. Monte-Carlo and accept-reject methods. The
Central Limit Theorem concludes discussion.

\section{Looking for Higgs Particles}
\index{Higgs Particles}

\subsection{Bumps in Histograms,  Experiments and Accelerators}

This unit is devoted to Python and Java experiments looking at
histograms of Higgs Boson production with various forms of shape of
signal and various background and with various event totals. The
lectures use Python but use of Java is described.

\slides{Physics}{20}{Higgs}{https://drive.google.com/open?id=0B8936_ytjfjmYXNoM3ZadGR6QlE}

\sourcecode{Physics}{HiggsClassI-Sloping.py}{examples/physics/mr-higgs/higgs-classI-sloping.py}


\subsection{Particle Counting}

We return to particle case with slides used in introduction and stress
that particles often manifested as bumps in histograms and those bumps
need to be large enough to stand out from background in a statistically
significant fashion.

\video{Physics}{13:49}{Discovery of Higgs Particle}{https://www.youtube.com/watch?v=iaypHlgFyuU}

We give a few details on one LHC experiment ATLAS. Experimental physics
papers have a staggering number of authors and quite big budgets.
Feynman diagrams describe processes in a fundamental fashion.

\video{Physics}{7:38}{Looking for Higgs Particle and Counting Introduction II}{http://youtu.be/UAMzmOgjj7I}

\subsection{Experimental Facilities}

We give a few details on one LHC experiment ATLAS. Experimental physics
papers have a staggering number of authors and quite big budgets.
Feynman diagrams describe processes in a fundamental fashion.

\video{Physics}{9:29}{Looking for Higgs Particle Experiments}{http://youtu.be/BW12d780qT8}

\subsection{Accelerator Picture Gallery of Big Science}

This lesson gives a small picture gallery of accelerators. Accelerators,
detection chambers and magnets in tunnels and a large underground
laboratory used fpr experiments where you need to be shielded from
background like cosmic rays.

\video{Physics}{11:21}{Accelerator Picture Gallery of Big Science}{http://youtu.be/WLJIxWWMYi8}

\subsection{Resources}

\begin{itemize}

\item
  \url{http://grids.ucs.indiana.edu/ptliupages/publications/Where\%20does\%20all\%20the\%20data\%20come\%20from\%20v7.pdf}
\item
  \url{http://www.sciencedirect.com/science/article/pii/S037026931200857X}
\item
  \url{http://www.nature.com/news/specials/lhc/interactive.html}
\end{itemize}

Looking for Higgs Particles: Python Event Counting for Signal and
Background (Part 2)


This unit is devoted to Python experiments looking at histograms of
Higgs Boson production with various forms of shape of signal and various
background and with various event totals.

\slides{Physics}{29}{Higgs II}{https://drive.google.com/open?id=0B8936_ytjfjmUHRpV2g2V28walE}{PDF}

Files:

\sourcecode{Physics}{HiggsClassI-Sloping.py}{examples/physics/mr-higgs/higgs-classI-sloping.py}

\sourcecode{Physics}{HiggsClassIII.py}{examples/physics/number-theory/higgs-classIII.py}

\sourcecode{Physics}{HiggsClassIIUniform.py}{examples/physics/mr-higgs/higgs-classII-uniform.py}


%\subsection{Physics Use Case II 1: Class
%Software}\label{physics-use-case-ii-1-class-software}

%We discuss how this unit uses Java and Python on both a backend server
%(FutureGrid) or a local client. WE point out useful book on Python for
%data analysis. This builds on technology training in Section 3.

%\video{Physics}{9:30}{Higgs Particle Events and Counting}{https://www.youtube.com/watch?v=L8j2qB4lSZ0}

%\begin{WARNING}
%This video contains Java information, but we are no longer using Java in
%this class.
%\end{WARNING}%

\subsection{Event Counting}
\index{Event Counting}

We define \textit{event counting} data collection environments. We discuss
the python and Java code to generate events according to a particular
scenario (the important idea of Monte Carlo data). Here a sloping
background plus either a Higgs particle generated similarly to LHC
observation or one observed with better resolution (smaller measurement
error).

\video{Physics}{7:02}{Event Counting}{http://youtu.be/h8-szCeFugQ}

\subsection{Monte Carlo}
\index{Monte Carlo}

This uses Monte Carlo data both to generate data like the experimental
observations and explore effect of changing amount of data and changing
measurement resolution for Higgs.

\video{Physics}{7:33}{With Python examples of Signal plus Background}{http://youtu.be/bl2f0tAzLj4}

This lesson continues the examination of Monte Carlo data looking at
effect of change in number of Higgs particles produced and in change
in shape of background.

\video{Physics}{7:01}{Change shape of background \& num of Higgs Particles}{http://youtu.be/bw3fd5cfQhk}

\subsection{Resources}

\begin{itemize}

\item
  Python for Data Analysis: Agile Tools for Real World Data By Wes
  McKinney, Publisher: O'Reilly Media, Released: October 2012, Pages:
  472.
\item
  \url{http://jwork.org/scavis/api/}
\item
  \url{https://en.wikipedia.org/wiki/DataMelt}
\end{itemize}

\subsection{Random Variables, Physics and
  Normal Distributions}

We introduce random variables and some simple principles of statistics
and explains why they are relevant to Physics counting experiments. The
unit introduces Gaussian (normal) distributions and explains why they
seen so often in natural phenomena. Several Python illustrations are
given. Java is currently not available in this unit.

\slides{Physics}{39}{Higgs}{https://drive.google.com/open?id=0B8936_ytjfjmNWhrS0xadk16SWM}

\sourcecode{Physics}{HiggsClassIII.py}{examples/physics/number-theory/higgs-classIII.py}

\subsection{Statistics Overview and Fundamental Idea: Random
  Variables}
\index{Random Variables}

We go through the many different areas of statistics covered in the
Physics unit. We define the statistics concept of a random variable.

\video{Physics}{8:19}{Random variables and normal distributions}{https://www.youtube.com/watch?v=_sLGyt4qWWk}

\subsection{Physics and Random Variables}


We describe the DIKW pipeline for the analysis of this type of physics
experiment and go through details of analysis pipeline for the LHC ATLAS
experiment. We give examples of event displays showing the final state
particles seen in a few events. We illustrate how physicists decide
whats going on with a plot of expected Higgs production experimental
cross sections (probabilities) for signal and background.

\video{Physics}{8:34}{Physics and Random Variables I}{http://youtu.be/Tn3GBxgplxg}

\video{Physics}{5:50}{Physics and Random Variables II}{http://youtu.be/qWEjp0OtvdA}

\subsection{Statistics of Events with Normal Distributions}
\index{Normal Distributions}

We introduce Poisson and Binomial distributions and define independent
identically distributed (IID) random variables. We give the law of large
numbers defining the errors in counting and leading to Gaussian
distributions for many things. We demonstrate this in Python
experiments.

\video{Physics}{11:25}{Statistics of Events with Normal Distributions}{http://youtu.be/LMBtpWOOQLo}

\subsection{Gaussian Distributions}
\index{Gaussian Distributions}

We introduce the Gaussian distribution and give Python examples of the
fluctuations in counting Gaussian distributions.

\video{Physics}{9:08}{Gaussian Distributions}{http://youtu.be/LWIbPa-P5W0}

\subsection{Using Statistics}
\index{Statistics}

We discuss the significance of a standard deviation and role of biases
and insufficient statistics with a Python example in getting incorrect
answers.

\video{Physics}{14:02}{Using Statistics}{http://youtu.be/n4jlUrGwgic}

\subsection{Resources}

\begin{itemize}

\item
  \url{http://indico.cern.ch/event/20453/session/6/contribution/15?materialId=slides}
\item
  \url{http://www.atlas.ch/photos/events.html}
\item
  \url{https://cms.cern/}
\end{itemize}

\subsection{Random Numbers, Distributions and Central Limit Theorem}
\index{Central Limit Theorem}

We discuss Random Numbers with their Generators and Seeds. It introduces
Binomial and Poisson Distribution. Monte-Carlo and accept-reject methods
are discussed. The Central Limit Theorem and Bayes law concludes
discussion. Python and Java (for student - not reviewed in class)
examples and Physics applications are given.

\slides{Physics}{44}{Higgs III}{https://drive.google.com/open?id=0B8936_ytjfjmTUxkZXVRRmlBSUk}{PDF}

Files:

\sourcecode{Physics}{HiggsClassIII.py}{examples/physics/calculated-dice-roll/higgs-classIV-seeds.py}

\subsubsection{Generators and Seeds}
\index{Generators}
\index{Seeds}

We define random numbers and describe how to generate them on the
computer giving Python examples. We define the seed used to define to
specify how to start generation.

\video{Physics}{6:28}{Higgs Particle Counting Errors}{https://www.youtube.com/watch?v=de4AQ9AFt54}

\video{Physics}{7:10}{Generators and Seeds II}{http://youtu.be/9QY5qkQj2Ag}

\subsubsection{Binomial Distribution}
\index{Binomial Distribution}


We define binomial distribution and give LHC data as an example of where
this distribution valid.

\video{Physics}{12:38}{Binomial Distribution: }{http://youtu.be/DPd-eVI_twQ}

\subsubsection{Accept-Reject}

We introduce an advanced method \textbf{accept/reject} for generating
random variables with arbitrary distributions.

\video{Physics}{5:54}{Accept-Reject}{http://youtu.be/GfshkKMKCj8}

\subsubsection{Monte Carlo Method}
\index{Monte Carlo Method}

We define Monte Carlo method which usually uses accept/reject method in
typical case for distribution.

\video{Physics}{2:23}{Monte Carlo Method}{http://youtu.be/kIQ-BTyDfOQ}

\subsubsection{Poisson Distribution}
\index{Poisson Distribution}

We extend the Binomial to the Poisson distribution and give a set of
amusing examples from Wikipedia.

\video{Physics}{4:37}{Poisson Distribution}{http://youtu.be/WFvgsVo-k4s}

\subsubsection{Central Limit Theorem}
\index{Central Limit Theorem}

We introduce Central Limit Theorem and give examples from Wikipedia.

\video{Physics}{4:47}{Central Limit Theorem}{http://youtu.be/ZO53iKlPn7c}

\subsubsection{Interpretation of Probability: Bayes v.  Frequency}
\index{Bayes}

This lesson describes difference between Bayes and frequency views of
probability. Bayes's law of conditional probability is derived and
applied to Higgs example to enable information about Higgs from multiple
channels and multiple experiments to be accumulated.

\video{Physics}{12:39}{Interpretation of Probability}{http://youtu.be/jzDkExAQI9M}

\subsubsection{Resources}

\TODO{integrate physics-references.bib}

\section{SKA -- Square Kilometer Array}
\index{Square Kilometer Array}
\index{SKA}


Professor Diamond, accompanied by Dr. Rosie Bolton from the SKA
Regional Centre Project gave a presentation at SC17 ``into the deepest
reaches of the observable universe as they describe the SKA’s
international partnership that will map and study the entire sky in
greater detail than ever before.''

\URL{http://sc17.supercomputing.org/presentation/?id=inspkr101&sess=sess263}

A summary article about this effort is available at:

\URL{https://www.hpcwire.com/2017/11/17/sc17-keynote-hpc-powers-ska-efforts-peer-deep-cosmos/}

The video is hosted at 

\URL{http://sc17.supercomputing.org/presentation/?id=inspkr101&sess=sess263}

Start at about 1:03:00 (e.g. the one hour mark)


\section{Radar}

The changing global climate is suspected to have long-term effects on
much of the world's inhabitants. Among the various effects, the rising
sea level will directly affect many people living in low-lying coastal
regions. While the ocean-s thermal expansion has been the dominant
contributor to rises in sea level, the potential contribution of
discharges from the polar ice sheets in Greenland and Antarctica may
provide a more significant threat due to the unpredictable response to
the changing climate. The Radar-Informatics unit provides a glimpse in
the processes fueling global climate change and explains what methods
are used for ice data acquisitions and analysis.

\slides{Radar}{58}{Radar}{https://drive.google.com/open?id=0B8936_ytjfjmZ0VzZ0ZIenpUMTQ}

\subsection{Introduction}

This lesson motivates radar-informatics by building on previous
discussions on why X-applications are growing in data size and why
analytics are necessary for acquiring knowledge from large data. The
lesson details three mosaics of a changing Greenland ice sheet and
provides a concise overview to subsequent lessons by detailing
explaining how other remote sensing technologies, such as the radar, can
be used to sound the polar ice sheets and what we are doing with radar
images to extract knowledge to be incorporated into numerical models.

\video{Radar}{3:31}{Radar Informatics}{https://youtu.be/LXOncC2AhsI}

\subsection{Remote Sensing}

This lesson explains the basics of remote sensing, the characteristics
of remote sensors and remote sensing applications. Emphasis is on image
acquisition and data collection in the electromagnetic spectrum.

\video{Radar}{6:43}{Remote Sensing}{https://youtu.be/TTrm9rmZySQ}

\subsection{Ice Sheet Science}

This lesson provides a brief understanding on why melt water at the base
of the ice sheet can be detrimental and why it's important for sensors
to sound the bedrock.

\video{Radar}{1:00}{Ice Sheet Science}{https://youtu.be/rDpjMLguVBc}

\subsection{Global Climate Change}

This lesson provides an understanding and the processes for the
greenhouse effect, how warming effects the Polar Regions, and the
implications of a rise in sea level.

\video{Radar}{2:51}{Global Climate Change}{https://youtu.be/f9hzzJX0qDs}

\subsection{Radio Overview}

This lesson provides an elementary introduction to radar and its
importance to remote sensing, especially to acquiring information about
Greenland and Antarctica.

\video{Radar}{4:16}{Radio Overview}{https://youtu.be/PuI7F-RMKCI}

\subsection{Radio Informatics}

This lesson focuses on the use of sophisticated computer vision
algorithms, such as active contours and a hidden markov model to support
data analysis for extracting layers, so ice sheet models can accurately
forecast future changes in climate.

\video{Radar}{3:35}{Radio Informatics}{https://youtu.be/q3Pwyt49syE}

\input{chapter/theory/python-tech}
%----------------------------------------------------------------------------------------
%	PART
%----------------------------------------------------------------------------------------
\chapterimage{images/python.jpeg} % Chapter heading image


\part{Python}


%----------------------------------------------------------------------------------------
%	CHAPTER 1
%----------------------------------------------------------------------------------------
\chapter{Introduction}
\label{C:python}

\FILENAME

\FILENAME

\section{Introduction to Python}\label{introduction-to-python}

Portions of this lesson have been adapted from the
\href{https://docs.python.org/2/tutorial/}{official Python Tutorial}
copyright \href{http://www.python.org/}{Python Software Foundation}.

Python is an easy to learn programming language. It has efficient
high-level data structures and a simple but effective approach to
object-oriented programming. Python's simple syntax and dynamic typing,
together with its interpreted nature, make it an ideal language for
scripting and rapid application development in many areas on most
platforms. The Python interpreter and the extensive standard library are
freely available in source or binary form for all major platforms from
the Python Web site, \url{https://www.python.org/}, and may be freely
distributed. The same site also contains distributions of and pointers
to many free third party Python modules, programs and tools, and
additional documentation. The Python interpreter can be extended with
new functions and data types implemented in C or C++ (or other languages
callable from C). Python is also suitable as an extension language for
customizable applications.

Python is an interpreted, dynamic, high-level programming language
suitable for a wide range of applications.

The philosophy of python is summarized in
\href{https://www.python.org/dev/peps/pep-0020/}{The Zen of Python} as
follows:

\begin{itemize}
\tightlist
\item
  Explicit is better than implicit
\item
  Simple is better than complex
\item
  Complex is better than complicated
\item
  Readability counts
\end{itemize}

The main features of Python are:

\begin{itemize}
\tightlist
\item
  Use of indentation whitespace to indicate blocks
\item
  Object orient paradigm
\item
  Dynamic typing
\item
  Interpreted runtime
\item
  Garbage collected memory management
\item
  a large standard library
\item
  a large repository of third-party libraries
\end{itemize}

Python is used by many companies (such as Google, Yahoo!, CERN, NASA)
and is applied for web development, scientific computing, embedded
applications, artificial intelligence, software development, and
information security, to name a few.

The material collected here introduces the reader to the basic concepts and
features of the Python language and system. After you have worked
tthrough the material you will be able to:

\begin{itemize}
\tightlist
\item
  use Python
\item
  use the interactive Python interface
\item
  understand the basic syntax of Python
\item
  write and run Python programs stored in a file
\item
  have an overview of the standard library
\item
  install Python libraries using pyenv or if it is not available
  virtualenv
\end{itemize}

This tutorial does not attempt to be comprehensive and cover every
single feature, or even every commonly used feature. Instead, it
introduces many of Python's most noteworthy features, and will give you
a good idea of the language's flavor and style. After reading it, you
will be able to read and write Python modules and programs, and you will
be ready to learn more about the various Python library modules.

In order to conduct this lesson you need

\begin{itemize}
\tightlist
\item
  A computer with Python 2.7.13 or 3.6.2
\item
  Familiarity with command line usage
\item
  A text editor such as
  \href{https://www.jetbrains.com/pycharm/}{PyCharm}, emacs, vi or
  others. You should identity which works best for you and set it up.
\end{itemize}

\subsection{Links}\label{links}

\begin{itemize}
\tightlist
\item
  \href{https://www.python.org/}{Python}
\item
  \href{https://pip.pypa.io/en/stable/}{Pip}
\item
  \href{https://virtualenv.pypa.io/en/stable/}{Virtualenv}
\item
  \href{http://www.numpy.org/}{NumPy}
\item
  \href{https://scipy.org/}{SciPy}
\item
  \href{http://matplotlib.org/}{Matplotlib}
\item
  \href{http://pandas.pydata.org/}{Pandas}
\item
  \href{https://github.com/pyenv/pyenv}{pyenv}
\item
  \href{https://github.com/pyenv/pyenv}{PyCharm}
\end{itemize}

Python module of the week is a Web site that provides a number of short
examples on how to use some elementary python modules. Not all modules
are equally useful and you should decide if there are better
alternatives. However for beginners this site provides a number of good
examples

\begin{itemize}
\tightlist
\item
  Python 2: \url{https://pymotw.com/2/}
\item
  Python 3: \url{https://pymotw.com/3/}
\end{itemize}


\chapter{Install}
\label{C:python-install}

\FILENAME

\section{Python Installation}\label{python-installation}
\index{Python!Install}

Python is easy to install and very good instructions for most platforms
can be found on the python.org Web page. We will be using Python 2.7.13
and/or Python 3 in our activities.

To manage python modules, it is useful to have
\href{https://pypi.python.org/pypi/pip}{pip} package installation tool
on your system.

In the tutorial, we assume that you have a computer with python
installed. However, we also recommend that for the class you use
Python's virtualenv (see below) to isolate your development Python from
the system installed Python.

\subsection{Managing custom Python
installs}\label{managing-custom-python-installs}

Often you have your own computer and you do not like to change its
environment to keep it in pristine condition. Python comes with many
libraries that could for example conflict with libraries that you have
installed. To avoid this it is bets to work in an isolated python we can
use tools such as virtualenv, pyenv or pyvenv for 3.6.4\footnote{check
  for the newest version}. Which you use
depends on you, but we highly recommend pyenv if you can.

\subsubsection{Managing Multiple Python Versions with
Pyenv}
\label{S:managing-multiple-python-versions-with-pyenv}
\index{pyenv}

Python has several versions that are used by the community. This
includes Python 2 and Python 3, but alls different management of the
python libraries. As each OS may have their own version of python
installed. It is not recommended that you modify that version. Instead
you may want to create a localized python installation that you as a
user can modify. To do that we recommend \emph{pyenv}. Pyenv allows
users to switch between multiple versions of Python
(\url{https://github.com/yyuu/pyenv}). To summarize:

\begin{itemize}

\item
  users to change the global Python version on a per-user basis;
\item
  users to enable support for per-project Python versions;
\item
  easy version changes without complex environment variable management;
\item
  to search installed commands across different python versions;
\item
  integrate with tox (\url{https://tox.readthedocs.io/}).
\end{itemize}

\paragraph{Instalation without pyenv}\label{instalation-without-pyenv}

If you need to have more than one python version installed and do not
want or can use pyenv, we recommend you download and install python
2.7.13 and 3.6.4\footnote{check
  for the newest version} from python.org
(\url{https://www.python.org/downloads/})

\paragraph{Disabeling wrong python installs on
OSX}\label{disabeling-wrong-python-installs-on-osx}

While working with students we have seen at times that they take other
classes either at universities or online that teach them how to program
in python. Unfortunately, although they seem to do that they often
ignore to teach you how to properly install python. I just recently had
a students that had installed python 7 times on his OSX machine, while
another student had 3 different installations, all of which confliced
with each other as they were not set up properly.

We recommend that you inspect if you have a files such as
\verb|~/.bashrc| or \verb|~/.bashrc_profile| in your
home directory and identify if it activates various versions of python
on your computer. If so you could try to deactivate them while
out-commenting the various versions with the \# character at the
beginning of the line, start a new terminal and see if the terminal
shell still works. Than you can follow our instructions here while using
an install on pyenv.

\paragraph{Install pyenv on OSX from
git}\label{install-pyenv-on-osx-from-git}

This is our recommended way to install pyenv on OSX:

\begin{verbatim}
$ git clone https://github.com/pyenv/pyenv.git ~/.pyenv
$ git clone https://github.com/pyenv/pyenv-virtualenv.git ~/.pyenv/plugins/pyenv-virtualenv
$ git clone https://github.com/yyuu/pyenv-virtualenvwrapper.git ~/.pyenv/plugins/pyenv-virtualenvwrapper
$ echo 'export PYENV_ROOT="$HOME/.pyenv"' >> ~/.bash_profile
$ echo 'export PATH="$PYENV_ROOT/bin:$PATH"' >> ~/.bash_profile
\end{verbatim}

\paragraph{Installation of Homebrew}\label{instalation-of-homebrew}

Before installing anything on your computer make sure you have enough
space. Use in the terminal the command:

\begin{verbatim}
$ df -h
\end{verbatim}

which gives your an overview of your file system. If you do not have
enough space, please make sure you free up unused files from your drive.

In many occasions it is beneficial to use readline as it provides nice
editing features for the terminal and xz for completion. First, make
sure you have xcode installed:

\begin{verbatim}
$ xcode-select --install
\end{verbatim}

Next install homebrew, pyenv, pyenv-virtualenv and pyenv-virtualwrapper.
Additionally install readline and some compression tools:

\begin{verbatim}
/usr/bin/ruby -e "$(curl -fsSL https://raw.githubusercontent.com/Homebrew/install/master/install)"
brew update
brew install readline xz
\end{verbatim}

\paragraph{Install pyenv on OSX with
Homebrew}\label{install-pyenv-on-osx-with-homebrew}

We describe here a mechanism of installing pyenv with homebrew. Other
mechanisms can be found on the pyenv documentation page
(\url{https://github.com/yyuu/pyenv-installer}). You must have homebrew
installed as discussed in the previous section.

To install pyenv with homebrew execute in the terminal:

\begin{verbatim}
brew install pyenv pyenv-virtualenv pyenv-virtualenvwrapper
\end{verbatim}

\paragraph{Install pyenv on Ubuntu}\label{install-pyenv-on-ubuntu}

The following steps will install pyenv in a new ubuntu 16.04
distribution.

Start up a terminal and execute in the terminal the following commands.
We recommend that you do it one command at a time so you can observe if
the command succeeds:

\begin{verbatim}
$ sudo apt-get update
$ sudo apt-get install git python-pip make build-essential libssl-dev
$ sudo apt-get install zlib1g-dev libbz2-dev libreadline-dev libsqlite3-dev
$ sudo pip install virtualenvwrapper

$ git clone https://github.com/yyuu/pyenv.git ~/.pyenv
$ git clone https://github.com/pyenv/pyenv-virtualenv.git ~/.pyenv/plugins/pyenv-virtualenv   
$ git clone https://github.com/yyuu/pyenv-virtualenvwrapper.git ~/.pyenv/plugins/pyenv-virtualenvwrapper

$ echo 'export PYENV_ROOT="$HOME/.pyenv"' >> ~/.bashrc
$ echo 'export PATH="$PYENV_ROOT/bin:$PATH"' >> ~/.bashrc
\end{verbatim}

Now that you have installed pyenv it is not yet activated in your
current terminal. The easiest thing to do is to start a new terminal and
typ in:

\begin{verbatim}
which pyenv
\end{verbatim}

If you see a response pyenv is installed and you can proceed with the
next steps.

\begin{description}
\item[Please remember whenever you modify \verb|.bashrc| or]
\verb|.bash_profile| you need to start a new terminal.
\end{description}

\paragraph{Install Different Python
Versions}\label{install-different-python-versions}

Pyenv provides a large list of different python versions. To see the
entire list please use the command:

\begin{verbatim}
$ pyenv install -l
\end{verbatim}

However, for us we only need to worry about python 2.7.13 and python
3.6.4\footnote{check
  for the newest version}. You can now
install different versions of python into your local environment with
the following commands:

\begin{verbatim}
$ pyenv install 2.7.13
$ pyenv install 3.6.4
\end{verbatim}

You can set the global python default version with:

\begin{verbatim}
$ pyenv global 2.7.13
\end{verbatim}

Type the following to determine which version you activated:

\begin{verbatim}
$ pyenv version
\end{verbatim}

Type the following to determine which versions you have available:

\begin{verbatim}
$ pyenv versions
\end{verbatim}

Associate a specific environment name with a certain python version, use
the following commands:

\begin{verbatim}
$ pyenv virtualenv 2.7.13 ENV2
$ pyenv virtualenv 3.6.4 ENV3
\end{verbatim}

In the example above, ENV2 would represent python 2.7.13 while ENV3
would represent python 3.6.4. Often it is easier to type the alias
rather than the explicit version.

\paragraph{Set up the Shell}\label{set-up-the-shell}

To make all work smoothly from your terminal, you can include the
following in your \verb|.bashrc| files:

\begin{verbatim}
export PYENV_VIRTUALENV_DISABLE_PROMPT=1
eval "$(pyenv init -)"
eval "$(pyenv virtualenv-init -)"

__pyenv_version_ps1() {
  local ret=$?;
  output=$(pyenv version-name)
  if [[ ! -z $output ]]; then
    echo -n "($output)"
  fi
  return $ret;
}

PS1="\$(__pyenv_version_ps1) ${PS1}"
\end{verbatim}

We recommend that you do this towards the end of your file.

\paragraph{Switching Environments}\label{switching-environments}

After setting up the different environments, switching between them is
now easy. Simply use the following commands:

\begin{verbatim}
(2.7.13) $ pyenv activate ENV2
(ENV2) $ pyenv activate ENV3
(ENV3) $ pyenv activate ENV2
(ENV2) $ pyenv deactivate ENV2
(2.7.13) $ 
\end{verbatim}

To make it even easier, you can add the following lines to your
\verb|.bash_profile| file:

\begin{verbatim}
alias ENV2="pyenv activate ENV2"
alias ENV3="pyenv activate ENV3"
\end{verbatim}

If you start a new terminal, you can switch between the different
versions of python simply by typing:

\begin{verbatim}
$ ENV2
$ ENV3
\end{verbatim}

\subsection{Updating Python Version List}

Pyenv maintains locally a list of available python versions. To see
the list use the command

\begin{lstlisting}
pyenv install -l
\end{lstlisting}

To obtain the newest list please use the command

\begin{lstlisting}
cd ~/.pyenv/plugins/python-build/../.. && git pull
\end{lstlisting}

Now when you call 

\begin{lstlisting}
pyenv install -l
\end{lstlisting}

You will see the updated list.

\subsection{Installation without pyenv}

If you need to have more than one python version installed and do not
want or can use pyenv, we recommend you download and install python
2.7.13 and 3.6.4 from python.org
(\url{https://www.python.org/downloads/})

\subsubsection{Make sure pip is up to date}

As you will want to install other packages, make sure pip is up to date:

\begin{verbatim}
pip install pip -U
\end{verbatim}

pyenv virtualenv anaconda3-4.3.1 ANA3 pyenv activate ANA3

\subsection{Anaconda and Miniconda}\label{anaconda-and-miniconda}

\begin{description}
\item[We do not recommend that you use anaconda or miniconda as it may]
interfere with your default python interpreters and setup.
\end{description}

Please note that beginners to pyton should always use anaconda or
miniconda only afterthey have installed pyenv and use it. For this class
neither anaconda nor miniconda is required. In fact we do not recommend
it. We keep this section as we know that other classes at IU may use
anaconda. We are not aware if these classes teach you the right way to
install it, with \emph{pyenv}.

\subsubsection{Miniconda}\label{miniconda}

\begin{description}
\item[This section about miniconda is experimental and has not]
been tested. We are looking for contributors that help completing it. If
you use anaconda or miniconda we recommend to manage it via pyenv.
\end{description}

To install mini conda you can use the following commands:

\begin{verbatim}
$ mkdir ana
$ cd ana
$ pyenv install miniconda3-latest
$ pyenv local miniconda3-latest
$ pyenv activate miniconda3-latest
$ conda create -n ana anaconda
\end{verbatim}

To activate use:

\begin{verbatim}
$ source activate ana
\end{verbatim}

To deactivate use:

\begin{verbatim}
$ source deactivate
\end{verbatim}

To install cloudmesh cmd5 please use:

\begin{verbatim}
$ pip install cloudmesh.cmd5
$ pip install cloudmesh.sys
\end{verbatim}

\subsubsection{Anaconda}\label{anaconda}

\begin{description}
\item[This section about anaconda is experimental and has not]
been tested. We are looking for contributors that help completing it.
\end{description}

You can add anaconda to your pyenv with the following commands:

\begin{verbatim}
pyenv install anaconda3-4.3.1
\end{verbatim}

To switch more easily we recommend that you use the following in your
\verb|.bash_profile| file:

\begin{verbatim}
alias ANA="pyenv activate anaconda3-4.3.1"
\end{verbatim}

Once you have done this you can easily switch to anaconda with the
command:

\begin{verbatim}
$ ANA
\end{verbatim}

Terminology in annaconda could lead to confusion. Thus we like to point
out that the version number of anaconda is unrelated to the python
version. Furthermore, anaconda uses the term root not for the root user,
but for the originating directory in which the anaconda program is
installed.

In case you like to build your own conda packages at a later time we
recommend that you install the conda-build package:

\begin{verbatim}
$ conda install conda-build
\end{verbatim}

When executing:

\begin{verbatim}
pyenv versions
\end{verbatim}

you will see after the install completed the anaconda versions
installed:

\begin{verbatim}
pyenv versions
system
2.7.13
2.7.13/envs/ENV2
3.6.4
3.6.4/envs/ENV3
ENV2 
ENV3
* anaconda3-4.3.1 (set by PYENV_VERSION environment variable)
\end{verbatim}

Let us now create virtualenv for anaconda:

\begin{verbatim}
$ pyenv virtualenv anaconda3-4.3.1 ANA
\end{verbatim}

To activate it you can now use:

\begin{verbatim}
$ pyenv ANA
\end{verbatim}

However, anaconda may modify your \verb|.bashrc| or \verb|.bash_profile| files and
may result in incompatibilities with other python versions. For this
reason we recommend not to use it. If you find ways to get it to work
reliably with other versions, please let us know and we update this
tutorial.

To install cloudmesh cmd5 please use:

\begin{verbatim}
$ pip install cloudmesh.cmd5
$ pip install cloudmesh.sys
\end{verbatim}

\paragraph{Exercise}

\begin{exercise}
Write installation instructions for an operating system of your choice
and add to this documentation.
\end{exercise}

\begin{exercise}
Replicate the steps above, so you can type in ENV2 and ENV3 in your
terminals to switch between python 2 and 3.
\end{exercise}

\subsubsection{virtualenv}\label{virtualenv}
\index{virtualenv}

environment while using virtualenv,. Documentation about it can be found
at:

\begin{verbatim}
* https://virtualenv.pypa.io
\end{verbatim}

The installation is simple once you have pip installed. If it is not
installed you can say:

\begin{verbatim}
$ easy_install pip
\end{verbatim}

After that you can install the virtual env with:

\begin{verbatim}
$ pip install virtualenv
\end{verbatim}

To setup an isolated environment for example in the directory
\textasciitilde{}/ENV please use:

\begin{verbatim}
$ virtualenv ~/ENV
\end{verbatim}

To activate it you can use the command:

\begin{verbatim}
$ source ~/ENV/bin/activate
\end{verbatim}

you can put this command in your \verb|.bashrc| or \verb|.bash_profile| files so you
do not forget to activate it. Instructions for this can be
found in our lesson on Linux \verb|bashrc|.


\FILENAME

\section{Interactive Python}\label{interactive-python}
\index{Python!REPL}
\index{Python!interactive}

Python can be used interactively. Start by entering the interactive loop
by executing the command:

\begin{verbatim}
$ python
\end{verbatim}

You should see something like the following:

\begin{verbatim}
Python 2.7.13 (default, Nov 19 2016, 06:48:10)
[GCC 5.4.0 20160609] on linux2
Type "help", "copyright", "credits" or "license" for more information.
>>>
\end{verbatim}

The \textgreater{}\textgreater{}\textgreater{} is the prompt for the
interpreter. This is similar to the shell interpreter you have been
using.

Often we show the prompt when illustrating an example. This is to
provide some context for what we are doing. If you are following along
you will not need to type in the prompt.

This interactive prompt does the following:

\begin{itemize}
\tightlist
\item
  \emph{read} your input commands
\item
  \emph{evaluate} your command
\item
  \emph{print} the result of evaluation
\item
  \emph{loop} back to the beginning.
\end{itemize}

This is why you may see the interactive loop referred to as a
\textbf{REPL}:
\textbf{R}ead-\textbf{E}valuate-\textbf{P}rint-\textbf{L}oop.

\section{REPL (Read Eval Print Loop)}\label{repl-read-eval-print-loop}
\index{Python!REPL}

We have so far seen a few examples of types: \textbf{string}s,
\textbf{bool}s, \textbf{int}s, and \textbf{float}s. A \textbf{type}
indicates that values of that type support a certain set of operations.
For instance, how would you exponentiate a string? If you ask the
interpreter, this results in an error:

\begin{verbatim}
>>> "hello"**3
Traceback (most recent call last):
  File "<stdin>", line 1, in <module>
TypeError: unsupported operand type(s) for ** or pow(): 'str' and 'int'
\end{verbatim}

There are many different types beyond what we have seen so far, such as
\textbf{dictionaries}s, \textbf{list}s, \textbf{set}s. One handy way of
using the interactive python is to get the type of a value using
`type():

::

   \textgreater{}\textgreater{}\textgreater{} type(42)
   \textless{}type 'int'\textgreater{}
   \textgreater{}\textgreater{}\textgreater{} type(hello)
   \textless{}type 'str'\textgreater{}
   \textgreater{}\textgreater{}\textgreater{} type(3.14)
   \textless{}type 'float'\textgreater{}

You can also ask for help about something using help():

::

   \textgreater{}\textgreater{}\textgreater{} help(int)
   \textgreater{}\textgreater{}\textgreater{} help(list)
   \textgreater{}\textgreater{}\textgreater{} help(str)

.. tip::

   Using help()` opens up a pager. To navigate you can use the spacebar
to go down a page w to go up a page, the arrow keys to go up/down
line-by-line, or q to exit.

\section{Python 3 Features in Python 2}\label{python-3-features-in-python-2}
\index{Python!2 and 3}


As mentioned earlier, we assume you will use Python 2.7.X because there
are still some libraries that haven't been ported to Python 3. However,
there are some features of Python 3 we can and want to use in Python
2.7. Before we do anything else, we need to make these features
available to any subsequent code we write:

\begin{verbatim}
>>> from __future__ import print_function, division
\end{verbatim}

The first of these imports allows us to use the print function to output
text to the screen, instead of the print statement, which Python 2 uses.
This is simply a \href{https://www.python.org/dev/peps/pep-3105/}{design
decision} that better reflects Python's underlying philosophy.

The second of these imports makes sure that the
\href{https://www.python.org/dev/peps/pep-0238/}{division operator}
behaves in a way a newcomer to the language might find more intruitive.
In Python 2, division / is \emph{floor division} when the arguments are
integers, meaning that 5 / 2 == 2, for example. In Python 3, division /
is \emph{true division}, thus 5 / 2 == 2.5.


\chapter{Language}
\label{C:python-language}
\FILENAME

\section{Statements and Strings}\label{statements-and-strings}
\index{Python!statements}
\index{Python!strings}

Let us explore the syntax of Python. Type into the interactive loop and
press Enter:

\begin{verbatim}
>>> print("Hello world from Python!")
Hello world from Python!
\end{verbatim}

What happened: the print function was given a \textbf{string} to
process. A string is a sequence of characters. A \textbf{character} can
be a alphabetic (A through Z, lower and upper case), numeric (any of the
digits), white space (spaces, tabs, newlines, etc), syntactic directives
(comma, colon, quotation, exclamation, etc), and so forth. A string is
just a sequence of the character and typically indicated by surrounding
the characters in double quotes.

Standard output is discussed in the ../../lesson/linux/shell lesson.

So, what happened when you pressed Enter? The interactive Python program
read the line print "Hello world from Python!", split it into the print
statement and the "Hello world from Python!" string, and then executed
the line, showing you the output.

\section{Variables}\label{variables}
\index{Python!variables}

You can store data into a \textbf{variable} to access it later. For
instance, instead of:

\begin{verbatim}
>>> print('Hello world from Python!')
\end{verbatim}

which is a lot to type if you need to do it multiple times, you can
store the string in a variable for convenient access:

\begin{verbatim}
>>> hello = 'Hello world from Python!'
>>> print(hello)
Hello world from Python!
\end{verbatim}

\section{Data Types}\label{data-types}
\index{Python!data types}

\subsection{Booleans}\label{booleans}

A \textbf{boolean} is a value that indicates \emph{truthness} of
something. You can think of it as a toggle: either ``on'' or ``off'',
``one'' or ``zero'', ``true'' or ``false''. In fact, the only possible
values of the \textbf{boolean} (or bool) type in Python are:

\begin{itemize}
\tightlist
\item
  True
\item
  False
\end{itemize}

You can combine booleans with \textbf{boolean operators}:

\begin{itemize}
\tightlist
\item
  and
\item
  or
\end{itemize}

\begin{verbatim}
>>> print(True and True)
True
>>> print(True and False)
False
>>> print(False and False)
False
>>> print(True or True)
True
>>> print(True or False)
True
>>> print(False or False)
False
\end{verbatim}

\subsection{Numbers}\label{numbers}
\index{Python!numbers}

The interactive interpreter can also be used as a calculator. For
instance, say we wanted to compute a multiple of 21:

\begin{verbatim}
>>> print(21 * 2)
42
\end{verbatim}

We saw here the print statement again. We passed in the result of the
operation 21 * 2. An \textbf{integer} (or \textbf{int}) in Python is a
numeric value without a fractional component (those are called
\textbf{floating point} numbers, or \textbf{float} for short).

The mathematical operators compute the related mathematical operation to
the provided numbers. Some operators are:

\begin{itemize}
\tightlist
\item
  * --- multiplication
\item
  / --- division
\item
  + --- addition
\item
  - --- subtraction
\item
  ** --- exponent
\end{itemize}

Exponentiation is read as x**y is x to the yth power:

\[x^y\]

You can combine \textbf{float}s and \textbf{int}s:

\begin{verbatim}
>>> print(3.14 * 42 / 11 + 4 - 2)
13.9890909091
>>> print(2**3)
8
\end{verbatim}

Note that \textbf{operator precedence} is important. Using parenthesis
to indicate affect the order of operations gives a difference results,
as expected:

\begin{verbatim}
>>> print(3.14 * (42 / 11) + 4 - 2)
11.42
>>> print(1 + 2 * 3 - 4 / 5.0)
6.2
>>> print( (1 + 2) * (3 - 4) / 5.0 )
-0.6
\end{verbatim}



\section{Module Management}\label{module-management}

A module allows you to logically organize your Python code. Grouping
related code into a module makes the code easier to understand and use.
A module is a Python object with arbitrarily named attributes that you
can bind and reference. A module is a file consisting of Python code. A
module can define functions, classes and variables. A module can also
include runnable code.

\subsection{Import Statement}\label{import-statement}

\begin{quote}
When the interpreter encounters an import statement, it imports the
module if the module is present in the search path. A search path is a
list of directories that the interpreter searches before importing a
module. The from\ldots{}import Statement Python's from statement lets
you import specific attributes from a module into the current namespace.
The from\ldots{}import has the following syntax - from modname:
\end{quote}

import name1{[}, name2{[}, \ldots{} nameN{]}{]}

When the interpreter encounters an import statement, it imports the
module if the module is present in the search path. A search path is a
list of directories that the interpreter searches before importing a
module.

\subsection{The from \ldots{} import
Statement}\label{the-from-import-statement}

Python's from statement lets you import specific attributes from a
module into the current namespace. The from \ldots{} import has the
following syntax:

\begin{verbatim}
::
\end{verbatim}

\begin{quote}
from module1 import name1{[}, name2{[}, \ldots{} nameN{]}{]}
\end{quote}

\section{Date Time in Python}\label{date-time-in-python}

The datetime module supplies classes for manipulating dates and times in
both simple and complex ways. While date and time arithmetic is
supported, the focus of the implementation is on efficient attribute
extraction for output formatting and manipulation. For related
functionality, see also the time and calendar modules.

The import Statement You can use any Python source file as a module by
executing an import statement in some other Python source file.

\begin{verbatim}
>>>from datetime import datetime
\end{verbatim}

This module offers a generic date/time string parser which is able to
parse most known formats to represent a date and/or time.

\begin{verbatim}
>>>from dateutil.parser import parse
\end{verbatim}

pandas is an open source Python library for data analysis that needs to
be imported.

\begin{verbatim}
>>>import pandas as pd
\end{verbatim}

Create a string variable with the class start time

\begin{verbatim}
>>>fall_start = '08-21-2017'
\end{verbatim}

Convert the string to datetime format

\begin{verbatim}
>>>datetime.strptime(fall_start, '%m-%d-%Y')
datetime.datetime(2017, 8, 21, 0, 0)
\end{verbatim}

Creating a list of strings as dates

\begin{verbatim}
>>>class_dates = ['8/25/2017', '9/1/2017', '9/8/2017', '9/15/2017', '9/22/2017', '9/29/2017']
\end{verbatim}

Convert Class\_dates strings into datetime format and save the list into
variable a

\begin{verbatim}
>>>a = [datetime.strptime(x, '%m/%d/%Y') for x in class_dates]
\end{verbatim}

Use parse() to attempt to auto-convert common string formats. Parser
must be a string or character stream, not list.

\begin{verbatim}
>>>parse(fall_start)
datetime.datetime(2017, 8, 21, 0, 0)
\end{verbatim}

Use parse() on every element of the Class\_dates string.

\begin{verbatim}
>>>[parse(x) for x in class_dates] 
[datetime.datetime(2017, 8, 25, 0, 0),
 datetime.datetime(2017, 9, 1, 0, 0),
 datetime.datetime(2017, 9, 8, 0, 0),
 datetime.datetime(2017, 9, 15, 0, 0),
 datetime.datetime(2017, 9, 22, 0, 0),
 datetime.datetime(2017, 9, 29, 0, 0)]  
\end{verbatim}

Use parse, but designate that the day is first.

\begin{verbatim}
>>>parse (fall_start, dayfirst=True)
datetime.datetime(2017, 8, 21, 0, 0)
\end{verbatim}

Create a dataframe.A DataFrame is a tablular data structure comprised of
rows and columns, akin to a spreadsheet, database table. DataFrame as a
group of Series objects that share an index (the column names).

\begin{verbatim}
>>>import pandas as pd
>>>data = {'class_dates': ['8/25/2017 18:47:05.069722', '9/1/2017 18:47:05.119994', 
                        '9/8/2017 18:47:05.178768', '9/15/2017 18:47:05.230071', 
                        '9/22/2017 18:47:05.230071', '9/29/2017 18:47:05.280592'], 
        'complete': [1, 0, 1, 1, 0, 1]} 
>>>df = pd.DataFrame(data, columns = ['class_dates', 'complete'])
>>>print(df)
                 class_dates  complete
0  8/25/2017 18:47:05.069722         1
1   9/1/2017 18:47:05.119994         0
2   9/8/2017 18:47:05.178768         1
3  9/15/2017 18:47:05.230071         1
4  9/22/2017 18:47:05.230071         0
5  9/29/2017 18:47:05.280592         1
\end{verbatim}

Convert df{[}`date'{]} from string to datetime

\begin{verbatim}
>>>import pandas as pd
>>>pd.to_datetime(df['class_dates'])
0   2017-08-25 18:47:05.069722
1   2017-09-01 18:47:05.119994
2   2017-09-08 18:47:05.178768
3   2017-09-15 18:47:05.230071
4   2017-09-22 18:47:05.230071
5   2017-09-29 18:47:05.280592
Name: class_dates, dtype: datetime64[ns]
\end{verbatim}

\section{Control Statements}\label{control-statements}

\subsection{Comparision}\label{comparision}

Computer programs do not only execute instructions. Occasionally, a
choice needs to be made. Such as a choice is based on a condition.
Python has several conditional operators:

\begin{verbatim}
>   greater than
<   smaller than
==  equals
!=  is not
\end{verbatim}

Conditions are always combined with variables. A program can make a
choice using the if keyword. For example:

\begin{verbatim}
>>> x = int(input("Guess x:"))
>>> if x == 4:
...    print('You guessed correctly!')
...    <ENTER>
\end{verbatim}

In this example, \emph{You guessed correctly!} will only be printed if
the variable x equals to four (see table above). Python can also execute
multiple conditions using the elif and else keywords.

\begin{verbatim}
>>> x = int(input("Guess x:"))
>>> if x == 4:
...     print('You guessed correctly!')
... elif abs(4 - x) == 1:
...     print('Wrong guess, but you are close!')
... else:
...     print('Wrong guess')
... <ENTER>
\end{verbatim}

\subsection{Iteration}\label{iteration}

To repeat code, the for keyword can be used. For example, to display the
numbers from 1 to 10, we could write something like this:

\begin{verbatim}
>>> for i in range(1, 11):
...    print('Hello!')
\end{verbatim}

The second argument to range, \emph{11}, is not inclusive, meaning that
the loop will only get to \emph{10} before it finishes. Python itself
starts counting from 0, so this code will also work:

\begin{verbatim}
>>> for i in range(0, 10):
...    print(i + 1)
\end{verbatim}

In fact, the range function defaults to starting value of \emph{0}, so
the above is equivalent to:

\begin{verbatim}
>>> for i in range(10):
...    print(i + 1)
\end{verbatim}

We can also nest loops inside each other:

\begin{verbatim}
>>> for i in range(0,10):
...     for j in range(0,10):
...         print(i,' ',j)
... <ENTER>
\end{verbatim}

In this case we have two nested loops. The code will iterate over the
entire coordinate range (0,0) to (9,9)

\section{Datatypes}\label{datatypes}

\subsection{Lists}\label{lists}

see: \url{https://www.tutorialspoint.com/python/python_lists.htm}

Lists in Python are ordered sequences of elements, where each element
can be accessed using a 0-based index.

To define a list, you simply list its elements between square brackest
`{[}{]}`:

\begin{verbatim}
>>> >>> names = ['Albert', 'Jane', 'Liz', 'John', 'Abby']
>>> names[0] # access the first element of the list
'Albert'
>>> names[2] # access the third element of the list
'Liz'
\end{verbatim}

You can also use a negative index if you want to start counting elements
from the end of the list. Thus, the last element has index \emph{-1},
the second before last element has index \emph{-2} and so on:

\begin{verbatim}
>>> names[-1] # access the last element of the list
'Abby'
>>> names[-2] # access the second last element of the list
'John'
\end{verbatim}

Python also allows you to take whole slices of the list by specifing a
beginning and end of the slice separated by a colon `::

::

  \textgreater{}\textgreater{}\textgreater{} names{[}1:-1{]} \# the middle elements, excluding first and last
  {[}'Jane', 'Liz', 'John'{]}

As you can see from the example above, the starting index in the slice
is inclusive and the ending one, exclusive.

Python provides a variety of methods for manipulating the members of a
list.

You can add elements with append`:

\begin{verbatim}
>>> names.append('Liz')
>>> names
['Albert', 'Jane', 'Liz', 'John', 'Abby', 'Liz']
\end{verbatim}

As you can see, the elements in a list need not be unique.

Merge two lists with `extend`:

\begin{verbatim}
>>> names.extend(['Lindsay', 'Connor'])
>>> names
['Albert', 'Jane', 'Liz', 'John', 'Abby', 'Liz', 'Lindsay', 'Connor']
\end{verbatim}

Find the index of the first occurrence of an element with `index`:

\begin{verbatim}
>>> names.index('Liz')
2
\end{verbatim}

Remove elements by value with `remove`:

\begin{verbatim}
>>> names.remove('Abby')
>>> names
['Albert', 'Jane', 'Liz', 'John', 'Liz', 'Lindsay', 'Connor']
\end{verbatim}

Remove elements by index with `pop`:

\begin{verbatim}
>>> names.pop(1)
'Jane'
>>> names
['Albert', 'Liz', 'John', 'Liz', 'Lindsay', 'Connor']
\end{verbatim}

Notice that pop returns the element being removed, while remove does
not.

If you are familiar with stacks from other programming languages, you
can use insert and `pop`:

\begin{verbatim}
>>> names.insert(0, 'Lincoln')
>>> names
['Lincoln', 'Albert', 'Liz', 'John', 'Liz', 'Lindsay', 'Connor']
>>> names.pop()
'Connor'
>>> names
['Lincoln', 'Albert', 'Liz', 'John', 'Liz', 'Lindsay']
\end{verbatim}

The Python documentation contains a \href{}{full list of list
operations}.

To go back to the range function you used earlier, it simply creates a
list of numbers:

\begin{verbatim}
>>> range(10)
[0, 1, 2, 3, 4, 5, 6, 7, 8, 9]
>>> range(2, 10, 2)
[2, 4, 6, 8]
\end{verbatim}

\subsection{Sets}\label{sets}

Python lists can contain duplicates as you saw above:

\begin{verbatim}
>>> names = ['Albert', 'Jane', 'Liz', 'John', 'Abby', 'Liz']
\end{verbatim}

When we don't want this to be the case, we can use a
\href{https://docs.python.org/2/library/stdtypes.html\#set}{set}:

\begin{verbatim}
>>> unique_names = set(names)
>>> unique_names
set(['Lincoln', 'John', 'Albert', 'Liz', 'Lindsay'])
\end{verbatim}

Keep in mind that the \emph{set} is an unordered collection of objects,
thus we can not access them by index:

\begin{verbatim}
>>> unique_names[0]
Traceback (most recent call last):
  File "<stdin>", line 1, in <module>
  TypeError: 'set' object does not support indexing
\end{verbatim}

However, we can convert a set to a list easily:

\textgreater{}\textgreater{}\textgreater{} unique\_names =
list(unique\_names) \textgreater{}\textgreater{}\textgreater{}
unique\_names {[}`Lincoln', `John', `Albert', `Liz', `Lindsay'{]}
\textgreater{}\textgreater{}\textgreater{} unique\_names{[}0{]}
`Lincoln'

Notice that in this case, the order of elements in the new list matches
the order in which the elements were displayed when we create the set
(we had set({[}'Lincoln', 'John', 'Albert', 'Liz',
'Lindsay'{]}) and now we have {[}'Lincoln', 'John', 'Albert', 'Liz',
'Lindsay'{]}). You should not assume this is the case in general. That
is, don't make any assumptions about the order of elements in a set when
it is converted to any type of sequential data structure.

You can change a set's contents using the add, remove and update methods
which correspond to the append, remove and extend methods in a list. In
addition to these, \emph{set} objects support the operations you may be
familiar with from mathematical sets: \emph{union}, \emph{intersection},
\emph{difference}, as well as operations to check containment. You can
read about this in the
\href{https://docs.python.org/2/library/stdtypes.html\#set}{Python
documentation for sets}.

\subsection{Removal and Testing for Membership in
Sets}\label{removal-and-testing-for-membership-in-sets}

One important advantage of a \emph{set} over a \emph{list} is that
\textbf{access to elements is fast}. If you are familiar with different
data structures from a Computer Science class, the Python list is
implemented by an array, while the set is implemented by a hash table.

We will demonstrate this with an example. Let's say we have a list and a
set of the same number of elements (approximately 100 thousand):

\begin{verbatim}
>>> import sys, random, timeit
>>> nums_set = set([random.randint(0, sys.maxint) for _ in range(10**5)])
>>> nums_list = list(nums_set)
>>> len(nums_set)
100000
\end{verbatim}

We will use the
\href{https://docs.python.org/2/library/timeit.html}{timeit} Python
module to time 100 operations that test for the existence of a member in
either the list or set:

\begin{verbatim}
>>> timeit.timeit('random.randint(0, sys.maxint) in nums', setup='import random; nums=%s' % str(nums_set), number=100)
0.0004038810729980469
>>> timeit.timeit('random.randint(0, sys.maxint) in nums', setup='import random; nums=%s' % str(nums_list), number=100)
0.3980541229248047
\end{verbatim}

The exact duration of the operations on your system will be different,
but the take away will be the same: searching for an element in a set is
orders of magnitude faster than in a list. This is important to keep in
mind when you work with large amounts of data.

\subsection{Dictionaries}\label{dictionaries}

One of the very important data structures in python is a dictionary also
referred to as \emph{dict}.

A dictionary represents a key value store:

\begin{verbatim}
>>> person = {'Name': 'Albert', 'Age': 100, 'Class': 'Scientist'}
>>> print("person['Name']: ", person['Name'])
person['Name']:  Albert
>>> print("person['Age']: ", person['Age'])
person['Age']:  100
\end{verbatim}

You can delete elements with the following commands:

\begin{verbatim}
>>> del person['Name'] # remove entry with key 'Name'
>>> person
{'Age': 100, 'Class': 'Scientist'}
>>> person.clear()     # remove all entries in dict
>>> person
{}
>>> del person         # delete entire dictionary
>>> person
Traceback (most recent call last):
  File "<stdin>", line 1, in <module>
  NameError: name 'person' is not defined
\end{verbatim}

You can iterate over a dict:

\begin{verbatim}
>>> person = {'Name': 'Albert', 'Age': 100, 'Class': 'Scientist'}
>>> for item in person:
...   print(item, person[item])
...   <ENTER>
Age 100
Name Albert
Class Scientist
\end{verbatim}

\subsection{Dictionary Keys and
Values}\label{dictionary-keys-and-values}

You can retrieve both the keys and values of a dictionary using the
keys() and values() methods of the dictionary, respectively:

\begin{verbatim}
>>> person.keys()
['Age', 'Name', 'Class']
>>> person.values()
[100, 'Albert', 'Scientist']
\end{verbatim}

Both methods return lists. Notice, however, that the order in which the
elements appear in the returned lists (Age, Name, Class) is different
from the order in which we listed the elements when we declared the
dictionary initially (Name, Age, Class). It is important to keep this in
mind: \textbf{you can't make any assumptions about the order in which
the elements of a dictionary will be returned by the keys() and values()
methods}.

However, you can assume that if you call keys() and values() in
sequence, the order of elements will at least correspond in both
methods. In the above example Age corresponds to 100, Name to 'Albert,
and Class to Scientist, and you will observe the same correspondence in
general as long as \textbf{keys() and values() are called one right
after the other}.

\subsection{Counting with
Dictionaries}\label{counting-with-dictionaries}

One application of dictionaries that frequently comes up is counting the
elements in a sequence. For example, say we have a sequence of coin
flips:

\begin{verbatim}
>>> import random
>>> die_rolls = [random.choice(['heads', 'tails']) for _ in range(10)]
>>> die_rolls
['heads', 'tails', 'heads', 'tails', 'heads', 'heads', 'tails', 'heads', 'heads', 'heads']
\end{verbatim}

The actual list die\_rolls will likely be different when you execute
this on your computer since the outcomes of the die rolls are random.

To compute the probabilities of heads and tails, we could count how many
heads and tails we have in the list:

\begin{verbatim}
>>> counts = {'heads': 0, 'tails': 0}
>>> for outcome in coin_flips:
...   assert outcome in counts
...   counts[outcome] += 1
...   <ENTER>
>>> print('Probability of heads: %.2f' % (counts['heads'] / len(coin_flips)))
Probability of heads: 0.70
>>> print('Probability of tails: %.2f' % (counts['tails'] / sum(counts.values())))
Probability of tails: 0.30
\end{verbatim}

In addition to how we use the dictionary counts to count the elements of
coin\_flips, notice a couple things about this example:

\begin{enumerate}
\tightlist
\item
  We used the assert outcome in counts statement. The assert statement
  in Python allows you to easily insert debugging statements in your
  code to help you discover errors more quickly. assert statements are
  executed whenever the internal Python \_\_debug\_\_ variable is set to
  True, which is always the case unless you start Python with the -O
  option which allows you to run \emph{optimized} Python.
\item
  When we computed the probability of tails, we used the built-in sum
  function, which allowed us to quickly find the total number of coin
  flips. sum is one of many built-in function you can
  \href{https://docs.python.org/2/library/functions.html}{read about
  here}.
\end{enumerate}

\section{Functions}\label{functions}

You can reuse code by putting it inside a function that you can call in
other parts of your programs. Functions are also a good way of grouping
code that logically belongs together in one coherent whole. A function
has a unique name in the program. Once you call a function, it will
execute its body which consists of one or more lines of code:

\begin{verbatim}
def check_triangle(a, b, c):
return \
    a < b + c and a > abs(b - c) and \
    b < a + c and b > abs(a - c) and \
    c < a + b and c > abs(a - b)

print(check_triangle(4, 5, 6))
\end{verbatim}

The def keyword tells Python we are defining a function. As part of the
definition, we have the function name, check\_triangle, and the
parameters of the function -- variables that will be populated when the
function is called.

We call the function with arguments 4, 5 and 6, which are passed in
order into the parameters a, b and c. A function can be called several
times with varying parameters. There is no limit to the number of
function calls.

It is also possible to store the output of a function in a variable, so
it can be reused.

\begin{verbatim}
def check_triangle(a, b, c):
 return \
     a < b + c and a > abs(b - c) and \
     b < a + c and b > abs(a - c) and \
     c < a + b and c > abs(a - b)

result = check_triangle(4, 5, 6)
print(result)
\end{verbatim}

\section{Classes}\label{classes}

A class is an encapsulation of data and the processes that work on them.
The data is represented in member variables, and the processes are
defined in the methods of the class (methods are functions inside the
class). For example, let's see how to define a Triangle class:

\begin{verbatim}
class Triangle(object):

 def __init__(self, length, width, height, angle1, angle2, angle3):
     if not self._sides_ok(length, width, height):
         print('The sides of the triangle are invalid.')
     elif not self._angles_ok(angle1, angle2, angle3):
         print('The angles of the triangle are invalid.')

     self._length = length
     self._width = width
     self._height = height

     self._angle1 = angle1
     self._angle2 = angle2
     self._angle3 = angle3

 def _sides_ok(self, a, b, c):
     return \
         a < b + c and a > abs(b - c) and \
         b < a + c and b > abs(a - c) and \
         c < a + b and c > abs(a - b)

 def _angles_ok(self, a, b, c):
     return a + b + c == 180

triangle = Triangle(4, 5, 6, 35, 65, 80)
\end{verbatim}

Python has full Aobject-oriented programming (OOP) capabilities, however
we can not cover all of them in a quick tutorial, so please refer to the
\href{https://docs.python.org/2.7/tutorial/classes.html}{Python docs on
classes and OOP}.

\chapter{Data Management}

\FILENAME

Obviously when dealing with big data we may not only be dealing with
data in one format but in many different formats. It is important that
you will be able to master such formats and seamlessly integrate in
your analysis. Thus we provide some simple examples on which different
data formats exist and how to use them.

\section{Formats}

\subsection{Pickle}

Python pickle allows you to save data in a python native format into a file
that can later be read in by other programs. However, the data format
may not be portable among different python versions thus the format is
often not suitable to store information. Instead we recommend for
standrad data to use either json or yaml.

\begin{verbatim}
import pickle

flavor = { "small": 100, 
           "medium": 1000,
           "large": 10000 }

pickle.dump( flavor, open( "data.p", "wb" ) )

\end{verbatim}

To read it back in use

\begin{verbatim}
flavor = pickle.load( open( "data.p", "rb" ) )
\end{verbatim}

\subsection{Text Files}

To read text files into a variable called content  you can use 

\begin{verbatim}
content = open(“filename.txt”, “r”).read() 
\end{verbatim}

You can also use the following code while using the convenient
\verb|with| statement

\begin{verbatim}
with open('filename.txt','r') as file:
    content = file.read()
\end{verbatim}

To split up the lines of the file into an array you can do

\begin{verbatim}
with open('filename.txt','r') as file:
    lines = file.read().splitlines()
\end{verbatim}


This cam aslo be done with the build in \verb|readlines| function
\begin{verbatim}
lines = open('filename.txt','r').readlines()
\end{verbatim}

In case the file is too big you will want to read the file line by
line:

\begin{verbatim}
with open('filename.txt','r') as file:
    line = file.readline()
    print (line)
\end{verbatim}


\subsection{CSV Files}

Often data is contained in comma separated values (CSV) within a
file. To read such files you can use the csv package.

\begin{verbatim}
import csv
with open(‘data.csv’, ‘rb’) as f:
   contents = csv.reader(f)
for row in content:
    print row
\end{verbatim}

Using pandas you can read them as follows.

\begin{verbatim}
import pandas as pd
df = pd.read_csv("example.csv") 
\end{verbatim}

There are many other modules and libraries that include CSV read
functions. IN case you need to split a single line by comma, you may
also use the \verb|split| function. However, remember it swill split
at every comma, including those contained in quotes. SO this method
although looking originally convenient has limitations.

\subsection{Excel spread sheets}

Pandas contains a method to read Excel files

\begin{verbatim}
import pandas as pd
filename = 'data.xlsx'
data = pd.ExcelFile(file)
df = data.parse('Sheet1')
\end{verbatim}

\subsection{YAML}

YAML is a very important format as it allows you easily to structure
data in hierarchical fileds It is frequently used to coordinate
programs while using yaml as the specification for configuration fils,
but also data files. To read in a yaml file the following code can be
used

\begin{verbatim}
import yaml
with open('data.yaml', 'r') as f:
    content = yaml.load(f)
\end{verbatim}

The nice part is that this code can also be used to verify if a file
is valid yaml. To write data out we can use

\begin{verbatim}
with open('data.yml', 'w') as f:
    yaml.dump(data, f, default_flow_style=False)
\end{verbatim}

The flow style set to false formats the data in a nice readable
fashion with indentations.


\subsection{JSON}

\begin{verbatim}
import json
with open('strings.json') as f:
    content = json.load(f)
\end{verbatim}

\subsection{XML}

\TODO{Tutorial: Please contribute a XML python tutorial.}

\subsection{RDF}

To read RDF files you will need to install RDFlib with 

\begin{verbatim}
pip install rdflib
\end{verbatim}

This will than allow you to read RDF files

\begin{verbatim}
from rdflib.graph import Graph
g = Graph()
g.parse("filename.rdf", format="format")
for entry in g:
   print(entry)
\end{verbatim}

Good examples on using RDF are provided on the RDFlib Web page
at~\url{https://github.com/RDFLib/rdflib}

From the Web page we showcase also how to directly process RDF data
from the Web

\begin{verbatim}
import rdflib
g=rdflib.Graph()
g.load('http://dbpedia.org/resource/Semantic_Web')

for s,p,o in g:
    print s,p,o
\end{verbatim}

\subsection{PDF}

The Portable Document Format (PDF) has been made available by Adobe
Inc. royalty free. This has enabled PDF to become a world wide adopted
format that also has been standardized in 2008 (ISO/IEC 32000-1:2008,
\url{https://www.iso.org/standard/51502.html}).  A lot of research is
published in papers making PDF one of the de-facto standards for
publishing. However, PDF is difficult to parse and is focused on high
quality output instead of data representation. Nevertheless,
tools to manipulate PDF exist:

\begin{description}
\item[PDFMiner] \url{https://pypi.python.org/pypi/pdfminer/} allows
  the simple translation of PDF into text that than can be further
  mined. The manual page helps to demonstrate some examples
  \url{http://euske.github.io/pdfminer/index.html}.

\item[pdf-parser.py]
  \url{https://blog.didierstevens.com/programs/pdf-tools/} parses pdf
  documents and identifies some structural elements that can than be
  further processed.

\end{description}

If you know about other tools, let us know.


\subsection{HTML}

A very powerful library to parse HTML Web pages is provided
with~\url{https://www.crummy.com/software/BeautifulSoup/}

More details about it are provided in the documentation page
\url{https://www.crummy.com/software/BeautifulSoup/bs4/doc/}

\TODO{Students: beautiful soup contribute tutorial}

\subsection{ConfigParser}

\URL{https://pymotw.com/2/ConfigParser/}

\subsection{ConfigDict}

\URL{https://github.com/cloudmesh/cloudmesh.common/blob/master/cloudmesh/common/ConfigDict.py}

\section{Encryption}

Often we need to protect the information stored in a file. This is
achieved with encryption. There are many methods of supporting
encryption and even if a file is encrypted it may be target to
attacks. Thus it is not only important to encrypt data that you do not
want others to se but also to make sure that the system on which the
data is hosted is secure. This is especially important if we talk
about big data having a potential large effect if it gets into the
wrong hands. 

To illustrate one type of encryption that is non trivial we have
chosen to demonstrate how to encrypt a file with an ssh key. In case
you have openssl installed on your system, this can be achieved as follows.


\begin{verbatim}
#! /bin/sh

# Step 1. Creating a file with data
echo "Big Data is the future." > file.txt

# Step 2. Create the pem 
openssl rsa -in ~/.ssh/id_rsa -pubout  > ~/.ssh/id_rsa.pub.pem

# Step 3. look at the pem file to illustrate how it looks like (optional)
cat ~/.ssh/id_rsa.pub.pem

# Step 4. encrypt the file into secret.txt
openssl rsautl -encrypt -pubin -inkey ~/.ssh/id_rsa.pub.pem -in file.txt -out secret.txt

# Step 5. decrypt the file and print the contents to stdout
openssl rsautl -decrypt -inkey ~/.ssh/id_rsa -in secret.txt
\end{verbatim}

Most important here are Step 4 that encrypts the file and Step 5 that
decrypts the file. Using the Python os module it is straight forward
to implement this. However, we are providing in cloudmesh a convenient
class that makes the use in python very simple.

\begin{verbatim}
from cloudmesh.common.ssh.encrypt import EncryptFile

e = EncryptFile('file.txt', 'secret.txt')
e.encrypt()
e.decrypt()
\end{verbatim}

In our class we initialize it with the locations of the file that is
to be encrypted and decrypted. To initiate that action just call the
methods \verb|encrypt| and \verb|decrypt|.

\section{Database Access}\label{database-access}

\TODO{Students: define conventional database access tutorial}

see: \url{https://www.tutorialspoint.com/python/python_database_access.htm}

\section{SQLite}

\TODO{Students: defineSQLite database access tutorial}

\url{https://www.sqlite.org/index.html}

\url{https://docs.python.org/3/library/sqlite3.html}

\subsection{Exercises}

\begin{exercise}
\label{E:Encryption.1} Test out the shell script to replicate how this
  example works
\end{exercise}

\begin{exercise}
  \label{E:Encryption.2} Test out the cloudmesh encryption class
\end{exercise}

\begin{exercise}
  \label{E:Encryption.3} What other encryption methods exist. Can you
  provide an example and contribute to the section?
\end{exercise}

\begin{exercise}
  \label{E:Encryption.4} What is the issue of encryption that make it
  challenging for Big Data
\end{exercise}

\begin{exercise}
  \label{E:Encryption.5} Given a test dataset with many files text
  files, how long will it take to encrypt and decrypt them on various
  machines. Write a benchmark that you test. Develop this benchmark as
  a group, test out the time it takes to execute it on a variety of
  platforms.
\end{exercise}


\chapter{Libraries}
\label{C:python-lib}




\section{Installing Libraries}\label{installing-libraries}

Often you may need functionality that is not present in Python's
standard library. In this case you have two option:

\begin{itemize}
\item  implement the features yourself
\item  use a third-party library that has the desired features.
\end{itemize}

Often you can find a previous implementation of what you need. Since
this is a common situation, there is a service supporting it: the
\href{https://pypi.python.org/pypi}{Python Package Index} (or PyPi for
short).

Our task here is to install the \href{}{autopep8} tool from PyPi. This
will allow us to illustrate the use if virtual environments using the
pyenv or virtualenv command, and installing and uninstalling PyPi
packages using pip.

\section{Using pip to Install
Packages}\label{using-pip-to-install-packages}

Let's now look at another important tool for Python development: the
Python Package Index, or PyPI for short. PyPI provides a large set of
third-party python packages. If you want to do something in python,
first check pypi, as odd are someone already ran into the problem and
created a package solving it.

In order to install package from PyPI, use the pip command. We can
search for PyPI for packages:

\begin{verbatim}
$ pip search --trusted-host pypi.python.org autopep8 pylint
\end{verbatim}

It appears that the top two results are what we want so install them:

\begin{verbatim}
$ pip install --trusted-host pypi.python.org autopep8 pylint
\end{verbatim}

This will cause pip to download the packages from PyPI, extract them,
check their dependencies and install those as needed, then install the
requested packages.

\begin{description}
\item[You can skip `--trusted-host pypi.python.org' option if you have]
patched urllib3 on Python 2.7.9.
\end{description}

\section{GUI}\label{gui}

\subsection{GUIZero}\label{guizero}

Install guizero with the following command:

\begin{verbatim}
sudo pip3 install guizero
\end{verbatim}

For a comprehensive tutorial on guizero,
\href{https://lawsie.github.io/guizero/howto/}{click here}.

\subsection{Kivy}\label{kivy}

You can install Kivy on OSX as followes:

\begin{verbatim}
brew install pkg-config sdl2 sdl2_image sdl2_ttf sdl2_mixer gstreamer
pip install -U Cython
pip install kivy
pip install pygame
\end{verbatim}

A hello world program for kivy is included in the cloudmesh.robot
repository. Which you can fine here

\begin{itemize}

\item
  \url{https://github.com/cloudmesh/cloudmesh.robot/tree/master/projects/kivy}
\end{itemize}

To run the program, please download it or execute it in cloudmesh.robot
as follows:

\begin{verbatim}
cd cloudmesh.robot/projects/kivy
python swim.py
\end{verbatim}

To create stand alone packages with kivy, please see:

\begin{verbatim}
-  https://kivy.org/docs/guide/packaging-osx.html
\end{verbatim}

\section{Formatting and Checking Python
Code}\label{formatting-and-checking-python-code}

First, get the bad code:

\begin{verbatim}
$ wget --no-check-certificate http://git.io/pXqb -O bad_code_example.py
\end{verbatim}

Examine the code:

\begin{verbatim}
$ emacs bad_code_example.py
\end{verbatim}

As you can see, this is very dense and hard to read. Cleaning it up by
hand would be a time-consuming and error-prone process. Luckily, this is
a common problem so there exist a couple packages to help in this
situation.

\section{Using autopep8}\label{using-autopep8}

We can now run the bad code through autopep8 to fix formatting problems:

\begin{verbatim}
$ autopep8 bad_code_example.py >code_example_autopep8.py
\end{verbatim}

Let us look at the result. This is considerably better than before. It
is easy to tell what the example1 and example2 functions are doing.

It is a good idea to develop a habit of using autopep8 in your
python-development workflow. For instance: use autopep8 to check a file,
and if it passes, make any changes in place using the -i flag:

\begin{verbatim}
$ autopep8 file.py    # check output to see of passes
$ autopep8 -i file.py # update in place
\end{verbatim}

If you use pyCharm you have the ability to use a similar function while
p;ressing on Inspect Code.

\section{Writing Python 3 Compatible
Code}\label{writing-python-3-compatible-code}

To write python 2 and 3 compatib;e code we recommend that you take a
look at: \url{http://python-future.org/compatible_idioms.html}

\section{Using Python on
FutureSystems}\label{using-python-on-futuresystems}

This is only important if you use Futuresystems resources.

In order to use Python you must log into your FutureSystems account.
Then at the shell prompt execute the following command:

\begin{verbatim}
$ module load python
\end{verbatim}

This will make the python and virtualenv commands available to you.

The details of what the module load command does are described in the
future lesson modules.

\section{Ecosystem}\label{ecosystem}

\subsection{pypi}\label{pypi}

Link: \href{https://pypi.python.org/pypi}{pypi}

The Python Package Index is a large repository of software for the
Python programming language containing a large number of packages
{[}link{]}. The nice think about pipy is that many packages can be
installed with the program `pip'.

To do so you have to locate the \textless{}package\_name\textgreater{}
for example with the search function in pypi and say on the commandline:

\begin{verbatim}
pip install <package_name>
\end{verbatim}

where pagage\_name is the string name of the package. an example would
be the package called cloudmesh\_client which you can install with:

\begin{verbatim}
pip install cloudmesh_client
\end{verbatim}

If all goes well the package will be installed.

\subsection{Alternative Installations}\label{alternative-installations}

The basic installation of python is provided by python.org. However
others claim to have alternative environments that allow you to install
python. This includes

\begin{itemize}

\item
  \href{https://store.enthought.com/downloads/\#default}{Canopy}
\item
  \href{https://www.continuum.io/downloads}{Anaconda}
\item
  \href{http://ironpython.net/}{IronPython}
\end{itemize}

Typically they include not only the python compiler but also several
useful packages. It is fine to use such environments for the class, but
it should be noted that in both cases not every python library may be
available for install in the given environment. For example if you need
to use cloudmesh client, it may not be available as conda or Canopy
package. This is also the case for many other cloud related and useful
python libraries. Hence, we do recommend that if you are new to python
to use the distribution form python.org, and use pip and virtualenv.

Additionally some python version have platform specific libraries or
dependencies. For example coca libraries, .NET or other frameworks are
examples. For the assignments and the projects such platform dependent
libraries are not to be used.

If however you can write a platform independent code that works on
Linux, OSX and Windows while using the python.org version but develop it
with any of the other tools that is just fine. However it is up to you
to guarantee that this independence is maintained and implemented. You
do have to write requirements.txt files that will install the necessary
python libraries in a platform independent fashion. The homework
assignment PRG1 has even a requirement to do so.

In order to provide platform independence we have given in the class a
``minimal'' python version that we have tested with hundreds of
students: python.org. If you use any other version, that is your
decision. Additionally some students not only use python.org but have
used iPython which is fine too. However this class is not only about
python, but also about how to have your code run on any platform. The
homework is designed so that you can identify a setup that works for
you.

However we have concerns if you for example wanted to use chameleon
cloud which we require you to access with cloudmesh. cloudmesh is not
available as conda, canopy, or other framework package. Cloudmesh client
is available form pypi which is standard and should be supported by the
frameworks. We have not tested cloudmesh on any other python version
then python.org which is the open source community standard. None of the
other versions are standard.

In fact we had students over the summer using canopy on their machines
and they got confused as they now had multiple python versions and did
not know how to switch between them and activate the correct version.
Certainly if you know how to do that, than feel free to use canopy, and
if you want to use canopy all this is up to you. However the homework
and project requires you to make your program portable to python.org. If
you know how to do that even if you use canopy, anaconda, or any other
python version that is fine. Graders will test your programs on a
python.org installation and not canpoy, anaconda, ironpython while using
virtualenv. It is obvious why. If you do not know that answer you may
want to think about that every time they test a program they need to do
a new virtualenv and run vanilla python in it. If we were to run two
instals in the same system, this will not work as we do not know if one
student will cause a side effect for another. Thus we as instructors do
not just have to look at your code but code of hundreds of students with
different setups. This is a non scalable solution as every time we test
out code from a student we would have to wipe out the OS, install it
new, install an new version of whatever python you have elected, become
familiar with that version and so on and on. This is the reason why the
open source community is using python.org. We follow best practices.
Using other versions is not a community best practice, but may work for
an individual.

We have however in regards to using other python version additional
bonus projects such as

\begin{itemize}

\item
  deploy run and document cloudmesh on ironpython
\item
  deploy run and document cloudmesh on anaconda, develop script to
  generate a conda packge form github
\item
  deploy run and document cloudmesh on canopy, develop script to
  generate a conda packge form github
\item
  deploy run and document cloudmesh on ironpython
\item
  other documentation that would be useful
\end{itemize}


\section{Resources}\label{resources}

If you are unfamiliar with programming in Python, we also refer you to
some of the numerous online resources. You may wish to start with
\href{https://www.learnpython.org}{Learn Python} or the book
\href{http://learnpythonthehardway.org/book/}{Learn Python the Hard
Way}. Other options include
\href{http://www.tutorialspoint.com/python/}{Tutorials Point} or
\href{http://www.codecademy.com/en/tracks/python}{Code Academy}, and the
Python wiki page contains a long list of
\href{https://wiki.python.org/moin/BeginnersGuide/Programmers}{references
for learning} as well. Additional resources include:

\begin{itemize}
\item
  \url{https://virtualenvwrapper.readthedocs.io}
\item
  \url{https://github.com/yyuu/pyenv}
\item
  \url{https://amaral.northwestern.edu/resources/guides/pyenv-tutorial}
\item
  \url{https://godjango.com/96-django-and-python-3-how-to-setup-pyenv-for-multiple-pythons/}
\item
  \url{https://www.accelebrate.com/blog/the-many-faces-of-python-and-how-to-manage-them/}
\item
  \url{http://ivory.idyll.org/articles/advanced-swc/}
\item
  \url{http://python.net/~goodger/projects/pycon/2007/idiomatic/handout.html}
\item
  \url{http://www.youtube.com/watch?v=0vJJlVBVTFg}
\item
  \url{http://www.korokithakis.net/tutorials/python/}
\item
  \url{http://www.afterhoursprogramming.com/tutorial/Python/Introduction/}
\item
  \url{http://www.greenteapress.com/thinkpython/thinkCSpy.pdf}
\item
  \url{https://docs.python.org/3.3/tutorial/modules.html}
\item
  \url{https://www.learnpython.org/en/Modules/_and/_Packages}
\item
  \url{https://docs.python.org/2/library/datetime.html}
\item
  \url{https://chrisalbon.com/python/strings/_to/_datetime.html}
\end{itemize}

A very long list of useful information are also available from

\begin{itemize}
\item
  \url{https://github.com/vinta/awesome-python}
\item
  \url{https://github.com/rasbt/python_reference}
\end{itemize}

This list may be useful as it also contains links to data visualization
and manipulation libraries, and AI tools and libraries. Please note that
for this class you can reuse such libraries if not otherwise stated.

\subsection{Jupyter Notebook Tutorials}\label{jupyter-notebook-tutorials}

A Short Introduction to Jupyter Notebooks and NumPy To view the
notebook, open this link in a background tab
\url{https://nbviewer.jupyter.org/} and copy and paste the following
link in the URL input area
\url{https://cloudmesh.github.io/classes/lesson/prg/Jupyter-NumPy-tutorial-I523-F2017.ipynb}
Then hit Go.


\section{Exercises}\label{exercises}

\begin{exercise}\label{E:Python.1}
Write a python program called iterate.py that accepts an integer n from
the command line. Pass this integer to a function called iterate.

The iterate function should then iterate from 1 to n. If the ith number
is a multiple of three, print ``multiple of 3'', if a multiple of 5
print ``multiple of 5'', if a multiple of both print ``multiple of 3 and
5'', else print the value.
\end{exercise}

\begin{exercise}\label{E:Python.2}
  \begin{enumerate}
  \item
    Create a pyenv or virtualenv \textasciitilde{}/ENV
  \item
    Modify your \textasciitilde{}/.bashrc shell file to activate your
    environment upon login.
  \item
    Install the docopt python package using pip
  \item
    Write a program that uses docopt to define a commandline program.
    Hint: modify the iterate program.
  \item
    Demonstrate the program works and submit the code and output.
  \end{enumerate}
\end{exercise}



\section{Python for Big Data}\label{python-for-big-data}

\subsection{An Example with Pandas, NumPy and Matplotlib}\label{an-example-with-pandas-numpy-and-matplotlib}

In this example, we will download some traffic citation data for the
city of Bloomington, IN, load it into Python and generate a histogram.
In doing so, you will be exposed to important Python libraries for
working with big data such as \href{www.numpy.org}{numpy},
\href{pandas.pydata.org}{pandas} and \href{matplotlib.org}{matplotlib}.

\subsubsection{Set Up Directories and Get Test
Data}\label{set-up-directories-and-get-test-data}

Data.gov is a government portal for open data and the
\href{https://catalog.data.gov/dataset?organization_type=City+Government\&organization=city-of-bloomington\&_organization_limit=0}{city
  of Bloomington, Indiana makes available a number of datasets there}.

We will use traffic citations data for 2016.

To start, let's create a separate directory for this project and
download the CSV data:

\begin{lstlisting}{bash}
$ cd ~/projects/i524
$ mkdir btown-citations
$ cd btown-citations
$ wget https://data.bloomington.in.gov/dataset/c543f0c1-1e37-46ce-a0ba-e0a949bd248a/resource/24841976-fd35-4483-a2b4-573bd1e77cfb/download/2016-first-quarter-citations.csv
\end{lstlisting}

Depending on your directory organization, the above might be slightly
different for you.

If you go to the link to data.gov for Bloomington above, you will see
that the citations data is organized per quarter, so there are a total
of four files. Above, we downloaded the data for the first quarter. Go
ahead and download the remaining three files with \texttt{wget}.

In this example, we will use three modules, \texttt{numpy},
\texttt{pandas} and \texttt{matplotlib}. If you set up
\texttt{virtualenv} as described in the
Python tutorial \textless{}python\_intro\textgreater{}, the first two of
these are already installed for you. To install \texttt{matplotlib},
make sure you've activated your \texttt{virtualenv} and use
\texttt{pip}:

\begin{lstlisting}{bash}
$ source ~/ENV/bin/activate
$ pip install matplotlib
\end{lstlisting}

If you are using a different distribution of Python, you will need to
make sure that all three of these modules are installed.

\subsubsection{Load Data in Pandas}\label{load-data-in-pandas}

From the same directory where you saved the citations data, let's start
the Python interpreter and load the citations data for Q1 2016

\begin{lstlisting}
$ python
>>> from __future__ import division, print_function
>>> import numpy as np
>>> import pandas as pd
>>> import matplotlib.pyplot as plt
>>> data = pd.read_csv('2016-first-quarter-citations.csv')
\end{lstlisting}
% $

If the first \texttt{import} statement seems confusing, take a look at
the Python tutorial \textless{}python\_intro\textgreater{}. The next
three \texttt{import} statements load each of the modules we will use in
this example. The final line uses Pandas' \verb|read_csv| function to
load the data into a Pandas \texttt{DataFrame} data structure.

\subsubsection{Working with DataFrames}\label{working-with-dataframes}

You can verify that you are working with a \texttt{DataFrame} and use
some of its methods to take a look at the structure of the data as
follows:

\begin{lstlisting}
>>> type(data)
<class 'pandas.core.frame.DataFrame'>
>>> data.index
Int64Index([  0,   1,   2,   3,   4,   5,   6,   7,   8,   9,
...
197, 198, 199, 200, 201, 202, 203, 204, 205, 206],
dtype='int64', length=200)
>>> data.columns
Index([u'Citation Number', u'Date Issued', u'Time Issued', u'Location ',
u'District', u'Cited Person Age', u'Cited Person Sex',
u'Cited Person Race', u'Offense Code', u'Offense Description',
u'Officer Age', u'Officer Sex', u'Officer Race'],
dtype='object')
>>> data.dtypes
Citation Number                object
Date Issued                    object
Time Issued                    object
Location                       object
District                       object
Cited Person Age              float64
Cited Person Sex               object
Cited Person Race              object
Offense Code                   object
Offense Description            object
Officer Age                   float64
Officer Sex                    object
Officer Race                   object
dtype: object
>>> data.shape
(200, 15)
\end{lstlisting}

As you can see from the \texttt{columns} field, when the CSV file was
read, the header line was used to populate the name of the columns in
the \texttt{DataFrame}. In addition, you will notice that
\verb|read_csv| correctly inferred the data type of some columns like
\emph{Age}, but not of others like \emph{Date Issued} and \emph{Time
Issued}. \verb|read_csv| is a very customizable function and in
general, you can correct issues like this using the \texttt{dtype} and
\texttt{converters} parameters. In this specific case, it makes more
sense to combine the \emph{Date Issued} and \emph{Time Issued} columns
into a new column containing a time stamp. We will see how to do this
shortly.

You can also look at the data itself with the \texttt{DataFrame}'s
\texttt{head()} and \texttt{tail()} methods:

\begin{lstlisting}
>>> data.head()
<Output omitted for brevity>
>>> data.tail()
<Output omitted for brevity>
\end{lstlisting}

In addition to letting you examine your data easily, \texttt{DataFrame}s
have methods that help you deal with missing values:

\begin{lstlisting}
>>> data = data.dropna(how='any')
>>> data.shape
\end{lstlisting}

Adding columns to the data is also easy. Here, we add two columns.
First, a
\href{https://docs.python.org/2/library/datetime.html}{datetime} column
that is a combination of the \texttt{Date\ Issued} and
\texttt{Time\ Issued} columns originally in the data. Second, a column
identifying what day of the week each citation was given. To understand
this example better, take a look at the Python docs for the
\texttt{strptime} and \texttt{strftime} functions in the
\texttt{datetime} module linked above.

\begin{lstlisting}
>>> from datetime import datetime
>>> data['DateTime Issued'] = data.apply(
...  lambda row: datetime.strptime(row['Date Issued'] + ':' + row['Time Issued'], '%m/%d/%y:%I:%M %p'), axis=1
... )
>>> data.columns
>>> data['Day of Week Issued'] = data.apply(
...  lambda row: datetime.strftime(row['DateTime Issued'], '%A'), axis=1
... )
\end{lstlisting}

\subsubsection{Plotting with Matplotlib and
NumPy}\label{plotting-with-matplotlib-and-numpy}

Let's say we want to see how many citations were given each day of the
week. We gather the data first:

\begin{lstlisting}
>>> days = ['Monday', 'Tuesday', 'Wednesday', 'Thursday', 'Friday', 'Saturday', 'Sunday']
>>> dow_data = [days.index(dow) for dow in data['Day of Week Issued']]
>>> dow_data
<Output omitted for brevity>
\end{lstlisting}

Then we use \texttt{matplotlib} to plot it:

\begin{lstlisting}
>>> fig = plt.figure()
>>> ax = fig.add_subplot(1, 1, 1)
>>> plt.hist(dow_data, bins=len(days))
>>> plt.xticks(range(len(days)), days)
>>> plt.show()
\end{lstlisting}

You should see something like this on your screen:

\includegraphics[width=4.16667in]{dow.png}

\subsubsection{\texorpdfstring{More \emph{DataFrame} Manipulation and
Plotting}{More DataFrame Manipulation and Plotting}}\label{more-dataframe-manipulation-and-plotting}

\texttt{DataFrame}s and \texttt{numpy} give us other ways to manipulate
data. For example, we can plot a histogram of the ages of violators like
this:

\begin{lstlisting}
>>> ages = data['Cited Person Age'].astype(int)
>>> fig = plt.figure()
>>> ax = fig.add_subplot(1, 1, 1)
>>> plt.hist(ages, bins=np.max(ages) - np.min(ages))
>>> plt.show()
\end{lstlisting}

\includegraphics[width=4.16667in]{ages.png}

Surprisingly, we see some 116 year-old violators! This is probably an
error in the data, so we can remove these data points easily and plot
the histogram again:

\begin{lstlisting}
>>> ages = ages[ages < 100]
>>> fig = plt.figure()
>>> ax = fig.add_subplot(1, 1, 1)
>>> plt.hist(ages, bins=np.max(ages) - np.min(ages))
>>> plt.show()
\end{lstlisting}

\includegraphics[width=4.16667in]{ages-filtered.png}

\subsubsection{Saving Plots to PDF}\label{saving-plots-to-pdf}

Oftentimes, you will want to save your \texttt{matplotlib} graph as a
PDF or an SVG file instead of just viewing it on your screen. For both,
we need to create a \texttt{figure} and plot the histogram as before:

\begin{lstlisting}
>>> fig = plt.figure()
>>> ax = fig.add_subplot(1, 1, 1)
>>> plt.hist(ages, bins=np.max(ages) - np.min(ages))
\end{lstlisting}

Then, instead of calling \texttt{plt.show()} we can invoke
\texttt{plt.savefig()} to save as SVG:

\begin{lstlisting}
>>> plt.savefig('hist.svg')
\end{lstlisting}

If we want to save the figure as PDF instead, we need to use the
\texttt{PdfPages} module together with \texttt{savefig()}:

\begin{lstlisting}
>>> import matplotlib.patches as mpatches
>>> from matplotlib.backends.backend_pdf import PdfPages   
>>> pp = PdfPages('hist.pdf')
>>> fig.savefig(pp, format='pdf')
>>> pp.close()
\end{lstlisting}

\subsubsection{Next Steps and Exercises}\label{next-steps-and-exercises}

There is a lot more to working with \texttt{pandas}, \texttt{numpy} and
\texttt{matplotlib} than we can show you here, but hopefully this
example has piqued your curiosity.

Don't worry if you don't understand everything in this example. For a
more detailed explanation on these modules and the examples we did,
please take a look at the tutorials below. The \texttt{numpy} and
\texttt{pandas} tutorials are mandatory if you want to be able to use
these modules, and the \texttt{matplotlib} gallery has many useful code
examples.

\subsection{Summary of Useful
Libraries}\label{summary-of-useful-libraries}

\subsubsection{Numpy}\label{s:numpy}

  \URL{http://www.numpy.org/}


According to the Numpy Web page ``NumPy is a package for scientific
computing with Python. It contains a powerful N-dimensional array
object, sophisticated (broadcasting) functions, tools for integrating
C/C++ and Fortran code, useful linear algebra, Fourier transform, and
random number capabilities''.

Tutorial:
\url{https://docs.scipy.org/doc/numpy-dev/user/quickstart.html}

\subsubsection{MatplotLib}\label{matplotlib}

  \URL{http://matplotlib.org/}

According the the Matplotlib Web page, ``matplotlib is a python 2D
plotting library which produces publication quality figures in a variety
of hardcopy formats and interactive environments across platforms.
matplotlib can be used in python scripts, the python and ipython shell
(ala MATLAB* or Mathematica), web application servers, and six
graphical user interface toolkits.''

Matplotlib Gallery: \url{http://matplotlib.org/gallery.html}

\subsubsection{Pandas}\label{pandas}

  \URL{http://pandas.pydata.org/}

According to the Pandas Web page, ``Pandas is a library library
providing high-performance, easy-to-use data structures and data
analysis tools for the Python programming language.''

In addition to access to charts via matplotlib it has elementary
functionality for conduction data analysis. Pandas may be very suitable
for your projects.

Tutorial: \url{http://pandas.pydata.org/pandas-docs/stable/10min.html}

Pandas Cheat Sheet:
\url{https://github.com/pandas-dev/pandas/blob/master/doc/cheatsheet/Pandas_Cheat_Sheet.pdf}

\subsection{Big Data Libraries}\label{other-useful-libraries}

\subsubsection{Scipy}\label{s:scipy}

  \URL{https://www.scipy.org/}

According to the Web page, SciPy (pronounced \textit{Sigh Pie}) is a
Python-based ecosystem of open-source software for mathematics, science,
and engineering. In particular, these are some of the core packages:

\begin{itemize}
\item NumPy
\item IPython
\item Pandas
\item Matplotlib
\item Sympy
\item SciPy library
\end{itemize}

It is thus an agglomeration of useful pacakes and will prbably sufice
for your projects in case you use Python.

\subsubsection{ggplot}\label{ggplot}

  \URL{http://ggplot.yhathq.com/}


According to the ggplot python Web page ggplot is a plotting system for
Python based on R's ggplot2. It allows to quickly generate some plots
quickly with little effort. Often it may be easier to use than
matplotlib directly.

\subsubsection{seaborn}\label{seaborn}

\URL{http://www.data-analysis-in-python.org/t_seaborn.html}

The good library for plotting is called seaborn which is build on top of
matplotlib. It provides high level templates for common statistical
plots.

\begin{itemize}
\item
  Gallery:
  \url{http://stanford.edu/~mwaskom/software/seaborn/examples/index.html}
\item
  Original Tutorial:
  \url{http://stanford.edu/~mwaskom/software/seaborn/tutorial.html}
\item
  Additional Tutorial:
  \url{https://stanford.edu/~mwaskom/software/seaborn/tutorial/distributions.html}
\end{itemize}

Here are some examples from a previous class:

\URL{https://github.com/bigdata-i523/hid231/blob/master/experiment/seaborn/seaborn-exercises.ipynb}
\URL{https://github.com/bigdata-i523/hid231/blob/master/experiment/learning-jupyter/learning_jupyter_notebook.ipynb}

\begin{exercise}\label{E:ipynb-export}
Take these examples and create sections in latex that can be added to
the book. Describe the process. 

1. export the ipynb as rst
2. use pandoc to export it to tex
3. do some cleanup on the tex files

Can this be automated with a cmd5 script such as
\begin{lstlisting}{bash}
cms ipynb [url=URL | file=FILE] --output FILENAME
\end{lstlisting}
\end{exercise}

\subsubsection{Bokeh}\label{bokeh}

Bokeh is an interactive visualization library with focus on web browsers
for display. Its goal is to provide a similar experience as D3.js

\begin{itemize}
\item
  URL: \url{http://bokeh.pydata.org/}
\item
  Gallery: \url{http://bokeh.pydata.org/en/latest/docs/gallery.html}
\end{itemize}

\subsubsection{pygal}\label{pygal}

Pygal is a simple API to produce graphs that can be easily embedded into
your Web pages. It contains annotations when you hover over data points.
It also allows to present the data in a table.

\URL{http://pygal.org/}


\subsubsection{Network and Graphs}\label{network-and-graphs}

\begin{itemize}

\item
  igraph: \url{http://www.pythonforsocialscientists.org/t_igraph.html}
\item
  networkx: \url{https://networkx.github.io/}
\end{itemize}

\subsubsection{REST}\label{rest}

\begin{itemize}

\item
  django REST FRamework \url{http://www.django-rest-framework.org/}
\item
  flask
  \url{https://blog.miguelgrinberg.com/post/designing-a-restful-api-with-python-and-flask}
\item
  requests
  \url{https://realpython.com/blog/python/api-integration-in-python/}
\item
  urllib2
  \url{http://rest.elkstein.org/2008/02/using-rest-in-python.html} (not
  recommended)
\item
  web
  \url{http://www.dreamsyssoft.com/python-scripting-tutorial/create-simple-rest-web-service-with-python.php}
  (not recommended)
\item
  bottle \url{http://bottlepy.org/docs/dev/index.html}
\item
  falcon \url{https://falconframework.org/}
\item
  eve \url{http://python-eve.org/}
\item
  \url{https://code.tutsplus.com/tutorials/building-rest-apis-using-eve--cms-22961}
\end{itemize}

\section{Parsing Data}

Being able to parse data is an important activity in the data analysis
process. Not all data may be following a specific format and the data
may need to be extracted.


\subsection{notebook.md Parser}

We are using a notebook.md to communicate what students have done
throughout the semester. We like to make a simple cmd5 command that
parses the notebook.md file and check it upon correctness.

An example for a notebook.md file is located here 

\URL{https://raw.githubusercontent.com/bigdata-i523/sample-hid000/master/notebook.md}

The following code may inspire you

\URL{https://github.com/bigdata-i523/hid203/tree/master/experiment}

We like to implement the following functionality and use docopts to
document the command.

\begin{lstlisting}
cms class notebook [--git=GITREPONAME] --verify hid

    verifies the correctness of the notebook.md file

cms class notebook [--git=GITREPONAME] --log

    displays the log of the notebook.md

cms class notebook [--git=GITREPONAME] --history

    displays a true or false for each week since the first occurance
    of the notebook.md file in the git repository
\end{lstlisting}

\begin{exercise}\label{E:notebook-md.1}
  Write a notebook.md parser
\end{exercise}

\begin{exercise}\label{E:notebook-md.2}
  How can this command be generalized to provide not only information
  for a student but to provide information for the class. Example: can
  we identify preferred days of when the notebooks are checked in. Can
  we identify a list of students that have not updated the notebok for
  a week?  Can we identify the list of student s that have updated the
  notebook for a week? 
\end{exercise}


\subsection{Video Length}

The Latex source of this class contains a macro to include videos.


Given a \LaTeX~file, can you create a table that includes the names of
all videos in that file and sums up the total viewing time. Previously
the document was stored in RST and the code from a previous student
may inspire you. Can you recreate it for \LaTeX? 

\URL{https://github.com/bigdata-i523/hid107/blob/master/cloudmesh/bar/command/mycommand.py}


\begin{lstlisting}
cms class video list FILENAME --output=[tabular|longtable|csv|txt]

    prints the videolist in the given format. txt means it is just ASCII
\end{lstlisting}

\begin{comment}
The previous work even summarized the video length by chapter.

see: https://piazza.com/class/j5wll7vzylg25j?cid=325
\end{comment}

\begin{exercise}
Write a tool that extracts the information for video length. 
\end{exercise}

\begin{exercise}
Write a tool that finds all youtube urls that are not in a video latex
macro.
\end{exercise}

\subsection{Dask}
\index{Dask}

Many times operations need to be done on data in parallel to utilize
modern processor architectures.


Dask provides a \textit{dynamic task scheduling} which is optimized for
computation. It is similar to other frameworks such as Airflow, Luigi,
Celery, or Make. However it is specializing in optimized interactive
computational workloads.

Furthermore, Dask targets Big Data \textit{collections} such as parallel
arrays, dataframes, and lists. These collections ar commonly found in
NumPy, Pandas, or Python iterators to larger-than-memory or
distributed environments. While using the Dask implementation we can
replace the original imports from the appropriate framework, replace
them with Dask imports and implicitly use parallel collections that
utilize internally the dynamic task schedulers.

More information can be found at:

\URL{https://dask.pydata.org}

\begin{exercise}
Conduct a performance study that showcases the difference of doing
parallel calculations in Dask, calculations in a framework such as
SciPy, and regular unthreaded python code.
\end{exercise}








\section{Radar Case Study}\label{radar-case-study}
\FILENAME

The changing global climate is suspected to have long-term effects on
much of the world's inhabitants. Among the various effects, the rising
sea level will directly affect many people living in low-lying coastal
regions. While the ocean-s thermal expansion has been the dominant
contributor to rises in sea level, the potential contribution of
discharges from the polar ice sheets in Greenland and Antarctica may
provide a more significant threat due to the unpredictable response to
the changing climate. The Radar-Informatics unit provides a glimpse in
the processes fueling global climate change and explains what methods
are used for ice data acquisitions and analysis.

\slides{Radar}{Radar}{58}{https://drive.google.com/open?id=0B8936_ytjfjmZ0VzZ0ZIenpUMTQ}

\subsection{Introduction}\label{introduction}

This lesson motivates radar-informatics by building on previous
discussions on why X-applications are growing in data size and why
analytics are necessary for acquiring knowledge from large data. The
lesson details three mosaics of a changing Greenland ice sheet and
provides a concise overview to subsequent lessons by detailing
explaining how other remote sensing technologies, such as the radar, can
be used to sound the polar ice sheets and what we are doing with radar
images to extract knowledge to be incorporated into numerical models.

\video{Radar}{3:31}{Radar Informatics}{https://youtu.be/LXOncC2AhsI}

\subsection{Remote Sensing}\label{remote-sensing}

This lesson explains the basics of remote sensing, the characteristics
of remote sensors and remote sensing applications. Emphasis is on image
acquisition and data collection in the electromagnetic spectrum.

\video{Radar}{6:43}{Remote Sensing}{https://youtu.be/TTrm9rmZySQ}

\subsection{Ice Sheet Science}\label{ice-sheet-science}

This lesson provides a brief understanding on why melt water at the base
of the ice sheet can be detrimental and why it's important for sensors
to sound the bedrock.

\video{Radar}{1:00}{Ice Sheet Science}{https://youtu.be/rDpjMLguVBc}

\subsection{Global Climate Change}\label{global-climate-change}

This lesson provides an understanding and the processes for the
greenhouse effect, how warming effects the Polar Regions, and the
implications of a rise in sea level.

\video{Radar}{2:51}{Global Climate Change}{https://youtu.be/f9hzzJX0qDs}

\subsection{Radio Overview}\label{radio-overview}

This lesson provides an elementary introduction to radar and its
importance to remote sensing, especially to acquiring information about
Greenland and Antarctica.

\video{Radar}{4:16}{Radio Overview}{https://youtu.be/PuI7F-RMKCI}

\subsection{Radio Informatics}\label{radio-informatics}

This lesson focuses on the use of sophisticated computer vision
algorithms, such as active contours and a hidden markov model to support
data analysis for extracting layers, so ice sheet models can accurately
forecast future changes in climate.

\video{Radar}{3:35}{Radio Informatics}{https://youtu.be/q3Pwyt49syE}



\chapter{Sensors}

\FILENAME\

We start with the Internet of Things IoT giving examples like monitors
of machine operation, QR codes, surveillance cameras, scientific
sensors, drones and self driving cars and more generally
transportation systems. We give examples of robots and drones. We
introduce the Industrial Internet of Things IIoT and summarize surveys
and expectations Industry wide. We give examples from General
Electric.  Sensor clouds control the many small distributed devices of
IoT and IIoT. More detail is given for radar data gathered by sensors;
ubiquitous or smart cities and homes including U-Korea; and finally
the smart electric grid.

\slides{Sensor}{31}{Sensor I}{https://drive.google.com/open?id=0B8936_ytjfjmVXZCUnR3TnVMMFk}

\slides{Sensor}{44}{Sensor II}{https://drive.google.com/open?id=0B8936_ytjfjmelMwSUl6Q1lLV1k}


\section{Internet of Things}\label{internet-of-things}

There are predicted to be 24-50 Billion devices on the Internet by 2020;
these are typically some sort of sensor defined as any source or sink of
time series data. Sensors include smartphones, webcams, monitors of
machine operation, barcodes, surveillance cameras, scientific sensors
(especially in earth and environmental science), drones and self driving
cars and more generally transportation systems. The lesson gives many
examples of distributed sensors, which form a Grid that is controlled by
a cloud.

\video{Sensor}{12:36}{Internet of Things}{https://www.youtube.com/watch?v=0O0-mz-CWtQ} 


\section{Robotics and IoT}

Examples of Robots and Drones.

\video{Sensor}{8:05}{Robotics and IoT Expectations}{https://www.youtube.com/watch?v=ABP0Yygw2Zg}


\section{Industrial Internet of Things}

We summarize surveys and expectations Industry wide.

\video{Sensor}{24:02}{Industrial Internet of Things}{https://www.youtube.com/watch?v=kxKzBfd62Og}


\section{Sensor Clouds}

We describe the architecture of a Sensor Cloud control environment and
gives example of interface to an older version of it. The performance of
system is measured in terms of processing latency as a function of
number of involved sensors with each delivering data at 1.8 Mbps rate.

\video{Sensor}{4:40}{Sensor Clouds}{https://youtu.be/0egT1FsVGrU}


\section{Earth/Environment/Polar Science data gathered by Sensors}

This lesson gives examples of some sensors in the
Earth/Environment/Polar Science field. It starts with material from the
CReSIS polar remote sensing project and then looks at the NSF Ocean
Observing Initiative and NASA's MODIS or Moderate Resolution Imaging
Spectroradiometer instrument on a satellite.

\video{Sensor}{4:58}{Earth/Environment/Polar Science data gathered by Sensors}{https://youtu.be/CS2gX7axWfI}


\section{Ubiquitous/Smart Cities}

For Ubiquitous/Smart cities we give two examples: Iniquitous Korea and
smart electrical grids.

\video{Sensor}{1:44}{Ubiquitous/Smart Cities}{https://youtu.be/MFFIItQ3SOo}


\section{U-Korea (U=Ubiquitous)}

Korea has an interesting positioning where it is first worldwide in
broadband access per capita, e-government, scientific literacy and total
working hours. However it is far down in measures like quality of life
and GDP. U-Korea aims to improve the latter by Pervasive computing,
everywhere, anytime i.e. by spreading sensors everywhere. The example of
a `High-Tech Utopia' New Songdo is given.

\video{Sensor}{2:49}{U-Korea (U=Ubiquitous)}{https://www.youtube.com/watch?v=U38zWbSI2n4} 


\section{Smart Grid}

The electrical Smart Grid aims to enhance USA's aging electrical
infrastructure by pervasive deployment of sensors and the integration of
their measurement in a cloud or equivalent server infrastructure. A
variety of new instruments include smart meters, power monitors, and
measures of solar irradiance, wind speed, and temperature. One goal is
autonomous local power units where good use is made of waste heat.

\video{Sensor}{6:04}{Smart Grid}{https://www.youtube.com/watch?v=UfEiIzaZzI8} 


\section{Resources}

\TODO{These resources have not all been checked to see if they still
  exist this is curretnly in progress}

\begin{itemize}
\item
  \url{https://www.gesoftware.com/minds-and-machines}
\item
  \url{https://www.gesoftware.com/predix}
\item
  \url{https://www.gesoftware.com/sites/default/files/the-industrial-internet/index.html}
\item
  \url{https://developer.cisco.com/site/eiot/discover/overview/}
\item
  \url{http://www.accenture.com/SiteCollectionDocuments/PDF/Accenture-Industrial-Internet-Changing-Competitive-Landscape-Industries.pdf}
\item
  \url{http://www.gesoftware.com/ge-predictivity-infographic}
\item
  \url{http://www.getransportation.com/railconnect360/rail-landscape}
\item
  \url{http://www.gesoftware.com/sites/default/files/GE-Software-Modernizing-Machine-to-Machine-Interactions.pdf}
\end{itemize}


\chapter{Sports}
\label{c:sports}
\index{Sports}

\FILENAME\

Sports sees significant growth in analytics with pervasive statistics
shifting to more sophisticated measures. We start with baseball as game
is built around segments dominated by individuals where detailed
(video/image) achievement measures including PITCHf/x and FIELDf/x are
moving field into big data arena. There are interesting relationships
between the economics of sports and big data analytics. We look at
Wearables and consumer sports/recreation. The importance of spatial
visualization is discussed. We look at other Sports: Soccer, Olympics,
NFL Football, Basketball, Tennis and Horse Racing.

\section{Basic Sabermetrics}

This unit discusses baseball starting with the movie Moneyball and the
2002-2003 Oakland Athletics. Unlike sports like basketball and soccer,
most baseball action is built around individuals often interacting in
pairs. This is much easier to quantify than many player phenomena in
other sports. We discuss Performance-Dollar relationship including new
stadiums and media/advertising. We look at classic baseball averages and
sophisticated measures like Wins Above Replacement.


\slides{Sport}{40}{Overview}{https://drive.google.com/open?id=0B8936_ytjfjmbWt6bGZuTFJ4TFE}


\subsection{Introduction and Sabermetrics (Baseball Informatics)
  Lesson}

Introduction to all Sports Informatics, Moneyball The 2002-2003 Oakland
Athletics, Diamond Dollars economic model of baseball, Performance -
Dollar relationship, Value of a Win.


\video{Sport}{31:4}{Introduction and Sabermetrics (Baseball Informatics) Lesson}{https://www.youtube.com/watch?v=Dd4zV__G5Q8} 


\subsection{Basic Sabermetrics}\label{basic-sabermetrics}

Different Types of Baseball Data, Sabermetrics, Overview of all data,
Details of some statistics based on basic data, OPS, wOBA, ERA, ERC,
FIP, UZR.


\video{Sport}{26:53}{Basic Sabermetrics}{https://www.youtube.com/watch?v=L0X-RQJZKrs} 



\subsection{Wins Above Replacement}\label{wins-above-replacement}

Wins above Replacement WAR, Discussion of Calculation, Examples,
Comparisons of different methods, Coefficient of Determination, Another,
Sabermetrics Example, Summary of Sabermetrics.


\video{Sport}{30:43}{Wins Above Replacement}{https://www.youtube.com/watch?v=D6PHqPor4LA} 




\section{Advanced  Sabermetrics}\label{sports-informatics-ii-sabermetrics-advanced}

This unit discusses `advanced sabermetrics' covering advances possible
from using video from PITCHf/X, FIELDf/X, HITf/X, COMMANDf/X and MLBAM.


\slides{Sport}{41}{Sporta II}{https://drive.google.com/open?id=0B8936_ytjfjmUDh0Y01GbW9tWnc}


\subsection{Pitching Clustering}\label{pitching-clustering}

A Big Data Pitcher Clustering method introduced by Vince Gennaro, Data
from Blog and video at 2013 SABR conference.


\video{Sport}{20:59}{Pitching Clustering}{https://www.youtube.com/watch?v=rZ9-b54aEvw} 


\subsection{Pitcher Quality}\label{pitcher-quality}

Results of optimizing match ups, Data from video at 2013 SABR
conference.


\video{Sport}{10:02}{Pitcher Quality}{https://www.youtube.com/watch?v=OkkUaySvXOY} 



\section{PITCHf/X}\label{pitchfx}

Examples of use of PITCHf/X.


\video{Sport}{10:39}{PITCHf/X}{https://www.youtube.com/watch?v=m7IXhsHgQmE} 


\subsection{Other Video Data Gathering in Baseball}

FIELDf/X, MLBAM, HITf/X, COMMANDf/X.


\video{Sport}{18:5}{Other Video Data Gathering in Baseball}{https://www.youtube.com/watch?v=nKZiOOGccms}



\section{Other Sports}

We look at Wearables and consumer sports/recreation. The importance of
spatial visualization is discussed. We look at other Sports: Soccer,
Olympics, NFL Football, Basketball, Tennis and Horse Racing.


\slides{Sport}{Sports III}{44}{https://drive.google.com/open?id=0B8936_ytjfjmUGdpUzFaRzhyWXM}


\subsection{Wearables}\label{wearables}

Consumer Sports, Stake Holders, and Multiple Factors.


\video{Sport}{22:2}{Wearables}{https://www.youtube.com/watch?v=F_cPq6xIXw0} 


\subsection{Soccer and the Olympics}\label{soccer-and-the-olympics}

Soccer, Tracking Players and Balls, Olympics.


\video{Sport}{8:28}{Soccer and the Olympics}{https://www.youtube.com/watch?v=AiZneaLJMTs} 



\subsection{Spatial Visualization in NFL and NBA}

NFL, NBA, and Spatial Visualization.


\video{Sport}{15:19}{Spatial Visualization in NFL and NBA}{https://www.youtube.com/watch?v=Uorh3RJLC1s}


\subsection{Tennis and Horse Racing}

Tennis, Horse Racing, and Continued Emphasis on Spatial Visualization.


\video{Sport}{8:52}{Tennis and Horse Racing}{https://www.youtube.com/watch?v=2P-pismFSrI}  

\subsection{Resources}

\TODO{These resources have not all been checked to see if they still
  exist this is curretnly in progress}


\begin{itemize}
\item
  \url{http://www.sloansportsconference.com/?page_id=481\&sort_cate=Research\%20Paper}
\item
  \url{http://www.slideshare.net/Tricon_Infotech/big-data-for-big-sports}
\item
  \url{http://www.slideshare.net/BrandEmotivity/sports-analytics-innovation-summit-data-powered-storytelling}
\item
  \url{http://www.liveathos.com/apparel/app}
\item
  \url{http://www.slideshare.net/elew/sport-analytics-innovation}
\item
  \url{http://www.wired.com/2013/02/catapault-smartball/}
\item
  \url{http://www.sloansportsconference.com/wp-content/uploads/2014/06/Automated_Playbook_Generation.pdf}
\item
  \url{http://autoscout.adsc.illinois.edu/publications/football-trajectory-dataset/}
\item
  \url{http://www.sloansportsconference.com/wp-content/uploads/2012/02/Goldsberry_Sloan_Submission.pdf}
\item
  \url{http://gamesetmap.com/}
\item
  \url{http://www.trakus.com/technology.asp\#tNetText}
\end{itemize}


\begin{itemize}
\item
  \url{http://www.slideshare.net/BrandEmotivity/sports-analytics-innovation-summit-data-powered-storytelling}
\item
  \url{http://www.sloansportsconference.com/}
\item
  \url{http://sabr.org/}
\item
  \url{http://en.wikipedia.org/wiki/Sabermetrics}
\item
  \url{http://en.wikipedia.org/wiki/Baseball_statistics}
\item
  \url{http://www.sportvision.com/baseball}
\item
  \url{http://m.mlb.com/news/article/68514514/mlbam-introduces-new-way-to-analyze-every-play}
\item
  \url{http://www.fangraphs.com/library/offense/offensive-statistics-list/}
\item
  \url{http://en.wikipedia.org/wiki/Component_ERA}
\item
  \url{http://www.fangraphs.com/library/pitching/fip/}
\item
  \url{http://nomaas.org/2012/05/a-look-at-the-defense-the-yankees-d-stinks-edition/}
\item
  \url{http://en.wikipedia.org/wiki/Wins_Above_Replacement}
\item
  \url{http://www.fangraphs.com/library/misc/war/}
\item
  \url{http://www.baseball-reference.com/about/war_explained.shtml}
\item
  \url{http://www.baseball-reference.com/about/war_explained_comparison.shtml}
\item
  \url{http://www.baseball-reference.com/about/war_explained_position.shtml}
\item
  \url{http://www.baseball-reference.com/about/war_explained_pitch.shtml}
\item
  \url{http://www.fangraphs.com/leaders.aspx?pos=all\&stats=bat\&lg=all\&qual=y\&type=8\&season=2014\&month=0\&season1=1871\&ind=0}
\item
  \url{http://battingleadoff.com/2014/01/08/comparing-the-three-war-measures-part-ii/}
\item
  \url{http://battingleadoff.com/2014/01/08/comparing-the-three-war-measures-part-ii/}
\item
  \url{http://en.wikipedia.org/wiki/Coefficient_of_determination}
\item
  \url{http://www.sloansportsconference.com/wp-content/uploads/2014/02/2014_SSAC_Data-driven-Method-for-In-game-Decision-Making.pdf}
\item
  \url{https://courses.edx.org/courses/BUx/SABR101x/2T2014/courseware/10e616fc7649469ab4457ae18df92b20/}
\end{itemize}


\begin{itemize}
\item
  \url{http://vincegennaro.mlblogs.com/}
\item
  \url{https://www.youtube.com/watch?v=H-kx-x_d0Mk}
\item
  \url{http://www.sportvision.com/media/pitchfx-how-it-works}
\item
  \url{http://www.baseballprospectus.com/article.php?articleid=13109}
\item
  \url{http://baseball.physics.illinois.edu/FastPFXGuide.pdf}
\item
  \url{http://baseball.physics.illinois.edu/FieldFX-TDR-GregR.pdf}
\item
  \url{http://www.sportvision.com/baseball/fieldfx}
\item
  \url{http://regressing.deadspin.com/mlb-announces-revolutionary-new-fielding-tracking-syste-1534200504}
\item
  \url{http://grantland.com/the-triangle/mlb-advanced-media-play-tracking-bob-bowman-interview/}
\item
  \url{http://www.sportvision.com/baseball/hitfx}
\item
  \url{https://www.youtube.com/watch?v=YkjtnuNmK74}
\end{itemize}




\chpater{Technology for Big Data Applications and
Analytics}\label{technology-for-big-data-applications-and-analytics}

\FILENAME

We use the K-means Python code in SciPy package to show real code for
clustering. After a simple example we generate 4 clusters of distinct
centers and various choice for sizes using Matplotlib tor visualization.
We show results can sometimes be incorrect and sometimes make different
choices among comparable solutions. We discuss the `'hill'' between
different solutions and rationale for running K-means many times and
choosing best answer. Then we introduce MapReduce with the basic
architecture and a homely example. The discussion of advanced topics
includes an extension to Iterative MapReduce from Indiana University
called Twister and a generalized Map Collective model. Some measurements
of parallel performance are given. The SciPy K-means code is modified to
support a MapReduce execution style. This illustrates the key ideas of
mappers and reducers. With appropriate runtime this code would run in
parallel but here the \emph{parallel} maps run sequentially. This simple
2 map version can be generalized to scalable parallelism. Python is used
to Calculate PageRank from Web Linkage Matrix showing several different
formulations of the basic matrix equations to finding leading
eigenvector. The unit is concluded by a calculation of PageRank for
general web pages by extracting the secret from Google.

\section{Technologypi: K-means}\label{technologypi-k-means}

We use the K-means Python code in SciPy package to show real code for
clustering. After a simple example we generate 4 clusters of distinct
centers and various choice for sizes using Matplotlib tor visualization.
We show results can sometimes be incorrect and sometimes make different
choices among comparable solutions. We discuss the \emph{hill} between
different solutions and rationale for running K-means many times and
choosing best answer.

Files:

\sourcecode{K-means}{xmean.py}{examples/python/k-means/xmean.py}

\sourcecode{K-means}{sample.csv}{examples/python/k-means/sample.csv}

\sourcecode{K-means}{parallel-kmeans.py}{examples/python/k-means/parallel-kmeans.py}

\sourcecode{K-means}{kmeans-extra.py}{examples/python/k-means/kmeans-extra.py}


\subsection{K-means in Python}\label{k-means-in-python}

We use the K-means Python code in SciPy package to show real code for
clustering and applies it a set of 85 two dimensional vectors --
officially sets of weights and heights to be clustered to find T-shirt
sizes. We run through Python code with Matplotlib displays to divide
into 2-5 clusters. Then we discuss Python to generate 4 clusters of
varying sizes and centered at corners of a square in two dimensions. We
formally give the K means algorithm better than before and make
definition consistent with code in SciPy.

\subsection{Analysis of 4 Artificial
Clusters}\label{analysis-of-4-artificial-clusters}

We present clustering results on the artificial set of 1000 2D points
described in previous lesson for 3 choices of cluster sizes \emph{small}
\emph{large} and \emph{very large}. We emphasize the SciPy always does
20 independent K means and takes the best result -- an approach to
avoiding local minima. We allow this number of independent runs to be
changed and in particular set to 1 to generate more interesting erratic
results. We define changes in our new K means code that also has two
measures of quality allowed. The slides give many results of clustering
into 2 4 6 and 8 clusters (there were only 4 real clusters). We show
that the \emph{very small} case has two very different solutions when
clustered into two clusters and use this to discuss functions with
multiple minima and a hill between them. The lesson has both discussion
of already produced results in slides and interactive use of Python for
new runs.

\section{Technology: MapReduce}\label{technology-mapreduce}

We describe the basic architecture of MapReduce and a homely example.
The discussion of advanced topics includes extension to Iterative
MapReduce from Indiana University called Twister and a generalized Map
Collective model. Some measurements of parallel performance are given.

\subsection{Introduction}\label{introduction}

This introduction uses an analogy to making fruit punch by slicing and
blending fruit to illustrate MapReduce. The formal structure of
MapReduce and Iterative MapReduce is presented with parallel data
flowing from disks through multiple Map and Reduce phases to be
inspected by the user.

\subsection{Advanced Topics}\label{advanced-topics}

This defines 4 types of MapReduce and the Map Collective model of Qiu.
The Iterative MapReduce model from Indiana University called Twister is
described and a few performance measurements on Microsoft Azure are
presented.

\section{Technology: Kmeans and MapReduce
Parallelism}\label{technology-kmeans-and-mapreduce-parallelism}

We modify the SciPy K-means code to support a MapReduce execution style
and runs it in this short unit. This illustrates the key ideas of
mappers and reducers. With appropriate runtime this code would run in
parallel but here the \emph{parallel} maps run sequentially. We stress
that this simple 2 map version can be generalized to scalable
parallelism.

Files:

\sourcecode{K-means}{ParallelKmeans}{examples/python/k-means/parallel-kmeans.py}

\subsection{MapReduce Kmeans in
Python}\label{mapreduce-kmeans-in-python}

We modify the SciPy K-means code to support a MapReduce execution style
and runs it in this short unit. This illustrates the key ideas of
mappers and reducers. With appropriate runtime this code would run in
parallel but here the \emph{parallel} maps run sequentially. We stress
that this simple 2 map version can be generalized to scalable
parallelism.

\section{Technology: PageRank}\label{technology-pagerank}

We use Python to Calculate PageRank from Web Linkage Matrix showing
several different formulations of the basic matrix equations to finding
leading eigenvector. The unit is concluded by a calculation of PageRank
for general web pages by extracting the secret from Google.

Files:

\sourcecode{K-means}{pagerank1.py}{examples/python/page-rank/pagerank1.py}

\sourcecode{K-means}{pagerank2.py}{examples/python/page-rank/pagerank2.py}


\subsection{Calculate PageRank from Web Linkage
Matrix}\label{calculate-pagerank-from-web-linkage-matrix}

We take two simple matrices for 6 and 8 web sites respectively to
illustrate the calculation of PageRank.

\subsection{Calculate PageRank of a Real
Page}\label{calculate-pagerank-of-a-real-page}

This tiny lesson presents a Python code that finds the Page Rank that
Google calculates for any page on the web.



\chapter{Big Data Use Cases Survey}\label{big-data-use-cases-survey}

\FILENAME

This section covers 51 values of X and an overall study of Big data that
emerged from a NIST (National Institute for Standards and Technology)
study of Big data. The section covers the NIST Big Data Public Working
Group (NBD-PWG) Process and summarizes the work of five subgroups:
Definitions and Taxonomies Subgroup, Reference Architecture Subgroup,
Security and Privacy Subgroup, Technology Roadmap Subgroup and the
Requirements andUse Case Subgroup. 51 use cases collected in this
process are briefly discussed with a classification of the source of
parallelism and the high and low level computational structure. We
describe the key features of this classification.

\section{Overview of NIST Big Data Public Working Group (NBD-PWG)
Process and1
Results}\label{overview-of-nist-big-data-public-working-group-nbd-pwg-process-and-results}

This unit covers the NIST Big Data Public Working Group (NBD-PWG)
Process and summarizes the work of five subgroups: Definitions and
Taxonomies Subgroup, Reference Architecture Subgroup, Security and
Privacy Subgroup, Technology Roadmap Subgroup and the Requirements and
Use Case Subgroup. The work of latter is continued in next two units.


\slides{Usecases}{Overview}{45}{https://drive.google.com/open?id=0B8936_ytjfjmODIxNGttU1pveWc}


\subsection{Introduction to NIST Big Data Public Working Group
(NBD-PWG)
Process}\label{introduction-to-nist-big-data-public-working-group-nbd-pwg-process}

The focus of the (NBD-PWG) is to form a community of interest from
industry, academia, and government, with the goal of developing a
consensus definitions, taxonomies, secure reference architectures, and
technology roadmap. The aim is to create vendor-neutral, technology and
infrastructure agnostic deliverables to enable big data stakeholders to
pick-and-choose best analytics tools for their processing and
visualization requirements on the most suitable computing platforms and
clusters while allowing value-added from big data service providers and
flow of data between the stakeholders in a cohesive and secure manner.


\video{Usecases}{13:02}{Introduction}{https://www.youtube.com/watch?v=3oKdmuH0N3k} 



\subsection{Definitions and Taxonomies
Subgroup}\label{definitions-and-taxonomies-subgroup}

The focus is to gain a better understanding of the principles of Big
Data. It is important to develop a consensus-based common language and
vocabulary terms used in Big Data across stakeholders from industry,
academia, and government. In addition, it is also critical to identify
essential actors with roles and responsibility, and subdivide them into
components and sub-components on how they interact/ relate with each
other according to their similarities and differences.

For Definitions: Compile terms used from all stakeholders regarding the
meaning of Big Data from various standard bodies, domain applications,
and diversified operational environments. For Taxonomies: Identify key
actors with their roles and responsibilities from all stakeholders,
categorize them into components and subcomponents based on their
similarities and differences. In particular data Science and Big Data
terms are discussed.


\video{Usecases}{7:42}{Taxonomies}{https://www.youtube.com/watch?v=7eOtuBV8udo} 


\subsection{Reference Architecture
Subgroup}\label{reference-architecture-subgroup}

The focus is to form a community of interest from industry, academia,
and government, with the goal of developing a consensus-based approach
to orchestrate vendor-neutral, technology and infrastructure agnostic
for analytics tools and computing environments. The goal is to enable
Big Data stakeholders to pick-and-choose technology-agnostic analytics
tools for processing and visualization in any computing platform and
cluster while allowing value-added from Big Data service providers and
the flow of the data between the stakeholders in a cohesive and secure
manner. Results include a reference architecture with well defined
components and linkage as well as several exemplars.


\video{Usecases}{10:05}{Architecture}{https://www.youtube.com/watch?v=h4ylW0vztDw} 


\subsection{Security and Privacy
Subgroup}\label{security-and-privacy-subgroup}

The focus is to form a community of interest from industry, academia,
and government, with the goal of developing a consensus secure reference
architecture to handle security and privacy issues across all
stakeholders. This includes gaining an understanding of what standards
are available or under development, as well as identifies which key
organizations are working on these standards. The Top Ten Big Data
Security and Privacy Challenges from the CSA (Cloud Security Alliance)
BDWG are studied. Specialized use cases include Retail/Marketing, Modern
Day Consumerism, Nielsen Homescan, Web Traffic Analysis, Healthcare,
Health Information Exchange, Genetic Privacy, Pharma Clinical Trial Data
Sharing, Cyber-security, Government, Military and Education.


\video{Usecases}{9:51}{Security}{https://www.youtube.com/watch?v=dHrHk-GvruY} 


\subsection{Technology Roadmap
Subgroup}\label{technology-roadmap-subgroup}

The focus is to form a community of interest from industry, academia,
and government, with the goal of developing a consensus vision with
recommendations on how Big Data should move forward by performing a good
gap analysis through the materials gathered from all other NBD
subgroups. This includes setting standardization and adoption priorities
through an understanding of what standards are available or under
development as part of the recommendations. Tasks are gather input from
NBD subgroups and study the taxonomies for the actors' roles and
responsibility, use cases and requirements, and secure reference
architecture; gain understanding of what standards are available or
under development for Big Data; perform a thorough gap analysis and
document the findings; identify what possible barriers may delay or
prevent adoption of Big Data; and document vision and recommendations.


\video{Usecases}{4:14}{Technology}{https://www.youtube.com/watch?v=va0UCR5gMTA} 


\subsection{Interfaces subgroup}\label{interfaces-subgroup}

This subgroup is working on the following document: \emph{NIST Big Data
Interoperability Framework: Volume 8, Reference Architecture Interface}.

This document summarizes interfaces that are instrumental for the
interaction with Clouds, Containers, and HPC systems to manage virtual
clusters to support the NIST Big Data Reference Architecture (NBDRA).
The Representational State Transfer (REST) paradigm is used to define
these interfaces allowing easy integration and adoption by a wide
variety of frameworks. . This volume, Volume 8, uses the work performed
by the NBD-PWG to identify objects instrumental for the NIST Big Data
Reference Architecture (NBDRA) which is introduced in the NBDIF: Volume
6, Reference Architecture.

This presentation was given at the \emph{2nd NIST Big Data Public
Working Group (NBD-PWG) Workshop} in Washington DC in June 2017. It
explains our thoughts on deriving automatically a refernce architecture
form the Refernce Architecture Interface specifications directly from
the document.

The workshop Web page is located at

\begin{itemize}
\item
  \url{https://bigdatawg.nist.gov/workshop2.php}
\end{itemize}

The agenda of teh workshop is as follows:

\begin{itemize}
\item
  \url{https://bigdatawg.nist.gov/2017_NIST_Big_Data_PWG_WorkshopAgenda_with_Speakers_Bio.pdf}
\end{itemize}

The Web cas of the presentation is given bellow, while you need to fast
forward to a particular time

\begin{itemize}
\item
  Webcast: Interface subgroup:
  \url{https://www.nist.gov/news-events/events/2017/06/2nd-nist-big-data-public-working-group-nbd-pwg-workshop}

  \begin{itemize}
    \item
    see: Big Data Working Group Day 1, part 2 Time start: 21:00 min,
    Time end: 44:00
  \end{itemize}
\item
  Slides:
  \url{https://github.com/cloudmesh/cloudmesh.rest/blob/master/docs/NBDPWG-vol8.pptx?raw=true}
\item
  Document:
  \url{https://github.com/cloudmesh/cloudmesh.rest/raw/master/docs/NIST.SP.1500-8-draft.pdf}
\end{itemize}

You are welcome to view other presentations if you are interested.

\subsection{Requirements and Use Case Subgroup
Introduction}\label{requirements-and-use-case-subgroup-introduction}

The focus is to form a community of interest from industry, academia,
and government, with the goal of developing a consensus list of Big Data
requirements across all stakeholders. This includes gathering and
understanding various use cases from diversified application
domains.Tasks are gather use case input from all stakeholders; derive
Big Data requirements from each use case; analyze/prioritize a list of
challenging general requirements that may delay or prevent adoption of
Big Data deployment; develop a set of general patterns capturing the
`'essence'' of use cases (not done yet) and work with Reference
Architecture to validate requirements and reference architecture by
explicitly implementing some patterns based on use cases. The progress
of gathering use cases (discussed in next two units) and requirements
systemization are discussed.


\video{Usecases}{27:28}{Requirements}{https://www.youtube.com/watch?v=f_vxmx3CmMU} 


\section{51 Big Data Use Cases}\label{big-data-use-cases}

This units consists of one or more slides for each of the 51 use cases -
typically additional (more than one) slides are associated with
pictures. Each of the use cases is identified with source of parallelism
and the high and low level computational structure. As each new
classification topic is introduced we briefly discuss it but full
discussion of topics is given in following unit.


\slides{Usecases}{51}{100}{https://drive.google.com/open?id=0B8936_ytjfjmYUlKckhLSUQxMUk}


\subsection{Government Use Cases}\label{government-use-cases}

This covers Census 2010 and 2000 - Title 13 Big Data; National Archives
and Records Administration Accession NARA, Search, Retrieve,
Preservation; Statistical Survey Response Improvement (Adaptive Design)
and Non-Traditional Data in Statistical Survey Response Improvement
(Adaptive Design).


\video{Usecases}{17:43}{Government Use Cases}{https://www.youtube.com/watch?v=e0ks_BuYUVM} 


\subsection{Commercial Use Cases}\label{commercial-use-cases}

This covers Cloud Eco-System, for Financial Industries (Banking,
Securities \& Investments, Insurance) transacting business within the
United States; Mendeley - An International Network of Research; Netflix
Movie Service; Web Search; IaaS (Infrastructure as a Service) Big Data
Business Continuity \& Disaster Recovery (BC/DR) Within A Cloud
Eco-System; Cargo Shipping; Materials Data for Manufacturing and
Simulation driven Materials Genomics.


\video{Usecases}{17:43}{Commercial Use Cases}{https://www.youtube.com/watch?v=URy9u8_34ww} 


\subsection{Defense Use Cases}\label{defense-use-cases}

This covers Large Scale Geospatial Analysis and Visualization; Object
identification and tracking from Wide Area Large Format Imagery (WALF)
Imagery or Full Motion Video (FMV) - Persistent Surveillance and
Intelligence Data Processing and Analysis.


\video{Usecases}{15:43}{Defense Use Cases}{https://www.youtube.com/watch?v=FXFfE8zcco8} 


\subsection{Healthcare and Life Science Use
Cases}\label{healthcare-and-life-science-use-cases}

This covers Electronic Medical Record (EMR) Data; Pathology
Imaging/digital pathology; Computational Bioimaging; Genomic
Measurements; Comparative analysis for metagenomes and genomes;
Individualized Diabetes Management; Statistical Relational Artificial
Intelligence for Health Care; World Population Scale Epidemiological
Study; Social Contagion Modeling for Planning, Public Health and
Disaster Management and Biodiversity and LifeWatch.


\video{Usecases}{30:11}{Healthcare and Life  Science Use Cases}{https://www.youtube.com/watch?v=uGeYrXENlpU}


\subsection{Deep Learning and Social Networks Use
Cases}\label{deep-learning-and-social-networks-use-cases}

This covers Large-scale Deep Learning; Organizing large-scale,
unstructured collections of consumer photos; Truthy: Information
diffusion research from Twitter Data; Crowd Sourcing in the Humanities
as Source for Bigand Dynamic Data; CINET: Cyberinfrastructure for
Network (Graph) Science and Analytics and NIST Information Access
Division analytic technology performance measurement, evaluations, and
standards.


\video{Usecases}{14:19}{Deep Learning and Social Networks Use Cases}{https://www.youtube.com/watch?v=bdWyhT8bvE4}


\subsection{Research Ecosystem Use
Cases}\label{research-ecosystem-use-cases}

DataNet Federation Consortium DFC; The `Discinnet process', metadata
-big data global experiment; Semantic Graph-search on Scientific
Chemical and Text-based Data and Light source beamlines.


\video{Usecases}{9:09}{Research Ecosystem Use Cases}{https://www.youtube.com/watch?v=jjyv4RmMIUU} 


\subsection{Astronomy and Physics Use
Cases}\label{astronomy-and-physics-use-cases}

This covers Catalina Real-Time Transient Survey (CRTS): a digital,
panoramic, synoptic sky survey; DOE Extreme Data from Cosmological Sky
Survey and Simulations; Large Survey Data for Cosmology; Particle
Physics: Analysis of LHC Large Hadron Collider Data: Discovery of Higgs
particle and Belle II High Energy Physics Experiment.


\video{Usecases}{17:33}{Astronomy and Physics Use Cases}{https://www.youtube.com/watch?v=MPEe8yDVwAo}



\subsection{Environment, Earth and Polar Science Use
Cases}\label{environment-earth-and-polar-science-use-cases}

EISCAT 3D incoherent scatter radar system; ENVRI, Common Operations of
Environmental Research Infrastructure; Radar Data Analysis for CReSIS
Remote Sensing of Ice Sheets; UAVSAR Data Processing, DataProduct
Delivery, and Data Services; NASA LARC/GSFC iRODS Federation Testbed;
MERRA Analytic Services MERRA/AS; Atmospheric Turbulence - Event
Discovery and Predictive Analytics; Climate Studies using the Community
Earth System Model at DOE's NERSC center; DOE-BER Subsurface
Biogeochemistry Scientific Focus Area and DOE-BER AmeriFlux and FLUXNET
Networks.


\video{Usecases}{25:29}{Environment, Earth and Polar Science Use Cases}{https://www.youtube.com/watch?v=YJGk-uvaUCg} 



\subsection{Energy Use Case}\label{energy-use-case}

This covers Consumption forecasting in Smart Grids.


\video{Usecases}{4:01}{Energy Use Case}{https://www.youtube.com/watch?v=5y_O-a8_Fbg} 



\section{Features of 51 Big Data Use
Cases}\label{features-of-51-big-data-use-cases}

This unit discusses the categories used to classify the 51 use-cases.
These categories include concepts used for parallelism and low and high
level computational structure. The first lesson is an introduction to
all categories and the further lessons give details of particular
categories.


\slides{Usecases}{Features}{43}{https://drive.google.com/open?id=0B8936_ytjfjmREJTMHhjMktXRHc}


\subsection{Summary of Use Case Classification
I}\label{summary-of-use-case-classification-i}

This discusses concepts used for parallelism and low and high level
computational structure. Parallelism can be over People (users or
subjects), Decision makers; Items such as Images, EMR, Sequences;
observations, contents of online store; Sensors -- Internet of Things;
Events; (Complex) Nodes in a Graph; Simple nodes as in a learning
network; Tweets, Blogs, Documents, Web Pages etc.; Files or data to be
backed up, moved or assigned metadata; Particles/cells/mesh points. Low
level computational types include PP (Pleasingly Parallel); MR
(MapReduce); MRStat; MRIter (Iterative MapReduce); Graph; Fusion; MC
(Monte Carlo) and Streaming. High level computational types include
Classification; S/Q (Search and Query); Index; CF (Collaborative
Filtering); ML (Machine Learning); EGO (Large Scale Optimizations); EM
(Expectation maximization); GIS; HPC; Agents. Patterns include Classic
Database; NoSQL; Basic processing of data as in backup or metadata; GIS;
Host of Sensors processed on demand; Pleasingly parallel processing; HPC
assimilated with observational data; Agent-based models; Multi-modal
data fusion or Knowledge Management; Crowd Sourcing.


\video{Usecases}{23:39}{Summary of Use Case Classification}{https://www.youtube.com/watch?v=X0vEmbn1Ld8}



\subsection{Database(SQL) Use Case
Classification}\label{databasesql-use-case-classification}

This discusses classic (SQL) datbase approach to data handling with
Search\&Query and Index features. Comparisons are made to NoSQL
approaches.


\video{Usecases}{11:13}{Database (SQL) Use Case Classification}{https://www.youtube.com/watch?v=jIVdQID11Q4}


\subsection{NoSQL Use Case
Classification}\label{nosql-use-case-classification}

This discusses NoSQL (compared in previous lesson) with HDFS, Hadoop and
Hbase. The Apache Big data stack is introduced and further details of
comparison with SQL.


\video{Usecases}{11:20}{NoSQL Use Case Classification}{https://www.youtube.com/watch?v=uGL8cFPrhoE} 


\subsection{Use Case Classifications
I}\label{use-case-classifications-i}

This discusses a subset of use case features: GIS, Sensors. the support
of data analysis and fusion by streaming data between filters.


\video{Usecases}{12:42}{Use Case Classifications I}{https://www.youtube.com/watch?v=79IwNCNjVWU} 



\subsection{Use Case Classifications
II}\label{use-case-classifications-ii}

This discusses a subset of use case features: Pleasingly parallel,
MRStat, Data Assimilation, Crowd sourcing, Agents, data fusion and
agents, EGO and security.


\video{Usecases}{20:18}{Use Case Classifications II}{https://www.youtube.com/watch?v=b-olNbWCJyg} 



\subsection{Use Case Classifications
III}\label{use-case-classifications-iii}

This discusses a subset of use case features: Classification, Monte
Carlo, Streaming, PP, MR, MRStat, MRIter and HPC(MPI), global and local
analytics (machine learning), parallel computing, Expectation
Maximization, graphs and Collaborative Filtering.


\video{Usecases}{17:25}{Use Case Classifications III}{https://www.youtube.com/watch?v=ewqoFGxyQmc} 



\TODO{These resources have not all been checked to see if they still
  exist this is curretnly in progress}

\section{Resources}\label{resources}

\begin{itemize}
\item
  NIST Big Data Public Working Group (NBD-PWG) Process
  \url{https://www.nist.gov/el/cyber-physical-systems/big-data-pwg}
\item
  Big Data Definitions: \url{http://dx.doi.org/10.6028/NIST.SP.1500-1}
  (link is external)
\item
  Big Data Taxonomies: \url{http://dx.doi.org/10.6028/NIST.SP.1500-2}
  (link is external)
\item
  Big Data Use Cases and Requirements:
  \url{http://dx.doi.org/10.6028/NIST.SP.1500-3} (link is external)
\item
  Big Data Security and Privacy:
  \url{http://dx.doi.org/10.6028/NIST.SP.1500-4} (link is external)
\item
  Big Data Architecture White Paper Survey:
  \url{http://dx.doi.org/10.6028/NIST.SP.1500-5} (link is external)
\item
  Big Data Reference Architecture:
  \url{http://dx.doi.org/10.6028/NIST.SP.1500-6} (link is external)
\item
  Big Data Standards Roadmap:
  \url{http://dx.doi.org/10.6028/NIST.SP.1500-7} (link is external)
\end{itemize}

Some of the links bellow may be outdated. Please let us know the new
links and notify us of the outdated links.



\begin{itemize}
\item
  DCGSA Standard
  Cloud:~\url{https://www.youtube.com/watch?v=l4Qii7T8zeg}
\item
  On line 51 Use Cases \url{http://bigdatawg.nist.gov/usecases.php}
\item
  Summary of Requirements Subgroup
  \url{http://bigdatawg.nist.gov/_uploadfiles/M0245_v5_6066621242.docx}
\item
  Use Case 6 Mendeley
  \url{http://mendeley.com\%20http//dev.mendeley.com}
\item
  Use Case 7 Netflix
  \url{http://www.slideshare.net/xamat/building-largescale-realworld-recommender-systems-recsys2012-tutoria}
\item
  Use Case 8 Search
  \url{http://www.slideshare.net/kleinerperkins/kpcb-internet-trends-2013},
  \url{http://webcourse.cs.technion.ac.il/236621/Winter2011-2012/en/ho_Lectures.html},
  \url{http://www.ifis.cs.tu-bs.de/teaching/ss-11/irws},
  \url{http://www.slideshare.net/beechung/recommender-systems-tutorialpart1intro},
  \url{http://www.worldwidewebsize.com/}
\item
  Use Case 9 IaaS (Infrastructure as a Service) Big Data Business
  Continuity \& Disaster Recovery (BC/DR) Within A Cloud Eco-System
  provided by Cloud Service Providers (CSPs) and Cloud Brokerage Service
  Providers (CBSPs) \url{http://www.disasterrecovery.org/}
\item
  Use Case 11 and Use Case 12 Simulation driven Materials Genomics
  \url{https://www.materialsproject.org/}
\item
  Use Case 13 Large Scale Geospatial Analysis and Visualization
  \url{http://www.opengeospatial.org/standards},~
  \url{http://geojson.org/}~,
  \url{http://earth-info.nga.mil/publications/specs/printed/CADRG/cadrg.html}~
\item
  Use Case 14 Object identification and tracking from Wide Area Large
  Format Imagery (WALF) Imagery or Full Motion Video (FMV) - Persistent
  Surveillance
  \url{http://www.militaryaerospace.com/topics/m/video/79088650/persistent-surveillance-relies-on-extracting-relevant-data-points-and-connecting-the-dots.htm},
  \url{http://www.defencetalk.com/wide-area-persistent-surveillance-revolutionizes-tactical-isr-45745/}
\item
  Use Case 15 Intelligence Data Processing and Analysis
  \url{http://www.afcea-aberdeen.org/files/presentations/AFCEAAberdeen_DCGSA_COLWells_PS.pdf},
  \url{http://stids.c4i.gmu.edu/papers/STIDSPapers/STIDS2012/_T14/_SmithEtAl/_HorizontalIntegrationOfWarfighterIntel.pdf},
  \url{http://stids.c4i.gmu.edu/STIDS2011/papers/STIDS2011_CR_T1_SalmenEtAl.pdf},
  \url{https://www.youtube.com/watch?v=l4Qii7T8zeg},
  \url{http://dcgsa.apg.army.mil/}


\item
  Use Case 16 Electronic Medical Record (EMR) Data:
  \href{http://www.regenstrief.org/}{Regenstrief Institute},
  \href{http://loinc.org/}{Logical observation identifiers names and
  codes}, \href{http://www.ihie.org/}{Indiana Health Information
  Exchange},
  \href{http://www.iom.edu/Activities/Quality/LearningHealthcare.aspx}{Institute
  of Medicine Learning Healthcare System}
\item
  Use Case 17 Pathology Imaging/digital pathology;
  \url{https://web.cci.emory.edu/confluence/display/PAIS}~,~https://web.cci.emory.edu/confluence/display/HadoopGIS
\item
  Use Case 19 Genome in a Bottle Consortium:
  \href{https://bigdatacoursespring2015.appspot.com/www.genomeinabottle.org}{www.genomeinabottle.org}
\item
  Use Case 20 Comparative analysis for metagenomes and genomes
  \url{http://img.jgi.doe.gov/}
\item
  Use Case 25
  \href{https://www.biodiversitycatalogue.org/}{Biodiversity} and
  \href{http://www.lifewatch.eu/web/guest/home}{LifeWatch}
\item
  Use Case 26 Deep Learning: Recent popular press coverage of deep
  learning technology:
  \url{http://www.nytimes.com/2012/11/24/science/scientists-see-advances-in-deep-learning-a-part-of-artificial-intelligence.html}~,
  \url{http://www.nytimes.com/2012/06/26/technology/in-a-big-network-of-computers-evidence-of-machine-learning.html}~,
  \url{http://www.wired.com/2013/06/andrew_ng/},~

  A recent research paper on HPC for Deep Learning:
  \url{http://www.stanford.edu/~acoates/papers/CoatesHuvalWangWuNgCatanzaro_icml2013.pdf},
  Widely-used tutorials and references for Deep Learning:
  \url{http://ufldl.stanford.edu/wiki/index.php/Main_Page},
  \url{http://deeplearning.net/}
\item
  Use Case 27 Organizing large-scale, unstructured collections of
  consumer photos \url{http://vision.soic.indiana.edu/projects/disco/}
\item
  Use Case 28 Truthy: Information diffusion research from Twitter Data
  \url{http://truthy.indiana.edu/}~,~http://cnets.indiana.edu/groups/nan/truthy/~,~http://cnets.indiana.edu/groups/nan/despic/
\item
  Use Case 30 CINET: Cyberinfrastructure for Network (Graph) Science and
  Analytics \url{http://cinet.vbi.vt.edu/cinet_new/}
\item
  Use Case 31 NIST Information Access Division analytic technology
  performance measurement, evaluations, and standards
  \url{http://www.nist.gov/itl/iad/}
\item
  Use Case 32 DataNet Federation Consortium DFC:
  \href{http://datafed.org/}{The DataNet Federation Consortium},
  \href{http://irods.org/}{iRODS}
\item
  Use Case 33 The `Discinnet process', metadata \textless{} -
  \textgreater{} big data global experiment
  \url{http://www.discinnet.org/}
\item
  Use Case 34 Semantic Graph-search on Scientific Chemical and
  Text-based Data
  \url{http://www.eurekalert.org/pub_releases/2013-07/aiop-ffm071813.php}
 , \url{http://xpdb.nist.gov/chemblast/pdb.pl}
\item
  Use Case 35 Light source beamlines
  \url{http://www-als.lbl.gov/}~,~https://www1.aps.anl.gov/
\item
  Use Case 36 \href{http://crts.caltech.edu/}{CRTS survey},
  \href{http://www.lpl.arizona.edu/css/}{CSS survey} ; For an overview
  of the classification challenges, see, e.g.,
  \url{http://arxiv.org/abs/1209.1681}
\item
  Use Case 37 DOE Extreme Data from Cosmological Sky Survey and
  Simulations
  \url{http://www.lsst.org/lsst/}~,~http://www.nersc.gov/~,~http://www.nersc.gov/assets/Uploads/HabibcosmosimV2.pdf
\item
  Use Case 38 Large Survey Data for Cosmology \url{http://desi.lbl.gov/}
 , \url{http://www.darkenergysurvey.org/}

\item
  Use Case 39 Particle Physics: Analysis of LHC Large Hadron Collider
  Data: Discovery of Higgs particle
  \url{http://grids.ucs.indiana.edu/ptliupages/publications/Where\%20does\%20all\%20the\%20data\%20come\%20from\%20v7.pdf},
  \url{http://www.es.net/assets/pubs_presos/High-throughput-lessons-from-the-LHC-experience.Johnston.TNC2013.pdf}
\item
  Use Case 40 Belle II High Energy Physics Experiment
  \url{http://belle2.kek.jp/}
\item
  Use Case 41 EISCAT 3D incoherent scatter radar system
  \url{https://www.eiscat3d.se/}

\item
  Use Case 42 ENVRI, Common Operations of Environmental Research
  Infrastructure, 
  \href{http://envri.eu/}{ENVRI Project website},
  \href{http://confluence.envri.eu:8090/display/ERM/Start}{ENVRI  Reference Model},
  \href{http://confluence.envri.eu:8090/download/attachments/327687/D3.3\%20Analysis\%20of\%20Requirements\%20V1.0.pdf?version=1\&modificationDate=1366965933706\&api=v2}{ENVRI
    deliverable D3.2 : Analysis of common requirements of
    Environmental  Research Infrastructures}, 
\href{https://www.icos-ri.eu/}{ICOS},
  \href{http://www.euro-argo.eu/}{Euro-Argo},
  \href{https://www.eiscat3d.se/node}{EISCAT 3D},
  \href{http://www.lifewatch.com/}{LifeWatch},
  \href{http://www.epos-eu.org/}{EPOS},
  \href{http://www.emso-eu.org/}{EMSO}

\item
  Use Case 43 Radar Data Analysis for CReSIS Remote Sensing of Ice
  Sheets \url{https://www.cresis.ku.edu/}
\item
  Use Case 44 UAVSAR Data Processing, Data Product Delivery, and Data
  Services
  \url{http://uavsar.jpl.nasa.gov/}, \url{http://www.asf.alaska.edu/program/sdc}, \url{http://geo-gateway.org/main.html}
\item
  Use Case 47 Atmospheric Turbulence - Event Discovery and Predictive
  Analytics
  \url{http://oceanworld.tamu.edu/resources/oceanography-book/teleconnections.htm},
  \url{http://www.forbes.com/sites/toddwoody/2012/03/21/meet-the-scientists-mining-big-data-to-predict-the-weather/}
\item
  Use Case 48 Climate Studies using the Community Earth System Model at
  DOE's NERSC center
  \url{http://www-pcmdi.llnl.gov/}, 
  \url{http://www.nersc.gov/}, 
  \url{http://science.energy.gov/ber/research/cesd/}, 
  \url{http://www2.cisl.ucar.edu/}

\item
  Use Case 50 DOE-BER AmeriFlux and FLUXNET Networks
  \url{http://ameriflux.lbl.gov/},
  \url{http://www.fluxdata.org/default.aspx}
\item
  Use Case 51 Consumption forecasting in Smart Grids
  \url{http://smartgrid.usc.edu/},
  \url{http://ganges.usc.edu/wiki/Smart_Grid},

  \url{https://www.ladwp.com/ladwp/faces/ladwp/aboutus/a-power/a-p-smartgridla?_afrLoop=157401916661989\&_afrWindowMode=0\&_afrWindowId=null\#\%40\%3F_afrWindowId\%3Dnull\%26_afrLoop\%3D157401916661989\%26_afrWindowMode\%3D0\%26_adf.ctrl-state\%3Db7yulr4rl_17},
  \url{http://ieeexplore.ieee.org/xpl/articleDetails.jsp?arnumber=6475927}


\end{itemize}

\section{Web Search and Text Mining}\label{web-search-and-text-mining}

\FILENAME

This section starts with an overview of data mining and puts our study
of classification, clustering and exploration methods in context. We
examine the problem to be solved in web and text search and note the
relevance of history with libraries, catalogs and concordances. An
overview of web search is given describing the continued evolution of
search engines and the relation to the field of Information.

The importance of recall, precision and diversity is discussed. The
important Bag of Words model is introduced and both Boolean queries and
the more general fuzzy indices. The important vector space model and
revisiting the Cosine Similarity as a distance in this bag follows. The
basic TF-IDF approach is dis cussed. Relevance is discussed with a
probabilistic model while the distinction between Bayesian and frequency
views of probability distribution completes this unit.

We start with an overview of the different steps (data analytics) in web
search and then goes key steps in detail starting with document
preparation. An inverted index is described and then how it is prepared
for web search. The Boolean and Vector Space approach to query
processing follow. This is followed by Link Structure Analysis including
Hubs, Authorities and PageRank. The application of PageRank ideas as
reputation outside web search is covered. The web graph structure,
crawling it and issues in web advertising and search follow. The use of
clustering and topic models completes the section.

\subsection{Web Search and Text Mining I}\label{web-search-and-text-mining-i}

The unit starts with the web with its size, shape (coming from the
mutual linkage of pages by URL's) and universal power laws for number of
pages with particular number of URL's linking out or in to page.
Information retrieval is introduced and compared to web search. A
comparison is given between semantic searches as in databases and the
full text search that is base of Web search. The origin of web search in
libraries, catalogs and concordances is summarized. DIKW -- Data
Information Knowledge Wisdom -- model for web search is discussed. Then
features of documents, collections and the important Bag of Words
representation. Queries are presented in context of an Information
Retrieval architecture. The method of judging quality of results
including recall, precision and diversity is described. A time line for
evolution of search engines is given.

Boolean and Vector Space models for query including the cosine
similarity are introduced. Web Crawlers are discussed and then the steps
needed to analyze data from Web and produce a set of terms. Building and
accessing an inverted index is followed by the importance of term
specificity and how it is captured in TF-IDF. We note how frequencies
are converted into belief and relevance.

\slides{Web}{Web Search and Text Mining}{56}{https://drive.google.com/open?id=0B8936_ytjfjmeWVSYk9RVXcyOFk}

\subsection{Web and Document/Text Search: The
Problem}\label{web-and-documenttext-search-the-problem}

\video{Web}{9:56}{Text Mining}{https://www.youtube.com/watch?v=RFBeAWBkUsI}

This lesson starts with the web with its size, shape (coming from the
mutual linkage of pages by URL's) and universal power laws for number of
pages with particular number of URL's linking out or in to page.



\subsection{Information Retrieval leading to Web
Search}\label{information-retrieval-leading-to-web-search}

\video{Web}{6:06}{Information Retrival}{https://youtu.be/KtWhk2cdRa4}

Information retrieval is introduced A comparison is given between
semantic searches as in databases and the full text search that is base
of Web search. The ACM classification illustrates potential complexity
of ontologies. Some differences between web search and information
retrieval are given.



\subsection{History behind Web Search}\label{history-behind-web-search}

\video{Web}{5:48}{Web Search History}{https://youtu.be/J7D61uH5gVM}

The origin of web search in libraries, catalogs and concordances is
summarized.




\subsection{Key Fundamental Principles behind Web
Search}\label{key-fundamental-principles-behind-web-search}

\video{Web}{9:30}{Principles}{https://youtu.be/yPFi6xFnDHE}

This lesson describes the DIKW -- Data Information Knowledge Wisdom --
model for web search. Then it discusses documents, collections and the
important Bag of Words representation.



\subsection{Information Retrieval (Web Search)
Components}\label{information-retrieval-web-search-components}

\video{Web}{5:06}{Fundametal Principals of Web
  Search}{https://youtu.be/EGsnonXgb3Y}

This describes queries in context of an Information Retrieval
architecture. The method of judging quality of results including recall,
precision and diversity is described.



\subsection{Search Engines}\label{search-engines}

\video{Web}{3:08}{Search Engines}{https://youtu.be/kBV-99N6f7k}

This short lesson describes a time line for evolution of search engines.
The first web search approaches were directly built on Information
retrieval but in 1998 the field was changed when Google was founded and
showed the importance of URL structure as exemplified by PageRank.



\subsection{Boolean and Vector Space
Models}\label{boolean-and-vector-space-models}

\video{Web}{6:17}{Boolean and Vector Space
  Model}{https://youtu.be/JzGBA0OhsIk}

This lesson describes the Boolean and Vector Space models for query
including the cosine similarity.



\subsection{Web crawling and Document
Preparation}\label{web-crawling-and-document-preparation}

\video{Web}{4:55}{Web crawling and Document
  Preparation}{https://youtu.be/Wv-r-PJ9lro}

This describes a Web Crawler and then the steps needed to analyze data
from Web and produce a set of terms.



\subsection{Indices}\label{indices}

\video{Web}{5:44}{Indices}{https://youtu.be/NY2SmrHoBVM}

This lesson describes both building and accessing an inverted index. It
describes how phrases are treated and gives details of query structure
from some early logs.



\subsection{TF-IDF and Probabilistic
Models}\label{tf-idf-and-probabilistic-models}

\video{Web}{3:57}{TF-IDF and Probabilistic
  Models}{https://youtu.be/9P_HUmpselU}

It describes the importance of term specificity and how it is captured
in TF-IDF. It notes how frequencies are converted into belief and
relevance.

\TODO{These resources have not all been checked to see if they still
  exist this is curretnly in progress}

\subsection{Resources}\label{resources}

\begin{itemize}

\item
  \url{http://saedsayad.com/data_mining_map.htm}
\item
  \url{http://webcourse.cs.technion.ac.il/236621/Winter2011-2012/en/ho_Lectures.html}
\item
  The Web Graph: an
  Overviews://www.youtube.com/watch?v=yPFi6xFnDHE\&feature=youtu.be
  Jean-Loup Guillaume and Matthieu Latapy
  \url{https://hal.archives-ouvertes.fr/file/index/docid/54458/filename/webgraph.pdf}
\item
  Constructing a reliable Web graph with information on browsing
  behavior, Yiqun Liu, Yufei Xue, Danqing Xu, Rongwei Cen, Min Zhang,
  Shaoping Ma, Liyun Ru
  \url{http://www.sciencedirect.com/science/article/pii/S0167923612001844}
\item
  \url{http://www.ifis.cs.tu-bs.de/teaching/ss-11/irws}
\end{itemize}

\subsection{Web Search and Text Mining
II}\label{web-search-and-text-mining-ii}

\slides{Web}{Text
  Mining}{33}{https://drive.google.com/open?id=0B6wqDMIyK2P7YmpLbzQ0X2xpbDg}{PDF}

We start with an overview of the different steps (data analytics) in web
search. This is followed by Link Structure Analysis including Hubs,
Authorities and PageRank. The application of PageRank ideas as
reputation outside web search is covered. Issues in web advertising and
search follow. his leads to emerging field of computational advertising.
The use of clustering and topic models completes unit with Google News
as an example.



\subsection{Data Analytics for Web Search}\label{data-analytics-for-web-search}

\video{Web}{6:11}{Web Search and Text Mining
  II}{https://www.youtube.com/watch?v=kHEFxhWwhx0}

This short lesson describes the different steps needed in web search
including: Get the digital data (from web or from scanning); Crawl web;
Preprocess data to get searchable things (words, positions); Form
Inverted Index mapping words to documents; Rank relevance of documents
with potentially sophisticated techniques; and integrate technology to
support advertising and ways to allow or stop pages artificially
enhancing relevance.




\subsection{Link Structure Analysis including PageRank}\label{link-structure-analysis-including-pagerank}

\video{Web}{17:24}{Realated
  Applications}{https://www.youtube.com/watch?v=ApDu-7_1LYk}

The value of links and the concepts of Hubs and Authorities are
discussed. This leads to definition of PageRank with examples.
Extensions of PageRank viewed as a reputation are discussed with journal
rankings and university department rankings as examples. There are many
extension of these ideas which are not discussed here although topic
models are covered briefly in a later lesson.



\subsection{Web Advertising and
Search}\label{web-advertising-and-search}

\video{Web}{9:02}{Web Advertising and
  Search}{https://www.youtube.com/watch?v=375sY1YMk5U}

Internet and mobile advertising is growing fast and can be personalized
more than for traditional media. There are several advertising types
Sponsored search, Contextual ads, Display ads and different models: Cost
per viewing, cost per clicking and cost per action. This leads to
emerging field of computational advertising.




\subsection{Clustering and Topic Models}\label{clustering-and-topic-models}

\video{Web}{6:21}{Clustering and Topic
  Models}{https://youtu.be/95cHMyZ-TUs}

We discuss briefly approaches to defining groups of documents. We
illustrate this for Google News and give an example that this can give
different answers from word-based analyses. We mention some work at
Indiana University on a Latent Semantic Indexing model.

\TODO{These resources have not all been checked to see if they still
  exist this is curretnly in progress}

\subsection{Resources}\label{resources-1}

\begin{itemize}
\item
  \url{http://www.ifis.cs.tu-bs.de/teaching/ss-11/irws}
\item
  \url{https://en.wikipedia.org/wiki/PageRank}
\item
  \url{http://webcourse.cs.technion.ac.il/236621/Winter2011-2012/en/ho_Lectures.html}
\item
  Meeker/Wu May 29 2013 Internet Trends D11 Conference
  \url{http://www.slideshare.net/kleinerperkins/kpcb-internet-trends-2013}
\end{itemize}


%%----------------------------------------------------------------------------------------
%	PACKAGES AND OTHER DOCUMENT CONFIGURATIONS
%----------------------------------------------------------------------------------------

\documentclass[11pt,fleqn]{book} 
\usepackage[final]{pdfpages}

\newcommand{\TITLE}{Lecture Notes\\Introduction to Cloud Computing}
\newcommand{\SUBTITLE}{Theory and Practice}
%\newcommand{\AUTHOR}{Gregor von Laszewski}
%\newcommand{\EMAIL}{laszewski@gmail.com}

\newcommand{\AUTHOR}{~}
\newcommand{\EMAIL}{~}

\newtheorem{thm}{Theorem}
\usepackage{multirow}
\usepackage{fullpage}
\usepackage{graphicx}
\usepackage{amsthm}
\usepackage{amssymb}
\usepackage{url}
\usepackage{amsfonts}
\usepackage{algpseudocode}
\usepackage{mathtools}
\usepackage{graphicx}
\usepackage{listings}
\usepackage{color}
\graphicspath{{./}{./images}{./section/icloud/assignment/problems/project1}}

%----------------------------------------------------------------------------------------
%	TITLE PAGE
%----------------------------------------------------------------------------------------

\begingroup
\thispagestyle{empty}
\begin{tikzpicture}[remember picture,overlay]
\node[inner sep=0pt] (background) at (current page.center) {\includegraphics[width=\paperwidth]{background.png}};
\draw (current page.center) node [fill=blue!2!white,fill
opacity=0.9,text opacity=1,inner
sep=1cm]{\Huge\centering\bfseries\sffamily\parbox[c][][t]{\paperwidth}{\centering
    \TITLE \\[15pt] % Book title
    {\Large \SUBTITLE}\\[20pt] % Subtitle
    {\huge \AUTHOR \\ \Large \EMAIL} \\
    {\normalsize \TODAY} \\
    {\normalsize   \url{https://tinyurl.com/vonLaszewski-handbook} } \\
  }
}; % Author name
\end{tikzpicture}
\vfill
\endgroup

%----------------------------------------------------------------------------------------
%	COPYRIGHT PAGE
%----------------------------------------------------------------------------------------

\newpage
~\vfill
\thispagestyle{empty}

\noindent Copyright \copyright\ 2017\\
\AUTHOR \\

\noindent \EMAIL \\ % Copyright notice

% \noindent \textsc{Indiana University}\\ % Publisher

\noindent \url{https://github.com/cloudmesh/classes}\\

\noindent \url{https://tinyurl.com/vonLaszewski-handbook}\\


\begin{comment}
\noindent Licensed under the Creative Commons
Attribution-NonCommercial 3.0 Unported License (the ``License''). You
may not use this file except in compliance with the License. You may
obtain a copy of the License at
\url{http://creativecommons.org/licenses/by-nc/3.0}. Unless required
by applicable law or agreed to in writing, software distributed under
the License is distributed on an \textsc{``as is'' basis, without
  warranties or conditions of any kind}, either express or
implied. See the License for the specific language governing
permissions and limitations under the License.\\ % License information

\end{comment}

\noindent \textit{First printing by Gregor von Laszewski, October 2017} % Printing/edition date

%----------------------------------------------------------------------------------------
%	TABLE OF CONTENTS
%----------------------------------------------------------------------------------------

%\usechapterimagefalse % If you don't want to include a chapter image, use this to toggle images off - it can be enabled later with \usechapterimagetrue

\chapterimage{TOC.png} % Table of contents heading image

\pagestyle{empty} % No headers

\shorttableofcontents{Short Table Of Contents}{0}
\newpage
\tableofcontents % Print the table of contents itself

\cleardoublepage % Forces the first chapter to start on an odd page so it's on the right

\pagestyle{fancy} % Print headers again









%\section{Plagiarism}

In academic life it is important to understand and avoid plagiarism.
Organizations and universities will have policies in place do address
plagiarism. An example is provided for Indiana University
\cite{www-iu-plagiarism}. We quote:

\begin{quotation}
``Honesty requires that any ideas or materials taken from
another source for either written or oral use must be fully
acknowledged. Offering the work of someone else as one’s own is
plagiarism. The language or ideas thus taken from another may range
from isolated formulas, sentences, or paragraphs to entire articles
copied from books, periodicals, speeches, or the writings of other
students. The offering of materials assembled or collected by others
in the form of projects or collections without acknowledgment also is
considered plagiarism. Any student who fails to give credit for ideas
or materials taken from another source is guilty of plagiarism. 

(Faculty Council, May 2, 1961; University Faculty Council, March 11,
1975; Board of Trustees, July 11, 1975)''
\end{quotation}

Faculty members at Universitys are also bound by policies that mandate
reporting. At Indiana University the following policy applies (for a
complete policy see the Web page):

\begin{quotation}
``Should
the faculty member detect signs of plagiarism or cheating, it is his
or her most serious obligation to investigate these thoroughly, to
take appropriate action with respect to the grades of students, and
{\em in any event} to report the matter to the Dean for Student Services [or
equivalent administrator]. The necessity to report every case of
cheating, whether or not further action is desirable, arises
particularly because of the possibility that this is not the student’s
first offense, or that other offenses may follow it. Equity also
demands that a uniform reporting practice be enforced; otherwise, some
students will be penalized while others guilty of the same actions
will go free.

(Faculty Council, May 2, 1961)''
\end{quotation}

Naturally if a student has any questions about understanding
plagiarism the University can provide assistance. If a student is in
doubt and asks for help this is not considered at that time
plagiarism. As you can see from the previous policies, the faculty do
not have any choice but reporting it to the university administration.
Thus you must not hold them personally responsible as this is part of
the tasks they are required to do if they like it or not. Instead, it
is {\bf the responsibility of the authors of the
  document} to assure no plagiarism occurs. If you are a student of
a class that writes a paper or project report this naturally also all
applies to you. In addition, if you work in a team you need to assure
the entire team addresses plagiarism appropriately.

%

\chapter{Sentient Architecture}

\FILENAME

\begin{figure}
\centering
\includegraphics[width=\columnwidth]{images/sentient.jpeg}
\caption{Sentinent Architecture: Source: \url{https://nicolatriscott.files.wordpress.com/2016/03/caaqt-euyam-jpm.jpg}} 
\end{figure}

\section{Introduction}

What is it

\section{Existing Deployments}

list	existing deployments

\section{Impact}

why is it important

not only science, 

\section{Integration into Cloud Computing and Big Data}

how dos it for to cloud computing and big data (Gregor can do that)


\section{Development}
S Architecture in practice


\subsection{Snowwhite and the Seven$^{+1}$ Dwarfs}

The ISE department at Indiana University has obtained eight dendrites
that were assambled by a number of students so they can be used in
class and in research projects. These dendrites can be accessed in
Smith Research Center and allow students and faculty members to
experiment with them. They are bare dendrites and have no
electronics on them. Hence, you will need a hardware device to
interact with.

The reason we named them \textit{Snowwhite and the Seven$^{+1}$ Dwarfs}
is based on the fact that the dendrites are white, and they need to be
interact with  somone. Beacuse of the white color we name the controll
unit snowwhite. The dendrites that are interacting with it are called
dwarfs as this just fits to the name snowwhite. As we actually have 8
and not 7 we added $^{+1}$.


We do not recommend to directly attach the
wires to boards, as they will draw too much power and destroy the
boards. Instead you will need a relay that you controll that itself
controlls the dendrite These can be:

\begin{description}

\item[Arduino:] These boards very simple but provide relative good
  protection of the output links. The disadvantage is that its
  interface is in C.

\item[Teensy] Whatever is now on it TBD

\item[Raspberry Pi] We recommend to use a Raspberry Pi as it has a
  great operating system and is more suited for additional analysis of
  data within an Edge Computing network than the other two choices. It
  also allows you to use python which is clearly a pluss as most of
  the material presented here are in Python.

\end{description}


\begin{figure}
\centering
\includegraphics[width=\columnwidth]{images/snowwhite.jpg}
\caption{Snowwhite and the Seven$^{+1}$  Dwarfs}
\label{F:snowwhite}
\end{figure}


\subsection{Liddy Hall Installation}

architecture drawing


\section{Sentinent Cloudmesh}

	Gregor provides introduction to cloudmesh (probably just pointer to other section)

	Gregor provides introduction to cloudmesh.pi 

	Introduction to parallel programming in python (TBD)

	Towards the Cloudmesh Parallel S development environment


\paragraph{Snowwhite and the Seven$^{+1}$ Dwarfs}

Test environment (8 dendrites)

\paragraph{Liddy Hall}

Andreas dsecribes how we program them 

(Andreas provides)


than we use cloudmesh to interface with them, maybe we need to just
show how we integrate them into mqtt (this is snowhwite) and than we  
can progrem them from another pi

	Test environment Liddy hall 

\section{Alternative Boards}

\subsection{Mega 2560}

``The MEGA 2560 is designed for more complex projects. With 54 digital
I/O pins, 16 analog inputs and a larger space for your sketch it is
the recommended board for 3D printers and robotics projects. This
gives your projects plenty of room and opportunities.'' 
\url{https://store.arduino.cc/usa/arduino-mega-2560-rev3}
A nice case such as the one offered at Amazon will provide good
protection \url{https://www.amazon.com/Eleduino-Arduino-Mega-2560-Enclosure/dp/B016QE46RQ}


\begin{figure}
\centering
\includegraphics[width=0.25\columnwidth]{images/mega2560.jpg}
\caption{MEGA 2560}
\label{F:mega2560}
\end{figure}


\subsection{Programming with Teensy}

``The Teensy is a complete USB-based microcontroller development
system, in a very small footprint, capable of implementing many types
of projects. All programming is done via the USB port.'' Source: \url{https://www.pjrc.com/teensy/}

Current state of programming (whateverthey have)

Which version



\section{Exercises}

\begin{description}

\item[Sentinent.1] Build dendrite. In this excersise you will be
  building a dendrite that you can add to the available pool of dendrites.
\item[Sentinent.2.1] Develop cloudmesh sensor/actuary. In this excersise
  you will be developing an actuator ar sensor interface in object
  oriented programming methodology. You can see many examples in
  cloudmesh.py on github,com. You will pick a sensor you have access
  to and that is not already included in cloudmesh.pi. 
\item[Sentinent.2.2] If you do
  prefer using another board, the option may exist do develp an
  interface for the sensor or actuator for this device. If OO
  programming is not available for that board, a clean design based on
  functions must be provided. However we believe this is mor complex
  than using the Pi. 
\item[Sentinent.3.1] Develop an mqtt based event publisher and
  subscription service. Use first LEDs to test your service before you
  hook up relays and dendrites.
\item[Sentinent.3.2] Hook up the dendrites to mqtt and controll them
\item[Sentinent.4] Develop sensors that interact with the dendrites
\item[Sentient.5] Explore the Page at
  \url{https://www.intorobotics.com/alternative-arduino-boards/} that
  lists a number of PI/Arduino alteranative boards provide a non
  plagiarized table for this chapter and evaluate which could be
  viable alternatives. If you have one of them we like you to provide
  a documentation on how to integrate them with the dendrites.

\end{description}

 

%%----------------------------------------------------------------------------------------
%	PART
%----------------------------------------------------------------------------------------
\chapterimage{cloud.jpeg} % Chapter heading image


\part{Container}

\FILENAME

%----------------------------------------------------------------------------------------
%	CHAPTERS
%----------------------------------------------------------------------------------------


\chapter{REST}

\section{Overview}

This section is accompanied by a video about REST.

\video{REST}{36:02}{REST}{https://youtu.be/xjFuA6q5N_U}

REST stands for {\bf RE}presentational {\bf S}tate {\bf
  T}ransfer. REST is an architecture style for designing networked
applications. It is based on stateless, client-server, cacheable
communications protocol. Although not based on http, in most cases,
the HTTP protocol is used. In contrast to what some others write or
say, REST is not a \emph{standard}. RESTful applications use HTTP
requests to (a) post data while creating and/or updating it, (b) read
data while making queries, and (c) delete data.

\url{https://www.ics.uci.edu/~fielding/pubs/dissertation/top.htm}

Hence REST uses HTTP for the four CRUD operations:

\begin{itemize}
\item  {\bf C}reate 
\item  {\bf R}ead
\item  {\bf U}pdate
\item  {\bf D}elete
\end{itemize}

As part of the HTTP protocol we have methods such as GET, PUT, POST, and
DELETE. These methods can than be used to implement a REST service. As
REST introduces collections and items we need to implement the CRUD
functions for them. The semantics is explained in the Table
illustrating how to implement them with HTTP methods.

\begin{description}

\item[http://.../resources/] ~\\

    \begin{description} 
    \item[GET] List the URIs and perhaps other details of the
      collection’s members
    \item[PUT] Replace the entire collection with another collection.
    \item[POST] Create a new entry in the collection. The new entry’s
      URI is assigned automatically and is usually returned by the
      operation.
    \item[DELETE] Delete the entire collection.
\end{description} 

\item[http://.../resources/item42] ~\\

    \begin{description} 

    \item[GET] Retrieve a representation of the addressed member of
      the collection, expressed in an appropriate Internet media type.

    \item[PUST] Replace the addressed member of the collection, or if
      it does not exist, create it.

    \item[POST] Not generally used. Treat the addressed member as a
      collection in its own right and create a new entry within it
                               
\item[DELETE] Delete the addressed member of the collection. 
\end{description} 
\end{description}

Source:
\URL{https://en.wikipedia.org/wiki/Representational_state_transfer}


Due to the structure that REST provides a number of tools have been
created that manage the creation of the specification for rest
services and their programming. We distinguish several different
categories:

\begin{description}
\item[REST programming language support:] These tools and services are
targeting a particular programming language. Such tools include Eve
which we will explore in more detail.

\item[REST documentation based tools:] These tools are primarily
  focussing on documenting REST specifications. Such tools include Swagger.

\item[REST design support tools:] These tools are used to support the
  design process of developing REST services while abstracting on top
  of the programming languages and define reusable specifications that
  can be used to create clients and servers for particular
  technology targets. Such tools include also swagger as additional
  tools are available that can generate code from swagger specifications
\end{description}

\section{Eve}

Next, we will focus on how to make a RESTful web service with Python
Eve. Eve makes the creation of a REST implementation in python easy. 


\URL{http://python-eve.org/}

\begin{WARNING}

Although we do recommend Ubuntu 17.04, at this time there is a bug
that forces us to use 16.04. Furthermore, we require you to follow the
instructions on how to install pyenv and use it to set up your python
environment. We recommend that you use either python 2.7.14 or 3.6.4.
We do not recommend you to use anaconda as it is not suited for cloud
computing but targets desktop computing. If you use pyenv you also
avoid the issue of interfering with your system wide python
install. We do recommend pyenv regardless if you use a virtual machine
or are working directly on your operating system. After you have set
up a proper python environment, make sure you have the newest version
of pip installed with 

\smallskip

\begin{lstlisting}
pip install pip -U
\end{lstlisting}
\end{WARNING}

To install Eve, you can say 

\begin{lstlisting}
  $ pip install eve
\end{lstlisting}

As Eve also needs a backend database and as MongoDB is an obvious
choice for this, we will also install MongoDB.  MongoDB is a Non-SQL
database which helps to store light weight data easily.

\subsection{Ubuntu}
On Ubuntu you can install MongoDB as follows

\begin{lstlisting}
$ sudo apt-key adv --keyserver hkp://keyserver.ubuntu.com:80 --recv 2930ADAE8CAF5059EE73BB4B58712A2291FA4AD5
$ echo "deb [ arch=amd64,arm64 ] https://repo.mongodb.org/apt/ubuntu xenial/
mongodb-org/3.6 multiverse" | sudo tee /etc/apt/sources.list.d/mongodb-org-3.6.list
$ sudo apt-get update
$ sudo apt-get install -y mongodb-org
\end{lstlisting}
% $

\subsection{OSX}

On OSX you can use the command

\begin{lstlisting}
brew update.
brew install mongodb
\end{lstlisting}

After downloading Mongo, create the {\em db} directory. This is where
the Mongo data files will live. You can create the directory in the
default location and assure it has the right permissions. Make sure
that the /data/db directory has the right permissions by running

\subsection{Verification}

In order to check the MongoDB installation, please run the following
commands in one terminal:

\begin{lstlisting}
$ mkdir -p ~/data/db
$ mongod --dbpath ~/data/db
\end{lstlisting}

In another terminal we try to connect to mongo and issue a mongo
command to show the databases:

\begin{lstlisting}
$ mongo --host 127.0.0.1:27017
$ show databases
\end{lstlisting}

If they execute without errors, you have successfully installed
MongoDB. In order to stop the running database instance run the
following command. simply CTRL-C the running mongod process

\subsection{Building a simple REST Service}

In this section we will focus on creating a simple rest service. To
organize our work we will create the following directory:


\begin{lstlisting}
$ mkdir ~/cloudmesh/eve
$ cd ~/cloudmesh/eve
\end{lstlisting}

As Eve needs a configuration and it is read in by default from the
file \verb|settings.py| we place the following content in the file
\verb|~/cloudmesh/eve/settings.py|:

\begin{lstlisting}
MONGO_HOST = 'localhost'
MONGO_PORT = 27017
MONGO_DBNAME = 'student_db'
DOMAIN = {
    'student': {
        'schema': {
            'firstname': {
                'type': 'string'
            },
            'lastname': {
                'type': 'string'
            },
            'school': {
                'type': 'string'
            },
            'university': {
                'type': 'string'
            },
            'email': {
                'type': 'string',
                 'unique': True
            }
        }
    }
}
RESOURCE_METHODS = ['GET', 'POST']
\end{lstlisting}

The DOMAIN object specifies the format of a \verb|student| object that
we are using as part of our REST service.  In addition we can specify
\verb|RESOURCE_METHODS| which methods are activated for the REST
service. This way the developer can restrict the available methods for
a REST service. To pass along the specification for mongoDB, we simply
specify the hostname, the port, as well as the database name.




Thus our object will be

Initially we need to create the python application to host the REST
API and we will use CURL to do the GET and POST requests in this
tutorial.

\subsection{Creating REST Service}

Create a file called \textbf{run.py} in the same directory where you
placed the \textbf{settings.py} and add the following code.

\begin{lstlisting}
from eve import Eve
app = Eve()

if __name__ == '__main__':
    app.run()
\end{lstlisting}

This is the minimal application which can host a REST
application. Let's start our database server and the REST service.

\subsection{Running REST Service}

Let's say we'll use terminal 1 to run the python REST API.

To start the rest service :
\begin{lstlisting}
$python run.py
\end{lstlisting}

\subsection{Running Database Server}

Next in terminal 2, run the following command to start the database server
\begin{lstlisting}
$ mongod --dbpath=<BASE-PATH>/data/db/
\end{lstlisting}

For example :
BASE-PATH can be replaced by any path you want. 
\begin{lstlisting}
$ mongod --dbpath=/home/john/e222/labs/lab1/data/db/
\end{lstlisting}

\subsection{Saving Data Using Curl}

We can use the commandline to save the data in the database using the
REST api. Here we use the endpoint we created as student in the schema
as the entry point to the rest service. The data is being saved using
JSON format.

Run the following command in the terminal 3, this will save data in
the database server.

\begin{lstlisting}
$ curl -H "Content-Type: application/json" -X POST -d 
'{"firtname":"John","lastname":"Doe", "school" : "ISE",
"university":"IUB", "email":"johndoe@iu.edu"}' http://127.0.0.1:5000/student/
\end{lstlisting}

\newline

In order to check the database please run the following commands in Terminal 4,

\begin{lstlisting}
Run mongo client
$ mongo
Query the databse list
$>show databases; (will display eve)  
Select database
$use eve  
$ show tables # query the table names
$ db.student.find().pretty()  # pretty will show the json in a clear way
\end{lstlisting}

\newline

Here we specify the header using \textbf{H} tag saying we need the
data to be saved using json format. And \textbf{X} tag says the HTTP
protocol and here we use POST method. And the tag \textbf{d} specifies
the data and make sure you use json format to enter the data. Finally,
the REST api endpoint to which we must save data. This allows us to
save the data in a table called \textbf{student} in MongoDB within a
database called \textbf{eve}.

\subsection{Retrieving Data Using Curl}

Run the following command to retrieve the saved data :

\begin{lstlisting}
$ curl http://127.0.0.1:5000/student?firstname=John
\end{lstlisting}



\newline

Here you can use any of the attributes in the saved object to call the
value. If you just used the endpoint itself without specifying the
attribute name, it will load all the saved objects.

\section{Exercises}
\subsection{Payloads}
For this section we will ensure that the proper payloads are received
upon a GET and DELETE request. For this to work ensure that you have a
MongoDB up and running.
\begin{lstlisting}
$ sudo service mongod start
\end{lstlisting}

We will now create the proper script to launch the API and an
appropriate configuration file. \\ The configuration file
(settings.py) should have the following line.
\begin{lstlisting}
DOMAIN = {'people': {}}
\end{lstlisting}

In the same directory as the the configuration file you need to create
a launch script i.e. run.py with the following lines.

\begin{lstlisting}
from eve import Eve
app = Eve()

if __name__ == '__main__':
    app.run()
\end{lstlisting}

Now that you have an instance of MongoDB running and the two newly
created .py files located in the same directory we can launch the
API. In a new terminal navigate to the directory containing the
configuration and launch files and execute the launch script. Again
make sure this is in a seperate terminal so you can see what's
happening as you make requests.

\begin{lstlisting}
$ python run.py
* Running on http://127.0.0.1:5000/
\end{lstlisting}

Now consume the API with the following command and inspect the payload
and identify what request was made.

\begin{lstlisting}
$ curl -i http://127.0.0.1:5000
\end{lstlisting}

Your payload should look like the one below, if your output isn't
formatted like below try adding ?pretty=1 (curl -i
http://127.0.0.1:5000?pretty=1) to the end of the curl command.
\begin{lstlisting}
HTTP/1.0 200 OK
Content-Type: application/json
Content-Length: 150
Server: Eve/0.7.6 Werkzeug/0.11.15 Python/2.7.5
Date: Wed, 17 Jan 2018 18:34:07 GMT

{
    "_links": {
        "child": [
            {
                "href": "people",
                "title": "people"
            }
        ]
    }
\end{lstlisting}

Remember that the API entry points adhere to the HATEOAS principle,
thus provide information regarding the resources accessible through
the API. In this case how many child resources are available through
our API?

\vspace{5mm}

If you said (1) you're correct! It's the one resource we defined in
the configuration file 'people'. Now lets try to request people.

\begin{lstlisting}
$ curl -i http://127.0.0.1:5000/people
\end{lstlisting}

How does the payload change? What does the \_links section describe?
Notice the difference between the payload when people is requested.

\begin{lstlisting}
{
    "_items": [],
    "_links": {
        "self": {
            "href": "people",
            "title": "people"
        },
        "parent": {
            "href": "/",
            "title": "home"
        }
    },
    "_meta": {
        "max_results": 25,
        "total": 0,
        "page": 1
    }
}
\end{lstlisting} 


\begin{exercise}
Write a RESTful service to determine a useful piece of information off
of your computer i.e. disk space, memory, RAM, etc.
\end{exercise}




\section{Object Management with Eve and Evegenie}

\url{http://python-eve.org/}

Eve makes the creation of a REST implementation in python easy.  We
will provide you with an implementation example that showcases that we
can create REST services without writing a single line of code. The
code for this is located at \url{https://github.com/cloudmesh/rest}

This code will have a master branch but will also have a dev branch in
which we will add gradually more objects. Objects in the dev branch will
include:

\begin{itemize}
\item
 virtual directories
\item
 virtual clusters
\item
 job sequences
\item
 inventories
\end{itemize}

You may want to check our active development work in the dev branch.
However for the purpose of this class the master branch will be
sufficient.

\subsection{Installation}\label{installation}

First we have to install mongodb. The installation will depend on your
operating system. For the use of the rest service it is not important to
integrate mongodb into the system upon reboot, which is focus of many
online documents. However, for us it is better if we can start and stop
the services explicitly for now.

On ubuntu, you need to do the following steps:

\begin{lstlisting}
TO BE CONTRIBUTED BY THE STUDENTS OF THE CLASS AS HOMEWORK
\end{lstlisting}

On windows 10, you need to do the following steps:

\begin{lstlisting}
TO BE CONTRIBUTED BY THE STUDENTS OF THE CLASS AS HOMEWORK. If you
elect Windows 10. YOu could be using the online documentation
provided by starting it on Windows, or running it in a docker container.
\end{lstlisting}

On OSX you can use home-brew and install it with:

\begin{lstlisting}
brew update
brew install mongodb
\end{lstlisting}

In future we may want to add ssl authentication in which case you
may
need to install it as follows:

\begin{lstlisting}
brew install mongodb --with-openssl
\end{lstlisting}

\subsection{Starting the service}\label{starting-the-service}

We have provided a convenient Makefile that currently only works for
OSX. It will be easy for you to adapt it to Linux. Certainly you can
look at the targes in the makefile and replicate them one by one.
Important targets are deploy and test.

When using the makefile you can start the services with:

\begin{lstlisting}
make deploy
\end{lstlisting}

IT will start two terminals. IN one you will see the mongo service, in
the other you will see the eve service. The eve service will take a file
called sample.settings.py that is base on sample.json for the start of
the eve service. The mongo servide is configured in sucj a way that it
only accepts incoming connections from the local host which will be
sufficient for our case. The mongo data is written into the
\verb|$USER/.cloudmesh| directory, so make sure it exists.

To test the services you can say:

\begin{lstlisting}
make test
\end{lstlisting}

YOu will se a number of json text been written to the screen.

\section{Creating your own objects}\label{creating-your-own-objects}

The example demonstrated how easy it is to create a mongodb and an eve
rest service. Now lets use this example to create your own. FOr this we
have modified a tool called evegenie to install it onto your system.

The original documentation for evegenie is located at:

\begin{itemize}
\tightlist
\item
 \url{http://evegenie.readthedocs.io/en/latest/}
\end{itemize}

However, we have improved evegenie while providing a commandline tool
based on it. The improved code is located at:

\begin{itemize}
\tightlist
\item
 \url{https://github.com/cloudmesh/evegenie}
\end{itemize}

You clone it and install on your system as follows:

\begin{lstlisting}
cd ~/github
git clone https://github.com/cloudmesh/evegenie
cd evegenie
python setup.py install
pip install .
\end{lstlisting}

This should install in your system evegenie. YOu can verify this by
typing:

\begin{lstlisting}
which evegenie
\end{lstlisting}

If you see the path evegenie is installed. With evegenie installed its
usaage is simple:

\begin{lstlisting}
$ evegenie

Usage:
  evegenie --help
  evegenie FILENAME
\end{lstlisting}

It takes a json file as input and writes out a settings file for the use
in eve. Lets assume the file is called sample.json, than the settings
file will be called sample.settings.py. Having the evegenie programm
will allow us to generate the settings files easily. You can include
them into your project and leverage the Makefile targets to start the
services in your project. In case you generate new objects, make sure
you rerun evegenie, kill all previous windows in which you run eve and
mongo and restart. In case of changes to objects that you have designed
and run previously, you need to also delete the mongod database.

\section{Towards cmd5 extensions to manage eve and
mongo}\label{towards-cmd5-extensions-to-manage-eve-and-mongo}

Naturally it is of advantage to have in cms administration commands to
manage mongo and eve from cmd instead of targets in the Makefile. Hence,
we \textbf{propose} that the class develops such an extension. We will
create in the repository the extension called admin and hobe that
students through collaborative work and pull requests complete such an
admin command.

The proposed command is located at:

 \URL{https://github.com/cloudmesh/rest/blob/master/cloudmesh/ext/command/admin.py}


It will be up to the class to implement such a command. Please
coordinate with each other.

The implementation based on what we provided in the Make file seems
straight forward. A great extension is to load the objects definitions
or eve e.g. settings.py not from the class, but forma place in
.cloudmesh. I propose to place the file at:

\begin{lstlisting}
.cloudmesh/db/settings.py
\end{lstlisting}

the location of this file is used when the Service class is initialized
with None. Prior to starting the service the file needs to be copied
there. This could be achieved with a set command.

\section{Responses}

\URL{https://dzone.com/refcardz/rest-foundations-restful}


\begin{comment}


\begin{lstlisting}
Code  Description

200 OK. The request has successfully executed. Response depends upon the verb invoked.
201 Created. The request has successfully executed and a new resource has been created in the process. The response body is either empty or contains a representation containing URIs for the resource created. The Location header in the response should point to the URI as well.
202 Accepted. The request was valid and has been accepted but has not yet been processed. The response should include a URI to poll for status updates on the request. This allows asynchronous REST requests
204 No Content. The request was successfully processed but the server did not have any response. The client should not update its display.
Table 1 - Successful Client Requests


Redirected Client Requests

CODE  DESCRIPTION
301 Moved Permanently. The requested resource is no longer located at the specified URL. The new Location should be returned in the response header. Only GET or HEAD requests should redirect to the new location. The client should update its bookmark if possible.
302 Found. The requested resource has temporarily been found somewhere else. The temporary Location should be returned in the response header. Only GET or HEAD requests should redirect to the new location. The client need not update its bookmark as the resource may return to this URL.
303 See Other. This response code has been reinterpreted by the W3C Technical Architecture Group (TAG) as a way of responding to a valid request for a non-network addressable resource. This is an important concept in the Semantic Web when we give URIs to people, concepts, organizations, etc. There is a distinction between resources that can be found on the Web and those that cannot. Clients can tell this difference if they get a 303 instead of 200. The redirected location will be reflected in the Location header of the response. This header will contain a reference to a document about the resource or perhaps some metadata about it.




Invalid Client Requests

Code  Description
405 Method Not Allowed.
406 Not Acceptable.
410 Gone.
411 Length Required.
412 Precondition Failed.
413 Entity Too Large.
414 URI Too Long.
415 Unsupported Media Type.
417 Expectation Failed.


Code  Description
500 Internal Server Error.
501 Not Implemented.
503 Service Unavailable.

\end{lstlisting}

\end{comment}



\section{Swagger}

Swagger \url{https://swagger.io/} is a tool for developing API
specifications based on the OpenAPI Specification (OAS). It allows not
only the specification, but the generation of code based on the
specification in a variety of languages.

Swagger itself has a number of tools which together build a framework
for developing REST services for a variety of languages.


\section{Swagger Tools}

The major Swagger tools of interest are:

\begin{description}

\item[Swagger Core] includes libraries for working with Swagger
 specifications \url{https://github.com/swagger-api/swagger-core}.

\item[Swagger Codegen] allows to generate code from the specifications
 to develop Client SDKs, servers, and documentation. \url{https://github.com/swagger-api/swagger-codegen}

\item[Swagger UI] is an HTML5 based UI for exploring and interacting
 with the specified APIs \url{https://github.com/swagger-api/swagger-ui}

\item[Swagger Editor] is a Web-browser based editor for composing 
 specifications using YAML \url{https://github.com/swagger-api/swagger-editor}

\end{description}

The developed APIs can be hosted and further developed on an
online repository named SwaggerHub \url{https://app.swaggerhub.com/home}
The convenient online editor is available which also can be installed
locally on a variety of operating systems including OSX, Linux, and
Windows. 


\section{FlaskRESTful}

Flask-RESTful provides a framework for developing REST APIs on top of
Flask. The abstraction introduce tha ability to use traditional existing
ORM libraries. The goal is to enable a minimal setup and allow
developers to leverage the familiarity they have with Flask and
develop Flask-RESTful services quickly.

\URL{https://flask-restful.readthedocs.io/en/latest/}

\section{Django REST Framework}

\URL{http://www.django-rest-framework.org/}

Django REST framework is a large toolkit to develop Web APIs. The
developers of the framework provide the following reasons for using it:

``(1) The Web browsable API is a huge usability win for your
developers.  (2) Authentication policies including packages for
OAuth1a and OAuth2.  (3) Serialization that supports both ORM and
non-ORM data sources.  (4) Customizable all the way down - just use
regular function-based views if you don't need the more powerful
features.  (5) Extensive documentation, and great community support.
(6) Used and trusted by internationally recognised companies including
Mozilla, Red Hat, Heroku, and Eventbrite.''


\chapter{Container}

\FILENAME

\section{Kubernetes}

\FIGURE{htb} 
    {1.0}
    {images/kubernetes.png}
    {Kubernetes (Source: Google)}
    {F:tas-arch} 



\chapter{Run Docker Locally on your Machine}\label{S:docker-local}

\FILENAME

\section{Installing Docker Community Edition}\label{installing-docker-community-edition}

To install docker on your computer, please visit the page:

\URL{https://www.docker.com/community-edition}{Docker Community Edition}

Here you will find a variety of packages, one of which will hopefully
suitable for your OS. The supported operating systems currently include:

\begin{itemize}
\item  OSX, Windows, Centos, Debian, Fedora, Ubuntu, AWS, Azure
\end{itemize}

Please chose the one most suitable for you.

\subsection{Instalation for OSX}\label{instalation-for-osx}

The docker community edition for OSX can be found at the following link

\HREF{https://store.docker.com/editions/community/docker-ce-desktop-mac?tab=description}{Information for OSX}

We recommend that at this time you get the version {\em Docker CE for MAC (stable)}

\URL{https://download.docker.com/mac/stable/Docker.dmg}

CLicking on the link will download a dmg file to your machine, that
you than will need to install by double clicking and allowing access
to the dmg file. Upon instalation a \texttt{whale} in the top status
bar shows that Docker is running, and you can acess it via a terminal.

\begin{figure}[htb]
\centering
\includegraphics[width=0.5\textwidth]{whale-in-menu-bar.png}
\caption{Docker integrated in the menu bar on OSX}
\end{figure}

\section{Testing if the install works}\label{testing-if-the-install-works}

To test if it works execute the following commands in a terminal:

\begin{verbatim}
docker version
\end{verbatim}

You should see an output similar to

\begin{verbatim}
docker version

Client:
  Version:      17.03.1-ce
  API version:  1.27
  Go version:   go1.7.5
  Git commit:   c6d412e
  Built:        Tue Mar 28 00:40:02 2017
  OS/Arch:      darwin/amd64

Server:
  Version:      17.03.1-ce
  API version:  1.27 (minimum version 1.12)
  Go version:   go1.7.5
  Git commit:   c6d412e
  Built:        Fri Mar 24 00:00:50 2017
  OS/Arch:      linux/amd64
  Experimental: true
\end{verbatim}

To see if you can run a container use

\begin{verbatim}
docker run hello-world
\end{verbatim}

Once executed you should see an outout similar to

\begin{verbatim}
Unable to find image 'hello-world:latest' locally
latest: Pulling from library/hello-world
78445dd45222: Pull complete 
Digest: sha256:c5515758d4c5e1e838e9cd307f6c6a .....
Status: Downloaded newer image for hello-world:latest

Hello from Docker!
This message shows that your installation appears to 
be working correctly.

To generate this message, Docker took the following steps:
1. The Docker client contacted the Docker daemon.
2. The Docker daemon pulled the "hello-world" image 
   from the Docker Hub.
3. The Docker daemon created a new container from that 
   image which runs the executable that produces the 
   output you are currently reading.
4. The Docker daemon streamed that output to the Docker 
   client, which sent it to your terminal.

To try something more ambitious, you can run an Ubuntu container 
with:

$ docker run -it ubuntu bash

Share images, automate workflows, and more with a free Docker ID:
https://cloud.docker.com/

For more examples and ideas, visit:
https://docs.docker.com/engine/userguide/
\end{verbatim}

% chktex-file 1
% chktex-file 8
% chktex-file 11
% chktex-file 32
% chktex-file 18
% chktex-file 25
% chktex-file 36

\chapter{Running Docker on FutureSystems}\label{S:docker-fg}
\index{Docker!FUturesystems}
\index{Futuresystems!Docker}

\FILENAME

\section{Overview}

\begin{IU}

This section is for IU students only that take classes with us.
\end{IU}

This documentation introduces how to run Docker container on
FutureSystems. Currently we have deployed Docker swarm on Echo.

\section{Getting Access}

You will need an account on FutureSystems and be enrolled in an active
project.  To verify, try to see if you can log into
india.futuresystems.org. You need to be a member of a valid
FutureSystems project, and had submitted an ssh public key via the
FutureSystems portal.

\begin{IU}
For 2018 you need to be in the following project:

\url{https://portal.futuresystems.org/project/537}

\end{IU}

If your access to the india host has been verified, try to login to
the docker swarm head node. To conveniently do this let us define some
Linux environment variables to simplify the access and the material
presented here. YOu can place them even in your \verb|.bashrc| or
\verb|.bash_profile| so the information gets populated whnever you start a
new terminal.

\begin{lstlisting}
export ECHO=149.165.150.76
export FS_USER=<put your futersystem here>
\end{lstlisting}


with the same username and key:

\begin{lstlisting}
ssh $FS_USER@$ECHO
\end{lstlisting}

However it is much more convenient to 

\begin{NOTE}
If you have access to india but not the docker swarm
system, your project may not have been authorized to access the docker
swarm cluster. Send a ticket to FutureSystems ticket system to request
this.
\end{NOTE}

Once logged in to the docker swarm head node, try to run:

\begin{lstlisting}
docker run hello-world
\end{lstlisting}

to verify \verb|docker run| works.

\section{Creating a service and deploy to the swarm cluster}

While \verb|docker run| can start a container and you may even attach to its
console, the recommended way to use a docker swarm cluster is to create
a service and have it run on the swarm cluster. The service will be
scheduled to one or many number of the nodes of the swarm cluster, based
on the configuration. It is also easy to scale up the service when more
swarm nodes are available. Docker swarm really makes it easier for
service/application developers to focus on the functionality development
but not worrying about how and where to bind the service to some
resources/server. The deployment, access, and scaling up/down when
necessary, are all managed transparently. Thus achieving the new
paradigm of \textit{serverless computing}.

As an example, the following command creates a service and deploy it to
the swarm cluster:

\begin{quote}
docker service create --name notebook\_test -p 9001:8888
jupyter/datascience-notebook start-notebook.sh
--NotebookApp.password=NOTEBOOK\_PASS\_HASH
\end{quote}

The NOTEBOOK\_PASS\_HASH can be generated in python:

\begin{lstlisting}
>>> import IPython
>>> IPython.lib.passwd("YOUR_SELECTED_PASSWROD")
'sha1:52679cadb4c9:6762e266af44f86f3d170ca1......'
\end{lstlisting}
%$

So pass through the string starting with 'sha1:......'.

The command pulls a published image from docker cloud, starts a container and
runs a script to start the service inside the container with necessary
parameters. The option ``-p 9001:8888'' maps the service port inside the
container (8888) to an external port of the cluster node (9001) so the
service could be accessed from the Internet. In this example, you can
then visit the URL\@:

\begin{quote}
\url{http://$ECHO:9001}
\end{quote}
%$

to access the Jupyter notebook. Using the specified password when you
create the service to login.

Please note the service will be dynamically deployed to a container
instance, which would be allocated to a swarm node based on the
allocation policy. Docker makes this process transparent to the user and
even created mesh routing so you can access the service using the IP
address of the management head node of the swarm cluster, no matter
which actual physical node the service was deployed to.

This also implies that the external port number used has to be free at
the time when the service was created.

Some useful related commands:


\begin{lstlisting}
docker service ls
\end{lstlisting}

lists the currently running services.

\begin{lstlisting}
docker service ps notebook_test
\end{lstlisting}

lists the detailed info of the container where the service is running.

\begin{lstlisting}
docker node ps NODE
\end{lstlisting}

lists all the running containers of a node.

\begin{lstlisting}
docker node ls
\end{lstlisting}

lists all the nodes in the swarm cluster.

To stop the service and the container:

\begin{lstlisting}
docker service rm noteboot_test
\end{lstlisting}


\subsection{Create your own service}

You can create your own service and run it. To do so, start from a base
image, e.g., a ubuntu image from the docker cloud. Then you could:

\begin{itemize}

\item Run a container from the image and attach to its console to develop
the service, and create a new image from the changed instance using
command `docker commit'.

\item Create a dockerfile, which has the step by step building process of
the service, and then build an image from it.

\end{itemize}

In reality, the first approach is probably useful when you are in the
phase of develop and debug your application/service. Once you have the
step by step instructions developped the latter approach is the
recommended way.

Publish the image to the docker cloud by following this documentation:

\begin{quote}
\url{https://docs.docker.com/docker-cloud/builds/push-images/}
\end{quote}

Please make sure no sensitive information is included in the image to
be published. Alternatively you could publish the image internally to
the swarm cluster.

\paragraph{Publish an image privately within the swarm cluster}

\TODO{Fugang: create image for distribution}

Once the image is published and available to the swarm cluster, you
could start a new service from the image similar to the Jupyter Notebook
example.

\subsection{Exercises}

\begin{exercise}

Obtain an account on future systems.

\end{exercise}

\begin{exercise}

Create a REST service with swagger codegen and run it on the echo cloud.

\end{exercise}


%----------------------------------------------------------------------------------------
%	PART
%----------------------------------------------------------------------------------------
\chapterimage{python-logo-generic.pdf} % Chapter heading image


\part{Python}

\FILENAME

%----------------------------------------------------------------------------------------
%	CHAPTER 1
%----------------------------------------------------------------------------------------

\chapter{TBD}



# section/icloud//assignment:
# section/icloud//assignment/files:
# section/icloud//course:
# section/icloud//evaluation:
# section/icloud//participation:

\chapter{Overview}
\label{s:overview}

\FILENAME

\section{Introduction to Cloud Computing}
\label{s:icloud-fundamentals}

Changes in computing technology for the past five decades are discussed.
The rise of Big Data is shown in terms of its growth and significance. A
prediction is made that the paradigm which has held `til now of
individual researchers with personal computers will give way to
communities of researchers organizing through clouds. A more in-depth
look at Unit 1 follows, focusing on the chapters from Distributed and
Cloud Computing: From Parallel Processing to the Internet of Things.

% Note that the following lectures are replaced by:
%
% Geoffrey's updated 22 lectures for e222 and cloud intro e516/616
% main location is at https://drive.google.com/drive/u/5/folders/1WLHvJ8qkJiXGC9NnOceTH9ZdFngC-xoO


\subsection{Introduction - Part A}\label{s:cloud-fundamentals-a}

\video{Cloud}{14:48}{Introduction - Part A}{https://drive.google.com/open?id=16gPOZ7EK6iaac2B-9KbM3OHITqHT8YxG}
\slides{Cloud}{7 Slides}{Introduction - Part A}{https://drive.google.com/open?id=17_voZxBdqiLicJKzroQ2DzjW4zvp0VYb} 

This lecture introductes you to the following concepts: Cloud
definitions (AWS, AZURE,etc) , Virtualization, Hype Cycle, Cloud
infrastructure, Cloud software, Clouds and Parallel computing,
Storage, HPC clouds, Jobs, Future Issues, and Fault tolerance.

\subsection{Introduction - Part B}\label{s:cloud-fundamentals-b}

\video{Cloud}{20:22}{Part B}{https://drive.google.com/open?id=1iGv6GG2b2th_RL3kQX6P04Y0AyuUxQcj}
\slides{Cloud}{13 Slides}{Part B}{https://drive.google.com/open?id=1SJZblEkhjgQBCDaDSiSCaXTLVPx_X8m4} 

This lecture introductes you to the following concepts: FAAS, Cloud
HPC, HPC-FAAS, public clouds, IOT, service oriented systems, cloud and
infrastructure description, virtualization in detail, Google and
microsoft cloud use cases, and Renting idle clouds.

\subsection{Introduction - Part C}\label{s:cloud-fundamentals-c}

\video{Cloud}{20:45}{Part C}{https://drive.google.com/open?id=1nU9HlqDe_vEZR1MlOq2XibnoHmaygo-H}
\slides{Cloud}{11 Slides}{Part C}{https://drive.google.com/open?id=1Thg2yOnKBQKdndgdItpviGw0hhxjY2K2}

This lecture introductes you to the following concepts: more on
clouds, Messaging model, SOA, FAAS, SAAS, PAAS, NAAS, etc. Garner
remarks on clouds, and Server computing.

\subsection{Introduction - Part D}\label{s:cloud-fundamentals-d}
 \video{Cloud}{9:08}{Part D}{https://drive.google.com/open?id=1Qayxfwuc_qSeCzaIpHj3-F0N7TONvM5L}
\slides{Cloud}{9 Slides}{Part D}{https://drive.google.com/open?id=1EsHbUn7xdjTrLXmY8HCIhZqLaWeODD_1}

This lecture introductes you to the following concepts: World wide
cloud market, IT growth by areas, IT infrastructure development,
Comparing Amazon and Cloud computing expenditures.

\subsection{Introduction - Part E}\label{s:cloud-fundamentals-e}
 \video{Cloud}{11:21}{Part E}{https://drive.google.com/open?id=1EC-eOYuBOV1qMojSg4Mg50doPqjAY0Hr}
\slides{Cloud}{8 Slides}{Part E}{https://drive.google.com/open?id=1K3RzVlfiwZAqwhCPxNEP8Q1JNxyOZhrv}

This lecture introductes you to the following concepts: Virtualization
in detail, Areas covered by Open Stack, and Virtualization technologies.

\subsection{Introduction - Part F}\label{s:cloud-fundamentals-f}
 \video{Cloud}{13:41}{Part F}{https://drive.google.com/open?id=1_M0HDemFmykAq4iPuQ0MHYIW0kffHuMa}
\slides{Cloud}{11 Slides}{Part F}{https://drive.google.com/open?id=18deLUm2zGlHcHHbyVhecwSf52RHAq9VT}

This lecture introductes you to the following concepts: Emerging
trends Move along hype cycle. Emerging technologies, Gartner priority
matrix for Emerging Technologies, Gartner Hype cycle for data center
infrastructure.

\subsection{Introduction - Part G}\label{s:cloud-fundamentals-g}
 \video{Cloud}{16:05}{Part G}{https://drive.google.com/open?id=19k6os58_OCCsmbusWQ6z6n9hwVmmd163}
\slides{Cloud}{15 Slides}{Part G}{https://drive.google.com/open?id=1ZD-h-dmQgAFROBEI1KMJROTxPVPAJ_P_}

This lecture introductes you to the following concepts: Gartner Hype
Cycle for technology emerging 2008, Gartner priority matrix for
emerging tech in 2008-2015, Digital business (Web, E- business, etc),
digital marketing, digital business, and Autonomous.

\subsection{Introduction - Part H}\label{s:cloud-fundamentals-h}

 \video{Cloud}{13:22}{Part H}{https://drive.google.com/open?id=1_lH7ou_UQOMMfdSSQLrqZr8yJZIr7rC0}
\slides{Cloud}{12 Slides}{Part H}{https://drive.google.com/open?id=1WOlwFtHNvFSaQX9WAXBwO2q1qsvBM6Yd}

This lecture introductes you to the following concepts: Cloud
infrastructure, Azure cloud data centers, Google cloud data centers,
IBM cloud data centers, Green clouds, Cloud and HPC

\subsection{Introduction - Part I}\label{s:cloud-fundamentals-i}
 \video{Cloud}{13:13}{Part I}{https://drive.google.com/open?id=1UwBzSKK-iHYW7aRCS_f4BoXNYxUb476i}
\slides{Cloud}{11 Slides}{Part I}{https://drive.google.com/open?id=1_1fL2L-tpjlL7og49XDxH_HLPD6Tupik} 

This lecture introductes you to the following concepts: Cloud
infrastructure Gartner view 2017, Containers vs virtual
machines. Serverless computing domain. Key trends in computer
infrastructure, and AI on data centers.

\subsection{Introduction - Part J}\label{s:cloud-fundamentals-j}
 \video{Cloud}{37:56}{Part J}{https://drive.google.com/open?id=1tasT1lJE_7pwowuk0D68M7rgW1gtv3im}
\slides{Cloud}{15 Slides}{Part J}{https://drive.google.com/open?id=1DKshrkb2bMOt5p0H2YDkOZhekAFIx5nJ}

This lecture introductes you to the following concepts: HPC-ABDS, Google Software,
SIMD, SPMD, MapReduce.

\subsection{Introduction - Part K}\label{s:cloud-fundamentals-k}
\video{Cloud}{11:58}{Part K}{https://drive.google.com/open?id=1BzpXNqs3Ai_QlSsjxSV6Q3FevZt1BPj-}
\slides{Cloud}{16 Slides}{Part K}{https://drive.google.com/open?id=1dQbzoVwDoqqnwSO8eHmu_WfiU2FVXtLL}

This lecture introductes you to the following concepts: NIST, Image and Video data growth,
scale of industrial Internet, Value of Data and Analytics. 

\subsection{Introduction - Part L}\label{s:cloud-fundamentals-l}
 \video{Cloud}{13:03}{Part L}{https://drive.google.com/open?id=1pIjcar6SZSelWiR6dv4lhnsmweDarNXR}
\slides{Cloud}{11 Slides}{Part L)}{https://drive.google.com/open?id=10bAckYz455Yd5tqDGpwkMZUmDn9FKl0n}

This lecture introductes you to the following concepts: CyberInfrastructure, 
Virtual Observatory, Gnome Sequencing Cost, Artificial Intelligence and Deep 
Learning

\subsection{Introduction - Part M}\label{s:cloud-fundamentals-m}
\video{Cloud}{24:12}{Part M}{https://drive.google.com/open?id=16rb0zZafTLoYhT638qc4IRorOk5qgRP-}
\slides{Cloud}{14 Slides}{Part M (Updated)}{https://drive.google.com/open?id=1TPgbfK6PhWjyr3u115Pq4jTPNPo2JU70}

This lecture introductes you to the following concepts: Applications on Clouds,
IOT in clouds, Cloud Fog, Data Streaming, Covergenece, Diamonds, Views, and Facets.

\subsection{Introduction - Part N}\label{s:cloud-fundamentals-n}
 \video{Cloud}{35:46}{Part N}{https://drive.google.com/open?id=1apuvCaKlQBZp8FjvqN9It5bJhfXmf1kg}
\slides{Cloud}{15 Slides}{Part N}{https://drive.google.com/open?id=1lBmpMry2FXwEjf93EMJPNDpw_G-yUcI1}

This lecture introductes you to the following concepts: Parallel Computing, 
Big Data and Simulations, Amdahl's Law, Concepts in Parallelism. 

\subsection{Introduction - Part O}\label{s:cloud-fundamentals-o}
 \video{Cloud}{19:22}{Part O}{https://drive.google.com/open?id=1_A51jrlzKYKhQvefXbJPs0J4GiCCQ1et}
\slides{Cloud}{10 Slides}{Part O}{https://drive.google.com/open?id=1JvIK-J4HUSvexmNKbBV_fdN8xvGBylR2}

This lecture introductes you to the following concepts: Cloud storage, 
Repositories, File Systems, Data lakes

\subsection{Introduction - Part P}\label{s:cloud-fundamentals-p}
 \video{Cloud}{19:29}{Part P)}{https://drive.google.com/open?id=1u5z9-GY1Hecp8nR47Gxe8506NnhHduF1}
\slides{Cloud}{8 Slides}{Part P}{https://drive.google.com/open?id=1ECR-m-pgW_7JnNDJ9ZLDYpicRrtxawwF}

This lecture introductes you to the following concepts: Branscomp Pyramid, 
Supercomputers versus clouds, Science Computing Environments. 

\subsection{Introduction - Part Q}\label{s:cloud-fundamentals-q}
 \video{Cloud}{16:19}{Part Q}{https://drive.google.com/open?id=1vmtpndTlmtV4DzUtl04_nWo9DCR84mIM}
\slides{Cloud}{10 Slides}{Part Q}{https://drive.google.com/open?id=1u_rugpZg6m2x0B7r4kifEUxMej71-wMh}

This lecture introductes you to the following concepts: Structure of different
applications, Simulation and Big Data, Software Implications, Languages. 

\subsection{Introduction - Part R}\label{s:cloud-fundamentals-r}
 \video{Cloud}{4:52}{Part R}{https://drive.google.com/open?id=1JMNPLdbS81Hfi5P7irDUy7kna35jpU6j}
\slides{Cloud}{6 Slides}{Part R}{https://drive.google.com/open?id=1Sev0VF7tDZo4Oxa_ghk6b4YhE4iTPkhE}

This lecture introductes you to the following concepts: Computer Engineering,
Clouds, Design, Data Science/Engineering. 

\subsection{Introduction - Part S}\label{s:cloud-fundamentals-s}
 \video{Cloud}{19:46}{Part S}{https://drive.google.com/open?id=1kgbrDiDNj0DKYi_ICvOXYeFZ32R1JH2_}
\slides{Cloud}{6 Slides}{Part S}{https://drive.google.com/open?id=1_LdE64DJqSKI7EmMbSsppHMO6KU4q4Sl} 

This lecture introductes you to the following concepts: Gartner Cloud COmputing
Hypercycle and priority matrix, Severless and Fass, Cloud Native, Microservices.

\subsection{Introduction - Part T}\label{s:cloud-fundamentals-t} 
 \video{Cloud}{11:29}{Part T}{https://drive.google.com/open?id=1TQsCuR-2C6D_OJeJH818WO5-OBrA8Spo}
\slides{Cloud}{13 Slides}{Part T}{https://drive.google.com/open?id=1fVelFbLaUbNXMGS6ENRn4QHLzFitAyUD}

This lecture introductes you to the following concepts: Security and Blockchains.

\subsection{Introduction - Part U}\label{s:cloud-fundamentals-u}
 \video{Cloud}{9:10}{Part U)}{https://drive.google.com/open?id=1jRpkdGCT-sQRb6pOGrW-qG58A_HYfr9W}
\slides{Cloud}{5 Slides}{Part U}{https://drive.google.com/open?id=1cR9YLcrtUMKINRFNMbdVre8Ydfa-lvAj}

This lecture introductes you to the following concepts: Fault Tolerance, Amazon 
S3 related Fault Tolerance. 

\chapter{IaaS}

\label{sec:icloud-iaas}

\FILENAME

Examples and definitions are given for SaaS, PaaS, and IaaS.
Computational models must be designed with the problems and effective
resources in mind. A demonstration of cloud use for Bioinformatics
shows how clouds offer advantages of provisioning and virtual cluster
support. Overhead and performance issues are touched upon through
charts showing the use of three different virtual clusters.


\begin{comment}
\TODO{This video is outdated, start wom abou 1:00 to end}
\TODO{The content seems redundant, we will provide an updated lecture.}

\video{Cloud}{7:45}{Course Expectations}{https://www.youtube.com/watch?v=j3sUW376pw8}

\slides{Cloud}{Page 1}{Course Expectations}{https://drive.google.com/open?id=0B88HKpainTSfQU1uQmxZWHdWQ1k}

\slides{Cloud}{Page 1}{Course Expectations - pptx}{https://drive.google.com/open?id=0B88HKpainTSfb1ZhWG4zTEg0SVk}
\end{comment}

\section{Growth of Virtual Machines}

Importance of virtualization is explored, including cross-platform
applications. Virtualization has seen rapid growth in recent years in
terms of use and services offered. Virtual machines differ from
traditional computers in that software virtualization layer (hypervisor)
runs on hardware, allowing guest OS to run on top of host OS. VMs can
run independent of hardware specifications. Four different types of VM
architecture, defined by the layer which the virtual machine monitor
(VMM) runs on. VM is identical to physical machines and can be saved and
stored, as well as migrated across hardware.

\video{Cloud}{10:16}{Growth of Virtual Machines}{https://www.youtube.com/watch?v=5oKoAPCXLws}

\slides{Cloud}{Page 28}{Growth of Virtual Machines}{https://drive.google.com/open?id=0B88HKpainTSfQU1uQmxZWHdWQ1k}

\slides{Cloud}{Page 28}{Growth of Virtual Machines - pptx}{https://drive.google.com/open?id=0B88HKpainTSfb1ZhWG4zTEg0SVk}

\section{Implementation Levels}

Virtualization can be implemented on five levels: application, library,
OS, hardware, and instruction. Their benefits are compared in terms of
performance, flexibility, complexity, and isolation. A layout is
provided for the Linux virtualization layer, OpenVZ (OS level), which
creates virtual private servers. CUDA is a high performance computing
library, not designed for VMs; vCUDA is a virtual layer that allows
interaction between CUDA and VMs, creating a virtual CUDA library.

\video{Cloud}{7:57}{Implementation Levels}{https://www.youtube.com/watch?v=Le-kv-eAhvg}

\slides{Cloud}{Page 41}{Implementation Levels}{https://drive.google.com/open?id=0B88HKpainTSfQU1uQmxZWHdWQ1k}

\slides{Cloud}{Page 41}{Implementation Levels - pptx}{https://drive.google.com/open?id=0B88HKpainTSfb1ZhWG4zTEg0SVk}

\section{Tools and Mechanisms}

A list of major hypervisors is given. Type 1 hypervisor resides on the
bare metal computer, while Type 2 runs over the host OS. XEN is an open
source hardware level hypervisor: consists of hypervisor, kernel, and
application. Domain0 in XEN is a VM that manages other VMs. Two types of
hardware virtualization: full virtualization and host-based
virtualization. Para-virtualization does not need to modify the guest OS
like full virtualization and works through hypercalls. An example is the
ESX server from VMware.

\video{Cloud}{7:32}{Tools and Mechanisms}{https://www.youtube.com/watch?v=VYz5rp5HDVE}

\slides{Cloud}{Page 47}{Tools and Mechanisms}{https://drive.google.com/open?id=0B88HKpainTSfQU1uQmxZWHdWQ1k}

\slides{Cloud}{Page 47}{Tools and Mechanisms - pptx}{https://drive.google.com/open?id=0B88HKpainTSfb1ZhWG4zTEg0SVk}

\section{CPU, Memory \& I/O Devices}

A hybrid approach to virtualization involves offloading some tasks to
the hardware to reduce overhead. This can be combined with
para-virtualization for even greater effects. In a guest OS, the VMM
provides shadow page tables to transfer virtual memory to machine
memory. An example is shown in the Intel Extended Page Table. A
virtualization layer for an I/O device is possible, allowing it to act
like a physical device and manage host and guest addresses, shown in a
detailed VMware example.

\video{Cloud}{6:41}{CPU, Memory \& I/O Devices}{https://www.youtube.com/watch?v=I_J4eUUavSY}

\slides{Cloud}{Page 58}{CPU, Memory \& I/O Devices}{https://drive.google.com/open?id=0B88HKpainTSfQU1uQmxZWHdWQ1k}

\slides{Cloud}{Page 58}{CPU, Memory \& I/O Devices - pptx}{https://drive.google.com/open?id=0B88HKpainTSfb1ZhWG4zTEg0SVk}

\section{Clusters and Resource Management}

Characteristics of VM clusters are listed, including the ability to run
multiple VMs on the same node and size alteration. Physical clusters are
linked through nodes, while virtual clusters can be linked through
physical or virtual nodes and can be replicated in virtual servers.
Prepackaged OS can be installed in a virtual cluster. Should a VM fail
for any reason, its image can be migrated to a new host so work is not
lost. An example of this is demonstrated with XEN.

\video{Cloud}{5:07}{Clusters and Resource Management}{https://www.youtube.com/watch?v=Mn9pgGtFy4g}

\slides{Cloud}{Page 66}{Clusters and Resource Management}{https://drive.google.com/open?id=0B88HKpainTSfQU1uQmxZWHdWQ1k}

\slides{Cloud}{Page 66}{Clusters and Resource Management - pptx}{https://drive.google.com/open?id=0B88HKpainTSfb1ZhWG4zTEg0SVk}

\section{Data Center Automation}

Whole data centers can be virtualized, enabling for the construction of
private clouds. Some tools for Infrastructure as a Service clouds are
Nimbus, Eucalyptus, OpenNebula, and vSphere. Eucalyptus is shown in
greater detail. Trust issues in cloud security are answered in virtual
machines. Suggested reading material is provided at the end.

\video{Cloud}{3:30}{Data Center Automation}{https://www.youtube.com/watch?v=mvXBRvTwAVg}

\slides{Cloud}{Page 74}{Data Center Automation}{https://drive.google.com/open?id=0B88HKpainTSfQU1uQmxZWHdWQ1k}

\slides{Cloud}{Page 74}{Data Center Automation - pptx}{https://drive.google.com/open?id=0B88HKpainTSfb1ZhWG4zTEg0SVk}

\section{Clouds in the Workplace}

Clouds run as servers for data storage and sharing on the Internet in an
on-demand capacity. Cloud services are scalable depending on the
client's needs, allowing for a seemingly limitless source of computing
power that can expand or shrink to meet financial demands. Some examples
of cloud services are LinkedIn, Amazon S3, and Google App Engine.
Different variations of clouds like IaaS and PaaS are offered by both
open source and commercial providers. Cloud systems are composed of
separate elements like Eucalyptus, Xen and VMware.

\video{Cloud}{7:13}{Clouds in the Workplace}{https://www.youtube.com/watch?v=Endt6mWUfEo}

\slides{Cloud}{Page 1}{Clouds in the Workplace}{https://drive.google.com/open?id=1kkTi8YXMR7cPR-9nWgnj9UgkXm4rUfHm}

\section{Checklists and Challenges}

The capabilities of several IaaS cloud structures like Amazon EC2 or
PaaS like Microsoft Azure are listed. Public and private clouds share
certain features; the main difference is public clouds are owned by
service providers while private clouds are offered by individual
corporations. Certain enabling technologies are required for clouds to
provide quick and scalable computing. These include virtual cluster
provisioning and multi-tenant environments. PaaS demands the capability
to process huge amounts of data as in the case of web searches. Some
challenges faced by cloud computing include vendor lock-in owing to lack
of standard APIs and metrics; for scientists, there is uncertainty about
whether experiments can be reproduced effectively in different cloud
environments. However there are distinct advantages clouds potentially
have to offer: standardized APIs can eliminate lock-in, and encryption
offers data confidentiality.

\video{Cloud}{9:08}{Checklists and Challenges}{https://www.youtube.com/watch?v=cwtWpZcWuQ0}

\slides{Cloud}{Page 11}{Checklists and Challenges}{https://drive.google.com/open?id=1kkTi8YXMR7cPR-9nWgnj9UgkXm4rUfHm}

\section{Data Center Setup}

Huge data centers enable cloud computing, containing up to a million
servers. Large data centers charge less for their services than small
ones. A diagram illustrates the typical setup of a cloud; rack space on
the bottom, on top of which are load balancers, then excess routers and
border routers. The next figure compares cost effectiveness in a
traditional IT model to a cloud. Other figures display small server
clusters and a typical data center arrangement, including emergency
power supply and cooling system. A chart shows the power consumption
based on CPU, disk, etc. Disks in warehouse servers may be onsite or
attached to outside connections like InfiniBand. Switches can form an
array of racks. The distribution of memory across a local, rack, or
array server in warehouse server setup is listed.

\video{Cloud}{7:49}{Data Center Setup}{https://www.youtube.com/watch?v=zBVtXzqF2ew}

\slides{Cloud}{Page 16}{Data Center Setup}{https://drive.google.com/open?id=1kkTi8YXMR7cPR-9nWgnj9UgkXm4rUfHm}

\section{Cultivating Clouds}

Power utilization effectiveness (PUE) for a warehouse is determined by
comparing it to IT power usage. Racks can contain 40 servers, shipping
containers can have up to 1,000 servers; a data center could take 2
years to construct. Warehouse scale computing has greater economy of
scale than data centers by reducing network and administrative costs.
Individual users can interact with clouds in the SaaS model, while
organizations use PaaS. Clouds generally use VMs to recover from system
failures. It is predicted that the cloud job market and demand for
clouds will experience great growth in the future. Clouds have become
ubiquitous in all aspects of the private and public sector. In the
future clouds must take into account user privacy, data security and
copyright protection.

\video{Cloud}{5:10}{Cultivating Clouds}{https://www.youtube.com/watch?v=zxoqRdvXM28}

\slides{Cloud}{Page 15}{Cultivating Clouds}{https://drive.google.com/open?id=1tTiWbi5_elBXmB--wMiCCB-3KtJa50AP}

\slides{Cloud}{Page 1}{Cultivating Clouds - Conclusions}{https://drive.google.com/open?id=15ofQSh3-BQNzTeycnEgKh5UXqGR3YMiz}


\FILENAME

\section{Assignments}\label{assignments}

\subsection{Assignment 0 - Identify Technologies}\label{assignment-0}

Assignment 0 will require you to identify the different software
tools/technologies included in the given attachment and then group
them into the correct layer of categories as indicated on the
left-hand side of the slide.

This homework is worth 5 points.

\begin{itemize}
\tightlist
\item
  pptx \textless{}files/assignment\_0.pptx\textgreater{}
\end{itemize}

\subsection{Project 1}\label{project-1}

You will need to complete the source code and write a report. Zip your
work into a file with the name username\_project1.zip (replace
`username' with your Group Contact member's username) and submit the
following:

\begin{itemize}
\item
  Complete source code
\item
  \begin{description}
  \item[A document with the following details:]
  \begin{itemize}
  \tightlist
  \item
    Transformation of data during the computations, i.e. data type of
    key, value
  \item
    The data structure used to transfer between Map and Reduce phases
  \item
    How the data flow happens through disk and memory during the
    computation
  \end{itemize}
  \end{description}
\end{itemize}

Only one submission per group is required for Project 1. It is due time
at 11:59 pm on Feburary 5.

\begin{itemize}
\tightlist
\item
  Project 1 \textless{}files/project1.pdf\textgreater{}
\item
  Input Data \textless{}files/project1\_input\_data.txt\textgreater{}
\end{itemize}

\subsection{Project 2}\label{project-2}

You are required to turn in the following items in a zip file
(username\_HadoopPageRank.zip) in this assignment:

\begin{itemize}
\item
  The source code of Hadoop PageRank you implemented.
\item
  \begin{description}
  \item[Technical report (username\_HadoopPageRank\_report.docx) that
  contains:]
  \begin{itemize}
  \tightlist
  \item
    The description of the main steps and data flow in your program.
  \item
    The output file (username\_HadoopPageRank\_output.txt) which
    contains the first 10 urls along with their ranks.
  \end{itemize}
  \end{description}
\item
  Project 2 \textless{}files/project2.pdf\textgreater{}
\end{itemize}

\subsection{Project 3}\label{project-3}

Project 3 asks you to implement a bioinformatics application using
Hadoop (Map only) MapReduce framework and write a report about the data
flow and your observations of the program.

You are required to turn in the following items in a zip file
(username\_HadoopBlast.zip) in this assignment:

\begin{itemize}
\tightlist
\item
  The source code of Hadoop Blast you implemented.
\item
  Technical report (username \_HadoopBlast\_report.docx) that answers
  the following questions. - What is Hadoop Distributed Cache and how is
  it used in this program? - Write the two lines that put and get values
  from Distributed cache. Also include the method and class information.
  - In previous projects we used Hadoop's TextInputFormat to feed in the
  file splits line by line to map tasks. In this program, however, we
  want to feed in a whole file to a single map task. What is the
  technique used to achieve this? Also, briefly explain what are the key
  and value pairs you receive as input to a map task and what methods
  are responsible for producing these pairs? - Do you think this
  particular implementation will work if the input files are larger than
  the default HDFS block size? Briefly explain why. {[}Hint: you can
  test what will happen by concatenating the same input file multiple
  times to create a larger input file in the resources/blast\_input
  folder{]} - If you wanted to extend this program such that all output
  files will be concatenated into a single file, what key and value
  pairs would you need to emit from the map task? Also, how would you
  use these in the reduce that you would need to add?
\item
  The 4 output FASTA files -- celllines\_1.fa to celllines\_4.fa.
\end{itemize}

Points will be reduced (maximum 0.5 points) if the filename or directory
structure are different from instructed above.

The point total for this project is 3, where the distribution is as
follows:

\begin{itemize}
\tightlist
\item
  Completeness of your code and output (1 points)
\item
  Correctness of written report (2 points)
\item
  Project 3 \textless{}files/project3.pdf\textgreater{}
\end{itemize}

\subsection{Project 4}\label{project-4}

Zip your source code and report in a file named username\_project4.zip

The point total for this project is 1.5, where the distribution is as
follows:

\begin{itemize}
\tightlist
\item
  Correctness of your code and output (1 points)
\item
  Completeness of written report (0.5 points)
\end{itemize}

Before you start this project, you need to complete the
Project4 Prerequisite\textless{}files/project4\_pre.pdf\textgreater{}
first. The submission folder for it will be published before the lab
session.

\begin{itemize}
\tightlist
\item
  Project 4 \textless{}files/project4.pdf\textgreater{}
\item
  Project 4 Prerequisite \textless{}files/project4\_pre.pdf\textgreater{}
\end{itemize}

\subsection{Project 5}\label{project-5}

Write an HBase FreqIndexBuilder program to build an inverted index table
which has the unique term's occurrences in all documents from the
clueWeb09 dataset. Zip your source code, results and report in a file
named username\_project5.zip. Submit this file to the Canvas submission
page.

\begin{itemize}
\tightlist
\item
  Complete source code
\item
  A written report describing the main steps
\end{itemize}

The point total for this project is 3, where the distribution is as
follows:

\begin{itemize}
\tightlist
\item
  Completeness of your code and output (2 points)
\item
  Correctness of written report (1 points)
\item
  Project 5 \textless{}files/project5.pdf\textgreater{}
\end{itemize}

\subsection{Project 6}\label{project-6}

After having familiarized yourself with the ``HBase Building an Inverted
Index'' homework and ``PageRank algorithms'' homework, you are ready to
use these applications to test the search engine function from the
packaged executable.

\subsubsection{Deliverables}\label{deliverables}

Zip your source code, library, and results in a file named
\href{mailto:username@test-search-engine.zip}{\nolinkurl{username@test-search-engine.zip}}.
Please submit this file to the Canvas Assignments page.

\subsubsection{Evaluation}\label{evaluation}

The point total for this project is 6, where the distribution is as
follows:

\begin{itemize}
\tightlist
\item
  Completeness of your code (5 points)
\item
  Correct output (1 points)
\item
  Project 6 \textless{}files/project6.pdf\textgreater{}
\end{itemize}

\subsection{Project 7}\label{project-7}

The goal of this project is to familiarize yourself with the concept of
map-collective applications. Harp is similar to MapReduce in terms of
programming with the exception that it provides collective communication
support across map tasks.

Zip your source code and output as username\_harp-pagerank.zip. Please
submit this file to the Assignments page.

The point total for this project is 6, where the distribution is as
follows:

\begin{itemize}
\tightlist
\item
  Completeness of your code (5 points)
\item
  Correct output (1 point)
\end{itemize}

We prepared a new VM for project7 and project8. Please download it from
\href{https://drive.google.com/file/d/0B2iFsq4CY1DteHhJUEk5cDNJajQ/view}{here}.

\begin{itemize}
\tightlist
\item
  Project 7 \textless{}files/project7.pdf\textgreater{}
\item
  \href{https://drive.google.com/file/d/0B2iFsq4CY1DteHhJUEk5cDNJajQ/view}{VirtualBox
  VM Download}
\end{itemize}

\begin{description}
\item[Do not copy and paste commands from pdf files. Please type them]
manually. Special characters cause problems in executing commands in a
terminal.
\end{description}

\subsection{Project 8}\label{project-8}

Zip your source code and report as username\_mbkmeans.zip.

The point total for this project is 6, where the distribution is as
follows: - Completeness of your code (5 points) - In the report,
describe your implementation and the output. (1 points)

You can get up to 4 bonus points based on your extra efforts.

\begin{itemize}
\tightlist
\item
  Project 8 \textless{}files/project8.pdf\textgreater{}
\end{itemize}

%----------------------------------------------------------------------------------------
%	PART
%----------------------------------------------------------------------------------------

\part{Draft: IoT}

\FILENAME

%----------------------------------------------------------------------------------------
%	CHAPTER 1
%----------------------------------------------------------------------------------------

\chapterimage{flock1.jpeg} % Chapter heading image

\input{section/iot/introduction}
\input{section/iot/hardware}
\input{section/iot/projects}
\input{section/iot/esp8266}
\input{section/iot/pi}
\input{section/iot/dexter}
\input{section/iot/grovepi}
\input{section/iot/sensors}
\input{section/iot/vnc}
\input{section/iot/turtle}
\input{section/iot/tools}


\input{section/icloud//course/iterative-mapreduce.tex}
\part{How to Run MapReduce (PaaS)}
\label{sec:icloud-mapreduce}


  15 Video lectures (1 hour 58 minutes 4 seconds)

\chapter{MapReduce}

\section{Apache Data Analysis OpenStack}

The buildup of Big Data has seen the development of new data storage
systems like MapReduce and Hadoop. Apache's Big Data Stack houses a host
of programs designed around Google's offerings like MapReduce. The
architecture of Hadoop 1.0 and 2.0 are compared, along with an
examination of the MapReduce concept. A demo video of Twister-MDS
includes a 3-dimensional representation of data cluster sorting through
the PlotViz program. Data analysis tool Twister boasts features like
in-memory support of tasks, data flow separation, and portability.

\video{Cloud}{12:01}{Apache Data Analysis OpenStack}{https://www.youtube.com/watch?v=6vkgvGtyv4Q}

\slides{Cloud}{Page 1}{Apache Data Analysis OpenStack}{https://drive.google.com/open?id=0B88HKpainTSfMnpCelpNQUpNdVE}
\slides{Cloud}{Page 1}{Apache Data Analysis OpenStack - pptx}{https://drive.google.com/open?id=0B88HKpainTSfTVlNRzRMemNaZEU}

\section{MapReduce}

MapReduce was designed by Google to address the problem of large-scale
data processing. A breakdown of basic MapReduce terms and functions
follows. Use of MapReduce has flourished since its premier, as
illustrated by an in-depth example of its use in WordCount. Finally the
basic process of MapReduce is shown.

\video{Cloud}{9:07}{MapReduce}{https://www.youtube.com/watch?v=sSIGaDaulvA}

\slides{Cloud}{Page 6}{MapReduce}{https://drive.google.com/open?id=0B88HKpainTSfMnpCelpNQUpNdVE}
\slides{Cloud}{Page 6}{MapReduce - pptx}{https://drive.google.com/open?id=0B88HKpainTSfTVlNRzRMemNaZEU}

\section{Hadoop Framework}

Hadoop is an open source version of MapReduce designed for broad
application in terms of code and settings. Storage is done in the Hadoop
Distributed File System through master and slave nodes. Compute is
handled by JobTracker and TaskTracker; the duties of these two
intertwined programs are then explored more fully.

\video{Cloud}{8:32}{Hadoop Framework}{https://www.youtube.com/watch?v=Vuroqly6FTE}

\slides{Cloud}{Page 15}{Hadoop Framework}{https://drive.google.com/open?id=0B88HKpainTSfMnpCelpNQUpNdVE}
\slides{Cloud}{Page 15}{Hadoop Framework - pptx}{https://drive.google.com/open?id=0B88HKpainTSfTVlNRzRMemNaZEU}

\section{Hadoop Tasks}

The Map stage of MapReduce is shown in greater detail. This process
starts with Hadoop Distributed File System, which handles the input
data. Key value pairs are assigned to the data blocks. Combiner reduces
data size and Partitioner determines distribution of keys among
reducers. Intermediate data is stored in a circular buffer before being
sent to reduce tasks. Shuffle and Merge are used to order and reduce
size of intermediate data. Reduce tasks take over then to determine the
output data format. A final chart illustrates the concept of parallelism
in MapReduce.

\video{Cloud}{11:01}{Hadoop Tasks}{https://www.youtube.com/watch?v=UN4t3tvdjms}

\slides{Cloud}{Page 24}{Hadoop Tasks}{https://drive.google.com/open?id=0B88HKpainTSfMnpCelpNQUpNdVE}
\slides{Cloud}{Page 24}{Hadoop Tasks - pptx}{https://drive.google.com/open?id=0B88HKpainTSfTVlNRzRMemNaZEU}

\section{Fault Tolerance}

Fault tolerance is a natural benefit of MapReduce. The master node pings
worker nodes regularly to verify they are working, and acts accordingly
if they do not respond. A diagram illustrates the files which are in
charge of things like number of map and reduce tasks, and what to do
when the limit is reached on the buffer. The lecture ends with a
discussion of class assignments.

\video{Cloud}{2:45}{Fault Tolerance}{https://www.youtube.com/watch?v=a3AlOTmD42k}

\slides{Cloud}{Page 36}{Fault Tolerance}{https://drive.google.com/open?id=0B88HKpainTSfMnpCelpNQUpNdVE}
\slides{Cloud}{Page 36}{Fault Tolerance - pptx}{https://drive.google.com/open?id=0B88HKpainTSfTVlNRzRMemNaZEU}

\section{Programming on a ComputerCluster}

Hadoop is now a large part of Yahoo!'s system setup, as well as handling
a tremendous variety of data in other areas like medicine and business.
A list of time spans for actions in system requirements is given. The
original MapReduce was designed to resolve problems like load balancing
and machine failures.

\video{Cloud}{6:01}{Programming on a ComputerCluster}{https://www.youtube.com/watch?v=rRR2ALa5CUA}

\slides{Cloud}{Page 1}{Programming on a ComputerCluster}{https://drive.google.com/open?id=0B88HKpainTSfd3hkTG4yY2FYUVE}
\slides{Cloud}{Page 1}{Programming on a ComputerCluster - pptx}{https://drive.google.com/open?id=0B88HKpainTSfcUkwN0l1VHBEdlU}

\section{How Hadoop Runs on a MapReduceJob}

A detailed diagram of the MapReduce job framework is given. This
includes task status updates, shuffling, and writing data to nodes.
MapReduce is a C++ framework, while Hadoop is written in Java. Shuffling
and sorting occurs in the map phase. Reduce reads and writes files to
HDFS, and the merger generates the final result. The second Quiz is
given at the end.

\video{Cloud}{9:25}{How Hadoop Runs on a MapReduceJob}{https://www.youtube.com/watch?v=KWLY_maNEPA}

\slides{Cloud}{Page 8}{How Hadoop Runs on a MapReduceJob}{https://drive.google.com/open?id=0B88HKpainTSfd3hkTG4yY2FYUVE}
\slides{Cloud}{Page 8}{How Hadoop Runs on a MapReduceJob - pptx}{https://drive.google.com/open?id=0B88HKpainTSfcUkwN0l1VHBEdlU}

\section{Literature Review}

This video deals primarily with scientific papers written on the topic
of MapReduce and related programs. There is a certain criteria for
judging scientific submissions. The first paper highlights Google File
System, covering topics like data chunks, metadata, and replicas. This
is followed by MapReduce and BigTable.

\video{Cloud}{9:43}{Literature Review}{https://www.youtube.com/watch?v=5YmjrhEFQsk}

\slides{Cloud}{Page 16}{Literature Review}{https://drive.google.com/open?id=0B88HKpainTSfd3hkTG4yY2FYUVE}
\slides{Cloud}{Page 16}{Literature Review - pptx}{https://drive.google.com/open?id=0B88HKpainTSfcUkwN0l1VHBEdlU}

\section{Introduction to BLAST}

There are four types of programming model communication patterns:
embarrassingly parallel (only map), classic map/reduce, iterative
map/reduce, and loosely synchronous. The basic bioinformatics BLAST
(Basic Local Alignment Sequence Tool) program data flow is illustrated.
An example of database creation comes from the Seattle Children's
Hospital. BLAST uses scores to find similar sequences in databases.

\video{Cloud}{8:27}{Introduction to BLAST}{https://www.youtube.com/watch?v=i3H9HmUYfq8}

\slides{Cloud}{Page 1}{Introduction to BLAST}{https://drive.google.com/open?id=0B88HKpainTSfdnFvY1V3dlFTRlE}
\slides{Cloud}{Page 1}{Introduction to BLAST - pptx}{https://drive.google.com/open?id=0B88HKpainTSfMDAwZm5uQlZWckU}

\section{BLAST Parallelization}

The role of master and worker nodes in BLAST multi-thread usage is
discussed. BLAST can be parallelized in several ways: multi-thread,
query segmentation, and database segmentation. BLAST is pleasingly
parallel in application, but many programs are not. Further information
about articles featuring BLAST is provided at the end.

\video{Cloud}{4:44}{BLAST Parallelization}{https://www.youtube.com/watch?v=isc0MjkwTlk}

\slides{Cloud}{Page 13}{BLAST Parallelization}{https://drive.google.com/open?id=0B88HKpainTSfdnFvY1V3dlFTRlE}
\slides{Cloud}{Page 13}{BLAST Parallelization - pptx}{https://drive.google.com/open?id=0B88HKpainTSfcUkwN0l1VHBEdlU}

\section{SIMD vs MIMD;SPMD vs MPMD}

Four types of parallel models: SISD (traditional PCs), SIMD (GPUs), MISD
(shuttle flight control computer), MIMD (distributed systems).
Point-to-point (P2P) communication in MPI is used as an example of
parallelization. Each successive process adds its own stamp to the data
before passing it on to the next. Matrix multiplication for scientific
applications differs from the norm in that data is sent in a matrix, not
a string. WordCount functions in a map/reduce pattern. These are all
types of SIMD. SPMD and MPMD are two other types of model.

\video{Cloud}{9:42}{SIMD vs MIMD;SPMD vs MPMD}{https://www.youtube.com/watch?v=zHQiR56Zmtc}

\slides{Cloud}{Page 1}{SIMD vs MIMD;SPMD vs MPMD}{https://drive.google.com/open?id=0B88HKpainTSfT28zLTdKYWhGdGM}
\slides{Cloud}{Page 1}{SIMD vs MIMD;SPMD vs MPMD - pptx}{https://drive.google.com/open?id=0B88HKpainTSfVGdyVzNjTzdfb3c}

\section{Data Locality}

A brief review is given of previous topics. As opposed to MPI and HPC,
MapReduce brings the computation to the data, rather than vice-versa.
This is done to limit energy usage and network congestion. Several
factors such as number of nodes and tasks can impact data locality. An
equation to improve data locality is tested in an experiment, whose
results are given. By default, Hadoop determines scheduling of tasks to
available slots in terms of best local composition, not global.

\video{Cloud}{8:36}{Data Locality}{https://www.youtube.com/watch?v=RqLA7_asK50}

\slides{Cloud}{Page 10}{Data Locality}{https://drive.google.com/open?id=0B88HKpainTSfT28zLTdKYWhGdGM}
\slides{Cloud}{Page 10}{Data Locality - pptx}{https://drive.google.com/open?id=0B88HKpainTSfVGdyVzNjTzdfb3c}

\section{Optimal Data Locality}

Global data optimization can be achieved through a proposed algorithm
given here. Task, slot, and cost are factors in this algorithm. Network
bandwidth must also be taken into consideration when assigning tasks to
slots. Linear Sum Assignment Problems require greater time to finish
when matrix size is increased. Two different scheduling algorithms were
designed to improve the original one in Hadoop. An experiment was run
comparing all three, with the network topology-aware algorithm clearly
outperforming the others.

\video{Cloud}{4:17}{Optimal Data Locality}{https://www.youtube.com/watch?v=Ok8vdrFXo5w}

\slides{Cloud}{Page 17}{Optimal Data Locality}{https://drive.google.com/open?id=0B88HKpainTSfT28zLTdKYWhGdGM}
\slides{Cloud}{Page 17}{Optimal Data Locality - pptx}{https://drive.google.com/open?id=0B88HKpainTSfVGdyVzNjTzdfb3c}

\section{Task Granularity}

Size of data blocks affects load balancing and overhead. Using Bag of
Divisible Tasks method, tasks can be split into sub-tasks and
distributed amongst slots to maximize efficiency. When splitting tasks,
one must take into account when and which tasks to split, as well as how
and how many. In our current proposed algorithm, tasks are split until
each slot is occupied. It also uses ASPK (Aggressive Scheduling with
Prior Knowledge) to split larger tasks first and when the performance
gain is deemed optimal. Optimal and Expected Remaining Job Execution
Time can help determine task splitting. Several examples are offered
with either single or multiple jobs.

\video{Cloud}{9:51}{Task Granularity}{https://www.youtube.com/watch?v=u9UpgTnOZz4}

\slides{Cloud}{Page 29}{Task Granularity}{https://drive.google.com/open?id=0B88HKpainTSfT28zLTdKYWhGdGM}
\slides{Cloud}{Page 29}{Task Granularity - pptx}{https://drive.google.com/open?id=0B88HKpainTSfVGdyVzNjTzdfb3c}

\section{Resource Utilization and Speculative Execution}

Resource stealing involves appropriating cores that are kept in reserve
on separate nodes and returning them when the computation is over.
Speculative execution addresses fault tolerance; when the master node
notices a task is running slowly, it will start a speculative task which
can take over if it is determined the original task will not finish in
time. Overuse of speculative tasks can lead to poor data locality and
higher energy demands.

\video{Cloud}{3:52}{Resource Utilization and Speculative Execution}{https://www.youtube.com/watch?v=wWyFiqDIYus}

\slides{Cloud}{Page 46}{Resource Utilization and Speculative Execution}{https://drive.google.com/open?id=0B88HKpainTSfT28zLTdKYWhGdGM}
\slides{Cloud}{Page 46}{Resource Utilization and Speculative Execution - pptx}{https://drive.google.com/open?id=0B88HKpainTSfVGdyVzNjTzdfb3c}

\FILENAME\

\section{How to Store Data (NoSQL)}\label{how-to-store-data-nosql}

\begin{itemize}

\item
  11 Video lectures (1 hour 26 minutes 8 seconds)
\end{itemize}

\subsection{RDBMS vs. NoSQL}\label{rdbms-vs.-nosql}

\begin{itemize}

\item
  Video: \href{https://www.youtube.com/watch?v=dJunqER9lb8}{Youtube}
  (9:22)
\item
  Slide:
  \href{https://drive.google.com/open?id=0B88HKpainTSfaDFNbjNiMm44bnc}{PDF
  (Page 1-10)}
\end{itemize}

\subsection{NoSQL Characteristics}\label{nosql-characteristics}

Clouds have arisen as an answer to the data demands of social media.
Three major programs for NoSQL are BigTable, Dynamo, and CAP theory.
NoSQL is not meant to replace SQL, but to tackle the large-data problems
SQL is not well equipped to handle. SQL ACID transactions are Atomic,
Consistent, Isolated, and Durable. Consistency can be either strong
(ACID) or weak (BASE). CAP theorem offers Consistency, Availability, and
Partition tolerance, only two of which can coexist for a shared-data
system. NoSQL comes in two varieties, each with pros and cons: Key-Value
or schema-less. Common advantages of NoSQL include their being open
source and fault tolerant.

\begin{itemize}

\item
  Video: \href{https://www.youtube.com/watch?v=BjtTDiKhqk8}{Youtube}
  (10:31)
\item
  Slide:
  \href{https://drive.google.com/open?id=0B88HKpainTSfaDFNbjNiMm44bnc}{PDF
  (Page 11-26)}
\end{itemize}

\subsection{BigTable}\label{bigtable}

Big Table is a key-value NoSQL model with data arranged in rows and
columns. It is composed of Data File System, Chubby, and SSTable. A
tablet is a range of rows in BigTable. The master node assigns tablets
to tablet servers and manages these servers. Memory is conserved by
making SSTables and memtables compact. BigTable is used in features of
Google like their search engine and Google Earth.

\begin{itemize}

\item
  Video: \href{https://www.youtube.com/watch?v=JAlz9AI5I-M}{Youtube}
  (6:55)
\item
  Slide:
  \href{https://drive.google.com/open?id=0B88HKpainTSfaDFNbjNiMm44bnc}{PDF
  (Page 28-42)}
\end{itemize}

\subsection{HBase}\label{hbase}

HBase is a NoSQL core component of the Hadoop Distributed File System.
It is a scalable distributed data store. A timeline of HBase and Hadoop
is shown. BigTable still has its uses but does not scale well to large
amounts of analytic processing. HBase has a row-column structure similar
to BigTable as well as master and slave nodes. Its place in the
architecture of HDFS is shown in a diagram.

\begin{itemize}

\item
  Video: \href{https://www.youtube.com/watch?v=i-ibhuVs-ck}{Youtube}
  (7:37)
\item
  Slide:
  \href{https://drive.google.com/open?id=0B88HKpainTSfaDFNbjNiMm44bnc}{PDF
  (Page 44-59)}
\end{itemize}

\subsection{HBase Coding}\label{hbase-coding}

This video gives an overview of the code used in the installation of
HBase and connecting to it.

\begin{itemize}

\item
  Video: \href{https://www.youtube.com/watch?v=KbFMpYRBTtU}{Youtube}
  (4:30)
\item
  Slide:
  \href{https://drive.google.com/open?id=0B88HKpainTSfaDFNbjNiMm44bnc}{PDF
  (Page 60-66)}
\end{itemize}

\subsection{Indexing Applications}\label{indexing-applications}

A brief summary of the course up to this point is given, followed by a
diagram showing the setup of a search engine. Google's search engine
contains three key technologies: Google File System, BigTable, and
MapReduce. However, research into big data remains difficult owing to
the scope of its size. Social media data in particular is a huge source
of data with numerous subsets, all of which demands specific approaches
in terms of search queries. There are three stages to this approach:
query, analysis, and visualization.

\begin{itemize}

\item
  Video: \href{https://www.youtube.com/watch?v=MxgabfoGH-M}{Youtube}
  (9:33)
\item
  Slide:
  \href{https://drive.google.com/open?id=0B88HKpainTSfWUh6dVNHcXloSnc}{PDF
  (Page 1-10)}
\end{itemize}

\subsection{Related Work}\label{related-work}

Indexing improves efficiency in querying data subsets and analysis.
Indices can be single (B+, Hash) or multi-dimensional (R, Quad). Four
databases which utilize indexing are HBase, Cassandra, Riak, and
MongoDB. Current indexing strategies have limits; for instance, they
cannot support range queries or only retrieve Top `n' most relevant
topics. Customizability of indexing among NoSQL databases is desirable.

\begin{itemize}

\item
  Video: \href{https://www.youtube.com/watch?v=NDjAdFSVzxo}{Youtube}
  (5:56)
\item
  Slide:
  \href{https://drive.google.com/open?id=0B88HKpainTSfWUh6dVNHcXloSnc}{PDF
  (Page 11-14)}
\end{itemize}

\subsection{Indexamples}\label{indexamples}

Mapping between metadata and raw index data is the essential issue with
indexing. Examples are shown for HBase, Riak, and MongoDB. An abstract
index structure contains index keys, entry IDs among multiple entries,
and additional fields. Index configuration allows for customizability
through choice of fields, which can be anything from timestamps, text,
or retweet status.

\begin{itemize}

\item
  Video: \href{https://www.youtube.com/watch?v=Ec3VFeTGuo8}{Youtube}
  (8:35)
\item
  Slide:
  \href{https://drive.google.com/open?id=0B88HKpainTSfWUh6dVNHcXloSnc}{PDF
  (Page 15-19)}
\end{itemize}

\subsection{Indexing 101}\label{indexing-101}

User-defined index allows a user to select the fields used in their
search. Data records are indexed or un-indexed. Index structure is made
up of key, entry ID, and entry fields. A walk-through customized index
creation is shown on HBase, called IndexedHBase. HBase is suited to
accommodate the creation of index tables. A performance test of
IndexedHBase is done on the Truthy Twitter repository, displaying the
various tables that can be created with different criteria. Loading time
for large-scale historical data can be reduced by adding nodes.
Streaming data can be handled by increasing loaders. A comparison of
query evaluation is made between IndexedHBase and Riak, with Riak being
more efficient with small data loads but IndexedHBase proving superior
for large-scale data.

\begin{itemize}

\item
  Video: \href{https://www.youtube.com/watch?v=eKQaLkw-HBU}{Youtube}
  (9:53)
\item
  Slide:
  \href{https://drive.google.com/open?id=0B88HKpainTSfWUh6dVNHcXloSnc}{PDF
  (Page 20-27)}
\end{itemize}

\subsection{Social Media Searches}\label{social-media-searches}

The Truthy Project archives social media data by way of metadata memes.
Some problems faced in analyzing this data include its large volume,
sparsity of information in tweets, and attempting to arrange streaming
tweets. Apache Open Stack upgrades Hadoop 2.0 with YARN and a new HDFS.
A diagram displays an indexing setup for social media data with YARN.

\begin{itemize}

\item
  Video: \href{https://www.youtube.com/watch?v=a3tcL-Qw9to}{Youtube}
  (6:19)
\item
  Slide:
  \href{https://drive.google.com/open?id=0B88HKpainTSfWUh6dVNHcXloSnc}{PDF
  (Page 28-34)}
\end{itemize}

\subsection{Analysis Algorithms}\label{analysis-algorithms}

Another method of use for inverted indices is in analysis algorithms.
The mathematics involved in this is explored, as well as how it relates
to index data, mapping, and reducing. Rather than scanning all raw data
present, indices allow for searching only the relevant data. An example
is given illustrating how this decreases the time needed to search
hashtags in Twitter.

\begin{itemize}

\item
  Video: \href{https://www.youtube.com/watch?v=MxoMd4mdshE}{Youtube}
  (6:57)
\item
  Slide:
  \href{https://drive.google.com/open?id=0B88HKpainTSfWUh6dVNHcXloSnc}{PDF
  (Page 35-40)}
\end{itemize}

\section{How to Build a Search Engine
(SaaS)}

  3 Video lectures (26 minutes 18 seconds)

\subsection{Google Components}

\video{Cloud}{7:02}{Google Components}{https://www.youtube.com/watch?v=IWMcv7HbbPM}

  Slide:
  \href{https://drive.google.com/open?id=0B88HKpainTSfYWZ0dDlrNThkVms}{PDF
  (Page 1-5)}

\subsection{Google Architecture}

\video{Cloud}{8:40}{Google Architecture}{https://www.youtube.com/watch?v=syZHezdbdRY}

  Slide:
  \href{https://drive.google.com/open?id=0B88HKpainTSfYWZ0dDlrNThkVms}{PDF
  (Page 6-13)}

\subsection{Google History}

\begin{quote}

  1.7.3. Google History: \url{https://youtu.be/Kg0NK0XUkHw?t=175}
  (starting 2:55)

  1.4.2. Google Search Engine 1:
  \url{https://www.youtube.com/watch?v=S2oT7uMw5Yg}

  1.4.3. Google Search Engine 2:
  \url{https://www.youtube.com/watch?v=pxos3Yt6y6I}

\end{quote}


\video{Cloud}{10:36}{Google History}{https://www.youtube.com/watch?v=Kg0NK0XUkHw}

  Slide:
  \href{https://drive.google.com/open?id=0B88HKpainTSfYWZ0dDlrNThkVms}{PDF
  (Page 14-26)}




%----------------------------------------------------------------------------------------

\end{document}

%


\section{Technology Training - kNN \&
Clustering}\label{technology-training---knn-clustering}
\FILENAME

This section is meant to provide a discussion on the kth Nearest
Neighbor (kNN) algorithm and clustering using K-means. Python version
for kNN is discussed in the video and instructions for both Java and
Python are mentioned in the slides. Plotviz is used for generating 3D
visualizations.

\subsection{Recommender Systems - K-Nearest
Neighbors}\label{recommender-systems---k-nearest-neighbors}

We discuss simple Python k Nearest Neighbor code and its application to
an artificial data set in 3 dimensions. Results are visualized in
Matplotlib in 2D and with Plotviz in 3D. The concept of training and
testing sets are introduced with training set pre-labelled.

Files:

\begin{itemize}

\item
  kNN.py \textless{}/files/python/knn/kNN.py\textgreater{}
\item
  kNN\_Driver.py \textless{}/files/python/knn/kNN\_Driver.py\textgreater{}
\item
  DatingTesting2.txt  \textless{}/files/python/knn/dating\_test\_set2.txt\textgreater{}
\item
  clusterFinal-M3-C3Dating-ReClustered.pviz \textless{}/files/python/knn/clusterFinal-M3-C3Dating-ReClustered.pviz\textgreater{}
\item
  DatingRating-OriginalLabels.pviz \textless{}/files/python/knn/dating\_rating\_original\_labels.pviz\textgreater{}
\item
  clusterFinal-M30-C28.pviz \textless{}/files/python/knn/clusterFinal-M30-C28.pviz\textgreater{}
\end{itemize}

\subsubsection{Python k'th Nearest Neighbor
Algorithms}\label{python-kth-nearest-neighbor-algorithms}

This lesson considers the Python k Nearest Neighbor code found on the
web associated with a book by Harrington on Machine Learning. There are
two data sets. First we consider a set of 4 2D vectors divided into two
categories (clusters) and use k=3 Nearest Neighbor algorithm to classify
3 test points. Second we consider a 3D dataset that has already been
classified and show how to normalize. In this lesson we just use
Matplotlib to give 2D plots.

\subsubsection{3D Visualization}\label{d-visualization}

The lesson modifies the online code to allow it to produce files
readable by PlotViz. We visualize already classified 3D set and rotate
in 3D.

\subsubsection{Testing k'th Nearest Neighbor
Algorithms}\label{testing-kth-nearest-neighbor-algorithms}

The lesson goes through an example of using k NN classification
algorithm by dividing dataset into 2 subsets. One is training set with
initial classification; the other is test point to be classified by k=3
NN using training set. The code records fraction of points with a
different classification from that input. One can experiment with
different sizes of the two subsets. The Python implementation of
algorithm is analyzed in detail.

\subsection{Clustering and heuristic
methods}\label{clustering-and-heuristic-methods}

We use example of recommender system to discuss clustering. The details
of methods are not discussed but k-means based clustering methods are
used and their results examined in Plotviz. The original labelling is
compared to clustering results and extension to 28 clusters given.
General issues in clustering are discussed including local optima, the
use of annealing to avoid this and value of heuristic algorithms.

Files:

\begin{itemize}

\item
  Fungi\_LSU\_3\_15\_to\_3\_26\_zeroidx.pviz \textless{}/files/python/plotviz/fungi\_lsu\_3\_15\_to\_3\_26\_zeroidx.pviz\textgreater{}
\item
  DatingRating-OriginalLabels.pviz \textless{}/files/python/plotviz/datingrating\_originallabels.pviz\textgreater{}
\item
  clusterFinal-M30-C28.pviz \textless{}/files/python/plotviz/clusterFinal-M30-C28.pviz\textgreater{}
\item
  clusterFinal-M3-C3Dating-ReClustered.pviz \textless{}/files/python/plotviz/clusterfinal\_m3\_c3dating\_reclustered.pviz\textgreater{}
\end{itemize}

\subsubsection{Kmeans Clustering}\label{kmeans-clustering}

We introduce the k means algorithm in a gentle fashion and describes its
key features including dangers of local minima. A simple example from
Wikipedia is examined.

\subsubsection{Clustering of Recommender System
Example}\label{clustering-of-recommender-system-example}

Plotviz is used to examine and compare the original classification with
an `'optimal'' clustering into 3 clusters using a fancy deterministic
annealing method that is similar to k means. The new clustering has
centers marked.

\subsubsection{Clustering of Recommender Example into more than 3
Clusters}\label{clustering-of-recommender-example-into-more-than-3-clusters}

The previous division into 3 clusters is compared into a clustering into
28 separate clusters that are naturally smaller in size and divide 3D
space covered by 1000 points into compact geometrically local regions.

\subsubsection{Local Optima in
Clustering}\label{local-optima-in-clustering}

This lesson introduces some general principles. First many important
processes are `'just'' optimization problems. Most such problems are
rife with local optima. The key idea behind annealing to avoid local
optima is described. The pervasive greedy optimization method is
described.

\subsubsection{Clustering in General}\label{clustering-in-general}

The two different applications of clustering are described. First find
geometrically distinct regions and secondly divide spaces into
geometrically compact regions that may have no `'thin air'' between
them. Generalizations such as mixture models and latent factor methods
are just mentioned. The important distinction between applications in
vector spaces and those where only inter-point distances are defined is
described. Examples are then given using PlotViz from 2D clustering of a
mass spectrometry example and the results of clustering genomic data
mapped into 3D with Multi Dimensional Scaling MDS.

\subsubsection{Heuristics}\label{heuristics}

Some remarks are given on heuristics; why are they so important why
getting exact answers is often not so important?

\subsubsection{Resources}\label{resources}

\begin{itemize}

\item
  \url{https://en.wikipedia.org/wiki/Kmeans}
\item
  \url{http://grids.ucs.indiana.edu/ptliupages/publications/DACIDR_camera_ready_v0.3.pdf}
\item
  \url{http://salsahpc.indiana.edu/millionseq/}
\item
  \url{http://salsafungiphy.blogspot.com/}
\item
  \url{https://en.wikipedia.org/wiki/Heuristic}
\end{itemize}




%----------------------------------------------------------------------------------------
%	PART
%----------------------------------------------------------------------------------------

\part{Python}

%----------------------------------------------------------------------------------------
%	CHAPTER 1
%----------------------------------------------------------------------------------------

\chapterimage{chapter_head_2.pdf} % Chapter heading image

\chapter{Introduction}

\FILENAME

\section{Introduction to Python}\label{introduction-to-python}

Portions of this lesson have been adapted from the
\href{https://docs.python.org/2/tutorial/}{official Python Tutorial}
copyright \href{http://www.python.org/}{Python Software Foundation}.

Python is an easy to learn programming language. It has efficient
high-level data structures and a simple but effective approach to
object-oriented programming. Python's simple syntax and dynamic typing,
together with its interpreted nature, make it an ideal language for
scripting and rapid application development in many areas on most
platforms. The Python interpreter and the extensive standard library are
freely available in source or binary form for all major platforms from
the Python Web site, \url{https://www.python.org/}, and may be freely
distributed. The same site also contains distributions of and pointers
to many free third party Python modules, programs and tools, and
additional documentation. The Python interpreter can be extended with
new functions and data types implemented in C or C++ (or other languages
callable from C). Python is also suitable as an extension language for
customizable applications.

Python is an interpreted, dynamic, high-level programming language
suitable for a wide range of applications.

The philosophy of python is summarized in
\href{https://www.python.org/dev/peps/pep-0020/}{The Zen of Python} as
follows:

\begin{itemize}
\tightlist
\item
  Explicit is better than implicit
\item
  Simple is better than complex
\item
  Complex is better than complicated
\item
  Readability counts
\end{itemize}

The main features of Python are:

\begin{itemize}
\tightlist
\item
  Use of indentation whitespace to indicate blocks
\item
  Object orient paradigm
\item
  Dynamic typing
\item
  Interpreted runtime
\item
  Garbage collected memory management
\item
  a large standard library
\item
  a large repository of third-party libraries
\end{itemize}

Python is used by many companies (such as Google, Yahoo!, CERN, NASA)
and is applied for web development, scientific computing, embedded
applications, artificial intelligence, software development, and
information security, to name a few.

The material collected here introduces the reader to the basic concepts and
features of the Python language and system. After you have worked
tthrough the material you will be able to:

\begin{itemize}
\tightlist
\item
  use Python
\item
  use the interactive Python interface
\item
  understand the basic syntax of Python
\item
  write and run Python programs stored in a file
\item
  have an overview of the standard library
\item
  install Python libraries using pyenv or if it is not available
  virtualenv
\end{itemize}

This tutorial does not attempt to be comprehensive and cover every
single feature, or even every commonly used feature. Instead, it
introduces many of Python's most noteworthy features, and will give you
a good idea of the language's flavor and style. After reading it, you
will be able to read and write Python modules and programs, and you will
be ready to learn more about the various Python library modules.

In order to conduct this lesson you need

\begin{itemize}
\tightlist
\item
  A computer with Python 2.7.13 or 3.6.2
\item
  Familiarity with command line usage
\item
  A text editor such as
  \href{https://www.jetbrains.com/pycharm/}{PyCharm}, emacs, vi or
  others. You should identity which works best for you and set it up.
\end{itemize}

\subsection{Links}\label{links}

\begin{itemize}
\tightlist
\item
  \href{https://www.python.org/}{Python}
\item
  \href{https://pip.pypa.io/en/stable/}{Pip}
\item
  \href{https://virtualenv.pypa.io/en/stable/}{Virtualenv}
\item
  \href{http://www.numpy.org/}{NumPy}
\item
  \href{https://scipy.org/}{SciPy}
\item
  \href{http://matplotlib.org/}{Matplotlib}
\item
  \href{http://pandas.pydata.org/}{Pandas}
\item
  \href{https://github.com/pyenv/pyenv}{pyenv}
\item
  \href{https://github.com/pyenv/pyenv}{PyCharm}
\end{itemize}

Python module of the week is a Web site that provides a number of short
examples on how to use some elementary python modules. Not all modules
are equally useful and you should decide if there are better
alternatives. However for beginners this site provides a number of good
examples

\begin{itemize}
\tightlist
\item
  Python 2: \url{https://pymotw.com/2/}
\item
  Python 3: \url{https://pymotw.com/3/}
\end{itemize}


\chapter{Install}

\FILENAME

\section{Python Installation}\label{python-installation}
\index{Python!Install}

Python is easy to install and very good instructions for most platforms
can be found on the python.org Web page. We will be using Python 2.7.13
and/or Python 3 in our activities.

To manage python modules, it is useful to have
\href{https://pypi.python.org/pypi/pip}{pip} package installation tool
on your system.

In the tutorial, we assume that you have a computer with python
installed. However, we also recommend that for the class you use
Python's virtualenv (see below) to isolate your development Python from
the system installed Python.

\subsection{Managing custom Python
installs}\label{managing-custom-python-installs}

Often you have your own computer and you do not like to change its
environment to keep it in pristine condition. Python comes with many
libraries that could for example conflict with libraries that you have
installed. To avoid this it is bets to work in an isolated python we can
use tools such as virtualenv, pyenv or pyvenv for 3.6.4\footnote{check
  for the newest version}. Which you use
depends on you, but we highly recommend pyenv if you can.

\subsubsection{Managing Multiple Python Versions with
Pyenv}
\label{S:managing-multiple-python-versions-with-pyenv}
\index{pyenv}

Python has several versions that are used by the community. This
includes Python 2 and Python 3, but alls different management of the
python libraries. As each OS may have their own version of python
installed. It is not recommended that you modify that version. Instead
you may want to create a localized python installation that you as a
user can modify. To do that we recommend \emph{pyenv}. Pyenv allows
users to switch between multiple versions of Python
(\url{https://github.com/yyuu/pyenv}). To summarize:

\begin{itemize}

\item
  users to change the global Python version on a per-user basis;
\item
  users to enable support for per-project Python versions;
\item
  easy version changes without complex environment variable management;
\item
  to search installed commands across different python versions;
\item
  integrate with tox (\url{https://tox.readthedocs.io/}).
\end{itemize}

\paragraph{Instalation without pyenv}\label{instalation-without-pyenv}

If you need to have more than one python version installed and do not
want or can use pyenv, we recommend you download and install python
2.7.13 and 3.6.4\footnote{check
  for the newest version} from python.org
(\url{https://www.python.org/downloads/})

\paragraph{Disabeling wrong python installs on
OSX}\label{disabeling-wrong-python-installs-on-osx}

While working with students we have seen at times that they take other
classes either at universities or online that teach them how to program
in python. Unfortunately, although they seem to do that they often
ignore to teach you how to properly install python. I just recently had
a students that had installed python 7 times on his OSX machine, while
another student had 3 different installations, all of which confliced
with each other as they were not set up properly.

We recommend that you inspect if you have a files such as
\verb|~/.bashrc| or \verb|~/.bashrc_profile| in your
home directory and identify if it activates various versions of python
on your computer. If so you could try to deactivate them while
out-commenting the various versions with the \# character at the
beginning of the line, start a new terminal and see if the terminal
shell still works. Than you can follow our instructions here while using
an install on pyenv.

\paragraph{Install pyenv on OSX from
git}\label{install-pyenv-on-osx-from-git}

This is our recommended way to install pyenv on OSX:

\begin{verbatim}
$ git clone https://github.com/pyenv/pyenv.git ~/.pyenv
$ git clone https://github.com/pyenv/pyenv-virtualenv.git ~/.pyenv/plugins/pyenv-virtualenv
$ git clone https://github.com/yyuu/pyenv-virtualenvwrapper.git ~/.pyenv/plugins/pyenv-virtualenvwrapper
$ echo 'export PYENV_ROOT="$HOME/.pyenv"' >> ~/.bash_profile
$ echo 'export PATH="$PYENV_ROOT/bin:$PATH"' >> ~/.bash_profile
\end{verbatim}

\paragraph{Installation of Homebrew}\label{instalation-of-homebrew}

Before installing anything on your computer make sure you have enough
space. Use in the terminal the command:

\begin{verbatim}
$ df -h
\end{verbatim}

which gives your an overview of your file system. If you do not have
enough space, please make sure you free up unused files from your drive.

In many occasions it is beneficial to use readline as it provides nice
editing features for the terminal and xz for completion. First, make
sure you have xcode installed:

\begin{verbatim}
$ xcode-select --install
\end{verbatim}

Next install homebrew, pyenv, pyenv-virtualenv and pyenv-virtualwrapper.
Additionally install readline and some compression tools:

\begin{verbatim}
/usr/bin/ruby -e "$(curl -fsSL https://raw.githubusercontent.com/Homebrew/install/master/install)"
brew update
brew install readline xz
\end{verbatim}

\paragraph{Install pyenv on OSX with
Homebrew}\label{install-pyenv-on-osx-with-homebrew}

We describe here a mechanism of installing pyenv with homebrew. Other
mechanisms can be found on the pyenv documentation page
(\url{https://github.com/yyuu/pyenv-installer}). You must have homebrew
installed as discussed in the previous section.

To install pyenv with homebrew execute in the terminal:

\begin{verbatim}
brew install pyenv pyenv-virtualenv pyenv-virtualenvwrapper
\end{verbatim}

\paragraph{Install pyenv on Ubuntu}\label{install-pyenv-on-ubuntu}

The following steps will install pyenv in a new ubuntu 16.04
distribution.

Start up a terminal and execute in the terminal the following commands.
We recommend that you do it one command at a time so you can observe if
the command succeeds:

\begin{verbatim}
$ sudo apt-get update
$ sudo apt-get install git python-pip make build-essential libssl-dev
$ sudo apt-get install zlib1g-dev libbz2-dev libreadline-dev libsqlite3-dev
$ sudo pip install virtualenvwrapper

$ git clone https://github.com/yyuu/pyenv.git ~/.pyenv
$ git clone https://github.com/pyenv/pyenv-virtualenv.git ~/.pyenv/plugins/pyenv-virtualenv   
$ git clone https://github.com/yyuu/pyenv-virtualenvwrapper.git ~/.pyenv/plugins/pyenv-virtualenvwrapper

$ echo 'export PYENV_ROOT="$HOME/.pyenv"' >> ~/.bashrc
$ echo 'export PATH="$PYENV_ROOT/bin:$PATH"' >> ~/.bashrc
\end{verbatim}

Now that you have installed pyenv it is not yet activated in your
current terminal. The easiest thing to do is to start a new terminal and
typ in:

\begin{verbatim}
which pyenv
\end{verbatim}

If you see a response pyenv is installed and you can proceed with the
next steps.

\begin{description}
\item[Please remember whenever you modify \verb|.bashrc| or]
\verb|.bash_profile| you need to start a new terminal.
\end{description}

\paragraph{Install Different Python
Versions}\label{install-different-python-versions}

Pyenv provides a large list of different python versions. To see the
entire list please use the command:

\begin{verbatim}
$ pyenv install -l
\end{verbatim}

However, for us we only need to worry about python 2.7.13 and python
3.6.4\footnote{check
  for the newest version}. You can now
install different versions of python into your local environment with
the following commands:

\begin{verbatim}
$ pyenv install 2.7.13
$ pyenv install 3.6.4
\end{verbatim}

You can set the global python default version with:

\begin{verbatim}
$ pyenv global 2.7.13
\end{verbatim}

Type the following to determine which version you activated:

\begin{verbatim}
$ pyenv version
\end{verbatim}

Type the following to determine which versions you have available:

\begin{verbatim}
$ pyenv versions
\end{verbatim}

Associate a specific environment name with a certain python version, use
the following commands:

\begin{verbatim}
$ pyenv virtualenv 2.7.13 ENV2
$ pyenv virtualenv 3.6.4 ENV3
\end{verbatim}

In the example above, ENV2 would represent python 2.7.13 while ENV3
would represent python 3.6.4. Often it is easier to type the alias
rather than the explicit version.

\paragraph{Set up the Shell}\label{set-up-the-shell}

To make all work smoothly from your terminal, you can include the
following in your \verb|.bashrc| files:

\begin{verbatim}
export PYENV_VIRTUALENV_DISABLE_PROMPT=1
eval "$(pyenv init -)"
eval "$(pyenv virtualenv-init -)"

__pyenv_version_ps1() {
  local ret=$?;
  output=$(pyenv version-name)
  if [[ ! -z $output ]]; then
    echo -n "($output)"
  fi
  return $ret;
}

PS1="\$(__pyenv_version_ps1) ${PS1}"
\end{verbatim}

We recommend that you do this towards the end of your file.

\paragraph{Switching Environments}\label{switching-environments}

After setting up the different environments, switching between them is
now easy. Simply use the following commands:

\begin{verbatim}
(2.7.13) $ pyenv activate ENV2
(ENV2) $ pyenv activate ENV3
(ENV3) $ pyenv activate ENV2
(ENV2) $ pyenv deactivate ENV2
(2.7.13) $ 
\end{verbatim}

To make it even easier, you can add the following lines to your
\verb|.bash_profile| file:

\begin{verbatim}
alias ENV2="pyenv activate ENV2"
alias ENV3="pyenv activate ENV3"
\end{verbatim}

If you start a new terminal, you can switch between the different
versions of python simply by typing:

\begin{verbatim}
$ ENV2
$ ENV3
\end{verbatim}

\subsection{Updating Python Version List}

Pyenv maintains locally a list of available python versions. To see
the list use the command

\begin{lstlisting}
pyenv install -l
\end{lstlisting}

To obtain the newest list please use the command

\begin{lstlisting}
cd ~/.pyenv/plugins/python-build/../.. && git pull
\end{lstlisting}

Now when you call 

\begin{lstlisting}
pyenv install -l
\end{lstlisting}

You will see the updated list.

\subsection{Installation without pyenv}

If you need to have more than one python version installed and do not
want or can use pyenv, we recommend you download and install python
2.7.13 and 3.6.4 from python.org
(\url{https://www.python.org/downloads/})

\subsubsection{Make sure pip is up to date}

As you will want to install other packages, make sure pip is up to date:

\begin{verbatim}
pip install pip -U
\end{verbatim}

pyenv virtualenv anaconda3-4.3.1 ANA3 pyenv activate ANA3

\subsection{Anaconda and Miniconda}\label{anaconda-and-miniconda}

\begin{description}
\item[We do not recommend that you use anaconda or miniconda as it may]
interfere with your default python interpreters and setup.
\end{description}

Please note that beginners to pyton should always use anaconda or
miniconda only afterthey have installed pyenv and use it. For this class
neither anaconda nor miniconda is required. In fact we do not recommend
it. We keep this section as we know that other classes at IU may use
anaconda. We are not aware if these classes teach you the right way to
install it, with \emph{pyenv}.

\subsubsection{Miniconda}\label{miniconda}

\begin{description}
\item[This section about miniconda is experimental and has not]
been tested. We are looking for contributors that help completing it. If
you use anaconda or miniconda we recommend to manage it via pyenv.
\end{description}

To install mini conda you can use the following commands:

\begin{verbatim}
$ mkdir ana
$ cd ana
$ pyenv install miniconda3-latest
$ pyenv local miniconda3-latest
$ pyenv activate miniconda3-latest
$ conda create -n ana anaconda
\end{verbatim}

To activate use:

\begin{verbatim}
$ source activate ana
\end{verbatim}

To deactivate use:

\begin{verbatim}
$ source deactivate
\end{verbatim}

To install cloudmesh cmd5 please use:

\begin{verbatim}
$ pip install cloudmesh.cmd5
$ pip install cloudmesh.sys
\end{verbatim}

\subsubsection{Anaconda}\label{anaconda}

\begin{description}
\item[This section about anaconda is experimental and has not]
been tested. We are looking for contributors that help completing it.
\end{description}

You can add anaconda to your pyenv with the following commands:

\begin{verbatim}
pyenv install anaconda3-4.3.1
\end{verbatim}

To switch more easily we recommend that you use the following in your
\verb|.bash_profile| file:

\begin{verbatim}
alias ANA="pyenv activate anaconda3-4.3.1"
\end{verbatim}

Once you have done this you can easily switch to anaconda with the
command:

\begin{verbatim}
$ ANA
\end{verbatim}

Terminology in annaconda could lead to confusion. Thus we like to point
out that the version number of anaconda is unrelated to the python
version. Furthermore, anaconda uses the term root not for the root user,
but for the originating directory in which the anaconda program is
installed.

In case you like to build your own conda packages at a later time we
recommend that you install the conda-build package:

\begin{verbatim}
$ conda install conda-build
\end{verbatim}

When executing:

\begin{verbatim}
pyenv versions
\end{verbatim}

you will see after the install completed the anaconda versions
installed:

\begin{verbatim}
pyenv versions
system
2.7.13
2.7.13/envs/ENV2
3.6.4
3.6.4/envs/ENV3
ENV2 
ENV3
* anaconda3-4.3.1 (set by PYENV_VERSION environment variable)
\end{verbatim}

Let us now create virtualenv for anaconda:

\begin{verbatim}
$ pyenv virtualenv anaconda3-4.3.1 ANA
\end{verbatim}

To activate it you can now use:

\begin{verbatim}
$ pyenv ANA
\end{verbatim}

However, anaconda may modify your \verb|.bashrc| or \verb|.bash_profile| files and
may result in incompatibilities with other python versions. For this
reason we recommend not to use it. If you find ways to get it to work
reliably with other versions, please let us know and we update this
tutorial.

To install cloudmesh cmd5 please use:

\begin{verbatim}
$ pip install cloudmesh.cmd5
$ pip install cloudmesh.sys
\end{verbatim}

\paragraph{Exercise}

\begin{exercise}
Write installation instructions for an operating system of your choice
and add to this documentation.
\end{exercise}

\begin{exercise}
Replicate the steps above, so you can type in ENV2 and ENV3 in your
terminals to switch between python 2 and 3.
\end{exercise}

\subsubsection{virtualenv}\label{virtualenv}
\index{virtualenv}

environment while using virtualenv,. Documentation about it can be found
at:

\begin{verbatim}
* https://virtualenv.pypa.io
\end{verbatim}

The installation is simple once you have pip installed. If it is not
installed you can say:

\begin{verbatim}
$ easy_install pip
\end{verbatim}

After that you can install the virtual env with:

\begin{verbatim}
$ pip install virtualenv
\end{verbatim}

To setup an isolated environment for example in the directory
\textasciitilde{}/ENV please use:

\begin{verbatim}
$ virtualenv ~/ENV
\end{verbatim}

To activate it you can use the command:

\begin{verbatim}
$ source ~/ENV/bin/activate
\end{verbatim}

you can put this command in your \verb|.bashrc| or \verb|.bash_profile| files so you
do not forget to activate it. Instructions for this can be
found in our lesson on Linux \verb|bashrc|.


\chapter{Language}

%----------------------------------------------------------------------------------------
%	PART
%----------------------------------------------------------------------------------------
\chapterimage{images/python.jpeg} % Chapter heading image


\part{Python}


%----------------------------------------------------------------------------------------
%	CHAPTER 1
%----------------------------------------------------------------------------------------
\chapter{Introduction}
\label{C:python}

\FILENAME

\FILENAME

\section{Introduction to Python}\label{introduction-to-python}

Portions of this lesson have been adapted from the
\href{https://docs.python.org/2/tutorial/}{official Python Tutorial}
copyright \href{http://www.python.org/}{Python Software Foundation}.

Python is an easy to learn programming language. It has efficient
high-level data structures and a simple but effective approach to
object-oriented programming. Python's simple syntax and dynamic typing,
together with its interpreted nature, make it an ideal language for
scripting and rapid application development in many areas on most
platforms. The Python interpreter and the extensive standard library are
freely available in source or binary form for all major platforms from
the Python Web site, \url{https://www.python.org/}, and may be freely
distributed. The same site also contains distributions of and pointers
to many free third party Python modules, programs and tools, and
additional documentation. The Python interpreter can be extended with
new functions and data types implemented in C or C++ (or other languages
callable from C). Python is also suitable as an extension language for
customizable applications.

Python is an interpreted, dynamic, high-level programming language
suitable for a wide range of applications.

The philosophy of python is summarized in
\href{https://www.python.org/dev/peps/pep-0020/}{The Zen of Python} as
follows:

\begin{itemize}
\tightlist
\item
  Explicit is better than implicit
\item
  Simple is better than complex
\item
  Complex is better than complicated
\item
  Readability counts
\end{itemize}

The main features of Python are:

\begin{itemize}
\tightlist
\item
  Use of indentation whitespace to indicate blocks
\item
  Object orient paradigm
\item
  Dynamic typing
\item
  Interpreted runtime
\item
  Garbage collected memory management
\item
  a large standard library
\item
  a large repository of third-party libraries
\end{itemize}

Python is used by many companies (such as Google, Yahoo!, CERN, NASA)
and is applied for web development, scientific computing, embedded
applications, artificial intelligence, software development, and
information security, to name a few.

The material collected here introduces the reader to the basic concepts and
features of the Python language and system. After you have worked
tthrough the material you will be able to:

\begin{itemize}
\tightlist
\item
  use Python
\item
  use the interactive Python interface
\item
  understand the basic syntax of Python
\item
  write and run Python programs stored in a file
\item
  have an overview of the standard library
\item
  install Python libraries using pyenv or if it is not available
  virtualenv
\end{itemize}

This tutorial does not attempt to be comprehensive and cover every
single feature, or even every commonly used feature. Instead, it
introduces many of Python's most noteworthy features, and will give you
a good idea of the language's flavor and style. After reading it, you
will be able to read and write Python modules and programs, and you will
be ready to learn more about the various Python library modules.

In order to conduct this lesson you need

\begin{itemize}
\tightlist
\item
  A computer with Python 2.7.13 or 3.6.2
\item
  Familiarity with command line usage
\item
  A text editor such as
  \href{https://www.jetbrains.com/pycharm/}{PyCharm}, emacs, vi or
  others. You should identity which works best for you and set it up.
\end{itemize}

\subsection{Links}\label{links}

\begin{itemize}
\tightlist
\item
  \href{https://www.python.org/}{Python}
\item
  \href{https://pip.pypa.io/en/stable/}{Pip}
\item
  \href{https://virtualenv.pypa.io/en/stable/}{Virtualenv}
\item
  \href{http://www.numpy.org/}{NumPy}
\item
  \href{https://scipy.org/}{SciPy}
\item
  \href{http://matplotlib.org/}{Matplotlib}
\item
  \href{http://pandas.pydata.org/}{Pandas}
\item
  \href{https://github.com/pyenv/pyenv}{pyenv}
\item
  \href{https://github.com/pyenv/pyenv}{PyCharm}
\end{itemize}

Python module of the week is a Web site that provides a number of short
examples on how to use some elementary python modules. Not all modules
are equally useful and you should decide if there are better
alternatives. However for beginners this site provides a number of good
examples

\begin{itemize}
\tightlist
\item
  Python 2: \url{https://pymotw.com/2/}
\item
  Python 3: \url{https://pymotw.com/3/}
\end{itemize}


\chapter{Install}
\label{C:python-install}

\FILENAME

\section{Python Installation}\label{python-installation}
\index{Python!Install}

Python is easy to install and very good instructions for most platforms
can be found on the python.org Web page. We will be using Python 2.7.13
and/or Python 3 in our activities.

To manage python modules, it is useful to have
\href{https://pypi.python.org/pypi/pip}{pip} package installation tool
on your system.

In the tutorial, we assume that you have a computer with python
installed. However, we also recommend that for the class you use
Python's virtualenv (see below) to isolate your development Python from
the system installed Python.

\subsection{Managing custom Python
installs}\label{managing-custom-python-installs}

Often you have your own computer and you do not like to change its
environment to keep it in pristine condition. Python comes with many
libraries that could for example conflict with libraries that you have
installed. To avoid this it is bets to work in an isolated python we can
use tools such as virtualenv, pyenv or pyvenv for 3.6.4\footnote{check
  for the newest version}. Which you use
depends on you, but we highly recommend pyenv if you can.

\subsubsection{Managing Multiple Python Versions with
Pyenv}
\label{S:managing-multiple-python-versions-with-pyenv}
\index{pyenv}

Python has several versions that are used by the community. This
includes Python 2 and Python 3, but alls different management of the
python libraries. As each OS may have their own version of python
installed. It is not recommended that you modify that version. Instead
you may want to create a localized python installation that you as a
user can modify. To do that we recommend \emph{pyenv}. Pyenv allows
users to switch between multiple versions of Python
(\url{https://github.com/yyuu/pyenv}). To summarize:

\begin{itemize}

\item
  users to change the global Python version on a per-user basis;
\item
  users to enable support for per-project Python versions;
\item
  easy version changes without complex environment variable management;
\item
  to search installed commands across different python versions;
\item
  integrate with tox (\url{https://tox.readthedocs.io/}).
\end{itemize}

\paragraph{Instalation without pyenv}\label{instalation-without-pyenv}

If you need to have more than one python version installed and do not
want or can use pyenv, we recommend you download and install python
2.7.13 and 3.6.4\footnote{check
  for the newest version} from python.org
(\url{https://www.python.org/downloads/})

\paragraph{Disabeling wrong python installs on
OSX}\label{disabeling-wrong-python-installs-on-osx}

While working with students we have seen at times that they take other
classes either at universities or online that teach them how to program
in python. Unfortunately, although they seem to do that they often
ignore to teach you how to properly install python. I just recently had
a students that had installed python 7 times on his OSX machine, while
another student had 3 different installations, all of which confliced
with each other as they were not set up properly.

We recommend that you inspect if you have a files such as
\verb|~/.bashrc| or \verb|~/.bashrc_profile| in your
home directory and identify if it activates various versions of python
on your computer. If so you could try to deactivate them while
out-commenting the various versions with the \# character at the
beginning of the line, start a new terminal and see if the terminal
shell still works. Than you can follow our instructions here while using
an install on pyenv.

\paragraph{Install pyenv on OSX from
git}\label{install-pyenv-on-osx-from-git}

This is our recommended way to install pyenv on OSX:

\begin{verbatim}
$ git clone https://github.com/pyenv/pyenv.git ~/.pyenv
$ git clone https://github.com/pyenv/pyenv-virtualenv.git ~/.pyenv/plugins/pyenv-virtualenv
$ git clone https://github.com/yyuu/pyenv-virtualenvwrapper.git ~/.pyenv/plugins/pyenv-virtualenvwrapper
$ echo 'export PYENV_ROOT="$HOME/.pyenv"' >> ~/.bash_profile
$ echo 'export PATH="$PYENV_ROOT/bin:$PATH"' >> ~/.bash_profile
\end{verbatim}

\paragraph{Installation of Homebrew}\label{instalation-of-homebrew}

Before installing anything on your computer make sure you have enough
space. Use in the terminal the command:

\begin{verbatim}
$ df -h
\end{verbatim}

which gives your an overview of your file system. If you do not have
enough space, please make sure you free up unused files from your drive.

In many occasions it is beneficial to use readline as it provides nice
editing features for the terminal and xz for completion. First, make
sure you have xcode installed:

\begin{verbatim}
$ xcode-select --install
\end{verbatim}

Next install homebrew, pyenv, pyenv-virtualenv and pyenv-virtualwrapper.
Additionally install readline and some compression tools:

\begin{verbatim}
/usr/bin/ruby -e "$(curl -fsSL https://raw.githubusercontent.com/Homebrew/install/master/install)"
brew update
brew install readline xz
\end{verbatim}

\paragraph{Install pyenv on OSX with
Homebrew}\label{install-pyenv-on-osx-with-homebrew}

We describe here a mechanism of installing pyenv with homebrew. Other
mechanisms can be found on the pyenv documentation page
(\url{https://github.com/yyuu/pyenv-installer}). You must have homebrew
installed as discussed in the previous section.

To install pyenv with homebrew execute in the terminal:

\begin{verbatim}
brew install pyenv pyenv-virtualenv pyenv-virtualenvwrapper
\end{verbatim}

\paragraph{Install pyenv on Ubuntu}\label{install-pyenv-on-ubuntu}

The following steps will install pyenv in a new ubuntu 16.04
distribution.

Start up a terminal and execute in the terminal the following commands.
We recommend that you do it one command at a time so you can observe if
the command succeeds:

\begin{verbatim}
$ sudo apt-get update
$ sudo apt-get install git python-pip make build-essential libssl-dev
$ sudo apt-get install zlib1g-dev libbz2-dev libreadline-dev libsqlite3-dev
$ sudo pip install virtualenvwrapper

$ git clone https://github.com/yyuu/pyenv.git ~/.pyenv
$ git clone https://github.com/pyenv/pyenv-virtualenv.git ~/.pyenv/plugins/pyenv-virtualenv   
$ git clone https://github.com/yyuu/pyenv-virtualenvwrapper.git ~/.pyenv/plugins/pyenv-virtualenvwrapper

$ echo 'export PYENV_ROOT="$HOME/.pyenv"' >> ~/.bashrc
$ echo 'export PATH="$PYENV_ROOT/bin:$PATH"' >> ~/.bashrc
\end{verbatim}

Now that you have installed pyenv it is not yet activated in your
current terminal. The easiest thing to do is to start a new terminal and
typ in:

\begin{verbatim}
which pyenv
\end{verbatim}

If you see a response pyenv is installed and you can proceed with the
next steps.

\begin{description}
\item[Please remember whenever you modify \verb|.bashrc| or]
\verb|.bash_profile| you need to start a new terminal.
\end{description}

\paragraph{Install Different Python
Versions}\label{install-different-python-versions}

Pyenv provides a large list of different python versions. To see the
entire list please use the command:

\begin{verbatim}
$ pyenv install -l
\end{verbatim}

However, for us we only need to worry about python 2.7.13 and python
3.6.4\footnote{check
  for the newest version}. You can now
install different versions of python into your local environment with
the following commands:

\begin{verbatim}
$ pyenv install 2.7.13
$ pyenv install 3.6.4
\end{verbatim}

You can set the global python default version with:

\begin{verbatim}
$ pyenv global 2.7.13
\end{verbatim}

Type the following to determine which version you activated:

\begin{verbatim}
$ pyenv version
\end{verbatim}

Type the following to determine which versions you have available:

\begin{verbatim}
$ pyenv versions
\end{verbatim}

Associate a specific environment name with a certain python version, use
the following commands:

\begin{verbatim}
$ pyenv virtualenv 2.7.13 ENV2
$ pyenv virtualenv 3.6.4 ENV3
\end{verbatim}

In the example above, ENV2 would represent python 2.7.13 while ENV3
would represent python 3.6.4. Often it is easier to type the alias
rather than the explicit version.

\paragraph{Set up the Shell}\label{set-up-the-shell}

To make all work smoothly from your terminal, you can include the
following in your \verb|.bashrc| files:

\begin{verbatim}
export PYENV_VIRTUALENV_DISABLE_PROMPT=1
eval "$(pyenv init -)"
eval "$(pyenv virtualenv-init -)"

__pyenv_version_ps1() {
  local ret=$?;
  output=$(pyenv version-name)
  if [[ ! -z $output ]]; then
    echo -n "($output)"
  fi
  return $ret;
}

PS1="\$(__pyenv_version_ps1) ${PS1}"
\end{verbatim}

We recommend that you do this towards the end of your file.

\paragraph{Switching Environments}\label{switching-environments}

After setting up the different environments, switching between them is
now easy. Simply use the following commands:

\begin{verbatim}
(2.7.13) $ pyenv activate ENV2
(ENV2) $ pyenv activate ENV3
(ENV3) $ pyenv activate ENV2
(ENV2) $ pyenv deactivate ENV2
(2.7.13) $ 
\end{verbatim}

To make it even easier, you can add the following lines to your
\verb|.bash_profile| file:

\begin{verbatim}
alias ENV2="pyenv activate ENV2"
alias ENV3="pyenv activate ENV3"
\end{verbatim}

If you start a new terminal, you can switch between the different
versions of python simply by typing:

\begin{verbatim}
$ ENV2
$ ENV3
\end{verbatim}

\subsection{Updating Python Version List}

Pyenv maintains locally a list of available python versions. To see
the list use the command

\begin{lstlisting}
pyenv install -l
\end{lstlisting}

To obtain the newest list please use the command

\begin{lstlisting}
cd ~/.pyenv/plugins/python-build/../.. && git pull
\end{lstlisting}

Now when you call 

\begin{lstlisting}
pyenv install -l
\end{lstlisting}

You will see the updated list.

\subsection{Installation without pyenv}

If you need to have more than one python version installed and do not
want or can use pyenv, we recommend you download and install python
2.7.13 and 3.6.4 from python.org
(\url{https://www.python.org/downloads/})

\subsubsection{Make sure pip is up to date}

As you will want to install other packages, make sure pip is up to date:

\begin{verbatim}
pip install pip -U
\end{verbatim}

pyenv virtualenv anaconda3-4.3.1 ANA3 pyenv activate ANA3

\subsection{Anaconda and Miniconda}\label{anaconda-and-miniconda}

\begin{description}
\item[We do not recommend that you use anaconda or miniconda as it may]
interfere with your default python interpreters and setup.
\end{description}

Please note that beginners to pyton should always use anaconda or
miniconda only afterthey have installed pyenv and use it. For this class
neither anaconda nor miniconda is required. In fact we do not recommend
it. We keep this section as we know that other classes at IU may use
anaconda. We are not aware if these classes teach you the right way to
install it, with \emph{pyenv}.

\subsubsection{Miniconda}\label{miniconda}

\begin{description}
\item[This section about miniconda is experimental and has not]
been tested. We are looking for contributors that help completing it. If
you use anaconda or miniconda we recommend to manage it via pyenv.
\end{description}

To install mini conda you can use the following commands:

\begin{verbatim}
$ mkdir ana
$ cd ana
$ pyenv install miniconda3-latest
$ pyenv local miniconda3-latest
$ pyenv activate miniconda3-latest
$ conda create -n ana anaconda
\end{verbatim}

To activate use:

\begin{verbatim}
$ source activate ana
\end{verbatim}

To deactivate use:

\begin{verbatim}
$ source deactivate
\end{verbatim}

To install cloudmesh cmd5 please use:

\begin{verbatim}
$ pip install cloudmesh.cmd5
$ pip install cloudmesh.sys
\end{verbatim}

\subsubsection{Anaconda}\label{anaconda}

\begin{description}
\item[This section about anaconda is experimental and has not]
been tested. We are looking for contributors that help completing it.
\end{description}

You can add anaconda to your pyenv with the following commands:

\begin{verbatim}
pyenv install anaconda3-4.3.1
\end{verbatim}

To switch more easily we recommend that you use the following in your
\verb|.bash_profile| file:

\begin{verbatim}
alias ANA="pyenv activate anaconda3-4.3.1"
\end{verbatim}

Once you have done this you can easily switch to anaconda with the
command:

\begin{verbatim}
$ ANA
\end{verbatim}

Terminology in annaconda could lead to confusion. Thus we like to point
out that the version number of anaconda is unrelated to the python
version. Furthermore, anaconda uses the term root not for the root user,
but for the originating directory in which the anaconda program is
installed.

In case you like to build your own conda packages at a later time we
recommend that you install the conda-build package:

\begin{verbatim}
$ conda install conda-build
\end{verbatim}

When executing:

\begin{verbatim}
pyenv versions
\end{verbatim}

you will see after the install completed the anaconda versions
installed:

\begin{verbatim}
pyenv versions
system
2.7.13
2.7.13/envs/ENV2
3.6.4
3.6.4/envs/ENV3
ENV2 
ENV3
* anaconda3-4.3.1 (set by PYENV_VERSION environment variable)
\end{verbatim}

Let us now create virtualenv for anaconda:

\begin{verbatim}
$ pyenv virtualenv anaconda3-4.3.1 ANA
\end{verbatim}

To activate it you can now use:

\begin{verbatim}
$ pyenv ANA
\end{verbatim}

However, anaconda may modify your \verb|.bashrc| or \verb|.bash_profile| files and
may result in incompatibilities with other python versions. For this
reason we recommend not to use it. If you find ways to get it to work
reliably with other versions, please let us know and we update this
tutorial.

To install cloudmesh cmd5 please use:

\begin{verbatim}
$ pip install cloudmesh.cmd5
$ pip install cloudmesh.sys
\end{verbatim}

\paragraph{Exercise}

\begin{exercise}
Write installation instructions for an operating system of your choice
and add to this documentation.
\end{exercise}

\begin{exercise}
Replicate the steps above, so you can type in ENV2 and ENV3 in your
terminals to switch between python 2 and 3.
\end{exercise}

\subsubsection{virtualenv}\label{virtualenv}
\index{virtualenv}

environment while using virtualenv,. Documentation about it can be found
at:

\begin{verbatim}
* https://virtualenv.pypa.io
\end{verbatim}

The installation is simple once you have pip installed. If it is not
installed you can say:

\begin{verbatim}
$ easy_install pip
\end{verbatim}

After that you can install the virtual env with:

\begin{verbatim}
$ pip install virtualenv
\end{verbatim}

To setup an isolated environment for example in the directory
\textasciitilde{}/ENV please use:

\begin{verbatim}
$ virtualenv ~/ENV
\end{verbatim}

To activate it you can use the command:

\begin{verbatim}
$ source ~/ENV/bin/activate
\end{verbatim}

you can put this command in your \verb|.bashrc| or \verb|.bash_profile| files so you
do not forget to activate it. Instructions for this can be
found in our lesson on Linux \verb|bashrc|.


\FILENAME

\section{Interactive Python}\label{interactive-python}
\index{Python!REPL}
\index{Python!interactive}

Python can be used interactively. Start by entering the interactive loop
by executing the command:

\begin{verbatim}
$ python
\end{verbatim}

You should see something like the following:

\begin{verbatim}
Python 2.7.13 (default, Nov 19 2016, 06:48:10)
[GCC 5.4.0 20160609] on linux2
Type "help", "copyright", "credits" or "license" for more information.
>>>
\end{verbatim}

The \textgreater{}\textgreater{}\textgreater{} is the prompt for the
interpreter. This is similar to the shell interpreter you have been
using.

Often we show the prompt when illustrating an example. This is to
provide some context for what we are doing. If you are following along
you will not need to type in the prompt.

This interactive prompt does the following:

\begin{itemize}
\tightlist
\item
  \emph{read} your input commands
\item
  \emph{evaluate} your command
\item
  \emph{print} the result of evaluation
\item
  \emph{loop} back to the beginning.
\end{itemize}

This is why you may see the interactive loop referred to as a
\textbf{REPL}:
\textbf{R}ead-\textbf{E}valuate-\textbf{P}rint-\textbf{L}oop.

\section{REPL (Read Eval Print Loop)}\label{repl-read-eval-print-loop}
\index{Python!REPL}

We have so far seen a few examples of types: \textbf{string}s,
\textbf{bool}s, \textbf{int}s, and \textbf{float}s. A \textbf{type}
indicates that values of that type support a certain set of operations.
For instance, how would you exponentiate a string? If you ask the
interpreter, this results in an error:

\begin{verbatim}
>>> "hello"**3
Traceback (most recent call last):
  File "<stdin>", line 1, in <module>
TypeError: unsupported operand type(s) for ** or pow(): 'str' and 'int'
\end{verbatim}

There are many different types beyond what we have seen so far, such as
\textbf{dictionaries}s, \textbf{list}s, \textbf{set}s. One handy way of
using the interactive python is to get the type of a value using
`type():

::

   \textgreater{}\textgreater{}\textgreater{} type(42)
   \textless{}type 'int'\textgreater{}
   \textgreater{}\textgreater{}\textgreater{} type(hello)
   \textless{}type 'str'\textgreater{}
   \textgreater{}\textgreater{}\textgreater{} type(3.14)
   \textless{}type 'float'\textgreater{}

You can also ask for help about something using help():

::

   \textgreater{}\textgreater{}\textgreater{} help(int)
   \textgreater{}\textgreater{}\textgreater{} help(list)
   \textgreater{}\textgreater{}\textgreater{} help(str)

.. tip::

   Using help()` opens up a pager. To navigate you can use the spacebar
to go down a page w to go up a page, the arrow keys to go up/down
line-by-line, or q to exit.

\section{Python 3 Features in Python 2}\label{python-3-features-in-python-2}
\index{Python!2 and 3}


As mentioned earlier, we assume you will use Python 2.7.X because there
are still some libraries that haven't been ported to Python 3. However,
there are some features of Python 3 we can and want to use in Python
2.7. Before we do anything else, we need to make these features
available to any subsequent code we write:

\begin{verbatim}
>>> from __future__ import print_function, division
\end{verbatim}

The first of these imports allows us to use the print function to output
text to the screen, instead of the print statement, which Python 2 uses.
This is simply a \href{https://www.python.org/dev/peps/pep-3105/}{design
decision} that better reflects Python's underlying philosophy.

The second of these imports makes sure that the
\href{https://www.python.org/dev/peps/pep-0238/}{division operator}
behaves in a way a newcomer to the language might find more intruitive.
In Python 2, division / is \emph{floor division} when the arguments are
integers, meaning that 5 / 2 == 2, for example. In Python 3, division /
is \emph{true division}, thus 5 / 2 == 2.5.


\chapter{Language}
\label{C:python-language}
\FILENAME

\section{Statements and Strings}\label{statements-and-strings}
\index{Python!statements}
\index{Python!strings}

Let us explore the syntax of Python. Type into the interactive loop and
press Enter:

\begin{verbatim}
>>> print("Hello world from Python!")
Hello world from Python!
\end{verbatim}

What happened: the print function was given a \textbf{string} to
process. A string is a sequence of characters. A \textbf{character} can
be a alphabetic (A through Z, lower and upper case), numeric (any of the
digits), white space (spaces, tabs, newlines, etc), syntactic directives
(comma, colon, quotation, exclamation, etc), and so forth. A string is
just a sequence of the character and typically indicated by surrounding
the characters in double quotes.

Standard output is discussed in the ../../lesson/linux/shell lesson.

So, what happened when you pressed Enter? The interactive Python program
read the line print "Hello world from Python!", split it into the print
statement and the "Hello world from Python!" string, and then executed
the line, showing you the output.

\section{Variables}\label{variables}
\index{Python!variables}

You can store data into a \textbf{variable} to access it later. For
instance, instead of:

\begin{verbatim}
>>> print('Hello world from Python!')
\end{verbatim}

which is a lot to type if you need to do it multiple times, you can
store the string in a variable for convenient access:

\begin{verbatim}
>>> hello = 'Hello world from Python!'
>>> print(hello)
Hello world from Python!
\end{verbatim}

\section{Data Types}\label{data-types}
\index{Python!data types}

\subsection{Booleans}\label{booleans}

A \textbf{boolean} is a value that indicates \emph{truthness} of
something. You can think of it as a toggle: either ``on'' or ``off'',
``one'' or ``zero'', ``true'' or ``false''. In fact, the only possible
values of the \textbf{boolean} (or bool) type in Python are:

\begin{itemize}
\tightlist
\item
  True
\item
  False
\end{itemize}

You can combine booleans with \textbf{boolean operators}:

\begin{itemize}
\tightlist
\item
  and
\item
  or
\end{itemize}

\begin{verbatim}
>>> print(True and True)
True
>>> print(True and False)
False
>>> print(False and False)
False
>>> print(True or True)
True
>>> print(True or False)
True
>>> print(False or False)
False
\end{verbatim}

\subsection{Numbers}\label{numbers}
\index{Python!numbers}

The interactive interpreter can also be used as a calculator. For
instance, say we wanted to compute a multiple of 21:

\begin{verbatim}
>>> print(21 * 2)
42
\end{verbatim}

We saw here the print statement again. We passed in the result of the
operation 21 * 2. An \textbf{integer} (or \textbf{int}) in Python is a
numeric value without a fractional component (those are called
\textbf{floating point} numbers, or \textbf{float} for short).

The mathematical operators compute the related mathematical operation to
the provided numbers. Some operators are:

\begin{itemize}
\tightlist
\item
  * --- multiplication
\item
  / --- division
\item
  + --- addition
\item
  - --- subtraction
\item
  ** --- exponent
\end{itemize}

Exponentiation is read as x**y is x to the yth power:

\[x^y\]

You can combine \textbf{float}s and \textbf{int}s:

\begin{verbatim}
>>> print(3.14 * 42 / 11 + 4 - 2)
13.9890909091
>>> print(2**3)
8
\end{verbatim}

Note that \textbf{operator precedence} is important. Using parenthesis
to indicate affect the order of operations gives a difference results,
as expected:

\begin{verbatim}
>>> print(3.14 * (42 / 11) + 4 - 2)
11.42
>>> print(1 + 2 * 3 - 4 / 5.0)
6.2
>>> print( (1 + 2) * (3 - 4) / 5.0 )
-0.6
\end{verbatim}



\section{Module Management}\label{module-management}

A module allows you to logically organize your Python code. Grouping
related code into a module makes the code easier to understand and use.
A module is a Python object with arbitrarily named attributes that you
can bind and reference. A module is a file consisting of Python code. A
module can define functions, classes and variables. A module can also
include runnable code.

\subsection{Import Statement}\label{import-statement}

\begin{quote}
When the interpreter encounters an import statement, it imports the
module if the module is present in the search path. A search path is a
list of directories that the interpreter searches before importing a
module. The from\ldots{}import Statement Python's from statement lets
you import specific attributes from a module into the current namespace.
The from\ldots{}import has the following syntax - from modname:
\end{quote}

import name1{[}, name2{[}, \ldots{} nameN{]}{]}

When the interpreter encounters an import statement, it imports the
module if the module is present in the search path. A search path is a
list of directories that the interpreter searches before importing a
module.

\subsection{The from \ldots{} import
Statement}\label{the-from-import-statement}

Python's from statement lets you import specific attributes from a
module into the current namespace. The from \ldots{} import has the
following syntax:

\begin{verbatim}
::
\end{verbatim}

\begin{quote}
from module1 import name1{[}, name2{[}, \ldots{} nameN{]}{]}
\end{quote}

\section{Date Time in Python}\label{date-time-in-python}

The datetime module supplies classes for manipulating dates and times in
both simple and complex ways. While date and time arithmetic is
supported, the focus of the implementation is on efficient attribute
extraction for output formatting and manipulation. For related
functionality, see also the time and calendar modules.

The import Statement You can use any Python source file as a module by
executing an import statement in some other Python source file.

\begin{verbatim}
>>>from datetime import datetime
\end{verbatim}

This module offers a generic date/time string parser which is able to
parse most known formats to represent a date and/or time.

\begin{verbatim}
>>>from dateutil.parser import parse
\end{verbatim}

pandas is an open source Python library for data analysis that needs to
be imported.

\begin{verbatim}
>>>import pandas as pd
\end{verbatim}

Create a string variable with the class start time

\begin{verbatim}
>>>fall_start = '08-21-2017'
\end{verbatim}

Convert the string to datetime format

\begin{verbatim}
>>>datetime.strptime(fall_start, '%m-%d-%Y')
datetime.datetime(2017, 8, 21, 0, 0)
\end{verbatim}

Creating a list of strings as dates

\begin{verbatim}
>>>class_dates = ['8/25/2017', '9/1/2017', '9/8/2017', '9/15/2017', '9/22/2017', '9/29/2017']
\end{verbatim}

Convert Class\_dates strings into datetime format and save the list into
variable a

\begin{verbatim}
>>>a = [datetime.strptime(x, '%m/%d/%Y') for x in class_dates]
\end{verbatim}

Use parse() to attempt to auto-convert common string formats. Parser
must be a string or character stream, not list.

\begin{verbatim}
>>>parse(fall_start)
datetime.datetime(2017, 8, 21, 0, 0)
\end{verbatim}

Use parse() on every element of the Class\_dates string.

\begin{verbatim}
>>>[parse(x) for x in class_dates] 
[datetime.datetime(2017, 8, 25, 0, 0),
 datetime.datetime(2017, 9, 1, 0, 0),
 datetime.datetime(2017, 9, 8, 0, 0),
 datetime.datetime(2017, 9, 15, 0, 0),
 datetime.datetime(2017, 9, 22, 0, 0),
 datetime.datetime(2017, 9, 29, 0, 0)]  
\end{verbatim}

Use parse, but designate that the day is first.

\begin{verbatim}
>>>parse (fall_start, dayfirst=True)
datetime.datetime(2017, 8, 21, 0, 0)
\end{verbatim}

Create a dataframe.A DataFrame is a tablular data structure comprised of
rows and columns, akin to a spreadsheet, database table. DataFrame as a
group of Series objects that share an index (the column names).

\begin{verbatim}
>>>import pandas as pd
>>>data = {'class_dates': ['8/25/2017 18:47:05.069722', '9/1/2017 18:47:05.119994', 
                        '9/8/2017 18:47:05.178768', '9/15/2017 18:47:05.230071', 
                        '9/22/2017 18:47:05.230071', '9/29/2017 18:47:05.280592'], 
        'complete': [1, 0, 1, 1, 0, 1]} 
>>>df = pd.DataFrame(data, columns = ['class_dates', 'complete'])
>>>print(df)
                 class_dates  complete
0  8/25/2017 18:47:05.069722         1
1   9/1/2017 18:47:05.119994         0
2   9/8/2017 18:47:05.178768         1
3  9/15/2017 18:47:05.230071         1
4  9/22/2017 18:47:05.230071         0
5  9/29/2017 18:47:05.280592         1
\end{verbatim}

Convert df{[}`date'{]} from string to datetime

\begin{verbatim}
>>>import pandas as pd
>>>pd.to_datetime(df['class_dates'])
0   2017-08-25 18:47:05.069722
1   2017-09-01 18:47:05.119994
2   2017-09-08 18:47:05.178768
3   2017-09-15 18:47:05.230071
4   2017-09-22 18:47:05.230071
5   2017-09-29 18:47:05.280592
Name: class_dates, dtype: datetime64[ns]
\end{verbatim}

\section{Control Statements}\label{control-statements}

\subsection{Comparision}\label{comparision}

Computer programs do not only execute instructions. Occasionally, a
choice needs to be made. Such as a choice is based on a condition.
Python has several conditional operators:

\begin{verbatim}
>   greater than
<   smaller than
==  equals
!=  is not
\end{verbatim}

Conditions are always combined with variables. A program can make a
choice using the if keyword. For example:

\begin{verbatim}
>>> x = int(input("Guess x:"))
>>> if x == 4:
...    print('You guessed correctly!')
...    <ENTER>
\end{verbatim}

In this example, \emph{You guessed correctly!} will only be printed if
the variable x equals to four (see table above). Python can also execute
multiple conditions using the elif and else keywords.

\begin{verbatim}
>>> x = int(input("Guess x:"))
>>> if x == 4:
...     print('You guessed correctly!')
... elif abs(4 - x) == 1:
...     print('Wrong guess, but you are close!')
... else:
...     print('Wrong guess')
... <ENTER>
\end{verbatim}

\subsection{Iteration}\label{iteration}

To repeat code, the for keyword can be used. For example, to display the
numbers from 1 to 10, we could write something like this:

\begin{verbatim}
>>> for i in range(1, 11):
...    print('Hello!')
\end{verbatim}

The second argument to range, \emph{11}, is not inclusive, meaning that
the loop will only get to \emph{10} before it finishes. Python itself
starts counting from 0, so this code will also work:

\begin{verbatim}
>>> for i in range(0, 10):
...    print(i + 1)
\end{verbatim}

In fact, the range function defaults to starting value of \emph{0}, so
the above is equivalent to:

\begin{verbatim}
>>> for i in range(10):
...    print(i + 1)
\end{verbatim}

We can also nest loops inside each other:

\begin{verbatim}
>>> for i in range(0,10):
...     for j in range(0,10):
...         print(i,' ',j)
... <ENTER>
\end{verbatim}

In this case we have two nested loops. The code will iterate over the
entire coordinate range (0,0) to (9,9)

\section{Datatypes}\label{datatypes}

\subsection{Lists}\label{lists}

see: \url{https://www.tutorialspoint.com/python/python_lists.htm}

Lists in Python are ordered sequences of elements, where each element
can be accessed using a 0-based index.

To define a list, you simply list its elements between square brackest
`{[}{]}`:

\begin{verbatim}
>>> >>> names = ['Albert', 'Jane', 'Liz', 'John', 'Abby']
>>> names[0] # access the first element of the list
'Albert'
>>> names[2] # access the third element of the list
'Liz'
\end{verbatim}

You can also use a negative index if you want to start counting elements
from the end of the list. Thus, the last element has index \emph{-1},
the second before last element has index \emph{-2} and so on:

\begin{verbatim}
>>> names[-1] # access the last element of the list
'Abby'
>>> names[-2] # access the second last element of the list
'John'
\end{verbatim}

Python also allows you to take whole slices of the list by specifing a
beginning and end of the slice separated by a colon `::

::

  \textgreater{}\textgreater{}\textgreater{} names{[}1:-1{]} \# the middle elements, excluding first and last
  {[}'Jane', 'Liz', 'John'{]}

As you can see from the example above, the starting index in the slice
is inclusive and the ending one, exclusive.

Python provides a variety of methods for manipulating the members of a
list.

You can add elements with append`:

\begin{verbatim}
>>> names.append('Liz')
>>> names
['Albert', 'Jane', 'Liz', 'John', 'Abby', 'Liz']
\end{verbatim}

As you can see, the elements in a list need not be unique.

Merge two lists with `extend`:

\begin{verbatim}
>>> names.extend(['Lindsay', 'Connor'])
>>> names
['Albert', 'Jane', 'Liz', 'John', 'Abby', 'Liz', 'Lindsay', 'Connor']
\end{verbatim}

Find the index of the first occurrence of an element with `index`:

\begin{verbatim}
>>> names.index('Liz')
2
\end{verbatim}

Remove elements by value with `remove`:

\begin{verbatim}
>>> names.remove('Abby')
>>> names
['Albert', 'Jane', 'Liz', 'John', 'Liz', 'Lindsay', 'Connor']
\end{verbatim}

Remove elements by index with `pop`:

\begin{verbatim}
>>> names.pop(1)
'Jane'
>>> names
['Albert', 'Liz', 'John', 'Liz', 'Lindsay', 'Connor']
\end{verbatim}

Notice that pop returns the element being removed, while remove does
not.

If you are familiar with stacks from other programming languages, you
can use insert and `pop`:

\begin{verbatim}
>>> names.insert(0, 'Lincoln')
>>> names
['Lincoln', 'Albert', 'Liz', 'John', 'Liz', 'Lindsay', 'Connor']
>>> names.pop()
'Connor'
>>> names
['Lincoln', 'Albert', 'Liz', 'John', 'Liz', 'Lindsay']
\end{verbatim}

The Python documentation contains a \href{}{full list of list
operations}.

To go back to the range function you used earlier, it simply creates a
list of numbers:

\begin{verbatim}
>>> range(10)
[0, 1, 2, 3, 4, 5, 6, 7, 8, 9]
>>> range(2, 10, 2)
[2, 4, 6, 8]
\end{verbatim}

\subsection{Sets}\label{sets}

Python lists can contain duplicates as you saw above:

\begin{verbatim}
>>> names = ['Albert', 'Jane', 'Liz', 'John', 'Abby', 'Liz']
\end{verbatim}

When we don't want this to be the case, we can use a
\href{https://docs.python.org/2/library/stdtypes.html\#set}{set}:

\begin{verbatim}
>>> unique_names = set(names)
>>> unique_names
set(['Lincoln', 'John', 'Albert', 'Liz', 'Lindsay'])
\end{verbatim}

Keep in mind that the \emph{set} is an unordered collection of objects,
thus we can not access them by index:

\begin{verbatim}
>>> unique_names[0]
Traceback (most recent call last):
  File "<stdin>", line 1, in <module>
  TypeError: 'set' object does not support indexing
\end{verbatim}

However, we can convert a set to a list easily:

\textgreater{}\textgreater{}\textgreater{} unique\_names =
list(unique\_names) \textgreater{}\textgreater{}\textgreater{}
unique\_names {[}`Lincoln', `John', `Albert', `Liz', `Lindsay'{]}
\textgreater{}\textgreater{}\textgreater{} unique\_names{[}0{]}
`Lincoln'

Notice that in this case, the order of elements in the new list matches
the order in which the elements were displayed when we create the set
(we had set({[}'Lincoln', 'John', 'Albert', 'Liz',
'Lindsay'{]}) and now we have {[}'Lincoln', 'John', 'Albert', 'Liz',
'Lindsay'{]}). You should not assume this is the case in general. That
is, don't make any assumptions about the order of elements in a set when
it is converted to any type of sequential data structure.

You can change a set's contents using the add, remove and update methods
which correspond to the append, remove and extend methods in a list. In
addition to these, \emph{set} objects support the operations you may be
familiar with from mathematical sets: \emph{union}, \emph{intersection},
\emph{difference}, as well as operations to check containment. You can
read about this in the
\href{https://docs.python.org/2/library/stdtypes.html\#set}{Python
documentation for sets}.

\subsection{Removal and Testing for Membership in
Sets}\label{removal-and-testing-for-membership-in-sets}

One important advantage of a \emph{set} over a \emph{list} is that
\textbf{access to elements is fast}. If you are familiar with different
data structures from a Computer Science class, the Python list is
implemented by an array, while the set is implemented by a hash table.

We will demonstrate this with an example. Let's say we have a list and a
set of the same number of elements (approximately 100 thousand):

\begin{verbatim}
>>> import sys, random, timeit
>>> nums_set = set([random.randint(0, sys.maxint) for _ in range(10**5)])
>>> nums_list = list(nums_set)
>>> len(nums_set)
100000
\end{verbatim}

We will use the
\href{https://docs.python.org/2/library/timeit.html}{timeit} Python
module to time 100 operations that test for the existence of a member in
either the list or set:

\begin{verbatim}
>>> timeit.timeit('random.randint(0, sys.maxint) in nums', setup='import random; nums=%s' % str(nums_set), number=100)
0.0004038810729980469
>>> timeit.timeit('random.randint(0, sys.maxint) in nums', setup='import random; nums=%s' % str(nums_list), number=100)
0.3980541229248047
\end{verbatim}

The exact duration of the operations on your system will be different,
but the take away will be the same: searching for an element in a set is
orders of magnitude faster than in a list. This is important to keep in
mind when you work with large amounts of data.

\subsection{Dictionaries}\label{dictionaries}

One of the very important data structures in python is a dictionary also
referred to as \emph{dict}.

A dictionary represents a key value store:

\begin{verbatim}
>>> person = {'Name': 'Albert', 'Age': 100, 'Class': 'Scientist'}
>>> print("person['Name']: ", person['Name'])
person['Name']:  Albert
>>> print("person['Age']: ", person['Age'])
person['Age']:  100
\end{verbatim}

You can delete elements with the following commands:

\begin{verbatim}
>>> del person['Name'] # remove entry with key 'Name'
>>> person
{'Age': 100, 'Class': 'Scientist'}
>>> person.clear()     # remove all entries in dict
>>> person
{}
>>> del person         # delete entire dictionary
>>> person
Traceback (most recent call last):
  File "<stdin>", line 1, in <module>
  NameError: name 'person' is not defined
\end{verbatim}

You can iterate over a dict:

\begin{verbatim}
>>> person = {'Name': 'Albert', 'Age': 100, 'Class': 'Scientist'}
>>> for item in person:
...   print(item, person[item])
...   <ENTER>
Age 100
Name Albert
Class Scientist
\end{verbatim}

\subsection{Dictionary Keys and
Values}\label{dictionary-keys-and-values}

You can retrieve both the keys and values of a dictionary using the
keys() and values() methods of the dictionary, respectively:

\begin{verbatim}
>>> person.keys()
['Age', 'Name', 'Class']
>>> person.values()
[100, 'Albert', 'Scientist']
\end{verbatim}

Both methods return lists. Notice, however, that the order in which the
elements appear in the returned lists (Age, Name, Class) is different
from the order in which we listed the elements when we declared the
dictionary initially (Name, Age, Class). It is important to keep this in
mind: \textbf{you can't make any assumptions about the order in which
the elements of a dictionary will be returned by the keys() and values()
methods}.

However, you can assume that if you call keys() and values() in
sequence, the order of elements will at least correspond in both
methods. In the above example Age corresponds to 100, Name to 'Albert,
and Class to Scientist, and you will observe the same correspondence in
general as long as \textbf{keys() and values() are called one right
after the other}.

\subsection{Counting with
Dictionaries}\label{counting-with-dictionaries}

One application of dictionaries that frequently comes up is counting the
elements in a sequence. For example, say we have a sequence of coin
flips:

\begin{verbatim}
>>> import random
>>> die_rolls = [random.choice(['heads', 'tails']) for _ in range(10)]
>>> die_rolls
['heads', 'tails', 'heads', 'tails', 'heads', 'heads', 'tails', 'heads', 'heads', 'heads']
\end{verbatim}

The actual list die\_rolls will likely be different when you execute
this on your computer since the outcomes of the die rolls are random.

To compute the probabilities of heads and tails, we could count how many
heads and tails we have in the list:

\begin{verbatim}
>>> counts = {'heads': 0, 'tails': 0}
>>> for outcome in coin_flips:
...   assert outcome in counts
...   counts[outcome] += 1
...   <ENTER>
>>> print('Probability of heads: %.2f' % (counts['heads'] / len(coin_flips)))
Probability of heads: 0.70
>>> print('Probability of tails: %.2f' % (counts['tails'] / sum(counts.values())))
Probability of tails: 0.30
\end{verbatim}

In addition to how we use the dictionary counts to count the elements of
coin\_flips, notice a couple things about this example:

\begin{enumerate}
\tightlist
\item
  We used the assert outcome in counts statement. The assert statement
  in Python allows you to easily insert debugging statements in your
  code to help you discover errors more quickly. assert statements are
  executed whenever the internal Python \_\_debug\_\_ variable is set to
  True, which is always the case unless you start Python with the -O
  option which allows you to run \emph{optimized} Python.
\item
  When we computed the probability of tails, we used the built-in sum
  function, which allowed us to quickly find the total number of coin
  flips. sum is one of many built-in function you can
  \href{https://docs.python.org/2/library/functions.html}{read about
  here}.
\end{enumerate}

\section{Functions}\label{functions}

You can reuse code by putting it inside a function that you can call in
other parts of your programs. Functions are also a good way of grouping
code that logically belongs together in one coherent whole. A function
has a unique name in the program. Once you call a function, it will
execute its body which consists of one or more lines of code:

\begin{verbatim}
def check_triangle(a, b, c):
return \
    a < b + c and a > abs(b - c) and \
    b < a + c and b > abs(a - c) and \
    c < a + b and c > abs(a - b)

print(check_triangle(4, 5, 6))
\end{verbatim}

The def keyword tells Python we are defining a function. As part of the
definition, we have the function name, check\_triangle, and the
parameters of the function -- variables that will be populated when the
function is called.

We call the function with arguments 4, 5 and 6, which are passed in
order into the parameters a, b and c. A function can be called several
times with varying parameters. There is no limit to the number of
function calls.

It is also possible to store the output of a function in a variable, so
it can be reused.

\begin{verbatim}
def check_triangle(a, b, c):
 return \
     a < b + c and a > abs(b - c) and \
     b < a + c and b > abs(a - c) and \
     c < a + b and c > abs(a - b)

result = check_triangle(4, 5, 6)
print(result)
\end{verbatim}

\section{Classes}\label{classes}

A class is an encapsulation of data and the processes that work on them.
The data is represented in member variables, and the processes are
defined in the methods of the class (methods are functions inside the
class). For example, let's see how to define a Triangle class:

\begin{verbatim}
class Triangle(object):

 def __init__(self, length, width, height, angle1, angle2, angle3):
     if not self._sides_ok(length, width, height):
         print('The sides of the triangle are invalid.')
     elif not self._angles_ok(angle1, angle2, angle3):
         print('The angles of the triangle are invalid.')

     self._length = length
     self._width = width
     self._height = height

     self._angle1 = angle1
     self._angle2 = angle2
     self._angle3 = angle3

 def _sides_ok(self, a, b, c):
     return \
         a < b + c and a > abs(b - c) and \
         b < a + c and b > abs(a - c) and \
         c < a + b and c > abs(a - b)

 def _angles_ok(self, a, b, c):
     return a + b + c == 180

triangle = Triangle(4, 5, 6, 35, 65, 80)
\end{verbatim}

Python has full Aobject-oriented programming (OOP) capabilities, however
we can not cover all of them in a quick tutorial, so please refer to the
\href{https://docs.python.org/2.7/tutorial/classes.html}{Python docs on
classes and OOP}.

\chapter{Data Management}

\FILENAME

Obviously when dealing with big data we may not only be dealing with
data in one format but in many different formats. It is important that
you will be able to master such formats and seamlessly integrate in
your analysis. Thus we provide some simple examples on which different
data formats exist and how to use them.

\section{Formats}

\subsection{Pickle}

Python pickle allows you to save data in a python native format into a file
that can later be read in by other programs. However, the data format
may not be portable among different python versions thus the format is
often not suitable to store information. Instead we recommend for
standrad data to use either json or yaml.

\begin{verbatim}
import pickle

flavor = { "small": 100, 
           "medium": 1000,
           "large": 10000 }

pickle.dump( flavor, open( "data.p", "wb" ) )

\end{verbatim}

To read it back in use

\begin{verbatim}
flavor = pickle.load( open( "data.p", "rb" ) )
\end{verbatim}

\subsection{Text Files}

To read text files into a variable called content  you can use 

\begin{verbatim}
content = open(“filename.txt”, “r”).read() 
\end{verbatim}

You can also use the following code while using the convenient
\verb|with| statement

\begin{verbatim}
with open('filename.txt','r') as file:
    content = file.read()
\end{verbatim}

To split up the lines of the file into an array you can do

\begin{verbatim}
with open('filename.txt','r') as file:
    lines = file.read().splitlines()
\end{verbatim}


This cam aslo be done with the build in \verb|readlines| function
\begin{verbatim}
lines = open('filename.txt','r').readlines()
\end{verbatim}

In case the file is too big you will want to read the file line by
line:

\begin{verbatim}
with open('filename.txt','r') as file:
    line = file.readline()
    print (line)
\end{verbatim}


\subsection{CSV Files}

Often data is contained in comma separated values (CSV) within a
file. To read such files you can use the csv package.

\begin{verbatim}
import csv
with open(‘data.csv’, ‘rb’) as f:
   contents = csv.reader(f)
for row in content:
    print row
\end{verbatim}

Using pandas you can read them as follows.

\begin{verbatim}
import pandas as pd
df = pd.read_csv("example.csv") 
\end{verbatim}

There are many other modules and libraries that include CSV read
functions. IN case you need to split a single line by comma, you may
also use the \verb|split| function. However, remember it swill split
at every comma, including those contained in quotes. SO this method
although looking originally convenient has limitations.

\subsection{Excel spread sheets}

Pandas contains a method to read Excel files

\begin{verbatim}
import pandas as pd
filename = 'data.xlsx'
data = pd.ExcelFile(file)
df = data.parse('Sheet1')
\end{verbatim}

\subsection{YAML}

YAML is a very important format as it allows you easily to structure
data in hierarchical fileds It is frequently used to coordinate
programs while using yaml as the specification for configuration fils,
but also data files. To read in a yaml file the following code can be
used

\begin{verbatim}
import yaml
with open('data.yaml', 'r') as f:
    content = yaml.load(f)
\end{verbatim}

The nice part is that this code can also be used to verify if a file
is valid yaml. To write data out we can use

\begin{verbatim}
with open('data.yml', 'w') as f:
    yaml.dump(data, f, default_flow_style=False)
\end{verbatim}

The flow style set to false formats the data in a nice readable
fashion with indentations.


\subsection{JSON}

\begin{verbatim}
import json
with open('strings.json') as f:
    content = json.load(f)
\end{verbatim}

\subsection{XML}

\TODO{Tutorial: Please contribute a XML python tutorial.}

\subsection{RDF}

To read RDF files you will need to install RDFlib with 

\begin{verbatim}
pip install rdflib
\end{verbatim}

This will than allow you to read RDF files

\begin{verbatim}
from rdflib.graph import Graph
g = Graph()
g.parse("filename.rdf", format="format")
for entry in g:
   print(entry)
\end{verbatim}

Good examples on using RDF are provided on the RDFlib Web page
at~\url{https://github.com/RDFLib/rdflib}

From the Web page we showcase also how to directly process RDF data
from the Web

\begin{verbatim}
import rdflib
g=rdflib.Graph()
g.load('http://dbpedia.org/resource/Semantic_Web')

for s,p,o in g:
    print s,p,o
\end{verbatim}

\subsection{PDF}

The Portable Document Format (PDF) has been made available by Adobe
Inc. royalty free. This has enabled PDF to become a world wide adopted
format that also has been standardized in 2008 (ISO/IEC 32000-1:2008,
\url{https://www.iso.org/standard/51502.html}).  A lot of research is
published in papers making PDF one of the de-facto standards for
publishing. However, PDF is difficult to parse and is focused on high
quality output instead of data representation. Nevertheless,
tools to manipulate PDF exist:

\begin{description}
\item[PDFMiner] \url{https://pypi.python.org/pypi/pdfminer/} allows
  the simple translation of PDF into text that than can be further
  mined. The manual page helps to demonstrate some examples
  \url{http://euske.github.io/pdfminer/index.html}.

\item[pdf-parser.py]
  \url{https://blog.didierstevens.com/programs/pdf-tools/} parses pdf
  documents and identifies some structural elements that can than be
  further processed.

\end{description}

If you know about other tools, let us know.


\subsection{HTML}

A very powerful library to parse HTML Web pages is provided
with~\url{https://www.crummy.com/software/BeautifulSoup/}

More details about it are provided in the documentation page
\url{https://www.crummy.com/software/BeautifulSoup/bs4/doc/}

\TODO{Students: beautiful soup contribute tutorial}

\subsection{ConfigParser}

\URL{https://pymotw.com/2/ConfigParser/}

\subsection{ConfigDict}

\URL{https://github.com/cloudmesh/cloudmesh.common/blob/master/cloudmesh/common/ConfigDict.py}

\section{Encryption}

Often we need to protect the information stored in a file. This is
achieved with encryption. There are many methods of supporting
encryption and even if a file is encrypted it may be target to
attacks. Thus it is not only important to encrypt data that you do not
want others to se but also to make sure that the system on which the
data is hosted is secure. This is especially important if we talk
about big data having a potential large effect if it gets into the
wrong hands. 

To illustrate one type of encryption that is non trivial we have
chosen to demonstrate how to encrypt a file with an ssh key. In case
you have openssl installed on your system, this can be achieved as follows.


\begin{verbatim}
#! /bin/sh

# Step 1. Creating a file with data
echo "Big Data is the future." > file.txt

# Step 2. Create the pem 
openssl rsa -in ~/.ssh/id_rsa -pubout  > ~/.ssh/id_rsa.pub.pem

# Step 3. look at the pem file to illustrate how it looks like (optional)
cat ~/.ssh/id_rsa.pub.pem

# Step 4. encrypt the file into secret.txt
openssl rsautl -encrypt -pubin -inkey ~/.ssh/id_rsa.pub.pem -in file.txt -out secret.txt

# Step 5. decrypt the file and print the contents to stdout
openssl rsautl -decrypt -inkey ~/.ssh/id_rsa -in secret.txt
\end{verbatim}

Most important here are Step 4 that encrypts the file and Step 5 that
decrypts the file. Using the Python os module it is straight forward
to implement this. However, we are providing in cloudmesh a convenient
class that makes the use in python very simple.

\begin{verbatim}
from cloudmesh.common.ssh.encrypt import EncryptFile

e = EncryptFile('file.txt', 'secret.txt')
e.encrypt()
e.decrypt()
\end{verbatim}

In our class we initialize it with the locations of the file that is
to be encrypted and decrypted. To initiate that action just call the
methods \verb|encrypt| and \verb|decrypt|.

\section{Database Access}\label{database-access}

\TODO{Students: define conventional database access tutorial}

see: \url{https://www.tutorialspoint.com/python/python_database_access.htm}

\section{SQLite}

\TODO{Students: defineSQLite database access tutorial}

\url{https://www.sqlite.org/index.html}

\url{https://docs.python.org/3/library/sqlite3.html}

\subsection{Exercises}

\begin{exercise}
\label{E:Encryption.1} Test out the shell script to replicate how this
  example works
\end{exercise}

\begin{exercise}
  \label{E:Encryption.2} Test out the cloudmesh encryption class
\end{exercise}

\begin{exercise}
  \label{E:Encryption.3} What other encryption methods exist. Can you
  provide an example and contribute to the section?
\end{exercise}

\begin{exercise}
  \label{E:Encryption.4} What is the issue of encryption that make it
  challenging for Big Data
\end{exercise}

\begin{exercise}
  \label{E:Encryption.5} Given a test dataset with many files text
  files, how long will it take to encrypt and decrypt them on various
  machines. Write a benchmark that you test. Develop this benchmark as
  a group, test out the time it takes to execute it on a variety of
  platforms.
\end{exercise}


\chapter{Libraries}
\label{C:python-lib}




\section{Installing Libraries}\label{installing-libraries}

Often you may need functionality that is not present in Python's
standard library. In this case you have two option:

\begin{itemize}
\item  implement the features yourself
\item  use a third-party library that has the desired features.
\end{itemize}

Often you can find a previous implementation of what you need. Since
this is a common situation, there is a service supporting it: the
\href{https://pypi.python.org/pypi}{Python Package Index} (or PyPi for
short).

Our task here is to install the \href{}{autopep8} tool from PyPi. This
will allow us to illustrate the use if virtual environments using the
pyenv or virtualenv command, and installing and uninstalling PyPi
packages using pip.

\section{Using pip to Install
Packages}\label{using-pip-to-install-packages}

Let's now look at another important tool for Python development: the
Python Package Index, or PyPI for short. PyPI provides a large set of
third-party python packages. If you want to do something in python,
first check pypi, as odd are someone already ran into the problem and
created a package solving it.

In order to install package from PyPI, use the pip command. We can
search for PyPI for packages:

\begin{verbatim}
$ pip search --trusted-host pypi.python.org autopep8 pylint
\end{verbatim}

It appears that the top two results are what we want so install them:

\begin{verbatim}
$ pip install --trusted-host pypi.python.org autopep8 pylint
\end{verbatim}

This will cause pip to download the packages from PyPI, extract them,
check their dependencies and install those as needed, then install the
requested packages.

\begin{description}
\item[You can skip `--trusted-host pypi.python.org' option if you have]
patched urllib3 on Python 2.7.9.
\end{description}

\section{GUI}\label{gui}

\subsection{GUIZero}\label{guizero}

Install guizero with the following command:

\begin{verbatim}
sudo pip3 install guizero
\end{verbatim}

For a comprehensive tutorial on guizero,
\href{https://lawsie.github.io/guizero/howto/}{click here}.

\subsection{Kivy}\label{kivy}

You can install Kivy on OSX as followes:

\begin{verbatim}
brew install pkg-config sdl2 sdl2_image sdl2_ttf sdl2_mixer gstreamer
pip install -U Cython
pip install kivy
pip install pygame
\end{verbatim}

A hello world program for kivy is included in the cloudmesh.robot
repository. Which you can fine here

\begin{itemize}

\item
  \url{https://github.com/cloudmesh/cloudmesh.robot/tree/master/projects/kivy}
\end{itemize}

To run the program, please download it or execute it in cloudmesh.robot
as follows:

\begin{verbatim}
cd cloudmesh.robot/projects/kivy
python swim.py
\end{verbatim}

To create stand alone packages with kivy, please see:

\begin{verbatim}
-  https://kivy.org/docs/guide/packaging-osx.html
\end{verbatim}

\section{Formatting and Checking Python
Code}\label{formatting-and-checking-python-code}

First, get the bad code:

\begin{verbatim}
$ wget --no-check-certificate http://git.io/pXqb -O bad_code_example.py
\end{verbatim}

Examine the code:

\begin{verbatim}
$ emacs bad_code_example.py
\end{verbatim}

As you can see, this is very dense and hard to read. Cleaning it up by
hand would be a time-consuming and error-prone process. Luckily, this is
a common problem so there exist a couple packages to help in this
situation.

\section{Using autopep8}\label{using-autopep8}

We can now run the bad code through autopep8 to fix formatting problems:

\begin{verbatim}
$ autopep8 bad_code_example.py >code_example_autopep8.py
\end{verbatim}

Let us look at the result. This is considerably better than before. It
is easy to tell what the example1 and example2 functions are doing.

It is a good idea to develop a habit of using autopep8 in your
python-development workflow. For instance: use autopep8 to check a file,
and if it passes, make any changes in place using the -i flag:

\begin{verbatim}
$ autopep8 file.py    # check output to see of passes
$ autopep8 -i file.py # update in place
\end{verbatim}

If you use pyCharm you have the ability to use a similar function while
p;ressing on Inspect Code.

\section{Writing Python 3 Compatible
Code}\label{writing-python-3-compatible-code}

To write python 2 and 3 compatib;e code we recommend that you take a
look at: \url{http://python-future.org/compatible_idioms.html}

\section{Using Python on
FutureSystems}\label{using-python-on-futuresystems}

This is only important if you use Futuresystems resources.

In order to use Python you must log into your FutureSystems account.
Then at the shell prompt execute the following command:

\begin{verbatim}
$ module load python
\end{verbatim}

This will make the python and virtualenv commands available to you.

The details of what the module load command does are described in the
future lesson modules.

\section{Ecosystem}\label{ecosystem}

\subsection{pypi}\label{pypi}

Link: \href{https://pypi.python.org/pypi}{pypi}

The Python Package Index is a large repository of software for the
Python programming language containing a large number of packages
{[}link{]}. The nice think about pipy is that many packages can be
installed with the program `pip'.

To do so you have to locate the \textless{}package\_name\textgreater{}
for example with the search function in pypi and say on the commandline:

\begin{verbatim}
pip install <package_name>
\end{verbatim}

where pagage\_name is the string name of the package. an example would
be the package called cloudmesh\_client which you can install with:

\begin{verbatim}
pip install cloudmesh_client
\end{verbatim}

If all goes well the package will be installed.

\subsection{Alternative Installations}\label{alternative-installations}

The basic installation of python is provided by python.org. However
others claim to have alternative environments that allow you to install
python. This includes

\begin{itemize}

\item
  \href{https://store.enthought.com/downloads/\#default}{Canopy}
\item
  \href{https://www.continuum.io/downloads}{Anaconda}
\item
  \href{http://ironpython.net/}{IronPython}
\end{itemize}

Typically they include not only the python compiler but also several
useful packages. It is fine to use such environments for the class, but
it should be noted that in both cases not every python library may be
available for install in the given environment. For example if you need
to use cloudmesh client, it may not be available as conda or Canopy
package. This is also the case for many other cloud related and useful
python libraries. Hence, we do recommend that if you are new to python
to use the distribution form python.org, and use pip and virtualenv.

Additionally some python version have platform specific libraries or
dependencies. For example coca libraries, .NET or other frameworks are
examples. For the assignments and the projects such platform dependent
libraries are not to be used.

If however you can write a platform independent code that works on
Linux, OSX and Windows while using the python.org version but develop it
with any of the other tools that is just fine. However it is up to you
to guarantee that this independence is maintained and implemented. You
do have to write requirements.txt files that will install the necessary
python libraries in a platform independent fashion. The homework
assignment PRG1 has even a requirement to do so.

In order to provide platform independence we have given in the class a
``minimal'' python version that we have tested with hundreds of
students: python.org. If you use any other version, that is your
decision. Additionally some students not only use python.org but have
used iPython which is fine too. However this class is not only about
python, but also about how to have your code run on any platform. The
homework is designed so that you can identify a setup that works for
you.

However we have concerns if you for example wanted to use chameleon
cloud which we require you to access with cloudmesh. cloudmesh is not
available as conda, canopy, or other framework package. Cloudmesh client
is available form pypi which is standard and should be supported by the
frameworks. We have not tested cloudmesh on any other python version
then python.org which is the open source community standard. None of the
other versions are standard.

In fact we had students over the summer using canopy on their machines
and they got confused as they now had multiple python versions and did
not know how to switch between them and activate the correct version.
Certainly if you know how to do that, than feel free to use canopy, and
if you want to use canopy all this is up to you. However the homework
and project requires you to make your program portable to python.org. If
you know how to do that even if you use canopy, anaconda, or any other
python version that is fine. Graders will test your programs on a
python.org installation and not canpoy, anaconda, ironpython while using
virtualenv. It is obvious why. If you do not know that answer you may
want to think about that every time they test a program they need to do
a new virtualenv and run vanilla python in it. If we were to run two
instals in the same system, this will not work as we do not know if one
student will cause a side effect for another. Thus we as instructors do
not just have to look at your code but code of hundreds of students with
different setups. This is a non scalable solution as every time we test
out code from a student we would have to wipe out the OS, install it
new, install an new version of whatever python you have elected, become
familiar with that version and so on and on. This is the reason why the
open source community is using python.org. We follow best practices.
Using other versions is not a community best practice, but may work for
an individual.

We have however in regards to using other python version additional
bonus projects such as

\begin{itemize}

\item
  deploy run and document cloudmesh on ironpython
\item
  deploy run and document cloudmesh on anaconda, develop script to
  generate a conda packge form github
\item
  deploy run and document cloudmesh on canopy, develop script to
  generate a conda packge form github
\item
  deploy run and document cloudmesh on ironpython
\item
  other documentation that would be useful
\end{itemize}


\section{Resources}\label{resources}

If you are unfamiliar with programming in Python, we also refer you to
some of the numerous online resources. You may wish to start with
\href{https://www.learnpython.org}{Learn Python} or the book
\href{http://learnpythonthehardway.org/book/}{Learn Python the Hard
Way}. Other options include
\href{http://www.tutorialspoint.com/python/}{Tutorials Point} or
\href{http://www.codecademy.com/en/tracks/python}{Code Academy}, and the
Python wiki page contains a long list of
\href{https://wiki.python.org/moin/BeginnersGuide/Programmers}{references
for learning} as well. Additional resources include:

\begin{itemize}
\item
  \url{https://virtualenvwrapper.readthedocs.io}
\item
  \url{https://github.com/yyuu/pyenv}
\item
  \url{https://amaral.northwestern.edu/resources/guides/pyenv-tutorial}
\item
  \url{https://godjango.com/96-django-and-python-3-how-to-setup-pyenv-for-multiple-pythons/}
\item
  \url{https://www.accelebrate.com/blog/the-many-faces-of-python-and-how-to-manage-them/}
\item
  \url{http://ivory.idyll.org/articles/advanced-swc/}
\item
  \url{http://python.net/~goodger/projects/pycon/2007/idiomatic/handout.html}
\item
  \url{http://www.youtube.com/watch?v=0vJJlVBVTFg}
\item
  \url{http://www.korokithakis.net/tutorials/python/}
\item
  \url{http://www.afterhoursprogramming.com/tutorial/Python/Introduction/}
\item
  \url{http://www.greenteapress.com/thinkpython/thinkCSpy.pdf}
\item
  \url{https://docs.python.org/3.3/tutorial/modules.html}
\item
  \url{https://www.learnpython.org/en/Modules/_and/_Packages}
\item
  \url{https://docs.python.org/2/library/datetime.html}
\item
  \url{https://chrisalbon.com/python/strings/_to/_datetime.html}
\end{itemize}

A very long list of useful information are also available from

\begin{itemize}
\item
  \url{https://github.com/vinta/awesome-python}
\item
  \url{https://github.com/rasbt/python_reference}
\end{itemize}

This list may be useful as it also contains links to data visualization
and manipulation libraries, and AI tools and libraries. Please note that
for this class you can reuse such libraries if not otherwise stated.

\subsection{Jupyter Notebook Tutorials}\label{jupyter-notebook-tutorials}

A Short Introduction to Jupyter Notebooks and NumPy To view the
notebook, open this link in a background tab
\url{https://nbviewer.jupyter.org/} and copy and paste the following
link in the URL input area
\url{https://cloudmesh.github.io/classes/lesson/prg/Jupyter-NumPy-tutorial-I523-F2017.ipynb}
Then hit Go.


\section{Exercises}\label{exercises}

\begin{exercise}\label{E:Python.1}
Write a python program called iterate.py that accepts an integer n from
the command line. Pass this integer to a function called iterate.

The iterate function should then iterate from 1 to n. If the ith number
is a multiple of three, print ``multiple of 3'', if a multiple of 5
print ``multiple of 5'', if a multiple of both print ``multiple of 3 and
5'', else print the value.
\end{exercise}

\begin{exercise}\label{E:Python.2}
  \begin{enumerate}
  \item
    Create a pyenv or virtualenv \textasciitilde{}/ENV
  \item
    Modify your \textasciitilde{}/.bashrc shell file to activate your
    environment upon login.
  \item
    Install the docopt python package using pip
  \item
    Write a program that uses docopt to define a commandline program.
    Hint: modify the iterate program.
  \item
    Demonstrate the program works and submit the code and output.
  \end{enumerate}
\end{exercise}



\section{Python for Big Data}\label{python-for-big-data}

\subsection{An Example with Pandas, NumPy and Matplotlib}\label{an-example-with-pandas-numpy-and-matplotlib}

In this example, we will download some traffic citation data for the
city of Bloomington, IN, load it into Python and generate a histogram.
In doing so, you will be exposed to important Python libraries for
working with big data such as \href{www.numpy.org}{numpy},
\href{pandas.pydata.org}{pandas} and \href{matplotlib.org}{matplotlib}.

\subsubsection{Set Up Directories and Get Test
Data}\label{set-up-directories-and-get-test-data}

Data.gov is a government portal for open data and the
\href{https://catalog.data.gov/dataset?organization_type=City+Government\&organization=city-of-bloomington\&_organization_limit=0}{city
  of Bloomington, Indiana makes available a number of datasets there}.

We will use traffic citations data for 2016.

To start, let's create a separate directory for this project and
download the CSV data:

\begin{lstlisting}{bash}
$ cd ~/projects/i524
$ mkdir btown-citations
$ cd btown-citations
$ wget https://data.bloomington.in.gov/dataset/c543f0c1-1e37-46ce-a0ba-e0a949bd248a/resource/24841976-fd35-4483-a2b4-573bd1e77cfb/download/2016-first-quarter-citations.csv
\end{lstlisting}

Depending on your directory organization, the above might be slightly
different for you.

If you go to the link to data.gov for Bloomington above, you will see
that the citations data is organized per quarter, so there are a total
of four files. Above, we downloaded the data for the first quarter. Go
ahead and download the remaining three files with \texttt{wget}.

In this example, we will use three modules, \texttt{numpy},
\texttt{pandas} and \texttt{matplotlib}. If you set up
\texttt{virtualenv} as described in the
Python tutorial \textless{}python\_intro\textgreater{}, the first two of
these are already installed for you. To install \texttt{matplotlib},
make sure you've activated your \texttt{virtualenv} and use
\texttt{pip}:

\begin{lstlisting}{bash}
$ source ~/ENV/bin/activate
$ pip install matplotlib
\end{lstlisting}

If you are using a different distribution of Python, you will need to
make sure that all three of these modules are installed.

\subsubsection{Load Data in Pandas}\label{load-data-in-pandas}

From the same directory where you saved the citations data, let's start
the Python interpreter and load the citations data for Q1 2016

\begin{lstlisting}
$ python
>>> from __future__ import division, print_function
>>> import numpy as np
>>> import pandas as pd
>>> import matplotlib.pyplot as plt
>>> data = pd.read_csv('2016-first-quarter-citations.csv')
\end{lstlisting}
% $

If the first \texttt{import} statement seems confusing, take a look at
the Python tutorial \textless{}python\_intro\textgreater{}. The next
three \texttt{import} statements load each of the modules we will use in
this example. The final line uses Pandas' \verb|read_csv| function to
load the data into a Pandas \texttt{DataFrame} data structure.

\subsubsection{Working with DataFrames}\label{working-with-dataframes}

You can verify that you are working with a \texttt{DataFrame} and use
some of its methods to take a look at the structure of the data as
follows:

\begin{lstlisting}
>>> type(data)
<class 'pandas.core.frame.DataFrame'>
>>> data.index
Int64Index([  0,   1,   2,   3,   4,   5,   6,   7,   8,   9,
...
197, 198, 199, 200, 201, 202, 203, 204, 205, 206],
dtype='int64', length=200)
>>> data.columns
Index([u'Citation Number', u'Date Issued', u'Time Issued', u'Location ',
u'District', u'Cited Person Age', u'Cited Person Sex',
u'Cited Person Race', u'Offense Code', u'Offense Description',
u'Officer Age', u'Officer Sex', u'Officer Race'],
dtype='object')
>>> data.dtypes
Citation Number                object
Date Issued                    object
Time Issued                    object
Location                       object
District                       object
Cited Person Age              float64
Cited Person Sex               object
Cited Person Race              object
Offense Code                   object
Offense Description            object
Officer Age                   float64
Officer Sex                    object
Officer Race                   object
dtype: object
>>> data.shape
(200, 15)
\end{lstlisting}

As you can see from the \texttt{columns} field, when the CSV file was
read, the header line was used to populate the name of the columns in
the \texttt{DataFrame}. In addition, you will notice that
\verb|read_csv| correctly inferred the data type of some columns like
\emph{Age}, but not of others like \emph{Date Issued} and \emph{Time
Issued}. \verb|read_csv| is a very customizable function and in
general, you can correct issues like this using the \texttt{dtype} and
\texttt{converters} parameters. In this specific case, it makes more
sense to combine the \emph{Date Issued} and \emph{Time Issued} columns
into a new column containing a time stamp. We will see how to do this
shortly.

You can also look at the data itself with the \texttt{DataFrame}'s
\texttt{head()} and \texttt{tail()} methods:

\begin{lstlisting}
>>> data.head()
<Output omitted for brevity>
>>> data.tail()
<Output omitted for brevity>
\end{lstlisting}

In addition to letting you examine your data easily, \texttt{DataFrame}s
have methods that help you deal with missing values:

\begin{lstlisting}
>>> data = data.dropna(how='any')
>>> data.shape
\end{lstlisting}

Adding columns to the data is also easy. Here, we add two columns.
First, a
\href{https://docs.python.org/2/library/datetime.html}{datetime} column
that is a combination of the \texttt{Date\ Issued} and
\texttt{Time\ Issued} columns originally in the data. Second, a column
identifying what day of the week each citation was given. To understand
this example better, take a look at the Python docs for the
\texttt{strptime} and \texttt{strftime} functions in the
\texttt{datetime} module linked above.

\begin{lstlisting}
>>> from datetime import datetime
>>> data['DateTime Issued'] = data.apply(
...  lambda row: datetime.strptime(row['Date Issued'] + ':' + row['Time Issued'], '%m/%d/%y:%I:%M %p'), axis=1
... )
>>> data.columns
>>> data['Day of Week Issued'] = data.apply(
...  lambda row: datetime.strftime(row['DateTime Issued'], '%A'), axis=1
... )
\end{lstlisting}

\subsubsection{Plotting with Matplotlib and
NumPy}\label{plotting-with-matplotlib-and-numpy}

Let's say we want to see how many citations were given each day of the
week. We gather the data first:

\begin{lstlisting}
>>> days = ['Monday', 'Tuesday', 'Wednesday', 'Thursday', 'Friday', 'Saturday', 'Sunday']
>>> dow_data = [days.index(dow) for dow in data['Day of Week Issued']]
>>> dow_data
<Output omitted for brevity>
\end{lstlisting}

Then we use \texttt{matplotlib} to plot it:

\begin{lstlisting}
>>> fig = plt.figure()
>>> ax = fig.add_subplot(1, 1, 1)
>>> plt.hist(dow_data, bins=len(days))
>>> plt.xticks(range(len(days)), days)
>>> plt.show()
\end{lstlisting}

You should see something like this on your screen:

\includegraphics[width=4.16667in]{dow.png}

\subsubsection{\texorpdfstring{More \emph{DataFrame} Manipulation and
Plotting}{More DataFrame Manipulation and Plotting}}\label{more-dataframe-manipulation-and-plotting}

\texttt{DataFrame}s and \texttt{numpy} give us other ways to manipulate
data. For example, we can plot a histogram of the ages of violators like
this:

\begin{lstlisting}
>>> ages = data['Cited Person Age'].astype(int)
>>> fig = plt.figure()
>>> ax = fig.add_subplot(1, 1, 1)
>>> plt.hist(ages, bins=np.max(ages) - np.min(ages))
>>> plt.show()
\end{lstlisting}

\includegraphics[width=4.16667in]{ages.png}

Surprisingly, we see some 116 year-old violators! This is probably an
error in the data, so we can remove these data points easily and plot
the histogram again:

\begin{lstlisting}
>>> ages = ages[ages < 100]
>>> fig = plt.figure()
>>> ax = fig.add_subplot(1, 1, 1)
>>> plt.hist(ages, bins=np.max(ages) - np.min(ages))
>>> plt.show()
\end{lstlisting}

\includegraphics[width=4.16667in]{ages-filtered.png}

\subsubsection{Saving Plots to PDF}\label{saving-plots-to-pdf}

Oftentimes, you will want to save your \texttt{matplotlib} graph as a
PDF or an SVG file instead of just viewing it on your screen. For both,
we need to create a \texttt{figure} and plot the histogram as before:

\begin{lstlisting}
>>> fig = plt.figure()
>>> ax = fig.add_subplot(1, 1, 1)
>>> plt.hist(ages, bins=np.max(ages) - np.min(ages))
\end{lstlisting}

Then, instead of calling \texttt{plt.show()} we can invoke
\texttt{plt.savefig()} to save as SVG:

\begin{lstlisting}
>>> plt.savefig('hist.svg')
\end{lstlisting}

If we want to save the figure as PDF instead, we need to use the
\texttt{PdfPages} module together with \texttt{savefig()}:

\begin{lstlisting}
>>> import matplotlib.patches as mpatches
>>> from matplotlib.backends.backend_pdf import PdfPages   
>>> pp = PdfPages('hist.pdf')
>>> fig.savefig(pp, format='pdf')
>>> pp.close()
\end{lstlisting}

\subsubsection{Next Steps and Exercises}\label{next-steps-and-exercises}

There is a lot more to working with \texttt{pandas}, \texttt{numpy} and
\texttt{matplotlib} than we can show you here, but hopefully this
example has piqued your curiosity.

Don't worry if you don't understand everything in this example. For a
more detailed explanation on these modules and the examples we did,
please take a look at the tutorials below. The \texttt{numpy} and
\texttt{pandas} tutorials are mandatory if you want to be able to use
these modules, and the \texttt{matplotlib} gallery has many useful code
examples.

\subsection{Summary of Useful
Libraries}\label{summary-of-useful-libraries}

\subsubsection{Numpy}\label{s:numpy}

  \URL{http://www.numpy.org/}


According to the Numpy Web page ``NumPy is a package for scientific
computing with Python. It contains a powerful N-dimensional array
object, sophisticated (broadcasting) functions, tools for integrating
C/C++ and Fortran code, useful linear algebra, Fourier transform, and
random number capabilities''.

Tutorial:
\url{https://docs.scipy.org/doc/numpy-dev/user/quickstart.html}

\subsubsection{MatplotLib}\label{matplotlib}

  \URL{http://matplotlib.org/}

According the the Matplotlib Web page, ``matplotlib is a python 2D
plotting library which produces publication quality figures in a variety
of hardcopy formats and interactive environments across platforms.
matplotlib can be used in python scripts, the python and ipython shell
(ala MATLAB* or Mathematica), web application servers, and six
graphical user interface toolkits.''

Matplotlib Gallery: \url{http://matplotlib.org/gallery.html}

\subsubsection{Pandas}\label{pandas}

  \URL{http://pandas.pydata.org/}

According to the Pandas Web page, ``Pandas is a library library
providing high-performance, easy-to-use data structures and data
analysis tools for the Python programming language.''

In addition to access to charts via matplotlib it has elementary
functionality for conduction data analysis. Pandas may be very suitable
for your projects.

Tutorial: \url{http://pandas.pydata.org/pandas-docs/stable/10min.html}

Pandas Cheat Sheet:
\url{https://github.com/pandas-dev/pandas/blob/master/doc/cheatsheet/Pandas_Cheat_Sheet.pdf}

\subsection{Big Data Libraries}\label{other-useful-libraries}

\subsubsection{Scipy}\label{s:scipy}

  \URL{https://www.scipy.org/}

According to the Web page, SciPy (pronounced \textit{Sigh Pie}) is a
Python-based ecosystem of open-source software for mathematics, science,
and engineering. In particular, these are some of the core packages:

\begin{itemize}
\item NumPy
\item IPython
\item Pandas
\item Matplotlib
\item Sympy
\item SciPy library
\end{itemize}

It is thus an agglomeration of useful pacakes and will prbably sufice
for your projects in case you use Python.

\subsubsection{ggplot}\label{ggplot}

  \URL{http://ggplot.yhathq.com/}


According to the ggplot python Web page ggplot is a plotting system for
Python based on R's ggplot2. It allows to quickly generate some plots
quickly with little effort. Often it may be easier to use than
matplotlib directly.

\subsubsection{seaborn}\label{seaborn}

\URL{http://www.data-analysis-in-python.org/t_seaborn.html}

The good library for plotting is called seaborn which is build on top of
matplotlib. It provides high level templates for common statistical
plots.

\begin{itemize}
\item
  Gallery:
  \url{http://stanford.edu/~mwaskom/software/seaborn/examples/index.html}
\item
  Original Tutorial:
  \url{http://stanford.edu/~mwaskom/software/seaborn/tutorial.html}
\item
  Additional Tutorial:
  \url{https://stanford.edu/~mwaskom/software/seaborn/tutorial/distributions.html}
\end{itemize}

Here are some examples from a previous class:

\URL{https://github.com/bigdata-i523/hid231/blob/master/experiment/seaborn/seaborn-exercises.ipynb}
\URL{https://github.com/bigdata-i523/hid231/blob/master/experiment/learning-jupyter/learning_jupyter_notebook.ipynb}

\begin{exercise}\label{E:ipynb-export}
Take these examples and create sections in latex that can be added to
the book. Describe the process. 

1. export the ipynb as rst
2. use pandoc to export it to tex
3. do some cleanup on the tex files

Can this be automated with a cmd5 script such as
\begin{lstlisting}{bash}
cms ipynb [url=URL | file=FILE] --output FILENAME
\end{lstlisting}
\end{exercise}

\subsubsection{Bokeh}\label{bokeh}

Bokeh is an interactive visualization library with focus on web browsers
for display. Its goal is to provide a similar experience as D3.js

\begin{itemize}
\item
  URL: \url{http://bokeh.pydata.org/}
\item
  Gallery: \url{http://bokeh.pydata.org/en/latest/docs/gallery.html}
\end{itemize}

\subsubsection{pygal}\label{pygal}

Pygal is a simple API to produce graphs that can be easily embedded into
your Web pages. It contains annotations when you hover over data points.
It also allows to present the data in a table.

\URL{http://pygal.org/}


\subsubsection{Network and Graphs}\label{network-and-graphs}

\begin{itemize}

\item
  igraph: \url{http://www.pythonforsocialscientists.org/t_igraph.html}
\item
  networkx: \url{https://networkx.github.io/}
\end{itemize}

\subsubsection{REST}\label{rest}

\begin{itemize}

\item
  django REST FRamework \url{http://www.django-rest-framework.org/}
\item
  flask
  \url{https://blog.miguelgrinberg.com/post/designing-a-restful-api-with-python-and-flask}
\item
  requests
  \url{https://realpython.com/blog/python/api-integration-in-python/}
\item
  urllib2
  \url{http://rest.elkstein.org/2008/02/using-rest-in-python.html} (not
  recommended)
\item
  web
  \url{http://www.dreamsyssoft.com/python-scripting-tutorial/create-simple-rest-web-service-with-python.php}
  (not recommended)
\item
  bottle \url{http://bottlepy.org/docs/dev/index.html}
\item
  falcon \url{https://falconframework.org/}
\item
  eve \url{http://python-eve.org/}
\item
  \url{https://code.tutsplus.com/tutorials/building-rest-apis-using-eve--cms-22961}
\end{itemize}

\section{Parsing Data}

Being able to parse data is an important activity in the data analysis
process. Not all data may be following a specific format and the data
may need to be extracted.


\subsection{notebook.md Parser}

We are using a notebook.md to communicate what students have done
throughout the semester. We like to make a simple cmd5 command that
parses the notebook.md file and check it upon correctness.

An example for a notebook.md file is located here 

\URL{https://raw.githubusercontent.com/bigdata-i523/sample-hid000/master/notebook.md}

The following code may inspire you

\URL{https://github.com/bigdata-i523/hid203/tree/master/experiment}

We like to implement the following functionality and use docopts to
document the command.

\begin{lstlisting}
cms class notebook [--git=GITREPONAME] --verify hid

    verifies the correctness of the notebook.md file

cms class notebook [--git=GITREPONAME] --log

    displays the log of the notebook.md

cms class notebook [--git=GITREPONAME] --history

    displays a true or false for each week since the first occurance
    of the notebook.md file in the git repository
\end{lstlisting}

\begin{exercise}\label{E:notebook-md.1}
  Write a notebook.md parser
\end{exercise}

\begin{exercise}\label{E:notebook-md.2}
  How can this command be generalized to provide not only information
  for a student but to provide information for the class. Example: can
  we identify preferred days of when the notebooks are checked in. Can
  we identify a list of students that have not updated the notebok for
  a week?  Can we identify the list of student s that have updated the
  notebook for a week? 
\end{exercise}


\subsection{Video Length}

The Latex source of this class contains a macro to include videos.


Given a \LaTeX~file, can you create a table that includes the names of
all videos in that file and sums up the total viewing time. Previously
the document was stored in RST and the code from a previous student
may inspire you. Can you recreate it for \LaTeX? 

\URL{https://github.com/bigdata-i523/hid107/blob/master/cloudmesh/bar/command/mycommand.py}


\begin{lstlisting}
cms class video list FILENAME --output=[tabular|longtable|csv|txt]

    prints the videolist in the given format. txt means it is just ASCII
\end{lstlisting}

\begin{comment}
The previous work even summarized the video length by chapter.

see: https://piazza.com/class/j5wll7vzylg25j?cid=325
\end{comment}

\begin{exercise}
Write a tool that extracts the information for video length. 
\end{exercise}

\begin{exercise}
Write a tool that finds all youtube urls that are not in a video latex
macro.
\end{exercise}

\subsection{Dask}
\index{Dask}

Many times operations need to be done on data in parallel to utilize
modern processor architectures.


Dask provides a \textit{dynamic task scheduling} which is optimized for
computation. It is similar to other frameworks such as Airflow, Luigi,
Celery, or Make. However it is specializing in optimized interactive
computational workloads.

Furthermore, Dask targets Big Data \textit{collections} such as parallel
arrays, dataframes, and lists. These collections ar commonly found in
NumPy, Pandas, or Python iterators to larger-than-memory or
distributed environments. While using the Dask implementation we can
replace the original imports from the appropriate framework, replace
them with Dask imports and implicitly use parallel collections that
utilize internally the dynamic task schedulers.

More information can be found at:

\URL{https://dask.pydata.org}

\begin{exercise}
Conduct a performance study that showcases the difference of doing
parallel calculations in Dask, calculations in a framework such as
SciPy, and regular unthreaded python code.
\end{exercise}







\chapter{Cloudmesh Command Shell}

\input{chapter/python-cmd5}

\begin{fileremark}\url{https://github.com/cloudmesh/classes/blob/master/docs/source/i523/2017/code.rst}\end{fileremark}
\section{Code Management}\label{code-management}


%----------------------------------------------------------------------------------------
%	LINUX
%----------------------------------------------------------------------------------------

\part{Linux}



\chapter{Linux}
\label{C:linux}

\FILENAME

\section{History}

LINUX is a reimplementation by the community of UNIX which was
developed in 1969 by Ken Thompson and Dennis Ritchie of Bell
Laboratories and rewritten in C. An important part of UNIX is what is
called the {\em kernel} which allows the software to talk to
the hardware and utilize it. 

In 1991 Linus Trovalds started developing a Linux Kernel that was
initially targeted for PC's. THis made it possible to run it on
Laptops and was later on further developed by making it a full
Operating system replacement for UNIX. 

\section{Shell}

One of the most important features for us will be to access the
computer with the help of a {\em shell}. THe shell is typically run in
what is called a terminal and allows interaction to the computer with
commandline programs. 

There are many good tutorials out there that explain why one needs a
linux shell and not just a GUI. Randomly we picked the first one that
came up with a google query (This is not an endorsement for the material
we point to, but could be a worth while read for someone that has no
experience in Shell programming:

\URL{http://linuxcommand.org/lc3_learning_the_shell.php}

Certainly you are welcome to use other resources that may suite you
best. We will however summarize in table form a number of useful
commands that you may als find even as a RefCard.

\URL{http://www.cheat-sheets.org/\#Linux}

We provide in Table \ref{T:shell-commands} a number of useful commands
that you want to explore. For more information simply type man and the
name of the command.


\begin{center}
\begin{longtable}{|p{4cm}|p{8cm}|}
\caption{Common commands}\label{T:shell-commands}\\

\hline
\multicolumn{1}{|p{4cm}|}{\textbf{Example}} & \multicolumn{1}{p{8cm}|}{\textbf{Description}} \\ 
\hline 
\endfirsthead

\multicolumn{2}{p{12cm}}%
{{\bfseries \tablename\ \thetable{} -- continued from previous page}} \\
\hline 
\hline \multicolumn{1}{|p{4cm}|}{\textbf{Example}} & \multicolumn{1}{p{8cm}|}{\textbf{Description}} \\ 
\hline 
\endhead

\hline 
\multicolumn{2}{|r|}{{Continued on next page}} \\
\hline
\endfoot

%\hline 
\hline
\endlastfoot

  \multicolumn{2}{|l|}{\cellcolor{blue!15} Help Commands}\\
  \hline
  man \emph{command} & manual page for the \emph{command} \\
  apropos {\em text} & list all commands that have text in it\\
  & \\

  \hline
  \multicolumn{2}{|l|}{\cellcolor{blue!15} File Commands}\\
  \hline
  ls & Directory listing\\
  ls -lisa & list details \\
  tree & list the directories in graphical form \\
  cd \emph{dirname} & Change directory to \emph{dirname} \\
  mkdir \emph{dirname} & create the directory \\
  pwd & print working directory \\
  rm \emph{file} & remove the file \\
  cp \emph{a} \emph{b} & copy file \emph{a} to \emph{b} \\
  mv \emph{a} \emph{b} & move/rename file \emph{a} to \emph{b}\\
  cat \emph{a} & print content of file\emph{a}\\
  cat - n &  (assignment) \\
  less \emph{a} & print paged content of file \emph{a}\\
  head -5 \emph{a} & Display first 5 lines of file \emph{a}\\
  tail -5 \emph{a} & Display last 5 lines of file \emph{a}\\
  du -hs . & show in human readable form the space used by the current
             directory\\
  wc &  (assignment) \\
  sort &  (assignment) \\
  tar &  (assignment) \\
  rsync &  (assignment) \\
  gzip &  (assignment) \\
  bzip2 &  (assignment) \\
  & \\
  
  %\hline
  %\multicolumn{2}{|l|}{\cellcolor{blue!15} Search Commands}\\
  %\hline
  chmod go-rwx {\em file} & changes the permission of the file \\
  chown {\em username} {\em file} & changes the ownership of the file \\
  chgrp {\em group} {\em file} & changes the group of a file\\
  & \\

  \hline
  \multicolumn{2}{|l|}{\cellcolor{blue!15} Search Commands}\\
  \hline
  fgrep ``text'' filename &  searches the text in the given file \\
  grep -R ``xyz'' . & recursively searches for xyz in all files \\
  find . -name ``*.py'' &  find all files with .py at the end \\
  & \\

  %\hline
  %\multicolumn{2}{|l|}{\cellcolor{blue!15} Process Commands}\\
  %\hline
  ps & list the running processes \\
  kill -9 1234 & kill the process with the id 1234 \\
  at &  (assignment) \\
  cron &  (assignment) \\
  crontab &  (assignment) \\
  & \\

  \hline
  \multicolumn{2}{|l|}{\cellcolor{blue!15} Device Commands}\\
  \hline
  mount /dev/cdrom /mnt/cdrom & mount a filesystem from a cd rom to /mnt/cdrom\\
  & \\

  \hline
  \multicolumn{2}{|l|}{\cellcolor{blue!15} System Commands}\\
  \hline
  users &  (assignment) \\
  who &  (assignment) \\
  whoami &  (assignment) \\
  dmesg &  (assignment) \\
  last &  (assignment) \\
  free -tm &  (assignment) \\
  uname &  (assignment) \\
  date &  (assignment) \\
  time &  (assignment) \\
  shutdown -h ``shut down'' & (assignment) \\
  & \\

  \hline
  \multicolumn{2}{|l|}{\cellcolor{blue!15} Networking Commands}\\
  \hline
  ping &  (assignment) \\
  netstat &  (assignment) \\
  hostname &  (assignment) \\
  traceroute &  (assignment) \\
  ifconfig &  (assignment) \\
  & \\

  \hline
  \multicolumn{2}{|l|}{\cellcolor{blue!15} Internet Commands}\\
  \hline
  host &  (assignment) \\
  whois &  (assignment) \\
  dig &  (assignment) \\
  wget &  (assignment) \\
  curl &  (assignment) \\
  & \\

  \hline
  \multicolumn{2}{|l|}{\cellcolor{blue!15} Remote Access Commands}\\
  \hline
  ssh &  (assignment) \\
  scp &  (assignment) \\
  sftp &  (assignment) \\
  & \\


\end{longtable}
\end{center}

\section{Multi-command execution}

One of the important features is that one can execute multiple
commands in the shell.

To execute command 2 once command 1 has finished use

\begin{verbatim}
command1; command2
\end{verbatim}

To execute command 2 as soon as command 1 forwards output to stdout use

\begin{verbatim}
command1; command2
\end{verbatim}

To execute command 1 in the background use

\begin{verbatim}
command1 &
\end{verbatim}



\section{Keyboard Shortcuts}\label{keyboard-shortcuts}

These shortcuts will come in handy. Note that many overlap with emacs
short cuts.

\begin{tabular}{ll}
Keys     & Description  \\
\hline
Up Arrow & Show the previous command\\
Ctrl + z & Stops the current command  \\
         & Resume with fg in the foreground \\
         & Resume with bg in the background \\
Ctrl + c & Halts the current command\\
Ctrl + l & Clear the screen\\
Ctrl + a & Return to the start of the line\\
Ctrl + e & Go to the end of the line\\
Ctrl + k & Cut everything after the cursor to a special clipboard\\
Ctrl + y & Paste from the special clipboard \\
Ctrl + d & Log out of current session, similar to exit \\
\end{tabular}

\section{.bashrc and .bash\_profile}\label{bashrc-and-.bash_profile}

Usage of a particular command and all the attributes associated with it,
use `man' command. Avoid using \verb|rm -r| command to delete files
recursively. A good way to avoid accidental deletion is to include the
following in your \verb|.bash_profile| file:

\begin{verbatim}
alias e=open_emacs
alias rm='rm -i'
alias mv='mv -i' 
alias h='history'
\end{verbatim}

More Information

\url{https://cloudmesh.github.io/classes/lesson/linux/refcards.html}

\section{Exercise}\label{exercise}

\begin{description}
\item[Linux.1:]
Familiarize yourself with the commands
\item[Linux.2:]
Find more commands that you find useful and add them to this page.
\item[Linux.3:]
Use the sort command to sort all lines of a file while removing
duplicates.
\item[Linux.4:] In Table \ref{T:shell-commands} you will find a number
  of commands with (assignment). Develop descriptions that you will
  contribute and add to the manual with a pull request. Work in a team
  so that only one pull request is issued. Do not only provide the
  description, but also a real example as showcased within the table.
\item[Linux.4:] Should there be other commands listed in the table. If
  so which? Create a pull request for them. 
\end{description}

\chapterimage{box.jpg} 
\chapter{Virtual Box}
\label{S:virtual-box}

\FILENAME

For development purposes we recommend tha you use for this class an
Ubuntu virtual machine that you set up with the help of virtualbox. We
recommend that you use the current version of ubuntu and do not
install or reuse a version that you have set up years ago.

As access to cloud resources requires some basic knowledge of linux
and security we will restrict access to our cloud services to thopse
that have demonstrated responsible use on their own
computers. Naturally as it is your own computer you must make sure you
follwo proper security. We have seen in the past students carelessly
working with virtual machines and introducing security vulnerabilities
on our clouds just becasue ``it was not their computer.'' Hence, we
will allow using of cloud resources only if you have demonstrated that
you responsibly use a linux virtual machine on your own computer.
Only after you have successfully used ubuntu in a virtual machine you
will be allowed to use virtual machines on clouds.

A ``cloud drivers license test'' will be conducted. Only after you
pass it we wil let you gain access to the cloud infrastructure. We
will announce this test. Before you have not passed the test, you will
not be able to use the clouds.  Furthermore, you do not have to ask us
for join requests to cloud projects before you have not passed the
test. Please be patient. Only students enrolled in the class can get
access to the cloud. 

If you however have access to other clouds yourself you are welcome to
use the, However, be reminded that projects need to be reproducable,
on our cloud. This will require you to make sure a TA can replicate it.

Let us now focus on using virtual box.

\section{Instalation}\label{creation}

First you will need to install virtualbox. It is easy to install and
details can be found at

  \URL{https://www.virtualbox.org/wiki/Downloads}

After you have installed virtualbox you also need to use an image. For
this class we will be using ubuntu Desktop 16.04 which you can find at:

  \URL{http://www.ubuntu.com/download/desktop}


Please note some hardware you may have may be too old or has too
little resources to be useful. We have heard from students that the
following is a minimal setup for the desktop machine:

\begin{itemize}
\item  multi core processor or better allowing to run hypervisors
\item  8 GB system memory
\item  50 GB of free hard drive space
\end{itemize}

For virtual machines you may need multiple, while the minimal
configuration may not work for all cases.

As configuration we often use

\begin{description}
\item[minimal] 1 core, 2GB Memory, 5 GB disk 
\item[latex] 2 core, 4GB Memory, 25 GB disk 
\end{description}

A video to showcase such an install is available at:

\video{Virtualbox}{seconds}{Video}{https://youtu.be/NWibDntN2M4}

\begin{NOTE}
  If you specify your machine too small you will not be able to
  install the development environment. Gregor used on his machine 8GB
  RAM and 25GB diskspace.
\end{NOTE}

Please let us know the smalest configuration that works.

\section{Guest additions}\label{guest-additions}

The virtual guest additions allow you to easily do the following tasks:

\begin{itemize}
\item  Resize the windows of the vm
\item  Copy and paste content between the Guest operating system and
  the host operating system windows.
\end{itemize}

This way you can use many native programs on you host and copy contents
easily into for example a terminal or an editor that you run in the Vm.

A video is located at

\video{Virtualbox}{4:46}{Video}{https://youtu.be/wdCoiNdn2jA}


Please reboot the machine after installation and configuration.

On OSX you can once you have enabled bidirectional copying in the Device
tab with

\begin{description}
\item[OSX to Vbox:] command c shift CONTRL v
\item[Vbox to OSX:] shift CONTRL v shift CONTRL v
\end{description}

\begin{NOTE}
  On Windows the key combination is naturally different. Please
  consult your windows manual. If you let us know TAs will add the
  information here.
\end{NOTE}




\section{Exercise}

\begin{description}
\item[Virtualbox.1:] Install ubuntu desktop on your computer with
  guest additions.
\item[Virtualbox.2:] Make sure you know how to paste and copy between
  your host and guest operating system.
\item[Virtualbox.3:] Install the programs defined by the development
  configuration.
\item[Virtualbox.4:] Provide us with the key combination to copy and
  paste between Windows and Vbox.
\end{description}




%----------------------------------------------------------------------------------------
%	PART
%----------------------------------------------------------------------------------------

\part{IoT}

%----------------------------------------------------------------------------------------
%	CHAPTER 1
%----------------------------------------------------------------------------------------

\chapterimage{chapter_head_2.pdf} % Chapter heading image

\FILENAME

\section{Introduction}\label{introduction}

\begin{description}
\item[You may find that some videos may have a different lesson,]
section or unit number. Please ignore this. In case the content does not
correspond to the title, please let us know.
\end{description}

This section has a technical overview of course followed by a broad
motivation for course hosted at www-cloudmesh-classes.

The course overview covers it's content and structure. It presents an
introduction to general field of Big Data and Analytics. We are
especially analysing the many different application areas in which Big
Data can be applied. As Big Datais typically not just used in isolation
but is part of a larger Informatics issue for a particular field we also
use the term X-Informatics, where X defines a usecase or area of
specialization in which Big Data is applied to. As such we organize the
class around the the \emph{Rallying Cry} of course: Use Clouds running
Data Analytics Collaboratively processing Big Data to solve problems in
X-Informatics.

The courses is set up as a number of lessons that are typically between
20 minutes to an hour. The lessons are either provided as written
documents or as video lectures. They are enhanced by an in person
meeting that takes place either in a lecture room for residential
students or as online meeting for online students.

The course covers a mix of applications (the X in X-Informatics) and
technologies needed to support the field electronically i.e. to process
the application data. The overview ends with a discussion of course
content at highest level. The course starts with a motivation
summarizing clouds and data science, then units describing applications
in areas such as Physics, e-Commerce, Web Search and Text mining,
Health, Sensors and Remote Sensing). These are interspersed with
discussions of infrastructure (clouds) and data analytics (algorithms
like clustering and collaborative filtering used in applications). The
course uses Python as primary programming language. We will be
introducing practical use of cloud resources so that you have the
oportunity to explore example analytics applications on smaller data
sets that you define.

The course motivation starts with striking examples of the data deluge
with examples from research, business and the consumer. The growing
number of jobs in data science is highlighted. He describes industry
trend in both clouds and big data. Then the cloud computing model
developed at amazing speed by industry is introduced. The 4 paradigms of
scientific research are described with growing importance of data
oriented version.He covers 3 major X-informatics areas: Physics,
e-Commerce and Web Search followed by a broad discussion of cloud
applications. Parallel computing in general and particular features of
MapReduce are described.

We discuss in this course include the following topics. We may change
the order of the topics to allow for maximal flexibility and parallel
learning experiences.

Writing Track:

\begin{itemize}
\item  Writing a short review article
\item  Writing a porject or term report
\end{itemize}

Theory Track:

\begin{itemize}
\item  Motivation: Big Data and the Cloud; Centerpieces of the Future Economy
\item  Introduction: What is Big Data, Data Analytics
\item  Use Cases: Big Data Use Cases Survey

  \begin{itemize}
  \item    Use Case, Physics Discovery of Higgs Particle
  \item    Use Case: e-Commerce and Lifestyle with recommender systems
  \item    Use Case: Web Search and Text Mining and their technologies
  \item    Use Case: Sports
  \item    Use Case: Health
  \item    Use Case: Sensors
  \item    Use Case: Radar for Remote Sensing.
  \end{itemize}

\item Parallel Computing Overview and familiar examples
\item Cloud Technology for Big Data Applications \& Analytics
\end{itemize}

Practice Track:

\begin{itemize}
\item
  Python for Big Data Applications and Analytics: NumPy, SciPy,
  MatPlotlib
\item
  Using FutureGrid for Big Data Applications and Analytics Course
\item
  Using Chameleon Cloud for Big Data Applications and Analytics Course
\item
  {[}optional{]} Using Plotviz Software for Displaying Point
  Distributions in 3D
\item
  Recommender Systems - K-Nearest Neighbors, Clustering and heuristic
  methods
\item
  PageRank
\item
  Kmeans
\item
  MapReduce
\item
  Kmeans and MapReduce Parallelism
\end{itemize}

\subsection{Course Motivation}\label{course-motivation}

We motivate the study of X-informatics by describing data science and
clouds. He starts with striking examples of the data deluge with
examples from research, business and the consumer. The growing number of
jobs in data science is highlighted. He describes industry trend in both
clouds and big data.

He introduces the cloud computing model developed at amazing speed by
industry. The 4 paradigms of scientific research are described with
growing importance of data oriented version. He covers 3 major
X-informatics areas: Physics, e-Commerce and Web Search followed by a
broad discussion of cloud applications. Parallel computing in general
and particular features of MapReduce are described. He comments on a
data science education and the benefits of using MOOC's.

\subsubsection{Emerging Technologies}\label{emerging-technologies}

This presents the overview of talk, some trends in computing and data
and jobs. Gartner's emerging technology hype cycle shows many areas of
Clouds and Big Data. We highlight 6 issues of importance: economic
imperative, computing model, research model, Opportunities in advancing
computing, Opportunities in X-Informatics, Data Science Education


\video{Introduction}{40:14}{Motivation}  {https://drive.google.com/file/d/0B1Of61fJF7WsV2RvMlFzSDNPZEU/view?usp=sharing}
  
\slides{Introduction}{30}  {Motivation}{https://drive.google.com/file/d/0B8936_ytjfjmOUZraHc4M1ptczA/view?usp=sharing}


\subsubsection{Data Deluge}\label{data-deluge}

We give some amazing statistics for total storage; uploaded video and
uploaded photos; the social media interactions every minute; aspects of
the business big data tidal wave; monitors of aircraft engines; the
science research data sizes from particle physics to astronomy and earth
science; genes sequenced; and finally the long tail of science. The next
slide emphasizes applications using algorithms on clouds. This leads to
the rallying cry ``Use Clouds running Data Analytics Collaboratively
processing Big Data to solve problems in X-Informatics educated in data
science'`with a catalog of the many values of X''Astronomy, Biology,
Biomedicine, Business, Chemistry, Climate, Crisis, Earth Science,
Energy, Environment, Finance, Health, Intelligence, Lifestyle,
Marketing, Medicine, Pathology, Policy, Radar, Security, Sensor, Social,
Sustainability, Wealth and Wellness''


\video{Introduction}{30:38}  {Data Deluge}{https://www.youtube.com/watch?v=7VHPXJv3DN4}


\slides{Introduction}{20}  {Data  Deluge}{https://drive.google.com/open?id=0B8936_ytjfjmUXY3anBaeU9lLVU}

\subsubsection{Jobs}\label{jobs}

Jobs abound in clouds and data science. There are documented shortages
in data science, computer science and the major tech companies advertise
for new talent.


\video{Introduction}{9:39}  {Jobs}{https://www.youtube.com/watch?v=KsjiQS8uXDA}


\slides{Introduction}{8}  {Jobs}{https://drive.google.com/open?id=0B8936_ytjfjmaG50YW9TeWdvUTg}


\subsubsection{Industrial Trends}\label{industrial-trends}

Trends include the growing importance of mobile devices and comparative
decrease in desktop access, the export of internet content, the change
in dominant client operating systems, use of social media, thriving
Chinese internet companies.


\video{Introduction}{19:25} 
  {Industrial Trends}{https://www.youtube.com/watch?v=32vD7uN7fqY}


\slides{Introduction}{16}
  {Industrial
  Trends}{https://drive.google.com/open?id=0B8936_ytjfjmWW1SdXgxWkRLYjg}



\video{Introduction}{16:54}   {Industrial Trends  II}{https://www.youtube.com/watch?v=O8fgXAQcnvw}

\slides{Introduction}{16}
  {Indusrial
  Trends II}{https://drive.google.com/open?id=0B8936_ytjfjmeEV2R19ORzhBQVE}



\video{Introduction}{30:13} 
  {Indusrial Trends
  III}{https://www.youtube.com/watch?v=kW38MG7ukzs}

\slides{Introduction}{21}
  {Industrial
  Trends III}{https://drive.google.com/open?id=0B8936_ytjfjmNDZKcE1MSU45ZG8}


\subsubsection{Digital Disruption of Old
Favorites}\label{digital-disruption-of-old-favorites}

Not everything goes up. The rise of the Internet has led to declines in
some traditional areas including Shopping malls and Postal Services.

\video{Introduction}{32:54} 
{Digital Distruption
and transformation}{https://www.youtube.com/watch?v=bw9yYXwe7Bs} 



\slides{Introduction}{28}
  {Digital
  Distruption and transformation}{https://drive.google.com/open?id=0B8936_ytjfjmdW5CYnBtME9FVTQ}


\subsubsection{Computing Model}\label{computing-model}

\emph{Industry adopted clouds which are attractive for data analytics}

Clouds and Big Data are transformational on a 2-5 year time scale.
Already Amazon AWS is a lucrative business with almost a \$4B revenue.
We describe the nature of cloud centers with economies of scale and
gives examples of importance of virtualization in server consolidation.
Then key characteristics of clouds are reviewed with expected high
growth in Infrastructure, Platform and Software as a Service.


\video{Introduction}{24:03} 
  {Computing Model I}{https://www.youtube.com/watch?v=oYKTCKFGTco}


\slides{Introduction}{14}
  {Computing
  Model I}{https://drive.google.com/open?id=0B8936_ytjfjmTU9nNml2bUlsUHM}



\video{Introduction}{28:18} 
  {Computing Model II}{https://www.youtube.com/watch?v=km_eXHq7m3o}


\slides{Introduction}{27}
  {Computing
  Model II}{https://drive.google.com/open?id=0B8936_ytjfjmNHhLYnI0X0YxdFE}

\subsubsection{Research Model}\label{research-model}

\emph{4th Paradigm; From Theory to Data driven science?}

We introduce the 4 paradigms of scientific research with the focus on
the new fourth data driven methodology.


\video{Introduction}{7:33}  {Research Model}{https://www.youtube.com/watch?v=xkeECe3mmjI}


\slides{Introduction}{4}  {Research  Model}{https//drive.google.com/open?id=0B8936_ytjfjma0pMbHJnek02dDA}


\subsubsection{Data Science Process}\label{data-science-process}

We introduce the DIKW data to information to knowledge to wisdom
paradigm. Data flows through cloud services transforming itself and
emerging as new information to input into other transformations.


\video{Introduction}{15:42} {Data Science Process}{https://www.youtube.com/watch?v=KstIH2aQ60Y}


\slides{Introduction}{10}
  {Data  Science Process}{https://drive.google.com/open?id=0B8936_ytjfjmVDVZa01keW0wQmc}


\subsubsection{Physics-Informatics}\label{physics-informatics}

\emph{Looking for Higgs Particle with Large Hadron Collider LHC}

We look at important particle physics example where the Large hadron
Collider has observed the Higgs Boson. He shows this discovery as a bump
in a histogram; something that so amazed him 50 years ago that he got a
PhD in this field. He left field partly due to the incredible size of
author lists on papers.


\video{Introduction}{13:27} 
  {Physics-informatics}{https://www.youtube.com/watch?v=2A7Z741FCHs}

\slides{Introduction}{6}
  {Physics-inforamtics}{https://drive.google.com/open?id=0B8936_ytjfjmc2J2TWgwWGRwaFk}


\subsubsection{Recommender Systems}\label{recommender-systems}

Many important applications involve matching users, web pages, jobs,
movies, books, events etc. These are all optimization problems with
recommender systems one important way of performing this optimization.
We go through the example of Netflix \textasciitilde{}\textasciitilde{}
everything is a recommendation and muses about the power of viewing all
sorts of things as items in a bag or more abstractly some space with
funny properties.


\video{Introduction}{12:21}
  {Recommender Systems  I}{https://www.youtube.com/watch?v=LXhng3fcG9o}



\slides{Introduction}{9}
  {Recommender  Systems I}{https://drive.google.com/open?id=0B8936_ytjfjmOXlVd2FsSUkwekk}



\video{Introduction}{9:44} 
  {Recommender Systems
  II}{https://www.youtube.com/watch?v=Y4S0jY0yfEE}

\slides{Introduction}{6}
  {Recommender
  Systems II}{https://drive.google.com/open?id=0B8936_ytjfjmMzM2M3RhMEJ4bjQ}


\subsubsection{Web Search and Information
Retrieval}\label{web-search-and-information-retrieval}

This course also looks at Web Search and here we give an overview of the
data analytics for web search, Pagerank as a method of ranking web pages
returned and uses material from Yahoo on the subtle algorithms for
dynamic personalized choice of material for web pages.


\video{Introduction}{12:05}   {Web Search and  Information Retrieval}{https://www.youtube.com/watch?v=p-0NtNTzoh8}


\slides{Introduction}{6}  {Web  Search and Information Retrieval}{https://drive.google.com/open?id=0B8936_ytjfjmSm8zNmZ5VFJxRms}


\subsubsection{Cloud Application in
Research}\label{cloud-application-in-research}

We describe scientific applications and how they map onto clouds,
supercomputers, grids and high throughput systems. He likes the cloud
use of the Internet of Things and gives examples.


\video{Introduction}{33:51}{Cloud Applications  in Research}{https://www.youtube.com/watch?v=U3ZG2qOFpxE}


\slides{Introduction}{20}  {Cloud  Applications in Research}{https://drive.google.com/open?id=0B8936_ytjfjma0RhdU0zdkxmczA}

\subsubsection{Parallel Computing and
MapReduce}\label{parallel-computing-and-mapreduce}

We define MapReduce and gives a homely example from fruit blending.


\video{Introduction}{14:02}  {Computing and  MapReduce}{https://www.youtube.com/watch?v=aQ8NMxe9IsU}


\slides{Introduction}{9}  {Computing  and MapReduce}{https://drive.google.com/open?id=0B8936_ytjfjmeTl4NWhHRjJMOGc}

\subsubsection{Data Science Education}\label{data-science-education}

We discuss one reason you are taking this course
\textasciitilde{}\textasciitilde{} Data Science as an educational
initiative and aspects of its Indiana University implementation. Then
general; features of online education are discussed with clear growth
spearheaded by MOOC's where we use this course and others as an example.
He stresses the choice between one class to 100,000 students or 2,000
classes to 50 students and an online library of MOOC lessons. In olden
days he suggested `'hermit's cage virtual university''
\textasciitilde{}\textasciitilde{} gurus in isolated caves putting
together exciting curricula outside the traditional university model.
Grading and mentoring models and important online tools are discussed.
Clouds have MOOC's describing them and MOOC's are stored in clouds; a
pleasing symmetry.


\video{Introduction}{28:08}   {Data Science  Education}{https://www.youtube.com/watch?v=bA_eNjJTmRQ}


\slides{Introduction}{19}  {Data  Science Education}{https://drive.google.com/open?id=0B8936_ytjfjmT0J1RjYwY1VwZ1k}


\subsubsection{Conclusions}\label{conclusions}

The conclusions highlight clouds, data-intensive methodology,
employment, data science, MOOC's and never forget the Big Data ecosystem
in one sentence ``Use Clouds running Data Analytics Collaboratively
processing Big Data to solve problems in X-Informatics educated in data
science''


\video{Introduction}{4:59}  {Conclusions}{https://www.youtube.com/watch?v=FmcR5mrhYvk}

\slides{Introduction}{4}  {Conclusions}{https://drive.google.com/open?id=0B8936_ytjfjmVjRNeG1pdUNnMlE}


\subsubsection{Resources}\label{resources}

\begin{itemize}
\item
  \url{http://www.gartner.com/technology/home.jsp} and many web links
\item
  Meeker/Wu May 29 2013 Internet Trends D11 Conference
  \url{http://www.slideshare.net/kleinerperkins/kpcb-internet-trends-2013}
\item
  \url{http://cs.metrostate.edu/~sbd/slides/Sun.pdf}
\item
  Taming The Big Data Tidal Wave: Finding Opportunities in Huge Data
  Streams with Advanced Analytics, Bill Franks Wiley ISBN:
  978-1-118-20878-6
\item
  Bill Ruh
  \url{http://fisheritcenter.haas.berkeley.edu/Big_Data/index.html}
\item
  \url{http://www.genome.gov/sequencingcosts/}
\item
  CSTI General Assembly 2012, Washington, D.C., USA Technical Activities
  Coordinating Committee (TACC) Meeting, Data Management, Cloud
  Computing and the Long Tail of Science October 2012 Dennis Gannon
\item
  \url{http://www.microsoft.com/en-us/news/features/2012/mar12/03-05CloudComputingJobs.aspx}
\item
  \url{http://www.mckinsey.com/mgi/publications/big_data/index.asp}
\item
  Tom Davenport
  \url{http://fisheritcenter.haas.berkeley.edu/Big_Data/index.html}
\item
  \url{http://research.microsoft.com/en-us/people/barga/sc09_cloudcomp_tutorial.pdf}
\item
  \url{http://research.microsoft.com/pubs/78813/AJ18_EN.pdf}
\item
  \url{http://www.google.com/green/pdfs/google-green-computing.pdf}
\item
  \url{http://www.wired.com/wired/issue/16-07}
\item
  \url{http://research.microsoft.com/en-us/collaboration/fourthparadigm/}
\item
  Jeff Hammerbacher
  \url{http://berkeleydatascience.files.wordpress.com/2012/01/20120117berkeley1.pdf}
\item
  \url{http://grids.ucs.indiana.edu/ptliupages/publications/Where\%20does\%20all\%20the\%20data\%20come\%20from\%20v7.pdf}
\item
  \url{http://www.interactions.org/cms/?pid=1032811}
\item
  \url{http://www.quantumdiaries.org/2012/09/07/why-particle-detectors-need-a-trigger/atlasmgg/}
\item
  \url{http://www.sciencedirect.com/science/article/pii/S037026931200857X}
\item
  \url{http://www.slideshare.net/xamat/building-largescale-realworld-recommender-systems-recsys2012-tutorial}
\item
  \url{http://www.ifi.uzh.ch/ce/teaching/spring2012/16-Recommender-Systems_Slides.pdf}
\item
  \url{http://en.wikipedia.org/wiki/PageRank}
\item
  \url{http://pages.cs.wisc.edu/~beechung/icml11-tutorial/}
\item
  \url{https://sites.google.com/site/opensourceiotcloud/}
\item
  \url{http://datascience101.wordpress.com/2013/04/13/new-york-times-data-science-articles/}
\item
  \url{http://blog.coursera.org/post/49750392396/on-the-topic-of-boredom}
\item
  \url{http://x-informatics.appspot.com/course}
\item
  \url{http://iucloudsummerschool.appspot.com/preview}
\item
  \url{https://www.youtube.com/watch?v=M3jcSCA9_hM}
\end{itemize}



\input{chapter/lesson/iot/hardware}
\FILENAME

\section{Projects}\label{projects}

Please see the introduction to the IoT section to get started.

Term project suggestion combining IoT and Big Data:

\begin{enumerate}
\def\labelenumi{\arabic{enumi}.}
\tightlist
\item
  Recognizing street sign in a car robot with a camera
\item
  Regognizing street lines in a car robot with camera
\item
  Driving a Robot car swarm without collisions
\item
  Simulating a City with robot cars
\item
  Control a robot fish with cameras
\item
  Build a distributed sensor system (with your classmates)
\end{enumerate}

Drones:

\begin{enumerate}
\def\labelenumi{\arabic{enumi}.}
\tightlist
\item
  Control a drone swarm with positioning system
\end{enumerate}

Suggest your own

\input{chapter/lesson/iot/esp8266}


\chpater{Raspberry PI 3}\label{raspberry-pi-3}

\FILENAME

\section{Raspbery PI for IOT
(Gregor)}\label{raspbery-pi-for-iot-gregor}

\section{Hardware}\label{hardware}

see hardware page we have

\section{Installation}\label{installation}

\subsection{Erasing the SD Card}\label{erasing-the-sd-card}

Before you can install an OS on your sd card, you must erase it and put
it in the proper format.

\begin{enumerate}
\def\labelenumi{\arabic{enumi}.}
\tightlist
\item
  Insert your sd card into your micro-sd adapter and open Disk Utility
  with a spotlight search.
\item
  In the Disk Utility, right click the name of the sd card and select
  erase.
\item
  Name the sd card and format it as MS-DOS (FAT). Then click erase.

  \includegraphics[width=0.5\textwidth]{images/diskutil.png}

\item
  If it does not erase the first time, try again. It sometimes takes
  multiple tries to work.
\end{enumerate}

\subsection{Installation of NOOBS}\label{installation-of-noobs}

NOOBS is an OS that includes Raspian. The official descrition of
Raspbian can be found
\href{https://www.raspberrypi.org/downloads/raspbian/}{here}. It comes
pre-packaged with many useful programming tools, and is easy to use.

\begin{enumerate}
\def\labelenumi{\arabic{enumi}.}
\tightlist
\item
  Download Noobs
  \href{https://www.raspberrypi.org/downloads/noobs/}{here}. This will
  take around 30 minutes.
\item
  Go to your Finder and in Downloads, search for NOOBS.
\item
  Open the NOOBS folder and drag its contents into the sd card in the
  devices section. There should be 20 files and folders in the NOOBS
  folder. The download should take about 3 minutes.
\item
  Once installed, eject the sd card and put it in your raspberry pi.
\item
  Power up your raspberry and you will see a menu like this
\end{enumerate}

\begin{figure}[htb]
\centering
\includegraphics[width=0.5\textwidth]{images/noobs.jpg}
\caption{Noobs}
\end{figure}

\begin{enumerate}
\def\labelenumi{\arabic{enumi}.}
\setcounter{enumi}{5}
\tightlist
\item
  Select Raspbian and click \texttt{Install\ (i)}
\end{enumerate}

\subsection{Installation of Dexter}\label{installation-of-dexter}

The version of Dexter that you want to flash onto your sd card is called
Raspbian for Robots. This is a Raspbian based os that is compatible with
the GrovePi board. It also comes with pre-installed Dexter Industries
software.

\begin{enumerate}
\def\labelenumi{\arabic{enumi}.}
\tightlist
\item
  First, download the most recent Dexter\_Industries\_jessie.zip file
  from
  \href{https://sourceforge.net/projects/dexterindustriesraspbianflavor/}{here}.
\item
  Once the file has downloaded, uncompress it and insert your sd card
  into the micro-sd adapter.
\item
  Open etcher and flash the uncompressed jessie image onto the sd card.
\end{enumerate}

\begin{figure}[htb]
\centering
\includegraphics[width=0.5\textwidth]{images/etcher.png}
\caption{Etcher}
\end{figure}

\begin{enumerate}
\def\labelenumi{\arabic{enumi}.}
\setcounter{enumi}{3}
\tightlist
\item
  Eject your sd card and insert it into your raspberry pi.
\end{enumerate}

\section{Configure}\label{configure}

\subsection{Prepare OS}\label{prepare-os}

\section{Update}\label{update}

The following are essential updates:

\begin{verbatim}
sudo apt-get update
sudo apt-get upgrade
sudo apt-get install emacs
dpkg -l > ~/Desktop/packages.list
pip freeze > ~/Desktop/pip-freeze-initial.list
\end{verbatim}

The following are necessary for the scientific libraries, but they
require lots of space. Our sd cards do not have enough space for them.

\begin{verbatim}
sudo apt-get install build-essential python-dev python-distlib python-setuptools python-pip python-wheel libzmq-dev libgdal-dev
sudo apt-get install xsel xclip libxml2-dev libxslt-dev python-lxml python-h5py python-numexpr python-dateutil python-six python-tz python-bs4 python-html5lib python-openpyxl python-tables python-xlrd python-xlwt cython python-sqlalchemy python-xlsxwriter python-jinja2 python-boto python-gflags python-googleapi python-httplib2 python-zmq libspatialindex-dev
sudo pip install bottleneck rtree
\end{verbatim}

add to .bashrc

\begin{verbatim}
cd
git clone git://github.com/yyuu/pyenv.git .pyenv
echo 'export PYENV_ROOT="$HOME/.pyenv"' >> ~/.bashrc
echo 'export PATH="$PYENV_ROOT/bin:$PATH"' >> ~/.bashrc
echo 'eval "$(pyenv init -)"' >> ~/.bashrc
source ~/.bashrc

export PATH="/home/pi/.pyenv/bin:$PATH"
eval "$(pyenv init -)"
eval "$(pyenv virtualenv-init -)"

curl -L https://raw.githubusercontent.com/pyenv/pyenv-installer/master/bin/pyenv-installer | bash
\end{verbatim}

source

\subsection{Update to Python 3.6.1}\label{update-to-python-3.6.1}

\section{change python version}\label{change-python-version}

\begin{itemize}
\tightlist
\item
  {[}https://linuxconfig.org/how-to-change-from-default-to-alternative-python-version-on-debian-linux{]}
  (https://linuxconfig.org/how-to-change-from-default-to-alternative-python-version-on-debian-linux)
\end{itemize}

Upgrade setuptools for pip install with

\begin{verbatim}
    $ pip3 install --upgrade setuptools
    
\end{verbatim}

Test your python version with

\begin{verbatim}
    $ python --version
    
\end{verbatim}

Check your python version alternatives

\begin{verbatim}
    $ update-alternatives --list python
    
\end{verbatim}

If python2.7 is not in your alternatives, add it with

\begin{verbatim}
    $ sudo update-alternatives --install /usr/bin/python python /usr/bin/python2.7 1
    
\end{verbatim}

If python3.4 is not in your alternatives, add it with

\begin{verbatim}
    $ sudo update-alternatives --install /usr/bin/python python /usr/bin/python3.4 2
    
\end{verbatim}

Now make python3.4 to your default with

\begin{verbatim}
    update-alternatives --config python
\end{verbatim}

Select python3.4

\section{install 3.6.1}\label{install-3.6.1}

To install python 3.6.1, follow steps 1 and 2. This is unnecessary for
our purposes.

\begin{itemize}
\tightlist
\item
  \href{https://gist.github.com/dschep/24aa61672a2092246eaca2824400d37f}{better
  get 3.6.1}
\end{itemize}

\section{install cloudmesh.pi}\label{install-cloudmesh.pi}

pip install cloudmesh.pi

pip install cloudmesh.pi with

\begin{verbatim}
    $ git clone https://github.com/cloudmesh/cloudmesh.pi.git
    $ cd cloudmesh.pi
    $ sudo pip3 install .
\end{verbatim}

see how we do this in osx/linux can this be done on raspbery? if not
document update from source with altinstall

\subsection{Install scientific
Libraries}\label{install-scientific-libraries}

check if they are already installed we don't have enough space to
install all of these.

\begin{verbatim}
sudo apt-get install python-numpy python-matplotlib python-scipy python-sklearn python-pandas
\end{verbatim}

numpy\\
matplotlib\\
scipy\\
scikitlearn

\subsection{cloudmesh.pi (Jon)}\label{cloudmesh.pi-jon}

cloudmesh.pi is a repository for our GrovePi module classes. These
classes require Dexter software, so you need to either have Raspian for
Robots or download the software separately.

If you have Raspian for Robots, run the following in your terminal:

\begin{verbatim}
cd
mkdir github
cd github
git clone https://github.com/cloudmesh/cloudmesh.pi.git
cd cloudmesh.pi
sudo pip install .
\end{verbatim}

\subsection{Install VNC}\label{install-vnc}

describe how to install and configure VNC

\section{Sensors (Jon)}\label{sensors-jon}

\subsection{Grove Sensors (Jon)}\label{grove-sensors-jon}

we already have draft

\subsection{Non Grove Sensors (Jon)}\label{non-grove-sensors-jon}

Elegoo as example

\section{Notes To integrates}\label{notes-to-integrates}

\subsection{Connecting}\label{connecting}

Hostnames:

\begin{itemize}
\tightlist
\item
  raspberrypi.local
\item
  raspberrypi.
\end{itemize}

change

recovery.cmdline

forgot what these were:

\begin{verbatim}
runinstaller quiet ramdisk_size=32768 root=/dev/ram0 init=/init vt.cur_default=1 elevator=deadline
silentinstall runinstaller quiet ramdisk_size=32768 root=/dev/ram0 init=/init vt.cur_default=1 elevator=deadline
\end{verbatim}

Connect the cable

You will see the activity LEDs flash while the OS installs. Depending on
your SD-Card this can take up to 40-60 minutes.

\section{VLC on OSX}\label{vlc-on-osx}

\begin{itemize}
\item
  \url{http://www.videolan.org/vlc/index.en_GB.html}
\item
  \url{http://get.videolan.org/vlc/2.2.6/macosx/vlc-2.2.6.dmg}
\item
  \url{http://www.mybigideas.co.uk/RPi/RPiCamera/}
\item ~
  \section{Camera on Pi}\label{camera-on-pi}

  sudo apt-get install vlc
\item
  \url{https://www.raspberrypi.org/learning/getting-started-with-picamera/worksheet/}
\item
  \url{https://www.hackster.io/bestd25/pi-car-016e66}
\end{itemize}

\section{Streaming video}\label{streaming-video}

\begin{itemize}
\tightlist
\item
  \url{https://blog.miguelgrinberg.com/post/stream-video-from-the-raspberry-pi-camera-to-web-browsers-even-on-ios-and-android}
\end{itemize}

\section{Linux Commandline}\label{linux-commandline}

\begin{itemize}
\tightlist
\item
  \url{http://www.computerworld.com/article/2598082/linux/linux-linux-command-line-cheat-sheet.html}
\end{itemize}

\section{Enable SPI}\label{enable-spi}

go to the configuration interfaces and enable

\section{RTIMUlib2}\label{rtimulib2}

git clone https://github.com/RTIMULib/RTIMULib2.git cd RTIMULib

Add the following two lines to /etc/modules

\begin{verbatim}
i2c-bcm2708
i2c-dev
\end{verbatim}

reboot

\begin{verbatim}
ls /dev/i2c-*
sudo apt-get install i2c-tools

sudo i2cdetect -y 1
         0  1  2  3  4  5  6  7  8  9  a  b  c  d  e  f
00:          -- -- -- -- -- -- -- -- -- -- -- -- -- 
10: -- -- -- -- -- -- -- -- -- -- -- -- -- -- -- -- 
20: -- -- -- -- -- -- -- -- -- -- -- -- -- -- -- -- 
30: -- -- -- -- -- -- -- -- -- -- -- -- -- -- -- -- 
40: -- -- -- -- -- -- -- -- -- -- -- -- -- -- -- -- 
50: -- -- -- -- -- -- -- -- -- -- -- -- -- -- -- -- 
60: -- -- -- -- -- -- -- -- 68 -- -- -- -- -- -- -- 
70: -- -- -- -- -- -- -- --
\end{verbatim}

\begin{figure}[htb]
\centering
\includegraphics[width=0.5\textwidth]{images/rasp3.jpg}
\caption{Pinout}
\end{figure}

create a file /etc/udev/rules.d/90-i2c.rules and add the line:

\begin{verbatim}
KERNEL=="i2c-[0-7]",MODE="0666"
\end{verbatim}

note: does not work

instead we do

\begin{verbatim}
sudo chmod 666 /dev/i2c-1 
\end{verbatim}

Set the I2C bus speed to 400KHz by adding to /boot/config.txt:

\begin{verbatim}
dtparam=i2c1_baudrate=400000
\end{verbatim}

reboot. In terminal change directories to

\begin{verbatim}
cd /home/pi/github/RTIMULib2/RTIMULib/IMUDrivers
\end{verbatim}

and open

\begin{verbatim}
emacs RTIMUDefs.h
\end{verbatim}

In RTIMUDefs.h change

\begin{verbatim}
#define MPU9250_ID 0x71
\end{verbatim}

To

\begin{verbatim}
#define MPU9250_ID 0x73



cd /home/pi/github/RTIMULib2/RTIMULib
\end{verbatim}

In terminal

\begin{verbatim}
mkdir build
cd build
cmake ..
make -j4
sudo make install
sudo ldconfig
\end{verbatim}

\section{Compile RTIMULib Apps}\label{compile-rtimulib-apps}

\begin{verbatim}
cd /home/pi/github/RTIMULib2/Linux/RTIMULibCal
make clean; make -j4
sudo make install
cd /home/pi/github/RTIMULib2/Linux/RTIMULibDrive
make clean; make -j4
sudo make install
cd /home/pi/github/RTIMULib2/Linux/RTIMULibDrive10
make clean; make -j4
sudo make install
cd /home/pi/github/RTIMULib2/Linux/RTIMULibDrive11
make clean; make -j4
sudo make install


cd /home/pi/github/RTIMULib2/Linux/RTIMULibDemo    
qmake clean
make clean
qmake
make -j4
sudo make install
cd /home/pi/github/RTIMULib2/Linux/RTIMULibDemoGL
qmake clean
make clean
qmake
make -j4
sudo make install
\end{verbatim}

\section{Camera}\label{camera}

\begin{itemize}
\tightlist
\item
  \href{https://www.raspberrypi.org/learning/getting-started-with-picamera/worksheet/}{Camera
  Tutorial}
\end{itemize}

.

\begin{verbatim}
sudo apt-get install libjpeg-dev libtiff5-dev libjasper-dev libpng12-dev
sudo apt-get install libavcodec-dev libavformat-dev libswscale-dev libv4l-dev

sudo apt-get install libxvidcore-dev libx264-dev

sudo pip install virtualenv virtualenvwrapper
sudo rm -rf ~/.cache/pip
\end{verbatim}

copy into \textasciitilde{}/.profile:

\begin{verbatim}
 echo -e "\n# virtualenv and virtualenvwrapper" >> ~/.profile
 echo "export WORKON_HOME=$HOME/.virtualenvs" >> ~/.profile
 echo "source /usr/local/bin/virtualenvwrapper.sh" >> ~/.profile
\end{verbatim}
%$

source \textasciitilde{}/.profile

\begin{verbatim}
mkvirtualenv cv -p python3
\end{verbatim}

workon cv

comandline has (cv) in front

\begin{verbatim}
pip install numpy

wget -O opencv.zip https://github.com/Itseez/opencv/archive/3.1.0.zip
wget -O opencv_contrib.zip https://github.com/Itseez/opencv_contrib/archive/3.1.0.zip
unzip opencv.zip
unzip opencv_contrib.zip
\end{verbatim}

\section{Lessons and Projects}\label{lessons-and-projects}

\begin{itemize}
\tightlist
\item
  \href{https://www.raspberrypi.org/learning/getting-started-with-guis/worksheet/}{Gui}\\
\item
  \href{https://www.raspberrypi.org/learning/getting-started-with-guis/}{Solder}\\
\item
  \href{https://www.raspberrypi.org/blog/an-image-processing-robot-for-robocup-junior/}{PI
  Camera Line Follower}\\
\item
  \href{https://circuitdigest.com/microcontroller-projects/web-controlled-raspberry-pi-surveillance-robot}{Pi
  car flask}
\end{itemize}

\section{OTHER TO BE INTEGRATED}\label{other-to-be-integrated}

\subsection{PI emulator on Windows}\label{pi-emulator-on-windows}

We have not yet tried it, but we like to hear back from you on
experiences with

\begin{itemize}
\tightlist
\item
  https://sourceforge.net/projects/rpiqemuwindows/
\end{itemize}

\subsection{Scratch}\label{scratch}

\begin{itemize}
\tightlist
\item
  \href{https://github.com/DexterInd/GrovePi/tree/master/Software/Scratch}{scratch}
\end{itemize}

\section{Web Server}\label{web-server}

\begin{itemize}
\tightlist
\item
  \href{https://www.raspberrypi.org/learning/python-web-server-with-flask/worksheet/}{Web
  Server Flask}
\end{itemize}

\FILENAME

\section{Dexter}\label{dexter}

\subsection{Creating an SD Card}\label{creating-an-sd-card}

\subsubsection{OSX}\label{osx}

First, install Etcher from \href{https://etcher.io/}{etcher.io} which
allows you to flash images onto the SD card. When flashing make sure you
only attach one USB SD card reader/wroiter or use the build in SD card
slot provided in some Mac's.

The version of etcher we used is

\begin{itemize}
\tightlist
\item
  \href{https://github.com/resin-io/etcher/releases/download/v1.1.1/Etcher-1.1.1-darwin-x64.dmg}{Etcher-1.1.1-darwin-x64.dmg}
\end{itemize}

Make sure to check if there is a newer version

\subsubsection{Dexter Rasbian}\label{dexter-rasbian}

Dexter provides a special image that contains the drivers and sample
programs for the GrovePi shield. We had some issues installing it on a
plain Raspbian OS, thus we recommend that you use dexters version if you
use the GrovePi shield. It is available from

\begin{itemize}
\tightlist
\item
  \href{http://sourceforge.net/projects/dexterindustriesraspbianflavor/}{Google
  Drive}
\item
  \href{http://sourceforge.net/projects/dexterindustriesraspbianflavor/}{Sourceforge}
\end{itemize}

Detailed information on how to generate an SD card while using your OS
is provided at

\begin{itemize}
\tightlist
\item
  \url{https://www.dexterindustries.com/howto/install-raspbian-for-robots-image-on-an-sd-card/}
\end{itemize}

\subsubsection{Github}\label{github}

Dexter maintains a github repository that includes their code for the
shield and many other projects at

\begin{itemize}
\tightlist
\item
  \url{https://github.com/DexterInd}
\end{itemize}

\subsubsection{Cloning Grove PI}\label{cloning-grove-pi}

To clone the GrovePI library on other computers you can use the command

\begin{verbatim}
git clone https://github.com/DexterInd/GrovePi.git
\end{verbatim}

\subsubsection{Dexter Sample programs}\label{dexter-sample-programs}

Dexter maintains all GrovePi related programs at

\begin{itemize}
\tightlist
\item
  \url{https://github.com/DexterInd/GrovePi}
\end{itemize}

The python related programs are in a subdirectory at

\begin{itemize}
\tightlist
\item
  \url{https://github.com/DexterInd/GrovePi/tree/master/Software/Python}
\end{itemize}

Here you find many programs and for a complete list visit that link.
Dependent on the sensors and actuators you have, inspect some programs.
Some of them may inspire you to purchase some sensors.

We have developed a partial library of GrovePi module classes at

\begin{itemize}
\tightlist
\item
  \url{https://github.com/cloudmesh/cloudmesh.pi/tree/master/cloudmesh/pi}
\end{itemize}



\chapter{GrovePi Modules}\label{grovepi-modules}

\FILENAME

\section{Introduction}\label{intro}

\begin{itemize}

\item
  \href{http://www.instructables.com/id/Basic-Electronics}{Electronics}:
  An introduction to the basic principals of electronics.
\item
  \href{https://learn.sparkfun.com/tutorials/voltage-current-resistance-and-ohms-law}{Volatage}:
  An introduction to the physics of electricity.
\item
  \href{https://info-ee.eps.surrey.ac.uk/Teaching/Unix/index.html}{Unix}:
  An introduction to the Unix os.
\item
  \href{https://github.com/DexterInd/GrovePi/tree/master/Software/Python}{grove
  examples}: A list of Dexter Industries example code for GrovePi
  modules.
\item
  \href{https://github.com/cloudmesh/cloudmesh.pi/tree/master/cloudmesh/pi}{GrovePi
  module classes}: A repository for the GrovePi module classes.
\end{itemize}

\section{LED}\label{led}

An LED is the simplest possible module for a raspberry pi, as it is
responsive only to the provided power. For an LED to emit light, it must
be exposed to a voltage greater than a certain threshold value. Above
this voltage, the conductivity of the diode increases exponentially and
its brightness increases likewise. If the current through the LED
becomes too high, the LED will burn out. The following link leads to a
tutorial from Dexter Industries for the LED module.

\begin{itemize}

\item
  \href{https://www.dexterindustries.com/GrovePi/projects-for-the-raspberry-pi/raspberry-pi-led-tutorial/}{Dexter
  LED tutorial}
\end{itemize}

Connect the LED To a digital port. The following code describes an LED
class. Since it is connected to a digital output, the voltage has only
two states, on and off. The default port for the LED class is D3. The
code for the \texttt{LED} class can be found here:

\begin{itemize}

\item
  \href{https://github.com/cloudmesh/cloudmesh.pi/blob/master/cloudmesh/pi/led.py}{LED
  Class}
\end{itemize}

\begin{figure}
\centering
\includegraphics{images/led.jpg}
\caption{LED}
\end{figure}

\section{Buzzer}\label{buzzer}

Connect the buzzer to a digital port. The default port for the Buzzer
class is D3. You will notice that the Buzzer class and the LED class are
interchangeable. This is because they work on the same digital
principal. Their two values are on and off. The code for the
\texttt{Buzzer} class can be found here:

\begin{itemize}

\item
  \href{https://github.com/cloudmesh/cloudmesh.pi/blob/master/cloudmesh/pi/buzzer.py}{Buzzer
  Class}
\end{itemize}

\begin{figure}
\centering
\includegraphics{../images/grovepi/buzzer.jpg}
\caption{Buzzer}
\end{figure}

\section{Relay}\label{relay}

The relay acts as a switch in a circuit. When the value on the relay is
1, it allows current to flow through it. When the value is 0, the relay
breaks the circuit and the current stops. Connect the relay to a digital
port. The default digital port is D4. The \texttt{Relay} class can be
found here:

\begin{itemize}

\item
  \href{https://github.com/cloudmesh/cloudmesh.pi/blob/master/cloudmesh/pi/relay.py}{Relay
  Class}
\end{itemize}

\begin{figure}
\centering
\includegraphics{../images/grovepi/relay.jpg}
\caption{Relay}
\end{figure}

\section{Light Sensor}\label{light-sensor}

The light sensor measures ligh intensity and returns a value between 0
and 1023. Connect the light sensor to an analog port. The default port
is A0. The analog port allows the light sensor to return a range of
values. The \texttt{LightSensor} class can be found here:

\begin{itemize}

\item
  \href{https://github.com/cloudmesh/cloudmesh.pi/blob/master/cloudmesh/pi/light.py}{LightSensor
  Class}
\end{itemize}

\begin{figure}
\centering
\includegraphics{../images/grovepi/light.jpg}
\caption{Light Sensor}
\end{figure}

\section{Rotary Angle Sensor}\label{rotary-angle-sensor}

The rotary angle sensor measures the angle to which it is turned.
Connect the sensor to an analog port. Port A0 is the default. The
\texttt{RotarySensor} class can be found here:

\begin{itemize}

\item
  \href{https://github.com/cloudmesh/cloudmesh.pi/blob/master/cloudmesh/pi/rotary.py}{RotarySensor
  Class}
\end{itemize}

\begin{figure}
\centering
\includegraphics{../images/grovepi/rotary.jpg}
\caption{Rotary Angle Sensor}
\end{figure}

\section{Barometer}\label{barometer}

Connect the barometer to an I2C port. In addition to pressure, the
GrovePi barometer measures temperature in Fahrenheit and Celcius. The
\texttt{Barometer} class can be found here.

\begin{itemize}

\item
  \href{https://github.com/cloudmesh/cloudmesh.pi/blob/master/cloudmesh/pi/barometer.py}{Barometer
  Class}
\end{itemize}

\begin{figure}
\centering
\includegraphics{../images/grovepi/barometer.jpg}
\caption{Barometer}
\end{figure}

\section{Distance Sensor}\label{distance-sensor}

Connect the distance sensor to a digital port. The grovepi module has a
built-in function to read the distance from the distance sensor, but it
is improperly calibrated, so this DistanceSensor class has a calibration
based on experimental data. The \texttt{DistanceSensor} class can be
found here:

\begin{itemize}

\item
  \href{https://github.com/cloudmesh/cloudmesh.pi/blob/master/cloudmesh/pi/distance.py}{DistanceSensor
  Class}
\end{itemize}

\begin{figure}
\centering
\includegraphics{../images/grovepi/distance.jpg}
\caption{Distance Sensor}
\end{figure}

\section{Temperature Sensor}\label{temperature-sensor}

The temperature sensor measures both temperature and humidity. Connect
the temperature sensor to a digital port. D7 is the default port. The
\texttt{TemperatureSensor} class can be found here:

\begin{itemize}

\item
  \href{https://github.com/cloudmesh/cloudmesh.pi/blob/master/cloudmesh/pi/temperature.py}{TemperatureSensor
  Class}
\end{itemize}

\begin{figure}
\centering
\includegraphics{../images/grovepi/temperature.jpg}
\caption{Temperature Sensor}
\end{figure}

\section{Heartbeat Sensor}\label{heartbeat-sensor}

Connect the heartbeat sensor to an I2C port. The heartbeat sensor
returns the heart rate of the wearer. The \texttt{HeartbeatSensor} class
can be found here:

\begin{itemize}

\item
  \href{https://github.com/cloudmesh/cloudmesh.pi/blob/master/cloudmesh/pi/heartbeat.py}{HearbeatSensor
  Class}
\end{itemize}

\begin{figure}
\centering
\includegraphics{../images/grovepi/heartbeat.jpg}
\caption{image}
\end{figure}

\section{Joystick}\label{joystick}

Connect the joystick to an analog port. A0 is the default port. The
joystick has an x, y, and click status based on the current state of the
module. The \texttt{Joystick} class can be found here:

\begin{itemize}

\item
  \href{https://github.com/cloudmesh/cloudmesh.pi/blob/master/cloudmesh/pi/joystick.py}{Joystick
  Class}
\end{itemize}

\begin{figure}
\centering
\includegraphics{../images/grovepi/joystick.jpg}
\caption{image}
\end{figure}

\section{LCD Screen}\label{lcd-screen}

The LCD screen can be used to display text and colors. In order to use
it, plug it into one of the I2C ports. The \texttt{LCD} class can be
found here:

\begin{itemize}

\item
  \href{https://github.com/cloudmesh/cloudmesh.pi/blob/master/cloudmesh/pi/lcd.py}{LCD
  Class}
\end{itemize}

\begin{figure}
\centering
\includegraphics{../images/grovepi/lcd.jpg}
\caption{LCD Screen}
\end{figure}

\section{Moisture Sensor}\label{moisture-sensor}

Connect the moisture sensor to an analog port. The default port is A0.
The \texttt{MoistureSensor} class can be found here:

\begin{itemize}

\item
  \href{https://github.com/cloudmesh/cloudmesh.pi/blob/master/cloudmesh/pi/moisture.py}{MoistureSensor
  Class}
\end{itemize}

\begin{figure}
\centering
\includegraphics{../images/grovepi/moisture.jpg}
\caption{Moisture Sensor}
\end{figure}

An example of the implimentation of the moisture sensor from Dexter
Industries can be found
\href{https://github.com/DexterInd/GrovePi/blob/master/Projects/plant_monitor/plant_project.py}{here}.
The program is meant to measure the environmental conditions that affect
plant growth.

\section{Water Sensor}\label{water-sensor}

The water sensor measures the amount of water in the environment of the
sensor. Connect the sensor to a digital point. D2 is the default port.
The \texttt{WaterSensor} class can be found here:

\begin{itemize}

\item
  \href{https://github.com/cloudmesh/cloudmesh.pi/blob/master/cloudmesh/pi/water.py}{WaterSensor
  Class}
\end{itemize}

\begin{figure}
\centering
\includegraphics{../images/grovepi/water.jpg}
\caption{Water Sensor}
\end{figure}

\FILENAME

\section{Sensors}\label{sensors}

This section is to be completed by the students of the class.

Task is to develop an object oriented class for one of the sensors. An
example for such a class can be found at:

\begin{itemize}
\tightlist
\item
  \url{https://github.com/cloudmesh/cloudmesh.pi/blob/master/cloudmesh/pi/led.py}
\end{itemize}

\subsection{Compass}\label{compass}

TODO: which compas sensor

The default pins are defined in variants/nodemcu/pins\_arduino.h as GPIO

\begin{verbatim}
SDA=4 
SCL=5
D1=5 
D2=4.
\end{verbatim}

You can also choose the pins yourself using the I2C constructor
Wire.begin(int sda, int scl);



\chapter{VNC}\label{vnc}

\FILENAME

\textbf{Note:} \emph{If you like to connect to your Raspberry from your
laptop, we recommend to use VNC. If you rather like to connect a monitor
and keyboard as well as a mouse to the Raspberry, you can skip the steps
with the VNC update.}

\section{Setting up VNC}\label{setting-up-vnc}

We had some issues with the installed version of VNC that is customized
for connecting a Laptop via the ethernet cable to the PI. However as we
connect wirelessly, our setup is slightly diffrent. The easiset way that
we found is to update the Raspbian OS as follows. In a terminal type

\begin{verbatim}
sudo apt-get update
sudo apt-get install realvnc-vnc-server 
sudo apt-get install realvnc-vnc-viewer
\end{verbatim}

Next you enable the VNC server in the configuration panel via the
Rasbian GUI by selecting

\begin{verbatim}
 Menu > 
    Preferences > 
       Raspberry Pi Configuration > 
          Interfaces.
\end{verbatim}

Here you toggle the VNC service to enabled. As we are already at it in
our setup we enabled all other services, especially those that deal with
Grove sensor related bins and wires.

Next reboot and double check if the settings are preseved after the
reboot

\subsection{Install VNC on OSX}\label{install-vnc-on-osx}

To install a vnc server of your liking on your Mac. You find one at

\begin{itemize}

\item
  \href{http://www.realvnc.com/download/vnc/latest/\%5D}{http://www.realvnc.com/download/vnc/latest/}
\end{itemize}

Be sure to download the version of the VNC Viewer for the computer you
are going to use to virtually control the Pi (there is a version listed
for Raspberry Pi-- don't download this one. For us this is the Mac
version.)

\subsection{Run VNC Viewer on OSX}\label{run-vnc-viewer-on-osx}

Once you have downloaded the VNC viewer installed it you can open the
program. Next you can start vnc viewer and enter the ip address of your
raspberry. Make sure you are on the same network. You can find the
address by using ifconfig.



\chpater{Turtle Graphics}\label{turtle-graphics}
\FILENAME

\section{Demo}\label{demo}

Python comes with a nice demonstartion program that allows you to learn
some simple programming concepts while moving a turtle on the screen. It
can be started with

\begin{verbatim}
python -m turtledemo
\end{verbatim}

\section{Program example}\label{program-example}

You can also create programs with your favorite editor and run it. Let
us put the following code into the program \texttt{turtle\_demo.py}.
Never save a file with the name \texttt{turtle.py} because python will
import it instead of the built-in turtle import that you need.

\begin{verbatim}
import turtle

window = turtle.Screen() 
robot = turtle.Turtle() 

robot.forward(50)   # Moves forward 50 pixels
robot.right(90)     # Rotate clockwise by 90 degrees

robot.forward(50)
robot.right(90)

robot.forward(50)
robot.right(90)

robot.forward(50)
robot.right(90)

turtle.done()

window.mainloop()
\end{verbatim}

After saving it you can run it from a terminal with

\begin{verbatim}
$ python turtle_demo.py
\end{verbatim}

\section{Shape}\label{shape}

\begin{verbatim}
shapes: “arrow”, “turtle”, “circle”, “square”, “triangle”, “classic”
\end{verbatim}

You can change the shape of your turtle to any of these shapes with the
Turtle method \texttt{shape(name)}. For example, if you have an instance
of the Turtle class called \texttt{robot}, you can make it appear as a
turtle by calling \texttt{robot.shape("turtle")}.

You can add your own shapes with the following functions:

\begin{verbatim}
turtle.register_shape(name, shape=None)

turtle.addshape(name, shape=None)
\end{verbatim}

There are three different ways to call this function:

name is the name of a gif-file and shape is None: Install the
corresponding image shape.

\begin{verbatim}
window.register_shape("turtle.gif")
\end{verbatim}

Note: Image shapes do not rotate when turning the turtle, so they do not
display the heading of the turtle!

name is an arbitrary string and shape is a tuple of pairs of
coordinates: Install the corresponding polygon shape.

\begin{verbatim}
window.register_shape("triangle", ((5,-3), (0,5), (-5,-3)))
\end{verbatim}

name is an arbitrary string and shape is a (compound) Shape object:
Install the corresponding compound shape.

Add a turtle shape to TurtleScreen's shapelist. Only thusly registered
shapes can be used by issuing the command shape(shapename).

\section{Links}\label{links}

\begin{itemize}
\tightlist
\item
  http://openbookproject.net/thinkcs/python/english3e/hello\_little\_turtles.html
\item
  \url{https://docs.python.org/3/library/turtle.html}
\end{itemize}

\section{Robot Dance Simulator}\label{robot-dance-simulator}

\begin{verbatim}
cms robot dance dance.txt
\end{verbatim}

\section{Scratch}\label{scratch}

\begin{itemize}
\tightlist
\item
  \href{https://scratch.mit.edu/scratchr2/static/sa/Scratch-456.0.2.dmg}{Scratch}
\end{itemize}

\section{MBlock}\label{mblock}

\begin{itemize}
\tightlist
\item
  \href{http://www.mblock.cc/download/}{MBlock}
\end{itemize}



\chapter{Tools}\label{tools}

\FILENAME

\begin{itemize}
\item
  \textbf{Terminal}: On OSX, when you navigate to the search
  magnification glass, you can type in \emph{terminal} to start it. A
  terminal allows you to execute a number of commands to interact with
  the computer from a commandline interface, e.g.~the terminal.
\item
  \href{https://linuxconfig.org/bash-scripting-tutorial}{Bash} it the
  command language used in terminal.
\item
  \href{https://cloudmesh.github.io/classes/lesson/prg/pyenv.html?highlight=xcode\#install-pyenv-on-osxhttps://cloudmesh.github.io/classes/lesson/prg/pyenv.html?highlight=xcode\#install-pyenv-on-osx}{Pyenv}
  allows to manage multiple versions of python easily.
  \href{https://github.com/pyenv/pyenv\#how-it-works}{Pyenv link}
\item
  \href{https://cloudmesh.github.io/classes/lesson/prg/pyenv.html?highlight=xcode\#install-pyenv-on-osxhttps://cloudmesh.github.io/classes/lesson/prg/pyenv.html?highlight=xcode\#install-pyenv-on-osx}{XCode}
  is an integrated development environment for macOS containing a suite
  of software development tools developed by Apple for developing
  software for macOS, iOS, watchOS and tvOS.
\item
  \href{https://brew.sh}{Homebrew} is a \emph{package manager} for OS X
  which lets the user \emph{install software} from \emph{UNIX} and
  \emph{open source software} that is not included in OSX.
\item
  \href{https://www.jetbrains.com/pycharm/download/download-thanks.html?platform=mac\&code=PCC}{pyCharm}:
  is an Integrated Development Environment for Python.
\item
  \emph{Matplotlib}: Matplotlib is a libarry that allows us to create
  nice graphs in python. As we typically install python with virtualenv,
  we need to configure matplotlib properly to use it. The easiest way to
  do this is to execute the following commands. After you run them you
  can use matplotlib.

\begin{verbatim}
$ pip install numpy
$ pip install matplotlib
$ echo "backend : TkAgg" > ~/.matplotlib/matplotlibrc
\end{verbatim}
\item
  \href{https://macdown.uranusjr.com/}{Macdown} a macdown editor for OSX
\item
  \href{https://blog.ghost.org/markdown/}{Markdown} (from Markdown)
\item
  \href{http://oracc.museum.upenn.edu/doc/help/usingemacs/aquamacs/}{AquaEmacs}
  (from Aquaemacs)
\item
  \href{http://marvelmind.com/}{Marvelmind} (from Marvelmind if you have
  marvelmind positioning sensors which are optional)
\item
  \href{https://www.arduino.cc/en/guide/macOSX}{Arduino} (from Arduino
  if you like to use their interface to access the esp8266 boards)
\item
  \href{https://computers.tutsplus.com/tutorials/40-terminal-tips-and-tricks-you-never-thought-you-needed--mac-51192}{40
  OSX Terminal Tricks}
\end{itemize}

\section{Markdown}\label{markdown}

MarkDown is a format convention that produces nicely formated text with
simple ASCII text. Markdown has very good support for editors that
render the final output in a view window next to the editor pane. Two
such editors are

\begin{itemize}
\tightlist
\item
  \href{https://macdown.uranusjr.com/}{Macdown}: MacDown provides a nice
  integrated editor that works well.
\item
  \href{https://www.jetbrains.com/pycharm/download/download-thanks.html?platform=mac\&code=PCC}{pyCharm}:
  We have successfully used Vladimir Schhneiders
  \href{https://plugins.jetbrains.com/plugin/7896-markdown-navigator}{Markdown
  Navigator plugin}. Once installes you click on a .md file pycharm will
  automatically ask to install the plugins from Markdown for you.
\end{itemize}

A detailed set of syntax rules can be found at: \textbf{BUG: LINK TO
MARKDOWN SYNTAX MISSING}

The following are some basic examples

\begin{itemize}
\tightlist
\item
  To \emph{empazise} a text you use \texttt{*empasize*}
\item
  To make text \textbf{bold} use \texttt{**bold**}
\item
  To make text \textbf{\emph{bold-and-emphasize}} use
  \texttt{***bold-and-emphasize***}
\item
  To create a hyperlink use \texttt{{[}Google{]}(https://google.com)}
  which will result in \href{https://google.com}{Google}
\item
  To include an image use \texttt{!{[}Bracketed\ Text{]}(link)}
\end{itemize}

A list can be created by item starting with ``*``, a''-``, or a''+" or a
number

\begin{verbatim}
1. one
2. two
\end{verbatim}

\begin{enumerate}
\def\labelenumi{\arabic{enumi}.}
\item
  one
\item
  two

\begin{verbatim}
* one
* two
\end{verbatim}
\end{enumerate}

\begin{itemize}
\tightlist
\item
  one
\item
  two
\end{itemize}

If you need to indent items underneath already bulleted items, precede
the indent items with four spaces and they will be nested under the item
above them.

To qoute textc precede it with a ``\textgreater{}''.

\begin{verbatim}
> Quote
\end{verbatim}

\begin{quote}
Quote
\end{quote}

Other syntax options can be found in the Format drop-down at the top of
the screen between View and Plug-ins of macdown.

\section{Aquamacs}\label{aquamacs}

There are many different versions of emacs available on OSX. Aquamacs is
often used as it integrates nicely with the OSX GUI interface.

\begin{itemize}
\tightlist
\item
  \href{http://aquamacs.org/download.shtml}{AquaEmacs}
\end{itemize}

\emph{Aquamacs} is a program for Mac devices which allows the user to
edit text, HTML, LaTeX, C++, Java, Python, R, Perl, Ruby, PHP, and more.
Aquaemacs integrates well with OSX and provides many functions through a
menu. You will mostly be using the File, Edit, menus or toolbar icons.

Emacs provides convenient keyboard shortcuts, most of which are
combinations with the Control or Meta key (The Meta key is the ESC key).
If you accidentally end up doing something wrong simply press
\texttt{CTRL-g} to get out without issue. Other Keyboard Shortcuts
include:

\begin{itemize}
\item
  \texttt{CTRL-x\ u} or File\textgreater{}Undo will cancel any command
  that you did not want done. (CHECK)
\item
  \texttt{ESC-g} will cancel any command you are in the middle of.
\item
  You can break paragraph lines with \texttt{Ctrl-x\ w}, where
  \texttt{w} will wrap text around word boundaries.
\item
  To delete text to the end of the current word, press \texttt{ESC-d}.
\item
  to delete the whole line from the position of the cursor to the end,
  press \texttt{CTRL-k}.
\end{itemize}

\section{Bash}\label{bash}

Bash is preinstalled in OSX. A \emph{bash} script contains
\emph{commands} in plain text. In order to create a bash script please
decide for a convenient name. Ltes assume we name our script
\emph{myscript}. Than you can create and edit such a script with

\begin{verbatim}
$ touch myscript.sh
$ emacs myscripts.sh
\end{verbatim}

Next you need to add the following line to the top ogf the script:

\begin{verbatim}
!# /bin/bash
\end{verbatim}

To demonstrate how to continue writing a script we will be using the
bash \texttt{echo} command that allows you to print text. Lets make the
second line

\begin{verbatim}
echo "Hello World"
\end{verbatim}

You can now save and start executing your script. Click ``File'' and
then ``Save''. Open Terminal and type in \texttt{cd} followed by the
name of the folder you put the document in. Now we need to execute the
script.

\emph{Executing} a Bash script is rather easy. In order to execute a
script, we need to first execute the \emph{permission set}. In order to
give Terminal permission to read/execute a Bash script, you have to type

\begin{verbatim}
chmod u+x myscript.sh
\end{verbatim}

After the script has been granted permission to be executed, you can
test it by typing

\begin{verbatim}
./myscript.sh
\end{verbatim}

into the terminal. You will see it prints

\begin{verbatim}
Hello World
\end{verbatim}

\section{Arduino}\label{arduino}

This instalation is optional. In the event that there is a TTY error,
you will need to install Arduino, since your Mac may be missing some
drivers that are included in Arduino. Simply go to
\href{https://www.arduino.cc/en/guide/macOSX}{Arduino} and follow the
instalation instructions.

\section{OSX Terminal}\label{osx-terminal}

\href{https://learn.sparkfun.com/tutorials/terminal-basics/coolterm-windows-mac-linux}{CoolTerm}

download \url{http://freeware.the-meiers.org/CoolTermMac.zip}


%----------------------------------------------------------------------------------------
%	COMMENT
%----------------------------------------------------------------------------------------


\begin{comment}

\part{Book Format}

%----------------------------------------------------------------------------------------
%	CHAPTER 1
%----------------------------------------------------------------------------------------

\chapterimage{chapter_head_2.pdf} % Chapter heading image

\chapter{LISTS}

\section{Paragraphs of Text}\index{Paragraphs of Text}

\lipsum[1-7] % Dummy text

%------------------------------------------------

\section{Citation}\index{Citation}

This statement requires citation \cite{article_key}; this one is more specific \cite[162]{book_key}.

%------------------------------------------------

\section{Lists}\index{Lists}

Lists are useful to present information in a concise and/or ordered way\footnote{Footnote example...}.

\subsection{Numbered List}\index{Lists!Numbered List}

\begin{enumerate}
\item The first item
\item The second item
\item The third item
\end{enumerate}

\subsection{Bullet Points}\index{Lists!Bullet Points}

\begin{itemize}
\item The first item
\item The second item
\item The third item
\end{itemize}

\subsection{Descriptions and Definitions}\index{Lists!Descriptions and Definitions}

\begin{description}
\item[Name] Description
\item[Word] Definition
\item[Comment] Elaboration
\end{description}


%----------------------------------------------------------------------------------------
%	CHAPTER 2
%----------------------------------------------------------------------------------------


\section{Theorems}\index{Theorems}

This is an example of theorems.

\subsection{Several equations}\index{Theorems!Several Equations}
This is a theorem consisting of several equations.

\begin{theorem}[Name of the theorem]
In $E=\mathbb{R}^n$ all norms are equivalent. It has the properties:
\begin{align}
& \big| ||\mathbf{x}|| - ||\mathbf{y}|| \big|\leq || \mathbf{x}- \mathbf{y}||\\
&  ||\sum_{i=1}^n\mathbf{x}_i||\leq \sum_{i=1}^n||\mathbf{x}_i||\quad\text{where $n$ is a finite integer}
\end{align}
\end{theorem}

\subsection{Single Line}\index{Theorems!Single Line}
This is a theorem consisting of just one line.

\begin{theorem}
A set $\mathcal{D}(G)$ in dense in $L^2(G)$, $|\cdot|_0$. 
\end{theorem}

%------------------------------------------------

\section{Definitions}\index{Definitions}

This is an example of a definition. A definition could be mathematical or it could define a concept.

\begin{definition}[Definition name]
Given a vector space $E$, a norm on $E$ is an application, denoted $||\cdot||$, $E$ in $\mathbb{R}^+=[0,+\infty[$ such that:
\begin{align}
& ||\mathbf{x}||=0\ \Rightarrow\ \mathbf{x}=\mathbf{0}\\
& ||\lambda \mathbf{x}||=|\lambda|\cdot ||\mathbf{x}||\\
& ||\mathbf{x}+\mathbf{y}||\leq ||\mathbf{x}||+||\mathbf{y}||
\end{align}
\end{definition}

%------------------------------------------------

\section{Notations}\index{Notations}

\begin{notation}
Given an open subset $G$ of $\mathbb{R}^n$, the set of functions $\varphi$ are:
\begin{enumerate}
\item Bounded support $G$;
\item Infinitely differentiable;
\end{enumerate}
a vector space is denoted by $\mathcal{D}(G)$. 
\end{notation}

%------------------------------------------------

\section{Remarks}\index{Remarks}

This is an example of a remark.

\begin{remark}
The concepts presented here are now in conventional employment in mathematics. Vector spaces are taken over the field $\mathbb{K}=\mathbb{R}$, however, established properties are easily extended to $\mathbb{K}=\mathbb{C}$.
\end{remark}

%------------------------------------------------

\section{Corollaries}\index{Corollaries}

This is an example of a corollary.

\begin{corollary}[Corollary name]
The concepts presented here are now in conventional employment in mathematics. Vector spaces are taken over the field $\mathbb{K}=\mathbb{R}$, however, established properties are easily extended to $\mathbb{K}=\mathbb{C}$.
\end{corollary}

%------------------------------------------------

\section{Propositions}\index{Propositions}

This is an example of propositions.

\subsection{Several equations}\index{Propositions!Several Equations}

\begin{proposition}[Proposition name]
It has the properties:
\begin{align}
& \big| ||\mathbf{x}|| - ||\mathbf{y}|| \big|\leq || \mathbf{x}- \mathbf{y}||\\
&  ||\sum_{i=1}^n\mathbf{x}_i||\leq \sum_{i=1}^n||\mathbf{x}_i||\quad\text{where $n$ is a finite integer}
\end{align}
\end{proposition}

\subsection{Single Line}\index{Propositions!Single Line}

\begin{proposition} 
Let $f,g\in L^2(G)$; if $\forall \varphi\in\mathcal{D}(G)$, $(f,\varphi)_0=(g,\varphi)_0$ then $f = g$. 
\end{proposition}

%------------------------------------------------

\section{Examples}\index{Examples}

This is an example of examples.

\subsection{Equation and Text}\index{Examples!Equation and Text}

\begin{example}
Let $G=\{x\in\mathbb{R}^2:|x|<3\}$ and denoted by: $x^0=(1,1)$; consider the function:
\begin{equation}
f(x)=\left\{\begin{aligned} & \mathrm{e}^{|x|} & & \text{si $|x-x^0|\leq 1/2$}\\
& 0 & & \text{si $|x-x^0|> 1/2$}\end{aligned}\right.
\end{equation}
The function $f$ has bounded support, we can take $A=\{x\in\mathbb{R}^2:|x-x^0|\leq 1/2+\epsilon\}$ for all $\epsilon\in\intoo{0}{5/2-\sqrt{2}}$.
\end{example}

\subsection{Paragraph of Text}\index{Examples!Paragraph of Text}

\begin{example}[Example name]
\lipsum[2]
\end{example}

%------------------------------------------------

\section{Exercises}\index{Exercises}

This is an example of an exercise.

\begin{exercise}
This is a good place to ask a question to test learning progress or further cement ideas into students' minds.
\end{exercise}

%------------------------------------------------

\section{Problems}\index{Problems}

\begin{problem}
What is the average airspeed velocity of an unladen swallow?
\end{problem}

%------------------------------------------------

\section{Vocabulary}\index{Vocabulary}

Define a word to improve a students' vocabulary.

\begin{vocabulary}[Word]
Definition of word.
\end{vocabulary}

\section{Table}\index{Table}

\begin{table}[h]
\centering
\begin{tabular}{l l l}
\toprule
\textbf{Treatments} & \textbf{Response 1} & \textbf{Response 2}\\
\midrule
Treatment 1 & 0.0003262 & 0.562 \\
Treatment 2 & 0.0015681 & 0.910 \\
Treatment 3 & 0.0009271 & 0.296 \\
\bottomrule
\end{tabular}
\caption{Table caption}
\end{table}

%------------------------------------------------

\section{Figure}\index{Figure}

\begin{figure}[h]
\centering\includegraphics[scale=0.5]{placeholder}
\caption{Figure caption}
\end{figure}
\end{comment}


%----------------------------------------------------------------------------------------
%	Appendix
%----------------------------------------------------------------------------------------

\part{Appendix}


\chapter*{Bibliography}
\addcontentsline{toc}{chapter}{\textcolor{ocre}{Bibliography}}

%------------------------------------------------

\section*{Articles}
\addcontentsline{toc}{section}{Articles}
\printbibliography[heading=bibempty,type=article]

%------------------------------------------------

\section*{Books}
\addcontentsline{toc}{section}{Books}
\printbibliography[heading=bibempty,type=book]

%----------------------------------------------------------------------------------------
%	INDEX
%----------------------------------------------------------------------------------------

\cleardoublepage
\phantomsection
\setlength{\columnsep}{0.75cm}
\addcontentsline{toc}{chapter}{\textcolor{ocre}{Index}}
\printindex

%----------------------------------------------------------------------------------------

\end{document}
