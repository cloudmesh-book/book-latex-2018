\section{Plagiarism}

In academic life it is important to understand and avoid plagiarism.
Organizations and universities will have policies in place do address
plagiarism. An example is provided for Indiana University
\cite{www-iu-plagiarism}. We quote:

\begin{quotation}
``Honesty requires that any ideas or materials taken from
another source for either written or oral use must be fully
acknowledged. Offering the work of someone else as one’s own is
plagiarism. The language or ideas thus taken from another may range
from isolated formulas, sentences, or paragraphs to entire articles
copied from books, periodicals, speeches, or the writings of other
students. The offering of materials assembled or collected by others
in the form of projects or collections without acknowledgment also is
considered plagiarism. Any student who fails to give credit for ideas
or materials taken from another source is guilty of plagiarism. 

(Faculty Council, May 2, 1961; University Faculty Council, March 11,
1975; Board of Trustees, July 11, 1975)''
\end{quotation}

Faculty members at Universitys are also bound by policies that mandate
reporting. At Indiana University the following policy applies (for a
complete policy see the Web page):

\begin{quotation}
``Should
the faculty member detect signs of plagiarism or cheating, it is his
or her most serious obligation to investigate these thoroughly, to
take appropriate action with respect to the grades of students, and
{\em in any event} to report the matter to the Dean for Student Services [or
equivalent administrator]. The necessity to report every case of
cheating, whether or not further action is desirable, arises
particularly because of the possibility that this is not the student’s
first offense, or that other offenses may follow it. Equity also
demands that a uniform reporting practice be enforced; otherwise, some
students will be penalized while others guilty of the same actions
will go free.

(Faculty Council, May 2, 1961)''
\end{quotation}

Naturally if a student has any questions about understanding
plagiarism the University can provide assistance. If a student is in
doubt and asks for help this is not considered at that time
plagiarism. As you can see from the previous policies, the faculty do
not have any choice but reporting it to the university administration.
Thus you must not hold them personally responsible as this is part of
the tasks they are required to do if they like it or not. Instead, it
is {\bf the responsibility of the authors of the
  document} to assure no plagiarism occurs. If you are a student of
a class that writes a paper or project report this naturally also all
applies to you. In addition, if you work in a team you need to assure
the entire team addresses plagiarism appropriately.
