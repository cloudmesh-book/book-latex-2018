
\chapter{ADVANCE REST}

\section{ADVANCED STUDY ON REST}\label{s:rest-advanced}

In the previous section we discussed the basic concepts about RESTful
web services. In this section, technical aspect on REST terminlogies
and JSON data structure will be discussed.

%\video{REST}{36:02}{REST}{https://youtu.be/xjFuA6q5N_U}

\subsection{HATEOAS}

HATEOAS stands for Hypermedia as the Engine of Application State and
this is enabled by defaule configuration. There are specific technical
terms associated with this configuration. We will take a look at each
technical term briefly. HATEOS explains how REST API endpoints are
defined and it priovides a clear description on how the API can be
consumed.

\begin{description} 

\item [\_links] The relation of current resource being accessed to the
  rest of the resources. It is like if we have a set of links to the
  set of objects or service endpoints that we are referring in the
  RESTful webservice. Here an endpoint refers to a service call which
  is responsible for executing one of the CRUD operations on a
  particular object or set of objects. More on the links, the links
  object contains the list of servable API endpoints or list of
  services.  When we are calling a GET request or any other request,
  we can use these service endpoints to execute different queries
  based on the user purpose.  For instance, a service call can be used
  to insert data or retrieve data from a remote database using a REST
  API call. About databases we'll discuss in detail in another
  chapter.

\item [title] The title in the rest end point is the name or topic
  that we are trying to address. It describes the nature of the object
  by a single word. For instance student, bank-statement, salary,etc
  can be a title.

\item [parent] The term parent referes to the very initial link or an
  API endpoint in a particular RESTful web service. Generally this is
  denoted with the primary address like http://example.com/api/v1/.

\item [href] The term href refers to the url segment that we use to
  access the a particular REST API endpoint. For instance
  ``student?page=1'' will return the first page of student list by
  retrieving a particular number of items from a remote database or a
  remote data source. The full url will look like this,
  ``http://www.exampleapi.com/student?page=1''.

\end{description}  
\subsection{Rendering Data in UI}

In this section, we are talking about the way of using HATEOAS
practically in user interface development in web applications.

\begin{lstlisting}
  <resource>
  <link rel='child' href='student' title='student'/>
  </resource>
\end{lstlisting}

This is the practical representation of the HATEOAS in real world
applications.
