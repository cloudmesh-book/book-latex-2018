\section{Chameleon OpenStack}
\label{chameleon-openstack}
\index{Chameleon!Openstack}

OpenStack on Chameleon provides plenty of compute resources to
provision Linux virtual machines with various server types thus we can
deploy tools and software needed for the class and projects. This
hands-on tutorial is to walk through a basic step of getting access to
OpenStack Chameleon cloud under the class allocation and to provide
introduction to command line tools which are flexible and powerful
once it is mastered. Web interface, OpenStack Horizon is offered but
the class does not recommend to use it intensively because it has
limited to utilize cloud computing resources and it is difficult to
resolve issues when their are technical problems. Command line tools
generate full debugging messages if necessary.


\subsection{Outages}

Any computer system maay undergo maintenance. Before filing tickets
with Chameleon cloud, make sure that the cloud is operationa. Outagaes
are postesed at 

\url{https://www.chameleoncloud.org/user/outages/}

To be notified by mail, you can subscribe to them at 

\url{https://www.chameleoncloud.org/user/profile/subscriptions/}

\subsection{Account Creation}

The fist step to get access Chameleon cloud is to create a user
account if you do not already have one. You can skip to the next
section if you have a chameleon cloud account.

The register web page is available at:

\url{https://www.chameleoncloud.org/user/register/}

For more details, please als consult the chameleon chapter in the handbook.

\subsection{Join a Project}

The active project is required to use allocated compute resources to
the project.  Each class has a particular project number to use. Ask
details on Piazza, or instructor. Otherwise use the following
information to join a project:

\begin{lstlisting}
Spring 2018: CH-819337
\end{lstlisting}

The instructor (PI) will authorize you to join a project. Once you
have approved, you can start to launch a new VM instance on Chameleon
OpenStack.

\subsection{Environment Variables}

To simplify the documentation it is beneficial to set two shell
variables. The first one is CCPROJECT, that specifies the project
number.
The second one is a prefix that you will use for VMS and keys as we ar
eusing a shared project. This way we can see which VMS and which keys
have been uploaded and keep the names of them unique.

\begin{lstlisting}
export CCPROJECT=CH-819337
export CCPREFIX=albert-111
\end{lstlisting}



\subsection{OpenStack RC File}

We will use Nova command line tools for Chameleon OpenStack and to
authorize our account on the command line tools, we need to obtain the
openstack RC file from the openstack web interface. Login to the
following page with your chameleon account:


\url{https://openstack.tacc.chameleoncloud.org/dashboard}

Confirm your project number and find \textbf{Access \& Security} on
the left menu.  The Access \& Security page has tabs and choose
\textbf{API Access} to download credentials on a local machine. Click
\textbf{Download OpenStack RC File} to download
\textit{CH-\$PROJECTID-openrc.sh} file on your machine.

\TODO{TA: REPLACE IMAGE WITH LSTISTING}

  \begin{figure}[!htbp]
    \includegraphics[width=8cm,height=8cm]{section/cloud/chameleon/images/openstack-chameleon-openrc.png}
    \centering
  \end{figure}

This file contains REST API URLs that we use for Chameleon
OpenStack.  Everytime you use nova command line tools, the file
should be loaded on your terminal.






\begin{lstlisting}
mkdir ~/cloudmesh/chameleon
mv CH-$CCPROJECT-openrc.sh ~/cloudmesh/chameleon
source ~/cloudmesh/chameleon/CH-$CCPROJECT-openrc.sh
\end{lstlisting}

Once you \textit{source} the file, you can use nova command line tools
without sourcing it again.  The environment variables are enabled
while your terminal is alive.

\subsection{PIP Nova installation}

OpenStack provides \textit{nova} command line tools written by Python
which we will use in this tutorial. Run the following command to
install nova CLI on your machine.

\begin{lstlisting}
pip install python-openstackclient
\end{lstlisting}

Try nova command to see if it installed successfully:

\begin{lstlisting}
nova image-list
\end{lstlisting}

\begin{lstlisting}
\$ nova image-list
+--------------------------------------+------------------+--------+---------+
| ID                                   | Name             | Status | Server  |
+--------------------------------------+------------------+--------+---------+
| be46bd5a-c4a5-4495-ad30-356186d8ff04 | CC-C7-autologin  | ACTIVE |         |
| 1fe5138b-300b-4b30-8d22-e728abbd7773 | CC-CentOS7       | ACTIVE |         |
...
\end{lstlisting}

The sample output messages look like above, if your installation is correctly
done.

\subsection{KeyPair Registration}

Once you have completed Nova CLI installation, SSH keypair registration is
required unless you already have one.

\begin{lstlisting}
nova keypair-add --pub-key $HOME/ssh/id_rsa.pub $CCPREFIX-key
\end{lstlisting}
%$ 

Once you register your key, you can confirm the registration by:

\begin{lstlisting}
nova keypair-list
\end{lstlisting}

The sample output looks like:

\begin{lstlisting}
+-------------+-------------------------------------------------+
| Name        | Fingerprint                                     |
+-------------+-------------------------------------------------+
| $CCPREFIX-key | cf:04:06:aa:8b:76:af:77:aa:0a:b5:87:ff:0f:ba:97 |
+-------------+-------------------------------------------------+
\end{lstlisting}

where 111 is the number from your hid.  Naturally you will need an ssh
key.  If you do not have an existing SSH keypair, you can create one,
but you also must go back to the keypair registration step that we
described earlier.

\begin{lstlisting}
ssh-keygen -t rsa -C $CCPREFIX-key
\end{lstlisting}


\subsection{Start a new VM instance}

To start new instances you can use the \textit{nova boot} command. It
will start a VM instance. You can use some parameters to specify which
base image and a server size we will use with a name. We use
\textit{CC-Ubuntu16.04} base image in this tutorial which is an
official Ubuntu 16.04 image provided by Chameleon project.

\begin{lstlisting}
nova boot --image CC-Ubuntu16.04 --key-name $CCPREFIX-key --flavor m1.small $CCPREFIX-01
\end{lstlisting}

where the 01 indicates the instance number. Note that we will be
terminating and deleting any VM in our project that does not follow
this naming convention.

\subsection{Floating IP Address}

If your new VM instance is up and running, it needs an external ip
address which is also called floating IP address. A floating IP allows
you to get access to this VM from the internet. Note that chameleon
has a limited number of floating IP addresses and it is best to return
them if not in use. If chameleon runs out of floating IP adresses,
please submit a ticket to chameleon.
However in amny cases the VM may only need a an internal IP address as a
default. IN case you need to access others, you could even tunnel all
connections through a single flaoting IP. naturally this would limit
data transfers in and out of chameleon, but is a recommended way to
deal with limited flaoting IPs.

Let us showcase how to associate a floating IP address and access it
via SSH.

\begin{tiny}
\begin{lstlisting}
 nova floating-ip-create ext-net
 +--------------------------------------+----------------+-----------+----------+---------+
 | Id                                   | IP             | Server Id | Fixed IP | Pool    |
 +--------------------------------------+----------------+-----------+----------+---------+
 | 13dc309e-9a82-45af-8a9a-7fbd74f4aec6 | 129.114.111.37 | -         | -        | ext-net |
 +--------------------------------------+----------------+-----------+----------+---------+
\end{lstlisting}
\end{tiny}

Now we have a IP address to assign to a VM instance. In this tutorial,
we will associate \textit{129.114.111.37} to our
albert-111-01 VM instance by:

\begin{lstlisting}
nova floating-ip-associate albert-111-01 129.114.111.37
\end{lstlisting}

Once you completed this step, you are now able to SSH into your VM
instance.  Confirm \textit{ACTIVE} state in your VM to get access.

\begin{tiny}
\begin{lstlisting}
| f19e0ba1-9b76-46f4-896f-378983ab2aa3 | albert-111-01 | ACTIVE  | - | Running  | $CCPROJECT-net=192.168.0.13, 129.114.111.37 |
\end{lstlisting}
\end{tiny}
%$

where 111 is the number from your hid and 01 is the instance number

\begin{lstlisting}
ssh cc@129.114.111.37
\end{lstlisting}

Note that \textit{cc} is login name your your VM if you start a VM
with the official Chameleon cloud image.

\subsection{Termination of VM Instance}

If you completed your work on your VM instance, you have to terminate
your VM and release a floating IP address associated with. For
example, we terminate our first instance and the IP address by:

\begin{lstlisting}
nova delete $CCPREFIX-01
nova floating-ip-delete 129.114.111.37
\end{lstlisting}


