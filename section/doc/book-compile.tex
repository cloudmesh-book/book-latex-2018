\chapter{Help}

\section{Compiling Cloudmesh Handbook}\label{s:help-compile-handbook}

Our handbook is conveniently designed so that new sections can be
added easily. This includes the support of draft chapters that use the
same style as the handbook.

To facilitate the draft documentation phase, we created a \LaTeX~ file
called single.tex that has also a target in the \verb|Makefile|.

We will walk you through the process of adding a new section to the handbook

Let us assume the section is placed in the directory

\begin{lstlisting}
  section/machine-learning/kmeans.tex 
\end{lstlisting}

First, we must add this folder path to the Makefile. To not effect the
existing makefile and to test it we copy the original makefile to a
convenient one that you can remember such as

\begin{lstlisting}
  cp Makefile Makefile.kmeans
\end{lstlisting}

Then open the new Makefile.kmeans and add the following target to the
makefile:

\begin{lstlisting}
albert: dest 
	latexmk -jobname=kmeans $(FLAGS) -pvc -view=pdf kmeans
\end{lstlisting}  

As the kmeans.tex file is in a new directory, we need to add the
directory name to the place where the LaTeX output files are placed.
Find in the dest target the alphabetical location where  this folder
belongs and add it.DO not remove the other folder names indicated by
...


\begin{lstlisting}
dest:
    ...
    mkdir -p dest/format
    mkdir -p dest/machine-learning
    mkdir -p dest/notebooks
    ...
\end{lstlisting}

Safe the Makefile.kmeans.
Next, copy the file called single.tex to kmeans.tex. 

\begin{lstlisting}
cp single.tex kmeans-main.tex
\end{lstlisting}

As we want to review the section we must however  add it to this LaTex
file with the line

\begin{lstlisting}
\include{section/machine-learning/kmeans}
\end{lstlisting}

You may eelect to outcomment other sections.
Now you can compile the file with 

\begin{lstlisting}
  make -f Makefile.kmeans kmeans-main
\end{lstlisting}

FROM HERE TO IMPROVE

The above command will make the file and it will load up a pdf.  In
that case if you still get errors, read the error line and install
xpdf reader by running the following command.

\begin{lstlisting}
 sudo apt-get install xpdf
\end{lstlisting}

It is very important to make sure all the packages are installed
beforehand jumping in to modifying or adding new content to the
handbook.

There is a way of adding the browser to view the pdf by configuring it
with the build. This will be discussed in a new chapter.

Finally if the expected output is displayed in the build pdf which is located in the dest folder, we can integrate the changes that we did for the albert.tex and Makefile.albert to the original Makefile and the single.tex file. 

