\FILENAME

\section{Basic Emacs}
\label{C:emacs}

One of the most useful short manuals for emacs is the following refrence
card. It takes some time to use this card efficiently, but the most
important commands are written on it. Generations of students have
litterally been just presented with this card and they learned emacs
from it.

\URL{https://www.gnu.org/software/emacs/refcards/pdf/refcard.pdf}


There is naturally also additional material available and a great
manual. You could also look at

\URL{https://www.gnu.org/software/emacs/tour/}


From the last page we have summarized the most useful and
\textbf{simple} features. And present them here. One of the hidden gems
of emacs is the ability to recreate replay able macros which we include
here also. You ought to try it and you will find that for data science
and the cleanup of data emacs (applied to smaller datasets) is a gem.

Notation

\begin{longtable}[]{@{}ll@{}}
\toprule
Key & Description\tabularnewline
\midrule
\endhead
C & Control\tabularnewline
M & Esc (meta character)\tabularnewline
\bottomrule
\end{longtable}

Here are some other ways on what to do if you have accidentally
pressed a wrong key:

\begin{itemize}
\item C-g If you pressed a prefix key (e.g. C-x) or you invoked
a command which is now prompting you for input (e.g. Find file:
\ldots{}), type C-g, repeatedly if necessary, to cancel. C-g also
cancels a long-running operation if it appears that Emacs has frozen.

\item C-/ If you executed a command and Emacs has modified your buffer, use C-/ to
undo that change. 
\end{itemize}

To save the current file say 

\begin{longtable}[]{ll}
\toprule
Key & Description\tabularnewline
\midrule
\endhead
C-x C-w & Write the buffer to file \tabularnewline
C-x C-s & Write the buffer to file and quit Emacs \tabularnewline
\bottomrule
\end{longtable}


Moving around in buffers can be done with cursor keys, or with the
following key combinations:

\begin{longtable}[]{ll}
\toprule
Key & Description\tabularnewline
\midrule
\endhead
C-f & Forward one character\tabularnewline
C-n & Next line\tabularnewline
C-b & Back one character\tabularnewline
C-p & Previous line\tabularnewline
\bottomrule
\end{longtable}

Here are some ways to move around in larger increments:

\begin{longtable}[]{ll}
\toprule
Key & Description\tabularnewline
\midrule
\endhead
C-a & Beginning of line\tabularnewline
M-f & Forward one word\tabularnewline
M-a & Previous sentence\tabularnewline
M-v & Previous screen\tabularnewline
M-\textless{} & Beginning of buffer\tabularnewline
C-e & End of line\tabularnewline
M-b & Back one word\tabularnewline
M-e & Next sentence\tabularnewline
C-v & Next screen\tabularnewline
M-\textgreater{} & End of buffer\tabularnewline
\bottomrule
\end{longtable}

You can jump directly to a particular line number in a buffer:

\begin{longtable}[]{ll}
\toprule
Key & Description\tabularnewline
\midrule
\endhead
M-g g & Jump to specified line\tabularnewline
\bottomrule
\end{longtable}

Searching is easy with the following commands

\begin{longtable}[]{ll}
\toprule
Key & Description\tabularnewline
\midrule
\endhead
C-s & Incremental search forward\tabularnewline
C-r & Incremental search backward\tabularnewline
\bottomrule
\end{longtable}

Replace

\begin{longtable}[]{ll}
\toprule
Key & Description\tabularnewline
\midrule
\endhead
M-\% & Query replace\tabularnewline
\bottomrule
\end{longtable}

Killing (``cutting'') text

\begin{longtable}[]{ll}
\toprule
Key & Description\tabularnewline
\midrule
\endhead
C-k & Kill line\tabularnewline
\bottomrule
\end{longtable}

Yanking

\begin{longtable}[]{ll}
\toprule
Key & Description\tabularnewline
\midrule
\endhead
C-y & Yanks last killed text\tabularnewline
\bottomrule
\end{longtable}

Macros

Keyboard Macros

Keyboard macros are a way to remember a fixed sequence of keys for later
repetition. They're handy for automating some boring editing tasks.

\begin{longtable}[]{ll}
\toprule
Key & Description\tabularnewline
\midrule
\endhead
M-x ( & Start recording macro\tabularnewline
M-x ) & Stop recording macro\tabularnewline
M-x e & Play back macro once\tabularnewline
M-5 C-x-e & Play back macro 5 times\tabularnewline
\bottomrule
\end{longtable}

Modes

``Every buffer has an associated major mode, which alters certain
behaviors, key bindings, and text display in that buffer. The idea is to
customize the appearance and features available based on the contents of
the buffer.'' modes are typically activated by ending such as .py,
.java, .rst, \ldots{}

\begin{longtable}[]{ll}
\toprule
Key & Description\tabularnewline
\midrule
\endhead
M-x python-mode & Mode for editing Python files\tabularnewline
M-x auto-fill-mode & Wraps your lines automatically when they get longer
than 70 characters.\tabularnewline
M-x flyspell-mode & Highlights misspelled words as you type.\tabularnewline
\bottomrule
\end{longtable}
