

\chapter{REST}

REST stands for {\bf RE}presentational {\bf S}tate {\bf
  T}ransfer. REST is an architecture style for designing networked
applications. It is based on stateless, client-server, cacheable
communications protocol. Although not based on http, in most cases,
the HTTP protocol is used. In contrast to what some others write or
say, REST is not a \emph{standard}. RESTful applications use HTTP
requests to (a) post data while creating and/or updating it, (b) read
data while making queries, and (c) delete data.

\url{https://www.ics.uci.edu/~fielding/pubs/dissertation/top.htm}

Hence REST uses HTTP for the four CRUD operations:

\begin{itemize}
\item  {\bf C}reate
\item  {\bf R}ead
\item  {\bf U}pdate
\item  {\bf D}elete
\end{itemize}

As part of the HTTP protocol we have methods such as GET, PUT, POST, and
DELETE. These methods can than be used to implement a REST service. As
REST introduces collections and items we need to implement the CRUD
functions for them. The semantics is explained in the Table
illustrating how to implement them with HTTP methods.

\begin{description}

\item[http://.../resources/] ~\\

    \begin{description} 
    \item[GET] List the URIs and perhaps other details of the
      collection’s members
    \item[PUT] Replace the entire collection with another collection.
    \item[POST] Create a new entry in the collection. The new entry’s
      URI is assigned automatically and is usually returned by the
      operation.
    \item[DELETE] Delete the entire collection.
\end{description} 

\item[http://.../resources/item17] ~\\

    \begin{description} 

    \item[GET] Retrieve a representation of the addressed member of
      the collection, expressed in an appropriate Internet media type.

    \item[PUST] Replace the addressed member of the collection, or if
      it does not exist, create it.

    \item[POST] Not generally used. Treat the addressed member as a
      collection in its own right and create a new entry within it
                               
\item[DELETE] Delete the addressed member of the collection. 
\end{description} 
\end{description}

Source:
\url{https://en.wikipedia.org/wiki/Representational_state_transfer}


Du to the structure that REST provides a number of toools have been
created that manage the creation of the specification for rest
services and their programming. We distinguisch several different
categories:

(1) REST programming language support: the tools and services are
targeting a particular programming language.


(3) REST dcumentation based tools that are primarily oriented towards
documenting REST specifications.

(2) REST design support: When abstracting the programming language out
dfine reusable specifications that can be used to to create clients
and servers for particular technology targets.

\section{Swagger}

Swagger \url{https://swagger.io/} is a tool for developing API
specifications based on the OpenAPI Specification (OAS). It allows not
only the specification, but the generation of code based on the
specification in a variety of languages.

Swagger itself has a number of tools which together 


\subsection{Tools}

The major Swagger tools of interest are:

\begin{description}

\item[Swagger Core] includes Java libraries for working with Swagger
 specifications \url{https://github.com/swagger-api/swagger-core}.

\item[Swagger Codegen] allows to generate code from the specifications
 to develop Client SDKs, servers, and documentatio. \url{https://github.com/swagger-api/swagger-codegen}

\item[Swagger UI] is an HTML5 based UI for exploring and interacting
 with the specified APIs \url{https://github.com/swagger-api/swagger-ui}

\item[Swagger Editor] is a Web-browser based editor for composing 
 specifications using YAML \url{https://github.com/swagger-api/swagger-editor}

\end{description}

The developped APIs can be hosted and further developed on an
online repository named SwaggerHub \url{https://app.swaggerhub.com/home}
The convenient online editor is available which also can be installed
locally on a variety of operating systems including OSX, Linux, and
Windows. 



\section{FlaskRESTful}

``Flask-RESTful integrates a framework for craeting 
REST APIs within Flask. The abstraction provided by it work
with your existing ORM/libraries. Thorugh its minimalistic approach It encourages best
practices with minimal setup. If you are familiar with Flask,
Flask-RESTful should be easy to pick up.''
\url{https://flask-restful.readthedocs.io/en/latest/}

\section{Django REST Framework}

\url{http://www.django-rest-framework.org/}

\section{Eve}

\url{http://python-eve.org/}

Eve makes the creation of a REST implemenation in python easy.  We
will provide you with an implementation example that showcases that we
can create REST services without writing a single line of code. The
code for this is located at \url{https://github.com/cloudmesh/rest}

This code will have a master branch but will also have a dev branch in
which we will add gradually more objects. Objects in the dev branch will
include:

\begin{itemize}
\tightlist
\item
 virtual directories
\item
 virtual clusters
\item
 job sequences
\item
 inventories
\end{itemize}

;You may want to check our active development work in the dev branch.
However for the purpose of this class the master branch will be
sufficient.

\subsubsection{Installation}\label{installation}

First we havt to install mongodb. The instalation will depend on your
operating system. For the use of the rest service it is not important to
integrate mongodb into the system upon reboot, which is focus of many
online documents. However, for us it is better if we can start and stop
the services explicitly for now.

On ubuntu, you need to do the following steps:

\begin{verbatim}
TO BE CONTRIBUTED BY THE STUDENTS OF THE CLASS as homework
\end{verbatim}

On windows 10, you need to do the following steps:

\begin{verbatim}
TO BE CONTRIBUTED BY THE STUDENTS OF THE CLASS as homework, if you
elect Windows 10. YOu could be using the online documentation
provided by starting it on Windows, or rinning it in a docker container.
\end{verbatim}

On OSX you can use homebrew and install it with:

\begin{verbatim}
brew update
brew install mongodb
\end{verbatim}

\begin{description}
\item[In future we may want to add ssl authentication in which case you
may]
need to install it as follows:
\end{description}

brew install mongodb --with-openssl

\subsubsection{Starting the service}\label{starting-the-service}

We have provided a convenient Makefile that currently only works for
OSX. It will be easy for you to adapt it to Linux. Certainly you can
look at the targes in the makefile and replicate them one by one.
Improtaht targest are deploy and test.

When using the makefile you can start the services with:

\begin{verbatim}
make deploy
\end{verbatim}

IT will start two terminals. IN one you will see the mongo service, in
the other you will see the eve service. The eve service will take a file
called sample.settings.py that is base on sample.json for the start of
the eve service. The mongo servide is configured in suc a wahy that it
only accepts incimming connections from the local host which will be
suffiicent fpr our case. The mongo data is written into the
\$USER/.cloudmesh directory, so make sure it exists.

To test the services you can say:

\begin{verbatim}
make test
\end{verbatim}

YOu will se a number of json text been written to the screen.

\subsection{Creating your own objects}\label{creating-your-own-objects}

The example demonstrated how easy it is to create a mongodb and an eve
rest service. Now lets use this example to creat your own. FOr this we
have modified a tool called evegenie to install it onto your system.

The original documentation for evegenie is located at:

\begin{itemize}
\tightlist
\item
 \url{http://evegenie.readthedocs.io/en/latest/}
\end{itemize}

However, we have improved evegenie while providing a commandline tool
based on it. The improved code is located at:

\begin{itemize}
\tightlist
\item
 \url{https://github.com/cloudmesh/evegenie}
\end{itemize}

You clone it and install on your system as follows:

\begin{verbatim}
cd ~/github
git clone https://github.com/cloudmesh/evegenie
cd evegenie
python setup.py install
pip install .
\end{verbatim}

This shoudl install in your system evegenie. YOu can verify this by
typing:

\begin{verbatim}
which evegenie
\end{verbatim}

If you see the path evegenie is installed. With evegenie installed its
usaage is simple:

\begin{verbatim}
$ evegenie

Usage:
  evegenie --help
  evegenie FILENAME
\end{verbatim}

It takes a json file as input and writes out a settings file for the use
in eve. Lets assume the file is called sample.json, than the settings
file will be called sample.settings.py. Having the evegenie programm
will allow us to generate the settings files easily. You can include
them into your project and leverage the Makefile targets to start the
services in your project. In case you generate new objects, make sure
you rerun evegenie, kill all previous windows in whcih you run eve and
mongo and restart. In case of changes to objects that you have designed
and run previously, you need to also delete the mongod database.

\subsection{Towards cmd5 extensions to manage eve and
mongo}\label{towards-cmd5-extensions-to-manage-eve-and-mongo}

Naturally it is of advantage to have in cms administration commands to
manage mongo and eve from cmd instead of targets in the Makefile. Hence,
we \textbf{propose} that the class develops such an extension. We will
create in the repository the extension called admin and hobe that
students through collaborative work and pull requests complete such an
admin command.

The proposed command is located at:

\begin{itemize}
\tightlist
\item
 \url{https://github.com/cloudmesh/rest/blob/master/cloudmesh/ext/command/admin.py}
\end{itemize}

It will be up to the class to implement such a command. Please
coordinate with each other.

The implementation based on what we provided in the Make file seems
straight forward. A great extensinion is to load the objects definitions
or eve e.g. settings.py not from the class, but forma place in
.cloudmesh. I propose to place the file at:

\begin{verbatim}
.cloudmesh/db/settings.py
\end{verbatim}

the location of this file is used whne the Service class is initialized
with None. Prior to starting the service the file needs to be copied
there. This could be achived with a set commad.


\subsection{Responses}

\URL{https://dzone.com/refcardz/rest-foundations-restful}

\begin{verbatim}
Code	Description

200	OK. The request has successfully executed. Response depends upon the verb invoked.
201	Created. The request has successfully executed and a new resource has been created in the process. The response body is either empty or contains a representation containing URIs for the resource created. The Location header in the response should point to the URI as well.
202	Accepted. The request was valid and has been accepted but has not yet been processed. The response should include a URI to poll for status updates on the request. This allows asynchronous REST requests
204	No Content. The request was successfully processed but the server did not have any response. The client should not update its display.
Table 1 - Successful Client Requests


Redirected Client Requests

CODE	DESCRIPTION
301	Moved Permanently. The requested resource is no longer located at the specified URL. The new Location should be returned in the response header. Only GET or HEAD requests should redirect to the new location. The client should update its bookmark if possible.
302	Found. The requested resource has temporarily been found somewhere else. The temporary Location should be returned in the response header. Only GET or HEAD requests should redirect to the new location. The client need not update its bookmark as the resource may return to this URL.
303	See Other. This response code has been reinterpreted by the W3C Technical Architecture Group (TAG) as a way of responding to a valid request for a non-network addressable resource. This is an important concept in the Semantic Web when we give URIs to people, concepts, organizations, etc. There is a distinction between resources that can be found on the Web and those that cannot. Clients can tell this difference if they get a 303 instead of 200. The redirected location will be reflected in the Location header of the response. This header will contain a reference to a document about the resource or perhaps some metadata about it.




Invalid Client Requests

Code	Description
405	Method Not Allowed.
406	Not Acceptable.
410	Gone.
411	Length Required.
412	Precondition Failed.
413	Entity Too Large.
414	URI Too Long.
415	Unsupported Media Type.
417	Expectation Failed.


Code	Description
500	Internal Server Error.
501	Not Implemented.
503	Service Unavailable.

\end{verbatim}