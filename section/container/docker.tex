\chapter{Run Docker Locally on your Machine}\label{S:docker-local}

\FILENAME

\section{Installing Docker Community Edition}\label{installing-docker-community-edition}

To install docker on your computer, please visit the page:

\URL{https://www.docker.com/community-edition}{Docker Community Edition}

Here you will find a variety of packages, one of which will hopefully
suitable for your OS. The supported operating systems currently include:

\begin{itemize}
\item  OSX, Windows, Centos, Debian, Fedora, Ubuntu, AWS, Azure
\end{itemize}

Please chose the one most suitable for you.

\subsection{Instalation for OSX}\label{instalation-for-osx}

The docker community edition for OSX can be found at the following link

\HREF{https://store.docker.com/editions/community/docker-ce-desktop-mac?tab=description}{Information for OSX}

We recommend that at this time you get the version {\em Docker CE for MAC (stable)}

\URL{https://download.docker.com/mac/stable/Docker.dmg}

CLicking on the link will download a dmg file to your machine, that
you than will need to install by double clicking and allowing access
to the dmg file. Upon instalation a \texttt{whale} in the top status
bar shows that Docker is running, and you can acess it via a terminal.

\begin{figure}[htb]
\centering
\includegraphics[width=0.5\textwidth]{whale-in-menu-bar.png}
\caption{Docker integrated in the menu bar on OSX}
\end{figure}

\section{Testing if the install works}\label{testing-if-the-install-works}

To test if it works execute the following commands in a terminal:

\begin{verbatim}
docker version
\end{verbatim}

You should see an output similar to

\begin{verbatim}
docker version

Client:
  Version:      17.03.1-ce
  API version:  1.27
  Go version:   go1.7.5
  Git commit:   c6d412e
  Built:        Tue Mar 28 00:40:02 2017
  OS/Arch:      darwin/amd64

Server:
  Version:      17.03.1-ce
  API version:  1.27 (minimum version 1.12)
  Go version:   go1.7.5
  Git commit:   c6d412e
  Built:        Fri Mar 24 00:00:50 2017
  OS/Arch:      linux/amd64
  Experimental: true
\end{verbatim}

To see if you can run a container use

\begin{verbatim}
docker run hello-world
\end{verbatim}

Once executed you should see an outout similar to

\begin{verbatim}
Unable to find image 'hello-world:latest' locally
latest: Pulling from library/hello-world
78445dd45222: Pull complete 
Digest: sha256:c5515758d4c5e1e838e9cd307f6c6a .....
Status: Downloaded newer image for hello-world:latest

Hello from Docker!
This message shows that your installation appears to 
be working correctly.

To generate this message, Docker took the following steps:
1. The Docker client contacted the Docker daemon.
2. The Docker daemon pulled the "hello-world" image 
   from the Docker Hub.
3. The Docker daemon created a new container from that 
   image which runs the executable that produces the 
   output you are currently reading.
4. The Docker daemon streamed that output to the Docker 
   client, which sent it to your terminal.

To try something more ambitious, you can run an Ubuntu container 
with:

$ docker run -it ubuntu bash

Share images, automate workflows, and more with a free Docker ID:
https://cloud.docker.com/

For more examples and ideas, visit:
https://docs.docker.com/engine/userguide/
\end{verbatim}
