\MDNAME\
%%%%%%%%%%%%%%%%%%%%%%%%%%%%%%%%%%%%%%%%%%%%%%%%%%%%%%%%%%%%%%%%%%%%%%%%%%%%%%%
% DO NOT MODIFY THIS FILE
%%%%%%%%%%%%%%%%%%%%%%%%%%%%%%%%%%%%%%%%%%%%%%%%%%%%%%%%%%%%%%%%%%%%%%%%%%%%%%%

\section{Using Kubernetes on FutureSystems}

\label{s:kubernetes-fs}

\index{Kubernetes!Futuresystems} \index{Futuresystems!Kubernetes}

This section introduces you on how to use the Kubernetes cluster on
FutureSystems. Currently we have deployed kubernetes on our cluster
called \emph{echo}.

\subsection{Getting Access}

You will need an account on FutureSystems and upload the ssh key to the
FutureSystems portal from the computer from which you want to login to
echo. To verify, if you have access try to see if you can log into
india.futuresystems.org. You need to be a member of a valid
FutureSystems project.

If you have verified that you haveaccess to the india, you can now try
to login to the kubernetes cluster head node with the same username and
key:

At this time we ask you to use the following IP address to communicate
with echo via its head node:

\begin{lstlisting}
149.165.150.85
\end{lstlisting}

To login to echo use the command

\begin{lstlisting}
ssh FS_USERNAME@149.165.150.85
\end{lstlisting}

where FS\_USERUSER is the username you have on futureSystems.

\textbf{NOTE: If you have access to india but not the kubernetes
software, your project may not have been authorized to access the
kubernetes cluster. Send a ticket to FutureSystems ticket system to
request this.}

Once you are logged in to the kubernetes cluster head node, try to run:

\begin{lstlisting}
kubectl get pods
\end{lstlisting}

This will let you know if you have access to kubernetes and verifies if
the kubectl command works for you. Naturally it will also list the pods.

\subsection{Example Use}

The following command runs an image called nginx with two replicas:

\begin{lstlisting}
kubectl run nginx --replicas=2 --image=nginx --port=80
\end{lstlisting}

As a result of this one deployment was created, and two PODs are created
and started. To see the deployment, please use the command

\begin{lstlisting}
kubectl get deployment
\end{lstlisting}

This will result in the following output

\begin{lstlisting}
NAME      DESIRED   CURRENT   UP-TO-DATE   AVAILABLE   AGE
nginx     2         2         2            2           7m
\end{lstlisting}

To see the pods please use the command

\begin{lstlisting}
kubectl get pods
\end{lstlisting}

This will result in the following output

\begin{lstlisting}
NAME                   READY STATUS  RESTARTS AGE
nginx-7587c6fdb6-4jnh6 1/1   Running 0        7m
nginx-7587c6fdb6-pxpsz 1/1   Running 0        7m
\end{lstlisting}

If we want to see more detailed information we cn use the command

\begin{lstlisting}
kubectl get pods -o wide
\end{lstlisting}

\begin{lstlisting}
NAME                   READY STATUS  RESTARTS AGE IP        NODE
nginx-75...-4jnh6 1/1   Running 0        8m  192.168.56.2   e003
nginx-75...-pxpsz 1/1   Running 0        8m  192.168.255.66 e005
\end{lstlisting}

Please note the IP address field. Now if we try to access the nginx
homepage with wget (or curl)

\begin{lstlisting}
wget 192.168.56.2
\end{lstlisting}

we see the following output:

\begin{lstlisting}
--2018-02-20 14:05:59--  http://192.168.56.2/
Connecting to 192.168.56.2:80... connected.
HTTP request sent, awaiting response... 200 OK
Length: 612 [text/html]
Saving to: 'index.html'

index.html    100%[=========>]     612  --.-KB/s    in 0s

2018-02-20 14:05:59 (38.9 MB/s) - 'index.html' saved [612/612]
\end{lstlisting}

It verifies that the specified image was running, and it is accessible
from within the cluster.

Next we need to start thinking about how we access this web server from
outside the cluster. We can explicitly exposing the service with the
following command

\begin{lstlisting}
kubectl expose deployment nginx --type=NodePort --name=999-nginx-ext
\end{lstlisting}

We will see the response

\begin{lstlisting}
service "nginx-external" exposed
\end{lstlisting}

To find the exposed ip adresses, we simply issue the command

\begin{lstlisting}
kubectl get svc
\end{lstlisting}

We se somthing like this

\begin{lstlisting}
NAME          TYPE      CLUSTER-IP    EXTERN PORT(S)      AGE
                                      AL-IP
kubernetes    ClusterIP 10.96.0.1     <none> 443/TCP      8h
999-nginx-ext NodePort  10.96.183.189 <none> 80:30275/TCP 6s
\end{lstlisting}

please note that we have given a unique name.

\begin{IU}

You could use your username or if you use one of our classes your
hid. The number part will typically be suficient.  For class users
that do not use the hid in the name we will terminate all instances
without notification. In addition, we like you explicitly to add
"-ext" to every container that is exposed to the internet. Naturally
we want you to shut down such services if they are not in use. Failure
to do so may result in termination of the service without notice, and
in the worst case revocation of your priveledges to use *echo*.

\end{IU}

In our example you will find the port on which our service is exposed
and remaped to. We find the port \textbf{30275} in the value
\textbf{80:30275/TCP} in the ports column for the running container.

Now if we visit this URL, which is the public IP of the head node
followed by the exposed port number

\begin{lstlisting}
http://149.165.150.85:30275
\end{lstlisting}

you should see the `Welcome to nginx' page.

\subsection{Exercises}

\begin{exercise}
Explore more complex service examples.
\end{exercise}

\begin{exercise}
Explore constructing a complex web app with multiple services.
\end{exercise}

\begin{exercise}
Define a deployment with a yaml file declaratively.
\end{exercise}

