% chktex-file 1
% chktex-file 29

\FILENAME

\section{Introduction to Kubernetes}
\index{Kubernetes}

\subsection{Topics Covered and Learning
Outcome}\label{topics-covered-and-learning-outcome-1}

\begin{itemize}

\item
  What is Kubernetes?
\item
  What are containers?
\item
  Cluster components in Kubernetes
\item
  Basic Units in Kubernetes
\item
  Run an example with Minikube
\item
  Interactive online tutorial
\item
  Have a solid understanding of Containers and Kubernetes
\item
  Understand the CLuster components of Kubernetes
\item
  Understand the terminology of Kubernetes
\item
  Gain practical experience with kubernetes
\item
  With minikube
\item
  With an interactive online tutorial
\end{itemize}

\subsection{What is Kubernetes?}\label{what-is-kubernetes}

Kubernetes is an open-source platform designed to automate deploying,
scaling, and operating application containers.
\url{https://kubernetes.io/docs/concepts/overview/what-is-kubernetes/}

With Kubernetes, you can:

\begin{itemize}

\item
  Deploy your applications quickly and predictably.
\item
  Scale your applications on the fly.
\item
  Roll out new features seamlessly.
\item
  Limit hardware usage to required resources only.
\item
  Run applications in public and private clouds.
\end{itemize}

Kubernetes is

\begin{itemize}

\item
  Portable: public, private, hybrid, multi-cloud
\item
  Extensible: modular, pluggable, hookable, composable
\item
  Self-healing: auto-placement, auto-restart, auto-replication,
  auto-scaling
\end{itemize}

\subsection{What are containers?}\label{what-are-containers}

\begin{figure}[htbp]
\centering
\includegraphics[width=0.8\textwidth]{why_containers.png}
\caption{Kubernetes Containers}
\end{figure}

Image source:
\url{https://d33wubrfki0l68.cloudfront.net/e7b766e0175f30ae37f7e0e349b87cfe2034a1ae/3e391/images/docs/why_containers.svg}

\subsection{Basic Units}\label{basic-units}

\subsubsection{Pods}\label{pods}

A pod (as in a pod of whales or pea pod) is a group of one or more
containers (such as Docker containers), with shared storage/network, and
a specification for how to run the containers. A pod's contents are
always co-located and co-scheduled, and run in a shared context. A pod
models an application-specific \textit{logical host}. It contains one or
more application containers which are relatively tightly coupled. In
a pre-container world, they would have executed on the same physical or
virtual machine.

\subsubsection{Services}\label{services}

Service is an abstraction which defines a logical set of Pods and a
policy by which to access them. Sometimes they are called a
micro-service. The set of Pods targeted by a Service is (usually)
determined by a Label Selector.

\subsubsection{Deployments}\label{deployments}

A Deployment controller provides declarative updates for Pods and
ReplicaSets. You describe a desired state in a Deployment object, and
the Deployment controller changes the actual state to the desired state
at a controlled rate. You can define Deployments to create new
ReplicaSets, or to remove existing Deployments and adopt all their
resources with new Deployments.

\subsection{Minikube}
\index{Kubernetes!Minikube}

\begin{enumerate}
\item
  minikube installation
\item
  minikube hello-minikube
\end{enumerate}

\subsubsection{\texorpdfstring{Install
\texttt{minikube}}{Install minikube}}\label{install-minikube}

\paragraph{OSX}\label{osx}

\begin{verbatim}
curl -Lo minikube https://storage.googleapis.com/minikube/releases/v0.25.0/minikube-darwin-amd64 && chmod +x minikube && sudo mv minikube /usr/local/bin/
\end{verbatim}

\paragraph{Windows 10}\label{windows-10}

We assume that you have installed Oracle VirtualBox in your machine
which must be a version 5.x.x.

Initially, we need to download two executables.

\href{http://storage.googleapis.com/kubernetes-release/release/v1.4.0/bin/windows/amd64/kubectl.exe}{Download
Kubectl}

\href{https://storage.googleapis.com/minikube/releases/v0.25.0/minikube-windows-amd64.exe}{Download
Minikube}

After downloading these two executables place them in the cloudmesh
directory we earlier created. Rename the \verb|minikube-windows-amd64.exe|
to \verb|minikube.exe|. Makesure minikube.exe and kubectl.exe lie in the
same directory.

\paragraph{Linux}\label{linux}

\begin{verbatim}
curl -Lo minikube https://storage.googleapis.com/minikube/releases/v0.25.0/minikube-linux-amd64 && chmod +x minikube && sudo mv minikube /usr/local/bin/
\end{verbatim}

Installing KVM2 is important for Ubuntu distributions

\begin{verbatim}
$ sudo apt install libvirt-bin qemu-kvm
$ sudo usermod -a -G libvirtd $(whoami)
$ newgrp libvirtd
\end{verbatim}

We are going to run minikube using KVM2 libraries instead of virtualbox
libraries for windows installation.

Then install the drivers for KVM2,

\begin{verbatim}
$ curl -LO https://storage.googleapis.com/minikube/releases/latest/docker-machine-driver-kvm2 && chmod +x docker-machine-driver-kvm2 && sudo mv docker-machine-driver-kvm2 /usr/bin/
\end{verbatim}

\subsubsection{Start a cluster using
Minikube}\label{start-a-cluster-using-minikube}

\paragraph{OSX Minikube Start}\label{osx-minikube-start}

\begin{verbatim}
$ minikube start
\end{verbatim}

\paragraph{Ubuntu Minikube Start}\label{ubuntu-minikube-start}

\begin{verbatim}
minikube start --vm-driver=kvm2
\end{verbatim}

\paragraph{Windows 10 Minikube Start}\label{windows-10-minikube-start}

In this case you must run Windows PowerShell as administrator. For
this search for the application in search and right click and click Run
as administrator. If you are an administrator it will run automatically
but if you are not please make sure you provide the admin login
information in the pop up.

\begin{verbatim}
$ cd  C:\Users\<username>\Documents\cloudmesh
$ .\minikube.exe start --vm-driver="virtualbox"
\end{verbatim}

\subsubsection{Create a deployment}\label{create-a-deployment}

\begin{verbatim}
$ kubectl run hello-minikube --image=k8s.gcr.io/echoserver:1.4 --port=8080
\end{verbatim}

\subsubsection{Expose the service}\label{expose-the-service}

\begin{verbatim}
$ kubectl expose deployment hello-minikube --type=NodePort
\end{verbatim}

\subsubsection{Check running status}\label{check-running-status}

This step is to make sure you have a pod up and running.

\begin{verbatim}
$ kubectl get pod
\end{verbatim}

\subsubsection{Call service api}\label{call-service-api}

\begin{verbatim}
$ curl $(minikube service hello-minikube --url)
\end{verbatim}

\subsubsection{Take a look from
Dashboard}\label{take-a-look-from-dashboard}

\begin{verbatim}
$ minikube dashboard
\end{verbatim}

If you want to get an interactive dashboard,

\begin{verbatim}
$ .\minikube.exe dashboard --url=true
http://192.168.99.101:30000
\end{verbatim}

Browse to http://192.168.99.101:30000 in your web browser and it will
provide a GUI dashboard regarding minikube.

\subsubsection{Delete the service and
deployment}\label{delete-the-service-and-deployment}

\begin{verbatim}
$ kubectl delete service hello-minikube
$ kubectl delete deployment hello-minikube
\end{verbatim}

\subsubsection{Stop the cluster}\label{stop-the-cluster}

For all platforms we can use the following command.

\begin{verbatim}
$ minikube stop
\end{verbatim}

\subsection{Interactive Tutorial
Online}\label{interactive-tutorial-online}

\begin{itemize}

\item
  Start cluster
  https://kubernetes.io/docs/tutorials/kubernetes-basics/cluster-interactive/
\item
  Deploy app
  https://kubernetes.io/docs/tutorials/kubernetes-basics/cluster-interactive
\item
  Explore
  https://kubernetes.io/docs/tutorials/kubernetes-basics/explore-intro/
\item
  Expose
  https://kubernetes.io/docs/tutorials/kubernetes-basics/expose-intro/
\item
  Scale
  https://kubernetes.io/docs/tutorials/kubernetes-basics/scale-intro/
\item
  Update
  https://kubernetes.io/docs/tutorials/kubernetes-basics/update-interactive/
\item
  MiniKube
  https://kubernetes.io/docs/tutorials/stateless-application/hello-minikube/
\end{itemize}
