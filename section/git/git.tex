\section{Git}\label{git}

This class uses open source technology and we like that you benefit from
material others in the class are developing or have developed. All
assignments are openly submitted to the class github for everyone to
see. As part of the goal of this class is to develop reusable
instructions, deployments, software, and examples. Such reuse is only
possible if the code is publicly available and others can benefit from
it. While using github.com we make sharing of information possible so
every one can benefit and achieve their best.

\subsection{Install}\label{install}

Information on how to install git can be found at

\URL{https://www.atlassian.com/git/tutorials/install-git}

\subsubsection{Config}\label{config}

Once you've got Git installed, several bits of configuration will
enhance your experience with the tool and better tune it to your
operating system. Let us tell you about settings for your username and
email address, line endings, and color, along with the settings'
associated configuration scopes.

\URL{https://www.youtube.com/watch?v=ZChtKFLiaNw}

It is important is that you always want to make sure that you want to
use the git config command to initialize git for the first time you use
it. This will ensure that you use on all resources the same Name and
e-mail so that git history and log will show consistently your checkins.
If you do not do this, your checkins in git do not show up in a
consistent fashion as a single user. This is done with the following
commands:

\begin{verbatim}
$ git config --global user.name "Albert Zweistein"
$ git config --global user.email albert.zweistein@gmail.com
\end{verbatim}

You can set also the editor with:

\begin{verbatim}
$ git config --global core.editor emacs
\end{verbatim}

You will also need to decide if you want to push branches individually
or all branches at the same time. It will be up to you to make what
whill work for you best. We found that the following seems to work best:

\begin{verbatim}
git config --global push.default matching
\end{verbatim}

More information about a first time setup is documented at:

\begin{verbatim}
* http://git-scm.com/book/en/Getting-Started-First-Time-Git-Setup
\end{verbatim}

To check your setup you can say:

\begin{verbatim}
$ git config --list
\end{verbatim}

In addition the tutorials from atlasian are a good source. However
remember that you may not use bitbucket as the repository, so ignore
those tutorials. We found the following useful

\begin{itemize}
\tightlist
\item
  What is git: \url{https://www.atlassian.com/git/tutorials/what-is-git}
\item
  Installing git:
  \url{https://www.atlassian.com/git/tutorials/install-git}
\item
  git config:
  \url{https://www.atlassian.com/git/tutorials/setting-up-a-repository\#git-config}
\item
  git clone:
  \url{https://www.atlassian.com/git/tutorials/setting-up-a-repository\#git-clone}
\item
  saving changes:
  \url{https://www.atlassian.com/git/tutorials/saving-changes}
\item
  collaborating with git:
  \url{https://www.atlassian.com/git/tutorials/syncing}
\end{itemize}

Please read the information on the screen when you set up

\subsection{Merge}\label{merge}

As we are allowing contribution by the community, they are best managed
through a merge with our upstream repository so you can update to the
newest status before you issue a pul request.

Make sure you have upstream repo defined:

\begin{verbatim}
$ git remote add upstream https://github.com/cloudmesh/classes
\end{verbatim}

Backup all your changed files - just in case you need them while merging
the changes back

Get latest from upstream:

\begin{verbatim}
$ git rebase upstream/master
\end{verbatim}

In this step, the conflicting file shows up (in my case it was
refs.bib):

\begin{verbatim}
$ git status
\end{verbatim}

should show the name of the conflicting file:

\begin{verbatim}
$ git diff <file name>
\end{verbatim}

should show the actual differences. May be in some cases, It is easy to
simply take latest version from upstream and reapply your changes.

So you can decide to checkout one version earlier of the specific file.
At this stage, the re-base should be complete. So, you need to commit
and push the changes to your fork:

\begin{verbatim}
$ git commit
$ git rebase origin/master
$ git push
\end{verbatim}

Then reapply your changes to refs.bib - simply use the backedup version
and use the editor to redo the changes.

At this stage, only refs.bib is changed:

\begin{verbatim}
$ git status
\end{verbatim}

should show the changes only in refs.bib. Commit this change using:

\begin{verbatim}
$ git commit -a -m "new:usr: <message>"
\end{verbatim}

And finally push the last commited change:

\begin{verbatim}
$ git push
\end{verbatim}

The changes in the file to resolve merge conflict automatically goes to
the original pull request and the pull request can be merged
automatically.

You still have to issue the pull request from the Github Web page so it
is registered with the upstream repository.
