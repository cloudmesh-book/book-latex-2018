

\chapter{Introduction}

\FILENAME\

Internet of Things is one of the driving forces in the modernisation of
todays world. It is based on connecting \emph{things} to the internet to
create a more aware world that can be interfaced with. This not only
includes us humans, but any \emph{thing} that can interact with other
things. It is clear that such a vision of interconnected devices will
result in billions of devices to communicate with each other. Some of
them may only communicate small number of items, while others will
communicate a large amount. Analysis of this data is dependent on the
capability of the \emph{thing}. If it is to small the analysis can be
conducted on a remote server or cloud while information to act are fead
back from the device. In other cases the device may be completely
autonomous and does not require any interaction. Yet in other cases the
collaborative information gathered from such devices is used to derive
decissions and actions.

Within this section we are trying to provide you with a small glimpse
into how IoT devices function and can be utilized on small projects.
Ideally if the class has all such a device we could even attempt to
build a cloud based service that collects and redistributes the data.

To keep things simple we are not providing a general introduction in
IoT. For that we offer other classes. However, we will introduce you to
two different devices. These are

\begin{itemize}

\item
  esp8266
\item
  Raspberry Pi
\end{itemize}

The reasons we chose them is that

\begin{enumerate}
\def\labelenumi{\arabic{enumi}.}

\item
  they are cheap,
\item
  we can program both in python allowing us to use a single programming
  langauage for all projects and assignments, and
\item
  they are sufficiently powerful and we can conduct real projects with
  them beyond toy projects.
\item
  the devices, especially the Raspberry PI can be used to also learn
  Linux in case you do not have access to a linux computer. Please note
  however the raspberry will have memory and space limitations that you
  need to deal with.
\end{enumerate}

Projects that you can do to test the devices are

esp8266 (easy-moderate, small memory):

\begin{enumerate}
\def\labelenumi{\arabic{enumi}.}

\item
  a LED blinker
\item
  a dendrite
\item
  a robot fish
\item
  a fish swarm
\item
  a robot swarm
\item
  an activity of your desire
\end{enumerate}

Raspberry Pi (easy-moderate, 32GB space limitation):

\begin{enumerate}
\def\labelenumi{\arabic{enumi}.}

\item
  a LED blinker
\item
  a robot car
\item
  a robot car with camera
\item
  a termerature service
\item
  a docker cluster
\end{enumerate}

Crazyflie 2.0 (difficult):

\begin{enumerate}
\def\labelenumi{\arabic{enumi}.}

\item
  programming a drone
\item
  programming a drone swarm
\end{enumerate}

Please note that for those at IU we do have a Lab in which you can use
some of the devices pointed out here. You can arrange for accessing the
infrastructure or you simply can buy it for yourself.

We have a hardware page that summarizes what you need. In case you want
to work on a swarm, we do have positioning sensors that simplify that
task.

In general we think that these platforms provide a wonderful
introduction into IoT and where it will move to. Such platforms were
just a decade ago not powerful enough or too expensive. However today
the provide a serious platform for developers. Sensors are available
easily as most Android comparible sensors can be used.

Before we jump right into programming the devices, we like to point out
that we dod not chose to use Arduinos, as their price advantage is no
longer valid. We also find that esp8266 and Raspberry can interface with
most sensors. Having the ability to easily use WiFi however is our
primary reason for using them. Furthermore being able to attach a camera
to the Raspberry is just superb. Image analysis will be one of the near
term future drivers for big data.
