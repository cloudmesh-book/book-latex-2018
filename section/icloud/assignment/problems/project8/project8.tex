\section*{Harp Mini\-Batch Kmeans}
\subsection*{Goal}
The goal for this project is to implement Harp\cite{Harp} Mini-batch Kmeans
from scratch. 

\subsection*{Deliverables}
Zip your source code and report as username\_mbkmeans.zip. Please submit this
file to the Canvas Assignments page.

\subsection*{Evaluation}

The point total for this project is 6, where the distribution is as follows:
\begin{itemize}
\item Completeness of your code (5 points)
\item In the report, describe your implementation and the output. (1 points)
\end{itemize}
 You can get up to 4 bonus points based on your  extra efforts.
\section*{Bonus credits}

Some options you may consider to get extra credits: 
\begin{itemize}
\item Perform experiments on various (small, medium, large, etc) datasets 
\item Test your algorithm on at least 2 nodes on FutureSystem.
\item Implement mini-batch kmeans using other tools/platforms
  (Spark\cite{Spark}, Flink\cite{Flink}, etc) and compare the performance
    between different tools/platforms.
\end{itemize}
You are encouraged to explore other options to get extra credits. Remember to
present all of your extra work in the report.
 
\subsection*{Dataset}
You can implement a script to  generate data randomly as your input datasets.
You are also free to use public datasets such as RCV1-v2\cite{RCV1-v2}.
  
\subsection*{Mini-batch Kmeans}
You can refer to the paper for sequential mini-batch
kmeans algorithm. You will need to design how to parallelize the algorithm so
that it can run with large scale datasets on distributed computing environment.

\begin{figure}[H]
\includegraphics[width=8cm]{section/icloud/assignment/problems/project8/mbkmeans}
\centering
\caption{Mini-batch Kmeans.}
\end{figure}  
