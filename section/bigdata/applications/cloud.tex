

\chapter{Cloud Computing (Outdated)}

\FILENAME

We describe the central role of Parallel computing in Clouds and Big
Data which is decomposed into lots of `'Little data'`running in
individual cores. Many examples are given and it is stressed that issues
in parallel computing are seen in day to day life for communication,
synchronization, load balancing and decomposition. Cyberinfrastructure
for e-moreorlessanything or moreorlessanything-Informatics and the
basics of cloud computing are introduced. This includes virtualization
and the important'`as a Service'' components and we go through several
different definitions of cloud computing.

Gartner's Technology Landscape includes hype cycle and priority matrix
and covers clouds and Big Data. Two simple examples of the value of
clouds for enterprise applications are given with a review of different
views as to nature of Cloud Computing. This IaaS (Infrastructure as a
Service) discussion is followed by PaaS and SaaS (Platform and Software
as a Service). Features in Grid and cloud computing and data are
treated. We summarize the 21 layers and almost 300 software packages in
the HPC-ABDS Software Stack explaining how they are used.

Cloud (Data Center) Architectures with physical setup, Green Computing
issues and software models are discussed followed by the Cloud Industry
stakeholders with a 2014 Gartner analysis of Cloud computing providers.
This is followed by applications on the cloud including data intensive
problems, comparison with high performance computing, science clouds and
the Internet of Things. Remarks on Security, Fault Tolerance and
Synchronicity issues in cloud follow. We describe the way users and data
interact with a cloud system. The Big Data Processing from an
application perspective with commercial examples including eBay
concludes section after a discussion of data system architectures.

\section{Parallel Computing (Outdated)}

We describe the central role of Parallel computing in Clouds and Big
Data which is decomposed into lots of `'Little data'' running in
individual cores. Many examples are given and it is stressed that issues
in parallel computing are seen in day to day life for communication,
synchronization, load balancing and decomposition.

\slides{Cloud}{33}{Parallel Computing}{https://drive.google.com/file/d/0B8936_ytjfjmZDRqMldzSVhVem8/view?usp=sharing}

\subsection{Decomposition}\label{decomposition}

We describe why parallel computing is essential with Big Data and
distinguishes parallelism over users to that over the data in problem.
The general ideas behind data decomposition are given followed by a few
often whimsical examples dreamed up 30 years ago in the early heady days
of parallel computing. These include scientific simulations, defense
outside missile attack and computer chess. The basic problem of parallel
computing -- efficient coordination of separate tasks processing
different data parts -- is described with MPI and MapReduce as two
approaches. The challenges of data decomposition in irregular problems
is noted.

\video{Cloud}{ 8:51}{Decomposition:}{https://drive.google.com/file/d/0B1Of61fJF7WsWXgtNndQN0Jydkk/view?usp=sharing}

\video{Cloud}{ 13:22}{Examples of Application:}{https://drive.google.com/file/d/0B1Of61fJF7WsQ1pMLWhXQV92OXM/view?usp=sharing}

\video{Cloud}{ 9:22}{Decomposition Strategies:}{https://drive.google.com/file/d/0B1Of61fJF7WsaVZLOEUzc0VuWjQ/view?usp=sharing}

\subsection{Parallel Computing in Society}

This lesson from the past notes that one can view society as an approach
to parallel linkage of people. The largest example given is that of the
construction of a long wall such as that (Hadrian's wall) between
England and Scotland. Different approaches to parallelism are given with
formulae for the speed up and efficiency. The concepts of grain size
(size of problem tackled by an individual processor) and coordination
overhead are exemplified. This example also illustrates Amdahl's law and
the relation between data and processor topology. The lesson concludes
with other examples from nature including collections of neurons (the
brain) and ants.

\video{cloud}{8:24}{Parallel Computing in Society I}{https://drive.google.com/file/d/0B1Of61fJF7WsY3hEeTJvTFYtN2s/view?usp=sharing}

\video{cloud}{8:01}{Parallel Computing in Society II}{https://drive.google.com/file/d/0B1Of61fJF7WsU1ROMmpNNTlUTUU/view?usp=sharing}

\subsection{Parallel Processing for Hadrian's Wall}

This lesson returns to Hadrian's wall and uses it to illustrate advanced
issues in parallel computing. First We describe the basic SPMD -- Single
Program Multiple Data -- model. Then irregular but homogeneous and
heterogeneous problems are discussed. Static and dynamic load balancing
is needed. Inner parallelism (as in vector instruction or the multiple
fingers of masons) and outer parallelism (typical data parallelism) are
demonstrated. Parallel I/O for Hadrian's wall is followed by a slide
summarizing this quaint comparison between Big data parallelism and the
construction of a large wall.

\video{cloud}{9:24}{Processing for Hadrian's Wall}{https://drive.google.com/file/d/0B1Of61fJF7WsNEtLOTNNN3dlNjQ/view?usp=sharing}

\subsection{Resources}

\begin{itemize}
\item
  Solving Problems in Concurrent Processors-Volume 1, with M. Johnson,
  G. Lyzenga, S. Otto, J. Salmon, D. Walker, Prentice Hall, March 1988.
\item
  Parallel Computing Works!, with P. Messina, R. Williams, Morgan
  Kaufman (1994). \url{http://www.netlib.org/utk/lsi/pcwLSI/text/}
\item
  The Sourcebook of Parallel Computing book edited by Jack Dongarra, Ian
  Foster, Geoffrey Fox, William Gropp, Ken Kennedy, Linda Torczon, and
  Andy White, Morgan Kaufmann, November 2002.
\item
  Geoffrey Fox Computational Sciences and Parallelism to appear in
  Encyclopedia on Parallel Computing edited by David Padua and published
  by Springer.
  \url{http://grids.ucs.indiana.edu/ptliupages/publications/SpringerEncyclopedia_Fox.pdf}
\end{itemize}

\section{Introduction}

We discuss Cyberinfrastructure for e-moreorlessanything or
moreorlessanything-Informatics and the basics of cloud computing. This
includes virtualization and the important `as a Service' components and
we go through several different definitions of cloud computing.Gartner's
Technology Landscape includes hype cycle and priority matrix and covers
clouds and Big Data. The unit concludes with two simple examples of the
value of clouds for enterprise applications. Gartner also has specific
predictions for cloud computing growth areas.

\slides{Cloud}{45}{Introduction}{https://drive.google.com/file/d/0B8936_ytjfjmdmF2Uy1vWS0xTFU/view?usp=sharing}

\subsection{Cyberinfrastructure for E-Applications}

This introduction describes Cyberinfrastructure or e-infrastructure and
its role in solving the electronic implementation of any problem where
e-moreorlessanything is another term for moreorlessanything-Informatics
and generalizes early discussion of e-Science and e-Business.

\video{cloud}{13:34}{Cloud Computing Introduction Part1}{https://drive.google.com/file/d/0B1Of61fJF7WsbXpEdF8zWFh4aXc/view?usp=sharing}

\subsection{What is Cloud Computing: Introduction}

Cloud Computing is introduced with an operational definition involving
virtualization and efficient large data centers that can rent computers
in an elastic fashion. The role of services is essential -- it underlies
capabilities being offered in the cloud. The four basic aaS's --
Software (SaaS), Platform (Paas), Infrastructure (IaaS) and Network
(NaaS) -- are introduced with Research aaS and other capabilities (for
example Sensors aaS are discussed later) being built on top of these.

\video{cloud}{12:01}{What is Cloud Computing Intro}{https://drive.google.com/file/d/0B1Of61fJF7WsdDdsYkw0dXdHS1U/view?usp=sharing}

\subsection{What and Why is Cloud Computing: Other Views I}

This lesson contains 5 slides with diverse comments on `'what is cloud
computing'' from the web.


\video{cloud}{5:25}{Other Views I}{https://drive.google.com/file/d/0B1Of61fJF7WsNm1jVVJMUVpCUlU/view?usp=sharing}

\video{cloud}{6:41}{Other Views II}{https://drive.google.com/file/d/0B1Of61fJF7WsV1RJcldzRlctRlk/view?usp=sharing}

\video{cloud}{7:27}{Other Views III}{https://drive.google.com/file/d/0B1Of61fJF7WsOUlxVHZ4MlN0RXc/view?usp=sharing}


\subsection{Gartner's Emerging Technology Landscape for Clouds and
  Big Data}\label{gartners-emerging-technology-landscape-for-clouds-and-big-data}

This lesson gives Gartner's projections around futures of cloud and Big
data. We start with a review of hype charts and then go into detailed
Gartner analyses of the Cloud and Big data areas. Big data itself is at
the top of the hype and by definition predictions of doom are emerging.
Before too much excitement sets in, note that spinach is above clouds
and Big data in Google trends.


\video{cloud}{11:26}{Gartners Emerging Technology Landscape}{https://drive.google.com/file/d/0B1Of61fJF7WsaTg5aEZ0cHJuM0k/view?usp=sharing}

\subsection{Simple Examples of use of Cloud Computing}

This short lesson gives two examples of rather straightforward
commercial applications of cloud computing. One is server consolidation
for multiple Microsoft database applications and the second is the
benefits of scale comparing gmail to multiple smaller installations. It
ends with some fiscal comments.


\video{cloud}{3:26}{Examples}{https://drive.google.com/file/d/0B1Of61fJF7WsLTBoM0NpYzVxOHc/view?usp=sharing}

\subsection{Value of Cloud Computing}\label{value-of-cloud-computing}

Some comments on fiscal value of cloud computing.


\video{cloud}{4:20}{Value of Cloud Computing}{https://drive.google.com/file/d/0B1Of61fJF7WsSFdfZ0hodDlnUGM/view?usp=sharing}

\subsection{Resources}

\begin{itemize}
\item
  \url{http://www.slideshare.net/woorung/trend-and-future-of-cloud-computing}
\item
  \url{http://www.slideshare.net/JensNimis/cloud-computing-tutorial-jens-nimis}
\item
  \url{https://setandbma.wordpress.com/2012/08/10/hype-cycle-2012-emerging-technologies/}
\item
  \url{http://insights.dice.com/2013/01/23/big-data-hype-is-imploding-gartner-analyst-2/}
\item
  \url{http://research.microsoft.com/pubs/78813/AJ18_EN.pdf}
\item
  \url{http://static.googleusercontent.com/media/www.google.com/en//green/pdfs/google-green-computing.pdf}
\end{itemize}

\section{Software and Systems}

We cover different views as to nature of architecture and application
for Cloud Computing. Then we discuss cloud software for the cloud
starting at virtual machine management (IaaS) and the broad Platform
(middleware) capabilities with examples from Amazon and academic
studies. We summarize the 21 layers and almost 300 software packages in
the HPC-ABDS Software Stack explaining how they are used.


\slides{Cloud}{32}{Software and Systems}{https://drive.google.com/file/d/0B8936_ytjfjmUHlEVG1wSUhDNnM/view?usp=sharing}

\subsection{What is Cloud Computing}\label{what-is-cloud-computing}

This lesson gives some general remark of cloud systems from an
architecture and application perspective.

\video{cloud}{6:20}{What is Cloud Computing}{https://drive.google.com/file/d/0B1Of61fJF7WsYlRhOHU5ci1seXc/view?usp=sharing}

\subsection{Introduction to Cloud Software Architecture: IaaS and  PaaS I}

We discuss cloud software for the cloud starting at virtual machine
management (IaaS) and the broad Platform (middleware) capabilities with
examples from Amazon and academic studies. We cover different views as
to nature of architecture and application for Cloud Computing. Then we
discuss cloud software for the cloud starting at virtual machine
management (IaaS) and the broad Platform (middleware) capabilities with
examples from Amazon and academic studies. We summarize the 21 layers
and almost 300 software packages in the HPC-ABDS Software Stack
explaining how they are used.


\video{cloud}{7:42}{Intro to IaaS and PaaS I}{https://drive.google.com/file/d/0B1Of61fJF7WsUm1XanBaaWtpQWM/view?usp=sharing}

\video{cloud}{6:42}{Intro to IaaS and PaaS II}{https://drive.google.com/file/d/0B1Of61fJF7WsMXpfTTlvNDBkbTQ/view?usp=sharing}

We discuss cloud software for the cloud starting at virtual machine
management (IaaS) and the broad Platform (middleware) capabilities with
examples from Amazon and academic studies. We cover different views as
to nature of architecture and application for Cloud Computing. Then we
discuss cloud software for the cloud starting at virtual machine
management (IaaS) and the broad Platform (middleware) capabilities with
examples from Amazon and academic studies. We summarize the 21 layers
and almost 300 software packages in the HPC-ABDS Software Stack
explaining how they are used.



\video{cloud}{7:42}{Software Architecture: }{https://youtu.be/1AnyJYyh490}

\video{cloud}{6:43}{IaaS and PaaS II: }{https://youtu.be/hVpFAUHcAd4}


\subsection{Using the HPC-ABDS Software Stack}

Using the HPC-ABDS Software Stack.


\video{cloud}{27:50}{ABDS}{https://drive.google.com/file/d/0B1Of61fJF7WsUTdlNmlYWDUyTlE/view?usp=sharing}

\subsection{Resources}

\begin{itemize}
\item
  \url{http://www.slideshare.net/JensNimis/cloud-computing-tutorial-jens-nimis}
\item
  \url{http://research.microsoft.com/en-us/people/barga/sc09_cloudcomp_tutorial.pdf}
\item
  \url{http://research.microsoft.com/en-us/um/redmond/events/cloudfutures2012/tuesday/Keynote_OpportunitiesAndChallenges_Yousef_Khalidi.pdf}
\item
  \url{http://cloudonomic.blogspot.com/2009/02/cloud-taxonomy-and-ontology.html}
\end{itemize}

\section{Architectures,  Applications and Systems}

We start with a discussion of Cloud (Data Center) Architectures with
physical setup, Green Computing issues and software models. We summarize
a 2014 Gartner analysis of Cloud computing providers. This is followed
by applications on the cloud including data intensive problems,
comparison with high performance computing, science clouds and the
Internet of Things. Remarks on Security, Fault Tolerance and
Synchronicity issues in cloud follow.


\slides{Cloud}{64}{Architectures}{https://drive.google.com/file/d/0B8936_ytjfjmTHlzcGN3SzFNTTA/view?usp=sharing}

\subsection{Cloud (Data Center) Architectures}

Some remarks on what it takes to build (in software) a cloud ecosystem,
and why clouds are the data center of the future are followed by
pictures and discussions of several data centers from Microsoft (mainly)
and Google. The role of containers is stressed as part of modular data
centers that trade scalability for fault tolerance. Sizes of cloud
centers and supercomputers are discussed as is ``green'' computing.


\video{cloud}{8:38}{Coud Architecture}{https://drive.google.com/file/d/0B1Of61fJF7WsYkxKelV2bTlMZ1k/view?usp=sharing}

\video{cloud}{9:59}{Cloud Data Center Architecture}{https://drive.google.com/file/d/0B1Of61fJF7WsRHJhN3VMaDJLTG8/view?usp=sharing}

\subsection{Analysis of Major Cloud Providers}\label{analysis-of-major-cloud-providers}

Gartner 2014 Analysis of leading cloud providers.

\video{cloud}{21:40}{Analysis of Major Cloud Providers}{https://drive.google.com/file/d/0B1Of61fJF7WsUXBjRUJpX1BaSjA/view?usp=sharing}

\subsection{Commercial Cloud Storage Trends}\label{commercial-cloud-storage-trends}

Use of Dropbox, iCloud, Box etc.


\video{cloud}{3:07}{Commercial Storage Trends}{https://drive.google.com/file/d/0B1Of61fJF7WsZjR5VHQ2MXFmbjg/view?usp=sharing}

\subsection{Cloud Applications I}\label{cloud-applications-i}

This short lesson discusses the need for security and issues in its
implementation. Clouds trade scalability for greater possibility of
faults but here clouds offer good support for recovery from faults. We
discuss both storage and program fault tolerance noting that parallel
computing is especially sensitive to faults as a fault in one task will
impact all other tasks in the parallel job.


\video{cloud}{7:57}{Cloud Applications I}{https://drive.google.com/file/d/0B1Of61fJF7WsYXlKVXk0aG8tZFk/view?usp=sharing}

\video{cloud}{7:44}{Cloud Applications II}{https://drive.google.com/file/d/0B1Of61fJF7WseGVUNHhGTHpZbVU/view?usp=sharing}

\subsection{Science Clouds}\label{science-clouds}

Science Applications and Internet of Things.

\video{cloud}{19:26}{Science Clouds}{https://drive.google.com/file/d/0B1Of61fJF7Wsd0lZejhPTkItZEE/view?usp=sharing}

\subsection{Security}\label{security}

This short lesson discusses the need for security and issues in its
implementation.


\video{cloud}{2:34}{Security}{https://drive.google.com/file/d/0B1Of61fJF7WsajE4QkljRUExLWM/view?usp=sharing}

\subsection{Comments on Fault Tolerance and Synchronicity Constraints}\label{comments-on-fault-tolerance-and-synchronicity-constraints}

Clouds trade scalability for greater possibility of faults but here
clouds offer good support for recovery from faults. We discuss both
storage and program fault tolerance noting that parallel computing is
especially sensitive to faults as a fault in one task will impact all
other tasks in the parallel job.

\video{cloud}{8:55}{Comments on Fault Tolerance and Synchronicity Constraints}{https://drive.google.com/file/d/0B1Of61fJF7WsdHRZV1VrTklWYVE/view?usp=sharing}

\subsection{Resources}

\begin{itemize}
\item
  \url{http://www.slideshare.net/woorung/trend-and-future-of-cloud-computing}
\item
  \url{http://www.eweek.com/c/a/Cloud-Computing/AWS-Innovation-Means-Cloud-Domination-307831}
\item
  CSTI General Assembly 2012, Washington, D.C., USA Technical Activities
  Coordinating Committee (TACC) Meeting, Data Management, Cloud
  Computing and the Long Tail of Science October 2012 Dennis Gannon.
\item
  \url{http://research.microsoft.com/en-us/um/redmond/events/cloudfutures2012/tuesday/Keynote_OpportunitiesAndChallenges_Yousef_Khalidi.pdf}
\item
  \url{http://www.datacenterknowledge.com/archives/2011/05/10/uptime-institute-the-average-pue-is-1-8/}
\item
  \url{https://loosebolts.wordpress.com/2008/12/02/our-vision-for-generation-4-modular-data-centers-one-way-of-getting-it-just-right/}
\item
  \url{http://www.mediafire.com/file/zzqna34282frr2f/koomeydatacenterelectuse2011finalversion.pdf}
\item
  \url{http://www.slideshare.net/JensNimis/cloud-computing-tutorial-jens-nimis}
\item
  \url{http://www.slideshare.net/botchagalupe/introduction-to-clouds-cloud-camp-columbus}
\item
  \url{http://www.venus-c.eu/Pages/Home.aspx}
\item
  Geoffrey Fox and Dennis Gannon Using Clouds for Technical Computing To
  be published in Proceedings of HPC 2012 Conference at Cetraro, Italy
  June 28 2012
  \url{http://grids.ucs.indiana.edu/ptliupages/publications/Clouds_Technical_Computing_FoxGannonv2.pdf}
\item
  \url{https://berkeleydatascience.files.wordpress.com/2012/01/20120119berkeley.pdf}
\item
  Taming The Big Data Tidal Wave: Finding Opportunities in Huge Data
  Streams with Advanced Analytics, Bill Franks Wiley ISBN:
  978-1-118-20878-6
\item
  Anjul Bhambhri, VP of Big Data, IBM
  \url{http://fisheritcenter.haas.berkeley.edu/Big_Data/index.html}
\item
  Conquering Big Data with the Oracle Information Model, Helen Sun,
  Oracle
\item
  Hugh Williams VP Experience, Search \& Platforms, eBay
  \url{http://businessinnovation.berkeley.edu/fisher-cio-leadership-program/}
\item
  Dennis Gannon, Scientific Computing Environments,
  \url{http://www.nitrd.gov/nitrdgroups/images/7/73/D_Gannon_2025_scientific_computing_environments.pdf}
\item
  \url{http://research.microsoft.com/en-us/um/redmond/events/cloudfutures2012/tuesday/Keynote_OpportunitiesAndChallenges_Yousef_Khalidi.pdf}
\item
  \url{http://www.datacenterknowledge.com/archives/2011/05/10/uptime-institute-the-average-pue-is-1-8/}
\item
  \url{https://loosebolts.wordpress.com/2008/12/02/our-vision-for-generation-4-modular-data-centers-one-way-of-getting-it-just-right/}
\item
  \url{http://www.mediafire.com/file/zzqna34282frr2f/koomeydatacenterelectuse2011finalversion.pdf}
\item
  \url{http://searchcloudcomputing.techtarget.com/feature/Cloud-computing-experts-forecast-the-market-climate-in-2014}
\item
  \url{http://www.slideshare.net/botchagalupe/introduction-to-clouds-cloud-camp-columbus}
\item
  \url{http://www.slideshare.net/woorung/trend-and-future-of-cloud-computing}
\item
  \url{http://www.venus-c.eu/Pages/Home.aspx}
\item
  \url{http://www.kpcb.com/internet-trends}
\end{itemize}

\section{Data Systems}

We describe the way users and data interact with a cloud system. The
unit concludes with the treatment of data in the cloud from an
architecture perspective and Big Data Processing from an application
perspective with commercial examples including eBay.

\slides{Cloud}{49}{Data Systems}{https://drive.google.com/file/d/0B1Of61fJF7WsN1RPVFRLUGJLZGs/view?usp=sharing}

\subsection{The 10 Interaction scenarios (access patterns)
I}\label{the-10-interaction-scenarios-access-patterns-i}

The next 3 lessons describe the way users and data interact with the
system.

\video{cloud}{10:26}{The 10 Interaction scenarios I}{https://drive.google.com/file/d/0B1Of61fJF7WsWldDNm1oNXdPQmc/view?usp=sharing}

\subsection{The 10 Interaction scenarios. Science Examples}

This lesson describes the way users and data interact with the system
for some science examples.

\video{cloud}{16:34}{The 10 Interaction scenarios. Science Examples}{https://drive.google.com/file/d/0B1Of61fJF7WsQTlvLWs4cm5NRE0/view?usp=sharing}

\subsection{Remaining general access patterns}

This lesson describe the way users and data interact with the system for
the final set of examples.


\video{cloud}{11:36}{Access Patterns}{https://drive.google.com/file/d/0B1Of61fJF7WsYVVRWmdpanV4Vlk/view?usp=sharing}

\subsection{Data in the Cloud}\label{data-in-the-cloud}

Databases, File systems, Object Stores and NOSQL are discussed and
compared. The way to build a modern data repository in the cloud is
introduced.

\video{cloud}{10:24}{Data in the Cloud}{https://drive.google.com/file/d/0B1Of61fJF7WsRzR6eHZwelVuOG8/view?usp=sharing}

\subsection{Applications Processing Big Data}

This lesson collects remarks on Big data processing from several
sources: Berkeley, Teradata, IBM, Oracle and eBay with architectures and
application opportunities.

\video{cloud}{8:45}{Processing Big Data}{https://drive.google.com/file/d/0B1Of61fJF7WsUG9UVGFOQXNXbnc/view?usp=sharing}

\section{Resources}

\begin{itemize}
\item
  \url{http://bigdatawg.nist.gov/_uploadfiles/M0311_v2_2965963213.pdf}
\item
  \url{https://dzone.com/articles/hadoop-t-etl}
\item
  \url{http://venublog.com/2013/07/16/hadoop-summit-2013-hive-authorization/}
\item
  \url{https://indico.cern.ch/event/214784/session/5/contribution/410}
\item
  \url{http://asd.gsfc.nasa.gov/archive/hubble/a_pdf/news/facts/FS14.pdf}
\item
  \url{http://blogs.teradata.com/data-points/announcing-teradata-aster-big-analytics-appliance/}
\item
  \url{http://wikibon.org/w/images/2/20/Cloud-BigData.png}
\item
  \url{http://hortonworks.com/hadoop/yarn/}
\item
  \url{https://berkeleydatascience.files.wordpress.com/2012/01/20120119berkeley.pdf}
\item
  \url{http://fisheritcenter.haas.berkeley.edu/Big_Data/index.html}
\end{itemize}
