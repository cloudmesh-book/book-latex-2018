\CHANGE
\FILENAME

\section{Advanced Coud Computing and Software Defined Systems}

\TODO{THIS WILL CHANGE}

This will significantly change.

Course number at Indiana University ENGR-E516 - 2018

\subsection{Course Description}

This course describes the emerging world of distributed intelligent
systems where each system component has internal, and external sources
of intelligence that are subject to collective control. Examples are
given from computers, networks, vehicle systems, digital
manufacturing, and robotics. Performance and fault tolerance and
system management are discussed.

\subsection{Course pre-requisites}

Python will be used as a Programming Languages. In some cases Java may
also be useful however it's use in class will be marginal. One of
I523, I524, ENGR-E516, E616 or existing knowledge to introduction to
cloud computing recommended. The class will be using Linux commandline
tools. Prior knowledge of Linux is of advantage but not required.
Studnets are expected to have access to a computer on which they can
execute Linux easily. As the OS requirements have recently increased
we recommend a computer with 8GB main memory and the ability to run
virtualbox and/or containers. If it turns out your machine is not
capable enough we attempt to provide access to IU linux machines.

\subsection{Justification}

Intelligent Systems Engineering is a department and a new major at
Indiana University. This course is for students who are interested in
any component of the Masters or Ph.D. in Intelligent Systems
Engineering. It is an advanced elective class.

\subsection{Teaching and learning methods}

\begin{itemize}
\item Lectures
\item Assignments including specific lab activities
\item Final project 
\end{itemize}

\subsection{Representative bibliography}

\begin{itemize}
\item	Machine to machine protocols \url{https://en.wikipedia.org/wiki/MQTT}
\item	Cloud software systems \url{http://hpc-abds.org/kaleidoscope/}
\item	GE Software defined machines \url{https://www.ge.com/digital/blog/software-defined-machines}
\item	Software Defined Networks \url{https://en.wikipedia.org/wiki/Software-defined_networking}
\item	There are a huge number of other web resources
\end{itemize}

\subsection{Student learning outcomes}

When students complete this course, they should be able to:

\begin{itemize}
\item	Have an advanced understanding of issues involved in designing and Software Defined Systems using the latest network and cloud technologies.
\item	Gain hands-on laboratory experience with several examples from academia and industry.
\item	Understand the application of Apache Big Data Software Stack and DevOps to Software Defined systems.
\item	Apply knowledge of mathematics, science, and engineering
\item	Understand research challenges and important issues with Software Defined Systems.
\item	Have advanced skills in teamwork with peers.
\end{itemize}

\subsection{Grading}

Grade Item	Percentage

\begin{tabular}{lr}
Assignments	  & 30\% \\
Final Project	& 60\% \\
Participation	& 10\% \\
\end{tabular}


\begin{table}[h]
\centering
\caption{This will change}
\label{T:e621}
\begin{tabular}{p{1cm}p{4cm}p{6cm}p{2cm}}
Weeks & Topic                    & Details and Activities  & Assignments          \\
\hline
1     & Introduction             & Software Defined Systems from Computers to Machines                                                                  & Problem set          \\
2     & Use case I               & Computers and Networks as Software Defined Systems                                                                   & Problem set          \\
3     & Big Data on Clouds       & Introduction to Apache Big Data Stack ABDS : Architecture and selected members                                       & Problem Set          \\
4     & Software Definitions I   & DevOps. Ansible, and other tools for software definition of systems                                                  & Problem Set          \\
5     & Solutions I              & Introduce Virtual Software Laboratory for software defined computers and networks                                    & Lab work             \\
6     & Software Definitions II  & OpenFlow and other tools for software definition of networks                                                         & Lab work             \\
7     & Use case II              & Industrial Internet of things and Industrial software defined machines with examples from GE, Johnson Controls, etc. & Problem set          \\
8     & Prepare for project      & Students read papers, discuss use cases, solutions. Get project started                                              & Define Project       \\
9     & Communication Techniques & MQTT and other communication protocols. Peer to Peer versus centralized coordination                                 & Problem Set          \\
10    & Use case II              & Linking instruments to the cloud – examples of lab instruments                                                       & Project Work         \\
11    & Use case III             & Vehicle networks and Robots                                                                                          & Problem Set          \\
12    & Integrated Systems       & Large scale integration and Systems Management                                                                       & Problem Set          \\
13    & Software Definitions II  & Other Software defined system specification languages                                                                & Project Work         \\
14    & System Properties        & Fault Tolerance and Performance                                                                                      & Project Work         \\
15    & Project Reports          & Demonstrate student projects                                                                                         & Final Project Report
\end{tabular}
\end{table}