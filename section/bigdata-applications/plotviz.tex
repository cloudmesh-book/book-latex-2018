

\chapter{Technology Training - Plotviz}\label{technology-training---plotviz}

\FILENAME

We introduce Plotviz, a data visualization tool developed at Indiana
University to display 2 and 3 dimensional data. The motivation is that
the human eye is very good at pattern recognition and can `'see''
structure in data. Although most Big data is higher dimensional than 3,
all can be transformed by dimension reduction techniques to 3D. He gives
several examples to show how the software can be used and what kind of
data can be visualized. This includes individual plots and the
manipulation of multiple synchronized plots.Finally, he describes the
download and software dependency of Plotviz.

\section{Using Plotviz Software for Displaying Point Distributions in
3D}\label{using-plotviz-software-for-displaying-point-distributions-in-3d}

We introduce Plotviz, a data visualization tool developed at Indiana
University to display 2 and 3 dimensional data. The motivation is that
the human eye is very good at pattern recognition and can `'see''
structure in data. Although most Big data is higher dimensional than 3,
all can be transformed by dimension reduction techniques to 3D. He gives
several examples to show how the software can be used and what kind of
data can be visualized. This includes individual plots and the
manipulation of multiple synchronized plots. Finally, he describes the
download and software dependency of Plotviz.

\slides{Plotviz}{34}{Plotvisz}{https://iu.app.box.com/s/jypomnrz755xgps5e6iw}

Files:


\sourcecode{Plotviz}{Fungi-LSU-3-15-to-3-26-zeroidx.pviz}{examples/python/plotviz/fungi-lsu-3-15-to-3-26-zeroidx.pviz}

\sourcecode{Plotviz}{DatingRatings-OriginalLabels.pviz}{examples/plotviz/datingrating-originallabels.pviz}

\sourcecode{Plotviz}{ClusterFinal-M30-C28.pviz{examples/plotviz/clusterFinal-M30-C28.pviz}

\sourcecode{Plotviz}{clusterFinal-M3-C3Dating-ReClustered.pviz}{examples/plotviz/clusterfinal-m3-c3dating-reclustered.pviz}


\subsection{Motivation and Introduction to
use}\label{motivation-and-introduction-to-use}

The motivation of Plotviz is that the human eye is very good at pattern
recognition and can `'see'' structure in data. Although most Big data is
higher dimensional than 3, all data can be transformed by dimension
reduction techniques to 3D and one can check analysis like clustering
and/or see structure missed in a computer analysis. The motivations
shows some Cheminformatics examples. The use of Plotviz is started in
slide 4 with a discussion of input file which is either a simple text or
more features (like colors) can be specified in a rich XML syntax.
Plotviz deals with points and their classification (clustering). Next
the protein sequence browser in 3D shows the basic structure of Plotviz
interface. The next two slides explain the core 3D and 2D manipulations
respectively. Note all files used in examples are available to students.

Video: 7:58: Motivation: \url{http://youtu.be/4aQlCmQ1jfY}

\subsection{Example of Use I: Cube and Structured
Dataset}\label{example-of-use-i-cube-and-structured-dataset}

Initially we start with a simple plot of 8 points -- the corners of a
cube in 3 dimensions -- showing basic operations such as
size/color/labels and Legend of points. The second example shows a
dataset (coming from GTM dimension reduction) with significant
structure. This has .pviz and a .txt versions that are compared.

Video: 9:45: Example I: \url{http://youtu.be/nCTT5mI_j_Q}

\subsection{Example of Use II: Proteomics and Synchronized
Rotation}\label{example-of-use-ii-proteomics-and-synchronized-rotation}

This starts with an examination of a sample of Protein Universe Browser
showing how one uses Plotviz to look at different features of this set
of Protein sequences projected to 3D. Then we show how to compare two
datasets with synchronized rotation of a dataset clustered in 2
different ways; this dataset comes from k Nearest Neighbor discussion.

Video: 9:14: Proteomics and Synchronized Rotation:
\url{http://youtu.be/lDbIhnLrNkk}

\subsection{Example of Use III: More Features and larger Proteomics
Sample}\label{example-of-use-iii-more-features-and-larger-proteomics-sample}

This starts by describing use of Labels and Glyphs and the Default mode
in Plotviz. Then we illustrate sophisticated use of these ideas to view
a large Proteomics dataset.

Video: 8:37: Larger Proteomics Sample: \url{http://youtu.be/KBkUW_QNSvs}

\subsection{Example of Use IV: Tools and
Examples}\label{example-of-use-iv-tools-and-examples}

This lesson starts by describing the Plotviz tools and then sets up two
examples -- Oil Flow and Trading -- described in PowerPoint. It finishes
with the Plotviz viewing of Oil Flow data.

Video: 10:17: Plotviz I: \url{http://youtu.be/zp_709imR40}

\subsection{Example of Use V: Final
Examples}\label{example-of-use-v-final-examples}

This starts with Plotviz looking at Trading example introduced in
previous lesson and then examines solvent data. It finishes with two
large biology examples with 446K and 100K points and each with over 100
clusters. We finish remarks on Plotviz software structure and how to
download. We also remind you that a picture is worth a 1000 words.

Video: 14:58: Plotviz II \url{http://youtu.be/FKoCfTJ_cDM}

\subsection{Resources}\label{resources}

Download files from \url{http://salsahpc.indiana.edu/pviz3/}
