\section{XGBoost} 

``XGBoost is an open-source software library which provides the gradient
boosting framework for C++, Java, Python,R, and Julia'' ~\cite{hid-sp18-401
-XGBoost-wiki}.

XGBoost stands for Extreme Gradient Boosting. Before talking about XGBoost, it
is best to give introduction to general gradient boosting. Gradient Boosting
is a machine learning technique used to build both regression and
classification models. It is primarily used in building decision trees. But
building gradient boosting models on huge datasets (that sometimes contain
more than 500,000 observations) is computationally onerous, not so efficient.
``The name xgboost, though, actually refers to the engineering goal to push
the limit of computations resources for boosted tree algorithms. Which is the
reason why many people use xgboost''.  - says Tianqi Chen, creator of XGBoost
(later received contributions from many developers)~\cite{hid-sp18-401
-XGBoost-gen}. The description of XGBoost according to the software repository
on github is ``Scalable, Portable and Distributed Gradient Boosting (GBDT,
GBRT or GBM) Library, for Python, R, Java, Scala, C++ and more''~\cite{hid-
sp18-401-XGBoost-git}.

