\section{Amazon RDS}

Amazon RDS~\cite{hid-sp18-420-amazon-RDS} stands for Amazon Relational Database
Service. Amazon RDS gives access to MySQL, MariaDB, Oracle, SQL Server, or
PostgreSQL database. It is a managed service provided by AWS which can be used
to manage different database administrative tasks. User can select the type of
RDS instance and accordingly AWS provides capacity. RDS has capacity to resize
as per requirement which enables user to change from one instance type to
another instance type without losing its data. It is cost effective and the
costing depends on the instance type~\cite{hid-sp18-420-amazon-RDS-FAQ}.

``Amazon RDS can automatically backup database and keep that database software
up to date with its latest version. RDS makes it easy to use replication to
enhance database availability, improve data durability, or scale beyond the
capacity constraints of a single database instance for read-heavy database
workloads''~\cite{hid-sp18-420-amazon-RDS-FAQ}. High availability is achieved by
built-in automated failover from primary database to a replicated secondary
database in case of any failure. This replicated secondary database in sync with
primary database.
