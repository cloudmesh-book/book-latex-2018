\chapter{Scipy}
\label{c:scipy}

SciPy is a library built above numpy and has a number of off the shelf
algorithms and operations implemented. These include algorithms from
calculus (such as integration), statistics, linear algebra,
image-processing, signal processing, machine learning.

To achieve this, SciPy bundels a number of useful open-source software
for mathematics, science, and engineering. It includes the follwoing
packages:

\begin{description}

\item[NumPy,] for managin N-dimensional arrays
\item[SciPy library,] to access fundamental scientific computing capabilities
\item[Matplotlib,] to conduct 2D plotting
\item[IPython,] for an Interactive console (see jupyter)
\item[Sympy,] for symbolic mathematics
\item[pandas,] for providing data structures and analysis

\end{description}


\section{Introduction}

First we add the usual scientific computing modules with the typical
abbreviations, including sp for scipy. We could invoke scipy's
statistical package as sp.stats, but for the sake of laziness we
abbreviate that too.

\begin{verbatim}
import numpy as np # import numpy
import scipy as sp # import scipy
from scipy import stats # refer directly to stats rather than sp.stats
import matplotlib as mpl # for visualization
from matplotlib import pyplot as plt # refer directly to pyplot 
                                     # rather than mpl.pyplot
\end{verbatim}

Now we create some random data to play with. We generate 100 samples
from a Gaussian distribution centered at zero.

\begin{verbatim}
s = sp.randn(100)
\end{verbatim}

How many elements are in the set?

\begin{verbatim}
print ('There are',len(s),'elements in the set')
\end{verbatim}

What is the mean (average) of the set?

\begin{verbatim}
print ('The mean of the set is',s.mean())
\end{verbatim}

What is the minimum of the set?

\begin{verbatim}
print ('The minimum of the set is',s.min())
\end{verbatim}

What is the maximum of the set?

\begin{verbatim}
print ('The maximum of the set is',s.max())
\end{verbatim}

We can use the scipy functions too. What's the median?

\begin{verbatim}
print ('The median of the set is',sp.median(s))
\end{verbatim}

What about the standard deviation and variance?

\begin{verbatim}
print ('The standard deviation is',sp.std(s),
       'and the variance is',sp.var(s))
\end{verbatim}

Isn't the variance the square of the standard deviation?

\begin{verbatim}
print ('The square of the standard deviation is',sp.std(s)**2)
\end{verbatim}

How close are the measures? The differences are close as the following
calculation shows

\begin{verbatim}
print ('The difference is',abs(sp.std(s)**2 - sp.var(s)))
\end{verbatim}

\begin{verbatim}
print ('And in decimal form, the difference is %0.16f' % 
       (abs(sp.std(s)**2 - sp.var(s))))
\end{verbatim}

How does this look as a histogram?

\begin{verbatim}
plt.hist(s) # yes, one line of code for a histogram
plt.show()
\end{verbatim}

\begin{center}
\includegraphics[width=0.5\textwidth]{scipy/output_23_0.png}
\end{center}

Let's add some titles.

\begin{verbatim}
plt.clf() # clear out the previous plot

plt.hist(s)
plt.title("Histogram Example")
plt.xlabel("Value")
plt.ylabel("Frequency")

plt.show()
\end{verbatim}

\begin{center}
\includegraphics[width=0.5\textwidth]{scipy/output_25_0.png}
\end{center}

Typically we do not include titles when we prepare images for inclusion
in LaTeX. There we use the caption to describe what the figure is about.

\begin{verbatim}
plt.clf() # clear out the previous plot

plt.hist(s)
plt.xlabel("Value")
plt.ylabel("Frequency")

plt.show()
\end{verbatim}

\begin{center}
\includegraphics[width=0.5\textwidth]{scipy/output_27_0.png}
\end{center}

Let's try out some linear regression, or curve fitting.

\begin{verbatim}
import random

def F(x):
    return 2*x - 2

def add_noise(x):
    return x + random.uniform(-1,1) 
\end{verbatim}

\begin{verbatim}
X = range(0,10,1)

Y = []
for i in range(len(X)):
    Y.append(add_noise(X[i]))

plt.clf() # clear out the old figure
plt.plot(X,Y,'.')
plt.show()
\end{verbatim}

\begin{center}
\includegraphics[width=0.5\textwidth]{scipy/output_30_0.png}
\end{center}

Now let's try linear regression to fit the curve.

\begin{verbatim}
m, b, r, p, est_std_err = stats.linregress(X,Y)
\end{verbatim}

What is the slope and y-intercept of the fitted curve?

\begin{verbatim}
print ('The slope is',m,'and the y-intercept is', b)
\end{verbatim}

\begin{verbatim}
def Fprime(x): # the fitted curve
    return m*x + b
\end{verbatim}

Now let's see how well the curve fits the data. We'll call the fitted
curve F'.

\begin{verbatim}
X = range(0,10,1)

Yprime = []
for i in range(len(X)):
    Yprime.append(Fprime(X[i]))
\end{verbatim}

\begin{verbatim}
plt.clf() # clear out the old figure

# the observed points, blue dots
plt.plot(X, Y, '.', label='observed points') 

# the interpolated curve, connected red line
plt.plot(X, Yprime, 'r-', label='estimated points')  

plt.title("Linear Regression Example") # title
plt.xlabel("x") # horizontal axis title
plt.ylabel("y") # vertical axis title
# legend labels to plot
plt.legend(['obsered points', 'estimated points']) 

# comment out so that you can save the figure
#plt.show()
\end{verbatim}

To save images into a PDF file for inclusion into \LaTeX~documents you
can save the images as follows. Other formats such as png are also
possible, but the quality is naturally not sufficient for inclusion in
papers and documents. For that you certainly want to use PDF. The save
of the figure has to occure before you use the \verb|show()| command.

\begin{verbatim}
plt.savefig("regression.pdf", bbox_inches='tight')
\end{verbatim}

\begin{verbatim}
plt.savefig('regression.png') 
\end{verbatim}

\begin{verbatim}
plt.show()
\end{verbatim}

\begin{center}
\includegraphics[width=0.5\textwidth]{scipy/output_40_0.png}
\end{center}


\section{References}\label{references}

For more information about SciPy we recommend that you visit the following link

\url{https://www.scipy.org/getting-started.html#learning-to-work-with-scipy}

Additional material and inspiration for this section are from

\begin{itemize}
\item ``Getting Started guide'' \url{https://www.scipy.org/getting-started.html}
\item
  Prasanth. ``Simple statistics with SciPy.'' Comfort at 1 AU. 
  February 28, 2011.   \url{https://oneau.wordpress.com/2011/02/28/simple-statistics-with-scipy/}.
\item
  SciPy Cookbook. Lasted updated: 2015. 
  \url{http://scipy-cookbook.readthedocs.io/}. 
\end{itemize}
